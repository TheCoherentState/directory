\documentclass[10pt, oneside]{amsart} 
\usepackage{amsmath, amsthm, amssymb, wasysym, verbatim, bbm, color, graphics, geometry, hyperref, biblatex, mathtools}
\usepackage{tcolorbox}

\hypersetup{
	colorlinks=true,
	linkcolor=blue,
	urlcolor=blue
}


\geometry{tmargin=1.25in, bmargin=1.25in, lmargin=1.25in, rmargin =1.25in}
\setlength\parindent{0pt}

\tcbuselibrary{theorems}
\newtcbtheorem
    []% init options
    {problem}% name
    {Problem}% title
    {%
      fonttitle=\bfseries,
    }% options
    {prob}% prefix

    \newcommand{\R}{\mathbb{R}}
    \newcommand{\C}{\mathbb{C}}
    \newcommand{\Z}{\mathbb{Z}}
    \newcommand{\N}{\mathbb{N}}
    \newcommand{\Q}{\mathbb{Q}}
    \newcommand{\Cdot}{\boldsymbol{\cdot}}
    \newcommand{\U}{\mathcal{U}}
    \newcommand{\V}{\mathcal{V}}

    \newtheorem{thm}{Theorem}
    \newtheorem{defn}{Definition}
    \newtheorem{conv}{Convention}
    \newtheorem{rem}{Remark}
    \newtheorem{lem}{Lemma}
    \newtheorem{cor}{Corollary}
    \newtheorem{prop}{Proposition}
    \newtheorem{prob}{Problem}

    \newcommand{\tr}{\mathrm{Tr}}
    \newcommand{\bm}{\boldsymbol}
    \DeclarePairedDelimiter\floor{\lfloor}{\rfloor}


    \title{Munkres Topology: Chapter 32}
    \author{Jack Ceroni}
    \date{December 2021}

    \begin{document}

    \maketitle

    \tableofcontents

    \vspace{.25in}

    \newpage

    \hrulefill

    \section{Problem 32.6}

    \textit{Show that $X$ is completely normal if and only if for every pair of sets $A$, $B$ with $A \cap \overline{B} = \overline{A} \cap B = \emptyset$, there exist disjoint open sets containing them.}
    \newline

    Suppose first that $X$ is completely normal. Note that $Y = X - (\overline{A} \cap \overline{B})$ is an open subspace of $X$. It is easy to see that both $A$ and $B$ are in $Y$: if this weren't the case then
    we could pick a point in both $A$ and $\overline{B}$, or $B$ and $\overline{A}$. Thus, we can pick open sets $U$ and $V$ in $Y$ containing $\overline{A} \cap Y$ and $\overline{B} \cap Y$, which themselves contain $A$ and $B$.
    \newline

    Since $Y$ is open, these sets will also be open in $X$.
    \newline

    Suppose the converse holds. Let $C$ be a subspace of $X$, and let $Q = C_1 \cap C$ and $P = C_2 \cap C$ be closed and disjoint in $C$, where $C_1$ and $C_2$ are closed
    in $X$. Clearly, $C_1$ and $C_2$ are separated, as they are equal to their closures.

    \hrulefill

    \end{document}
