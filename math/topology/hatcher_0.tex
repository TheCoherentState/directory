\documentclass[10pt, oneside]{amsart} 
\usepackage{amsmath, amsthm, amssymb, wasysym, verbatim, bbm, color, graphics, geometry, hyperref, biblatex, mathtools}
\usepackage{tcolorbox}
\usepackage{tikz-cd} 

\hypersetup{
	colorlinks=true,
	linkcolor=blue,
	urlcolor=blue
}


\geometry{tmargin=1.25in, bmargin=1.25in, lmargin=1.25in, rmargin =1.25in}
\setlength\parindent{0pt}

\tcbuselibrary{theorems}
\newtcbtheorem
    []% init options
    {problem}% name
    {Problem}% title
    {%
      fonttitle=\bfseries,
    }% options
    {prob}% prefix

    \newcommand{\R}{\mathbb{R}}
    \newcommand{\C}{\mathbb{C}}
    \newcommand{\Z}{\mathbb{Z}}
    \newcommand{\N}{\mathbb{N}}
    \newcommand{\Q}{\mathbb{Q}}
    \newcommand{\Cdot}{\boldsymbol{\cdot}}
    \newcommand{\U}{\mathcal{U}}
    \newcommand{\V}{\mathcal{V}}

    \newtheorem{thm}{Theorem}
    \newtheorem{defn}{Definition}
    \newtheorem{conv}{Convention}
    \newtheorem{rem}{Remark}
    \newtheorem{lem}{Lemma}
    \newtheorem{cor}{Corollary}
    \newtheorem{prop}{Proposition}
    \newtheorem{prob}{Problem}

    \newcommand{\tr}{\mathrm{Tr}}
    \newcommand{\bm}{\boldsymbol}
    \DeclarePairedDelimiter\floor{\lfloor}{\rfloor}


    \title{Hatcher Algebraic Topology: Chapter 0}
    \author{Jack Ceroni}
    \date{December 2021}

    \begin{document}

    \maketitle

    \tableofcontents

    \vspace{.25in}

    \newpage

    \hrulefill

    \section{Quotient Spaces}

    \textit{Strictly speaking, this is not a section in Hatcher. However, quotient spaces are so important that we will spend some time going over the basic results outlined in Munkres.}
    \newline

    

    \hrulefill

    \section{Cell Complexes}

    A cell-complex is a way that we can build a space by succesively gluing together $n$-dimensional disks. This can be accomplished as follows:

    \begin{itemize}
    \item We start with a collection of points, which we call $X^0$. We can regard each of these points as a $0$-cell.
    \item Glue lines (which we call $1$-cells) to the points in $X^0$.
      Formally, we denote these $1$-cells by $e^{1}_{\alpha}$. For each $1$-cell,
      we can define a corresponding attaching map $\phi_{\alpha} : S^{0} \rightarrow X^0$, which simply takes the boundary points of $e^{1}_{\alpha}$ and sends them to points in $X^0$.
      In this case, this will simply be the end points of the line. We then can define $X^1$ as:

      $$X^1 = \left( X^0 \displaystyle\bigsqcup_{\alpha} e^{1}_{\alpha} \right) / \sim$$

      \noindent
      where $\sim$ is the equivalence relation defined by identifying point $x \in X^0$ with all $y$ such that $\phi_{\alpha}(y) = x$, for some $\alpha$.
    \item Inductively continue the above procedure, for the case of $n$-cells, defined to by $n$-dimensional disks. Thus, a $2$-cell is an open disk, a $3$-cell an open, solid ball,
      and so on. We define each $X^n$ an an analogous way.
    \item Terminate this procedure after a finite number of steps (say, $N$), and take our final topological space to be $X^N$, or continue indefinitely and take the final topological
      space to be $\cup_{n \in \mathbb{Z}^{+}} X^{n}$.
    \end{itemize}

    One of the more interesting examples of cell-complexes in action is constructing projective spaces.
    \newline

    We define the real projective space $\mathbb{R} P^n$ to be the quotient space formed by taking all lines through the origin in $\mathbb{R}^{n + 1} - \{0\}$.
    That is to say, $x \sim y$ if and only if $x = \lambda y$ for some $\lambda \neq 0$. Clearly, this topological space is homeomorphic to $S^{n}$ with antipodal points identified, take:

    $$f : \mathbb{R}^{n + 1} - \{0\} \rightarrow S^{n} \ \ \ \ \ f(x) = \frac{x}{||x||}$$

    and let $\tilde{f} : \mathbb{R}P^n \rightarrow S^{n}/\sim$ be defined as $\tilde{f}([x]) = [f(x)]$, for $x \in \mathbb{R}P^n$, where we take $S^n/\sim$ to be a subspace of the projective space.
    It is easy to check this is a homeomorphism, using the facts about quotient spaces proved in the first section.
    \newline

    Now, brining together the antipodal points of $S^n$, it is clear that $S^n / \sim$ is homeomorphic to the disk $B^{n}$ with antipodal boundary points identified (we will not formally write out why this
    is the case, but it is geometrically obvious).
    \newline

    We wish to define a cell-structure which gives $\mathbb{R}P^{n}$. 
    
    \hrulefill

    \end{document}
