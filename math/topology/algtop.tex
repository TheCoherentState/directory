\documentclass[10pt, oneside]{amsart} 
\usepackage{amsmath, amsthm, amssymb, wasysym, verbatim, bbm, color, graphics, geometry, hyperref, biblatex, mathtools}
\usepackage{tcolorbox}

\hypersetup{
	colorlinks=true,
	linkcolor=blue,
	urlcolor=blue
}


\geometry{tmargin=1.25in, bmargin=1.25in, lmargin=1.25in, rmargin =1.25in}
\setlength\parindent{0pt}

\tcbuselibrary{theorems}
\newtcbtheorem
    []% init options
    {problem}% name
    {Problem}% title
    {%
      fonttitle=\bfseries,
    }% options
    {prob}% prefix

    \newcommand{\R}{\mathbb{R}}
    \newcommand{\C}{\mathbb{C}}
    \newcommand{\Z}{\mathbb{Z}}
    \newcommand{\N}{\mathbb{N}}
    \newcommand{\Q}{\mathbb{Q}}
    \newcommand{\Cdot}{\boldsymbol{\cdot}}
    \newcommand{\U}{\mathcal{U}}
    \newcommand{\V}{\mathcal{V}}

    \newtheorem{thm}{Theorem}
    \newtheorem{defn}{Definition}
    \newtheorem{conv}{Convention}
    \newtheorem{rem}{Remark}
    \newtheorem{lem}{Lemma}
    \newtheorem{cor}{Corollary}
    \newtheorem{prop}{Proposition}
    \newtheorem{prob}{Problem}
    \newtheorem{ex}{Example}

    \newcommand{\tr}{\mathrm{Tr}}
    \newcommand{\bm}{\boldsymbol}
    \DeclarePairedDelimiter\floor{\lfloor}{\rfloor}


    \title{Algebraic Topology}
    \author{Jack Ceroni}
    \date{November 2021}

    \begin{document}

    \maketitle

    \tableofcontents

    \vspace{.25in}

    \section{Introduction}

    Notes taken while studying algebraic topology in MAT327, in Fall 2021

    \newpage

    \section{Lecture 1: November 4th}

    \begin{rem}
      Two spaces are equivalent if they have the same shape.
    \end{rem}

    \begin{defn}[Homotopy Equivalence]
      Let $f$ and $f'$ be continuous maps from $X$ to $Y$. Then $f$ is said to be \textbf{homotopic} to $f'$ if there is a
      continuous map $F : X \times I \rightarrow Y$ such that $F(x, 0) = f(x)$, and
      $F(x, 1) = f'(x)$ for all $x \in X$. $F$ is called a \textbf{homotopy}.
    \end{defn}

    \begin{rem}
      This can be visualized as a continuous deformation from one map to another.
    \end{rem}

    \begin{defn}
      If $f'$ is constant, then $f$ is called \textbf{nullhomotopic}
    \end{defn}

    We are interested in situations in which $f$ and $f'$ are \textbf{paths}. More specifically, situations where $f$ and $f'$
    have the same starting point, and the same endpoint.
    If there is a homotopy that preserves the endpoints, then $f$ and $f'$ are said to be path-homotopic: $f \cong f'$.
    \newline

    \begin{lem}
      Path-homotopy and homotopy are equivalence relations. The equivalence classes are called homotopy classes and path-homotopy classes.
    \end{lem}

    \begin{proof}
      Clearly, reflexivity and symmetry hold (we simply ``run time in reverse'').
      \newline

      Suppose $f \simeq f'$, and $f' \simeq f''$. Let $F$ and $G$ be corresponding homotopies. Then define $H(x, t) = F(x, 2t)$ for
      $t \in \left[0, \frac{1}{2}\right]$ and $H(x, t) = G(x, 2t - 1)$ for $t \in \left[\frac{1}{2}, 1\right]$.
      The continuity of this map follows from the pasting lemma. Thus, $H$ is a homotopy for $f$ and $f''$.
    \end{proof}

    \begin{ex}
      Consider continuous maps $f, g : X \rightarrow \mathbb{R}^2$, with $X \subset \mathbb{R}$. $f$ is homotopic to $g$, as we can define the homotopy $F(x, t) = (1 - t) f(x) + t g(x)$. This is true for any convex subset of $\mathbb{R}^n$.
    \end{ex}

    \begin{ex}
      Consider the punctured plane $\mathbb{R}^2 - \{0\}$. Consider the paths $f(s) = (\cos \pi s, \sin \pi s)$, as well as $g(s) = (\cos \pi s, 2 \sin \pi 2)$ and $h(s) = (\cos \pi s, - \sin \pi s)$.
      \end{ex}

    \begin{defn}
      Let $f$ be a path from $x_0$ to $x_1$, and $g$ be a path from $x_1$ to $x_2$. Then:

      $$(f * g)(x) = \begin{cases}
        f(2s) & s \in \left[0, \frac{1}{2} \right] \\
        g(2s - 1) & s \in \left[ \frac{1}{2}, 1 \right]
        \end{cases}$$
    \end{defn}

    We want to be able to define an operation $*$ between path-homotopy/homotopy classes such that $[f] * [g] = [f * g]$. We need
    to show that given $f'$ such that $[f'] = [f]$, and $g'$ such that $[g'] = [g]$, that $[f' * g'] = [f * g]$. In other words,
    if $f'$ and $f$ are path-homotopic, and $g'$ and $g$ are path-homotopic, then we need to show that $f' * g'$ and $f * g$ are
    path-homotopic.
    \newline

    \begin{thm}
      The operation $*$ acting on homotopy classes has the following properties:

      \begin{enumerate}
      \item It is associative.
      \item There exist right and left identities. In other words, we can find $[e_{x_1}]$ and $[e_{x_2}]$ such that $[f] * [e_{x_1}] = [f]$, and $[e_{x_2}] * [f] = [f]$.
        \item There exist inverses. In other words, given $[f]$, there is $[\overline{f}]$ such that $[\overline{f}] * [f] = [e_{x_1}]$, and $[f] * [\overline{f}] = [e_{x_2}]$.
      \end{enumerate}

    \end{thm}

    \begin{rem}
      If $k$ is a continuous map between topological spaces $X$ and $Y$, $F$ is a path-homotopy in $X$ between $f$ and $f'$, then
      $K \circ F$ is a path-homotopy in $Y$ between $k \circ f$ and $k \circ f'$.
    \end{rem}

    \begin{proof}
      Consider the interval $I$, along with $\text{id} : I \rightarrow I$, and $\overline{\text{id}}(s) = 1 - s$. We want
      to show that ``going backwards'' is like an inverse. Note that $\text{id} * \overline{\text{id}} = e_0$. Let $F$
      be the path-homotopy between the identity and identity-bar.
      \newline

      We then consider $f \circ F$. Clearly, $(f \circ H)(0) = e_{x_0}$, and $(f \circ H)(1) = f * \overline{f}$.
    \end{proof}

    \section{Munkres: Section 51}

    \begin{prob}
      If a space is contractible, then it is path-connected.
    \end{prob}

    \begin{proof}
      Let $F$ be a homotopy between $i$ and the constant map $e_c$, for some $c$. Pick two points $x$ and $y$ in $X$. Let $F_x(t) = F(x, t)$ and $F_y(t) = F(y, 1 - t)$. Clearly,
      both maps are paths, from $x$ to $c$, and then from $c$ to $y$, with $F_x(1) = F_y(0)$. Hence, $F_x * F_y$ is a path from $x$ to $y$, so $X$ is path-connected.
    \end{proof}

    \begin{prob}
      If $Y$ is contractible, then $[X, Y]$ has a single element, for any $X$.
    \end{prob}

    \begin{proof}
      We need to show that all continuous maps are homotopic. Pick $f$ and $f'$. Clearly, $f = i_Y \circ f$, and $f' = i_Y \circ f'$. But since $i_Y \simeq e_c$, then $f \simeq e_c \circ f = e_c$, and
      $f' \simeq e_c$. Thus, $f \simeq f'$, so all continuous maps are homotopic.
    \end{proof}

    \begin{prob}
      Show that if $X$ is contractible and $Y$ is path-connected, then $[X, Y]$ has a single element.
    \end{prob}

    \section{Lecture 2: November 16th}

    \begin{def}
      Let $X$ be a topological space. Let $x_0 \in X$. A path that begins and ends at $x_0$ is called a \textit{loop}.
    \end{def}

    \begin{def}
    The set of (path) homotopy classes of loops at $x$ together with $*$ (as defined on equivalence classes of paths) is a group, and is called the fundamental ground at $x_0$, denoted
    $\pi_1(X, x_0)$.
    \end{def}

    \begin{ex}
      The fundamental group $\pi_1(\mathbb{R}^{n}, x_0)$ is trivial, as everything is path-homotopic in $\mathbb{R}^{n}$. The same goes for any convex subset of $\mathbb{R}^{n}$.
    \end{ex}

    \begin{def}
      A space with a trivial fundamental group is said to be \textit{simply-connected}.
    \end{def}

    We make the claim that in a path-connected space, the choice of base point of a fundamental group does not matter, up to an isomorphism.
    \newline

    \begin{def}
      Let $\alpha$ be a path from $x_0$ to $x_1$ in a path-connected space $X$. We define $\hat{\alpha}$ to be the map such that:

      $$\hat{\alpha}( [f] ) = [\bar{\alpha}] * [f] * [\alpha]$$
      \end{def}

    We claim that this is in fact an isomorphism. We first show that it is a homomorphism. Indeed:

    $$\hat{\alpha} ( [f] * [g] ) = \hat{\alpha} ([f * g]) = [\bar{\alpha}] * [f * g] * [\alpha] = [\bar{\alpha}] * [f] * [\alpha] * [\bar{\alpha}] * [g] * [\alpha] = \hat{\alpha}([f]) * \hat{\alpha}([g])$$

    Now, we claim that the map is bijective, thus proving it is an isomorphism. Let $\hat{\alpha}^{-1} = \hat{\bar{\alpha}}$. It is easy to see that this is an inverse.
    \newline

    \begin{lem}
      In a simply-connected space $X$, any two paths having the same initial and final points are path-homotopic.
    \end{lem}

    \begin{proof}
      The composition of the maps is a loop. Thus, the loop is homotopic to a constant loop, which immediately proves the theorem from simple algebra:

      $$[\bar{\alpha}] * [\beta] = [e_p] \Rightarrow [\beta] = [\alpha]$$
    \end{proof}

    \begin{lem}
      The fundamental group of a topological space is a topological invariant: if two spaces are homeomorphic, their fundamental groups will be the same
      up to isomorphism.
    \end{lem}

    \begin{proof}
      Let $f$ be a homeomorphism between $X$ and $Y$. Let $\textbf{f}$ map a loop $[g]$ in $X$ to a loop $[f \circ g]$ in $Y$. We know that $\textbf{f}$ is will defined because
      if $h \simeq g$, then $f \circ h \simeq f \circ g$, so $[f \circ h] = [f \circ g]$. Thus, our map is independent of choice of representative.
      \newline

      We can prove that this is a group homomorphism:

      $$\textbf{f}([g] * [h]) = \textbf{f}([g * h]) = [f \circ g * f \circ h] = [f \circ g] * [f \circ h] = \textbf{f}([g]) * \textbf{f}([h])$$

      Since $f$ is continuous both ways, we can define an inverse, showing that $\textbf{f}$ is an isomorphism, and we are done.
    \end{proof}

    \textbf{Note to self:} Go back and prove all of the basic stuff of homotopy and path-homotopy for yourself. Draw some pictures!
    \newline

    We now begin a discussion of covering spaces. Let $p : E \rightarrow B$ be a surjective, continuous map. The open set $U \subset B$ is said to be evenly covered
    by $p$ if $p^{-1}(U)$ can be written as the union of disjoint open sets, $V_{\alpha}$, such that $p|_{V_{\alpha}}$ is a homeomorphism between $V_{\alpha}$ and $U$.
    \newline

    If every point has a neighbourhood that is evenly covered by $p$, $p$ is said to be a covering map and $E$ is said to be a covering space.
    \newline

    Here is a simple example: if $X$ is a topological space, and $X \times \{1, \ ..., \ n\}$ is given the discrete topology, then it is clearly a covering space when
    $p$ is the map $\pi_1$.
    \newline

    \section{MAT327 Lecture 3}

    \begin{rem}
      Every covering map is a local homeomorphism but not every local hoemeomorphism is a covering map.
    \end{rem}

    \begin{prop}
      Let $p : E \rightarrow B$ be a covering map. If $B_0$ is a subspace of $B$ and $E_0 = p^{-1}(B_0)$, then $p : E_0 \rightarrow B_0$ is a covering map.
    \end{prop}

    \begin{prop}
      If $p$ and $p'$ are covering maps, then so is their product.
    \end{prop}

    \textbf{Uniqueness of the lifting of a covering map}
    \newline
    
    

    \end{document}
