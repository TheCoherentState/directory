\documentclass[10pt, oneside]{amsart} 
\usepackage{amsmath, amsthm, amssymb, wasysym, verbatim, bbm, color, graphics, geometry, hyperref, biblatex, mathtools}
\usepackage{tcolorbox}

\hypersetup{
	colorlinks=true,
	linkcolor=blue,
	urlcolor=blue
}


\geometry{tmargin=1.25in, bmargin=1.25in, lmargin=1.25in, rmargin =1.25in}
\setlength\parindent{0pt}

\tcbuselibrary{theorems}
\newtcbtheorem
    []% init options
    {problem}% name
    {Problem}% title
    {%
      fonttitle=\bfseries,
    }% options
    {prob}% prefix

    \newcommand{\R}{\mathbb{R}}
    \newcommand{\C}{\mathbb{C}}
    \newcommand{\Z}{\mathbb{Z}}
    \newcommand{\N}{\mathbb{N}}
    \newcommand{\Q}{\mathbb{Q}}
    \newcommand{\Cdot}{\boldsymbol{\cdot}}
    \newcommand{\U}{\mathcal{U}}
    \newcommand{\V}{\mathcal{V}}

    \newtheorem{thm}{Theorem}
    \newtheorem{defn}{Definition}
    \newtheorem{conv}{Convention}
    \newtheorem{rem}{Remark}
    \newtheorem{lem}{Lemma}
    \newtheorem{cor}{Corollary}
    \newtheorem{prop}{Proposition}
    \newtheorem{prob}{Problem}

    \newcommand{\tr}{\mathrm{Tr}}
    \newcommand{\bm}{\boldsymbol}
    \DeclarePairedDelimiter\floor{\lfloor}{\rfloor}


    \title{MAT327 Exam Prep}
    \author{Jack Ceroni}
    \date{November 2021}

    \begin{document}

    \maketitle

    \tableofcontents

    \vspace{.25in}

    \newpage

    \section{Practice Exam}

    \hrulefill

    \subsection{Problem 1} \textit{Let $J$ be uncountable. Is $\mathbb{R}^{J}$ with the product topology metrizable?}
    \newline

    $\mathbb{R}^{J}$ with the product topology is not metrizable. This is due to the fact that we can form a contradiction to the sequence lemma, which in
    turn implies that the space is not metrizable.
    \newline

    We let $A$ be the set of all points $\bm{x}$ in $\mathbb{R}^J$ that are $1$ at all except finitely many of their coordinates. We let $\bm{0}$ be the point
    with all coordinates equal to $0$.
    \newline

    Given some neighbourhood of the form $U = \prod_{\alpha} U_{\alpha}$ around $\bm{0}$, note that $U_{\alpha} \neq \mathbb{R}$ for only finitely many $\alpha$, so the point $\bm{y}$,
    with $y_{\beta} = 0$ for this finite collection of indices and $1$ otherwise is in $U$ and $A$. Thus, $\bm{0} \in \bar{A}$.
    \newline

    We assert that there is no sequence of points of $A$ that converges to $\bm{0}$. This is due to the fact that for any sequence $\bm{a}_n$, the set of coordinates that differ from $1$
    for some element of the sequence is finite. Taking the union of all such indices over $n$ yields a countable set. But the set of all indices, $J$, is uncountable, so we can choose some
    coordinate $\beta$ such that every $a \in \bm{a}_n$ is $1$ at that coordinate.
    \newline

    We then take the open set $\pi^{-1}_{\beta}(-1, 1)$ in $\mathbb{R}^{J}$. This contains $\bm{0}$, but no element of $\bm{a}_n$. Thus, $\bm{a}_n$ does not converge to $\bm{0}$.

    \hrulefill

    \subsection{Problem 2} \textit{Let $A$ be countable. Is $\mathbb{R}^{\omega} - A$ path-connected, where $\mathbb{R}^{\omega}$ is given the product topology?}
    \newline

    Let $a$ and $b$ be points of $\mathbb{R}^{\omega}$. Let $S_x = \{ L_{x}^{m} \ | \ m \in \mathbb{R} \}$ be the set of all lines passing through $x$ lying on the plane defined by $a$ and $b$, so:

    $$L_x^{m} = \{ x + t(a + mb) \ | \ t \in \mathbb{R} \}$$

    Clearly, given $m \neq n$, we have $L_x^{m} \cap L_x^{n} = \{x\}$. Since $A$ is countable, and the sets $S_a, S_b$ are uncountable, it follows that there exist infinite
    subsets $V_a \subset S_a$ and $V_b \subset S_b$ of lines that do not intersect $A$. Pick $L_a^{m} \in V_a$, and $L_b^{n} \in V_b$ that is not parallel to $L_a^{m}$. It follows that $L_a^{m} \cap L_b^{n} = \{c\}$.
    \newline

    We then define $F : [0, 2] \rightarrow \mathbb{R}^{\omega} - A$ as:

    $$F = \begin{cases}
      (1 - t)a + tc & t \in \left[ 0, 1 \right] \\
      t c + (t - 1)b & t \in \left[ 1, 2 \right]
    \end{cases}
    $$

    It is easy to see that $F([0, 1]) \subset L_a^{m}$ and $F([1, 2]) \subset L_b^{n}$, so $F([0, 2])$ lies in $\mathbb{R}^{\omega} - A$ and $F$ is well-defined. In addition, it is clear that
    $F$ is a path from $a$ to $b$, so the space is path-connected.

    \hrulefill

    \subsection{Problem 3} \textit{Suppose $A$ is a subset of $X$, a compact metric space, and every continuous function $f : A \rightarrow \mathbb{R}$ has a maximum in $A$. Prove that $A$ is compact.}
    \newline

    We will show that $A$ is closed, and therefore compact (as we are operating in a compact metric space).
    \newline

    Suppose $A$ is not closed. Then $X - A$ is not open. Thus, there must exist some $a \in X - A$ such that $a \in \bar{A}$.
    \newline

    Define $f : A \rightarrow \mathbb{R}$ such that $f(x) = \frac{1}{d(x, a)}$. Clearly, this function is continuous on $A$, as $g(x) = d(x, a)$ is continuous. However, it is obvious that $f$ is not bounded, as for any $N$, we can choose
    a point $y \in A$ such that $y \in B_{1/N}(a)$, so $f(y) > N$. Thus, we have a contradiction, so $A$ must be closed, and is therefore compact.

    \hrulefill

    \subsection{Problem 4} \textit{Calculate the fundamental group of $S^{1}$ with the north and south poles identified}
    \newline

    It is easy to see through a sketch that this space is homotopy equivalent to a sphere with a line connecting the north and south poles. This space is itself homotopy equivalent to
    $S^{2} \vee S^{1}$, which can been seen be gradually sliding the end points of the diameter together.
    \newline
   
    Let $x_0$ be the attachment point. We proceed via Seifert-van Kampen. Removing a point from $S^{1}$ gives an open set $U$ that deform retracts to $S^{2}$, so $\pi_1(U, x_0)$ is trivial.
    Removing a point from $S^{2}$ gives an open set $V$ that deform retracts to $S^{1}$, so $\pi_1(V, x_0) = \langle a \rangle$.
    \newline

    The intersection $U \cap V$ gives a space which deform retracts to a point, and therefore has a trivial fundamental group.
    \newline

    Thus, by Seifert-van Kampen, $\pi_1(X, x_0) = \langle a \ | \ \emptyset \rangle = \mathbb{Z}$.

    \hrulefill

    \subsection{Problem 5} \textit{Describe the universal covering space and covering map of the above space.}
    \newline

    Don't know, universal covering spaces were never once mentioned in class, so I'm not sure why this is on the practice exam...

    \hrulefill

    \subsection{Problem 6}

    \textbf{Part A}
    \newline

    \textit{Find the fundamental group of the doubly-punctured torus.}
    \newline

    It is easy to see that the doubly-punctured torus deform retracts to the boundary of the rectangle defining it, with a line passing down the middle. This is itself clearly the wedge of three circles: $X = S^{1} \vee S^{1} \vee S^{1}$.
    \newline

    We once again use Seifert-van Kampen. Let $a, b, c \in X$ be distinct, lying on each of the circles in the wedge (not at the attachment point $x_0$). Let $U = X - \{a\}$ and $V = X - \{b, c\}$. Clearly, $U$ deform retracts to $S^{1} \vee S^{1}$,
    so $\pi_1(U, x_0) = \langle a, b \rangle$. In addition, $V$ clearly deform retracts to $S^{1}$, so $\pi_1(V, x_0) = \langle c \rangle$. It follows that:

    $$\pi_1(X, x_0) = \pi_1(U \cup V, x_0) = \langle a, b, c \rangle = \mathbb{Z} * \mathbb{Z} * \mathbb{Z}$$

    as $U \cap V$ deform retracts to a point, and therefore is simply connected.
    \newline
    
    \textbf{Part B}
    \newline

    \textit{Find the fundamental group of $\mathbb{R}^3$ with $n$ lines through the origin removed.}
    \newline

    \begin{prop}
      $\mathbb{R}^2$ with $n$ points removed (which we call $R_n$) has a fundamental group $\mathbb{Z} * \cdots * \mathbb{Z}$, repeated $n$ time.
    \end{prop}

    \begin{proof}
      We proceed by induction. For $n = 1$, the claim clearly holds. Assume the case of $n$. We prove the case of $n + 1$. Choose open sets $U$ and $V$, with $U$ containing the $n + 1$-th hole, and
      $V$ containing the other $n$ holes, such that $U \cup V = X$, and $U \cap V$ is some unpunctured open set, and is therefore simply connected.
      \newline

      It follows that $\pi_1(U, x_0)$ is the free group with $1$ generator, while $\pi_1(V, x_0)$ is the free group of $n$ generators, from the inductive hypothesis. Thus, $\pi_1(R_n, x_0)$ is the
      free group with $n + 1$ generators.
      \end{proof}

    $\mathbb{R}^3 - X$ clearly deform retracts to $S^{2}$, with $2n$ holes. We know that $S^{2} - \{p\}$ is homeomorphic to $\mathbb{R}^{2}$. Thus, subtracting an additional $2n - 1$ points implies that our
    space is homeomorphic to $\mathbb{R}^{2}$ with $2n - 1$ distinct points removed, so from above:

    $$\pi_1(\mathbb{R}^{3} - X, x_0) = \pi(R_{2n - 1}, x_0) = \mathbb{Z} * \cdots * \mathbb{Z}$$

    repeated $2n - 1$ times.

    \hrulefill

    \subsection{Problem 7} \textit{Show that every continuous map $f : \mathbb{R} P^2 \rightarrow S^1$ is nulhomotopic.}
    \newline

    We utilize the general lifting lemma. Consider a continuous map $f : Y \rightarrow B$, and covering map $p : E \rightarrow B$, with $p(e_0) = b_0$ and $f(y_0) = b_0$. Suppose $Y$ is path-connected locally path-connected.
    Then there exists a lift of $f$ to $\tilde{f}$ with $\tilde{f}(y_0) = e_0$ if and only if:

    $$f_{*}(\pi_1(Y, y_0)) \subset p_{*}(\pi_1(E, e_0))$$

    We know that, given $y_0 \in \mathbb{R} P^2$, we have $\pi_1(\mathbb{R} P^2, y_0) = \mathbb{Z}/\mathbb{Z}2$. We also know that $\pi_1(S^1, b_0) = \mathbb{Z}$. Clerarly, if $\phi$ is a homomorphism
    from $\mathbb{Z}/\mathbb{Z}2$ to $\mathbb{Z}$, then $\phi(0) = 0$. Suppose $\phi(1) = n \neq 0$. Then:

    $$\phi(0) = \phi(1 + 1) = 2\phi(1) = 2n \neq 0$$

    a contradiction, so we must have $\phi(1) = 0$. Thus, $f_{*}$ must be the trivial homomorphism, so we clearly have $f_{*}(\pi_1(\mathbb{R} P^2, y_0)) \subset p_{*}(\pi_1(\mathbb{R}, e_0))$, where
    $p : \mathbb{R} \rightarrow S^1$ is the usual covering map.
    \newline

    From the general lifting lemma, we can lift $f$ to $\tilde{f} : \mathbb{R} P^2 \rightarrow \mathbb{R}$. Let $F(x, t) = (1 - t) \tilde{f}(x) + t e_0$. Clearly, this is a homotopy of $\tilde{f}(x)$ and
    $e_0$. Then $p \circ F$ is a homtopy of $p \circ \tilde{f} = f$ and $p(e_0)$, so $f$ is nulhomotopic.

    \hrulefill

    \subsection{Problem 8} \textit{Prove that there exists no retraction from the solid torus to its boundary torus.}
    \newline

    Recall that we proved that there exists no retraction from $D^2$ to $S^1$.
    \newline

    Suppose there is a retraction $r$ from $S^{1} \times D^2$ to $S^1 \times S^1$. Then define $g : D^2 \rightarrow S^1$ as:

    $$g(x) = (\pi_2 \circ r)(x_0, x)$$

    where $x_0 \in S^{1}$ and $\pi_2 : S^{1} \times S^{1} \rightarrow S^{1}$ is a projection. Clearly, this map is continuous, as both $\pi_2$ and $r$ are continuous. Note that for
    $x \in S^{1}$, we have $(x_0, x) \in S^{1} \times S^{1}$, so $r(x_0, x) = (x_0, x)$ (by definition of $r$), so:

    $$g(x) = (\pi_2 \circ r)(x_0, x) = \pi_2(x_0, x) = x$$

    so $g$ is a retraction from $D^2$ to $S^1$, a contradiction. It follows that there is no retraction from $S^{1} \times D^2$ to $S^1 \times S^1$.
    \newline

    Alternatively, we could have completed this proof through more algebraic means. Note that $S^{1} \times S^{1}$ has fundamental group $\mathbb{Z} \times \mathbb{Z}$, while
    $S^{1} \times D^{2}$ deform retracts to $S^{1}$, so it has fundamental group $S^{1}$.
    \newline

    If a retract exists, then inclusion of $S^{1} \times S^{1}$ into $S^{1} \times D^{2}$ induces an injective homomorphism of fundamental groups. However, there is
    no injective homomorphism of $\mathbb{Z} \times \mathbb{Z}$ into $\mathbb{Z}$. Suppose $\phi$ is such a homomorphism. Then let:

    $$\phi(1, 0) = a \ \ \ \ \text{and} \ \ \ \ \phi(0, 1) = b$$

    But we then have $\phi(b, 0) = ab$ and $\phi(0, a) = ab$, so $\phi(b, 0) = \phi(0, a)$, a clear contradiction to injectivity. Thus, not such retraction can exist.

    \hrulefill

    \subsection{Problem 9} \textit{Let $F_1 \cup F_2 \cup F_3 = S^{2}$, where each $F_i$ is closed. Show that some $F_i$ contains a pair of antipodal points.}
    \newline

    \begin{lem}
      Let $X$ be a metric space and $f(x) = d(x, A)$. If $A$ is closed, $f(x) = 0$ implies that $x \in A$.
    \end{lem}

    \begin{proof}
      Suppose $f(x) = 0$, so $\inf_{a \in A} d(x, a) = 0$. Suppose $x \notin A$, so $x \notin \bar{A}$. Then there exists some $\epsilon$-ball around
      $x$ disjoint from $A$, so $d(x, a) \geq \epsilon$ for all $a \in A$. This is a contradiction, so $x \in A$.
      \end{proof}

    We wish to apply Borsuk-Ulam. Consider the functions $f(x) = (d(x, F_1), d(x, F_2))$. Borsuk-Ulam says that there is $x$ such that $f(x) = f(-x)$. If $f(x)_1 = 0$
    or $f(x)_2 = 0$, then we automatically have $x, -x \in F_1$ or $x, -x \in F_2$. If both components are non-zero, then we must have $x, -x \in F_3$, as they are not in
    $F_1$ nor $F_2$.

    \hrulefill

    \subsection{Problem 10} \textit{Let $x_0$ and $x_1$ be points of a path-connected space. Show that $\pi(X, x_0)$ is abelian if and only if $\hat{\alpha} = \hat{\beta}$ for every pair
      of paths $\alpha$, $\beta$ from $x_0$ to $x_1$. Recall that $\hat{\alpha}([f]) = [\overline{\alpha} * f * \alpha]$.}
    \newline

    First, let us suppose that $\pi_1(X, x_0)$ is abelian. Let $\alpha$ and $\beta$ be paths from $x_0$ to $x_1$. Then $\alpha * \overline{\beta}$ and $\beta * \overline{\alpha}$ are loops based as $x_0$. Let $f$ be a loop at $x_0$. We therefore have:

    $$[\alpha * \overline{\beta}] * [f] * [\beta * \overline{\alpha}] = [\alpha * \overline{\beta}] * [\beta * \overline{\alpha}] * [f] = [f]$$

    where each element of the product is a loop in $\pi_1(X, x_0)$. It follows that:

    $$\hat{\alpha}([f]) = [\overline{\alpha}] * [f] * [\alpha] = [\overline{\alpha}] * [\alpha * \overline{\beta}] * [f] * [\beta * \overline{\alpha}] * [\alpha] = [\overline{\beta}] * [f] * [\beta] = \hat{\beta}([f])$$

    from above.
    \newline

    Now, suppose that $\hat{\alpha} = \hat{\beta}$ for every pair of paths. Suppose $[f], [g] \in \pi_1(X, x_0)$. Let $\alpha$ be a path from $x_0$ to $x_1$. We also have $h = f * \alpha$ a path from
    $x_0$ to $x_1$. Thus:

    $$\hat{\alpha}([g]) = [\overline{\alpha} * g * \alpha] = \hat{h}([g]) = [\overline{\alpha} * \overline{f} * g * f * \alpha]$$

    This then implies that $[g] = [\overline{f} * g * f]$, so $[f] * [g] = [g] * [f]$. Thus, the group is abelian.

    \hrulefill

    \newpage

    \section{Extra Hatcher Problems}

    \hrulefill

    \subsection{Problem 0.1} \textit{Construct an explicit deformation retraction from the torus $T$ with a point removed to its boundary circles.}
    \newline

    Not really sure what this problem wants, literally an explicit formula for a deformation retraction?

    \newpage

    \hrulefill

    \subsection{Problem 0.3}

    \textbf{Part A}
    \newline

    \textit{Show that the composition of homotopy equivalences is a homotopy equivalence.}
    \newline

    Let $f : X \rightarrow Y$ and $g : Y \rightarrow Z$ be homotopy equivalences. We wish to show that $g \circ f : X \rightarrow Z$ is a homotopy equivalence. Note that
    there exist maps $f' : Y \rightarrow X$ and $g' : Z \rightarrow Y$ such that $f f'$, $f' f$, $g g'$, and $g' g$ are homotopic to the identity map on their respective
    domains/codomains.
    \newline

    We claim that $(g \circ f) \circ (f' \circ g') \simeq \text{id}_{Z}$ and $(f' \circ g') \circ (g \circ f) \simeq \text{id}_{X}$ (we don't assume transitivity of homotopy, this is Part B of the question, so we have to
    construct an explicit homotopy).
    \newline

    If $F$ is a homotopy between $f \circ f'$ and $\text{id}_{Y}$, and $G$ is a homotopy between $g \circ g'$ and $\text{id}_{Z}$, define:

    $$H(x, t) = \begin{cases}
      g \circ F(x, 2t) \circ g' & t \in \left[0, \frac{1}{2} \right] \\
      G(x, 2t - 1) & t \in \left[\frac{1}{2}, 1 \right]
    \end{cases}
    $$

    This function is continuous, by the pasting lemma. Also, note that $H(x, 0) = g \circ F(x, 0) \circ g' = g \circ f \circ f' \circ g'$, and $H(x, 1) = G(x, 1) = \text{id}_{Z}$. Almost identical
    reasoning shows $(f' \circ g') \circ (g \circ f) \simeq \text{id}_{X}$, so it follows that $g \circ f$ is a homotopy equivalence.
    \newline

    \textbf{Part B}
    \newline

    \textit{Show that the relation of homotopy among maps is an equivalence relation.}
    \newline

    Given maps $f \simeq g$ and $g \simeq h$, we wish to show that $f \simeq h$. This is easy. If $F$ is a homotopy between the first pair and $G$ is a homotopy between the second pair, then define $H$ like we did above:

    $$H(x, t) = \begin{cases}
      F(x, 2t) & t \in \left[ 0, \frac{1}{2} \right] \\
      G(x, 2t - 1) & t \in \left[ \frac{1}{2}, 1 \right]
    \end{cases}
    $$

    Clearly, $H(x, 0) = F(x, 0) = f(x)$ and $H(x, 1) = G(x, 1) = h(x)$. Continuity is given by the pasting lemma, at $t = \frac{1}{2}$. Thus, homotopy is an equivalence relation, as maps are clearly homotopic to
    themselves, and if $f \simeq g$ with homotopy $F(x, t)$, then $F(x, 1 - t)$ is a homotopy from $g$ to $f$, so $g \simeq f$.
    \newline

    \textbf{Part C}
    \newline

    \textit{Show that a map homotopic to a homotopy equivalence is a homotopy equivalence.}
    \newline

    Let $f : X \rightarrow Y$ be a homotopy equivalence, so there is $f' : Y \rightarrow X$ with $f f'$ and $f' f$ homotopic to the identity. Let us consider $g \simeq f$. We prove a small lemma:

    \begin{lem}
      If $k_1 \simeq k_2$ (with $k_i : Y \rightarrow Z$), then given continuous $f : Z \rightarrow W$ and $g : X \rightarrow Y$, then:

      $$f \circ k_1 \circ g \simeq f \circ k_2 \circ g$$
    \end{lem}

    \begin{proof}
      Since $k_1 \simeq k_2$, there exists a homotopy $F$ between them. Define $G(x, t) = (f \circ F)(g(x), t)$. We claim that this is a homotopy between $f \circ k_1 \circ g$
      and $f \circ k_2 \circ g$. Clearly, this function is continuous, as $f$ and $g$ are. Note that:

      $$G(x, 0) = (f \circ F)(g(x), 0) = (f \circ k_1 \circ g)(x) \ \ \ \ \text{and} \ \ \ \ G(x, 1) = (f \circ F)(g(x), 1) = (f \circ k_2 \circ g)(x)$$

      and we are done.
    \end{proof}

    The proof is now trivial, we have:

    $$g f' \simeq f f' \simeq \text{id}_{Y} \ \ \ \ \text{and} \ \ \ \ f' g \simeq f' f \simeq \text{id}_{X}$$

    so $g$ is a homotopy equivalence, and we are done.

    \hrulefill

    \subsection{Problem 0.4}

    \textit{A deformation retraction in the weak sense of a space $X$ to a subspace $A$ is a homotopy such that $F(x, 0) = x$, $F(x, 1) \subset A$, and $F(A, t) \subset A$ for all $t$. Show that if $X$ deformation retracts to $A$ in this weak sense, then the inclusion from $A$ to $X$
      is a homotopy equivalence.}
    \newline

    Let $j : A \rightarrow X$ denote inclusion. Let $r(x) = F(x, 1)$. We claim that $jr$ and $rj$ are homotopic to the identity. Indeed, let $r : X \rightarrow A$ be defined as $r(x) = F(x, 1)$.
    Indeed, this is well-defined as $F(x, 1) \in A$ for all $x \in X$.
    \newline

    To show $r \circ j \simeq \text{id}_A$, we let $G : A \times I \rightarrow A$ be defined as $G(x, t) = F(x, t)$, which is well defined as $G(x, t) \in A$ for $x \in A$. It is easy to check
    this is a homotopy. Similarly, to show $j \circ r \simeq \text{id}_X$, we use the original homotopy $F$.
    \newline

    Thus, $j$ is a homotopy equivalence.

    \hrulefill

    \subsection{Problem 0.9}

    \textit{Show that a retract of a contractible space is contractible.}
    \newline

    Let $A$ be a retract of a contractible space $X$. We know that $\text{id}_X \simeq e_c$, where $e_c$ is the identity map. If $r$ is our retract, then note that:

    $$\text{id}_{A} = r \circ \text{id}_{X} \circ j$$

    If $F$ is a homotopy between $\text{id}_{X}$ and $e_c$, we claim that $G : A \rightarrow A$ defined as $G(x, t) = (r \circ F)(j(x), t)$ is a homotopy between $\text{id}_{A}$ and
    $e_{r(c)}$. Indeed, note that $G$ is continuous, well-defined, $G(x, 0) = (r \circ j)(x) = \text{id}_{A}(x)$ (from above), and $G(x, 1) = r(c)$. Thus, $\text{id}_{A} \simeq e_{r(c)}$, so $A$ is contractible.

    \hrulefill

    \subsection{Problem 0.10}

    \textit{Show that a space $X$ is contractible iff every map from $X$ to $Y$ for arbitrary $Y$, is nullhomotopic. Similarly, show $X$ is contractible iff every map from $Y$ to $X$ is nullhomotopic.}
    \newline

    First, suppose that every $f : X \rightarrow Y$, for arbitrary $Y$, is nulhomotopic. Then setting $Y = X$ and $f = \text{id}_{X}$ shows that $X$ is contractible. Conversely,
    suppose $X$ is contractible. Pick some $Y$ and some $f$. Note that:

    $$f = f \circ \text{id}_{X} \simeq f \circ e_{c} = e_{f(c)}$$

    so $f$ is nulhomotopic.
    \newline

    Now, suppose every $f : Y \rightarrow X$ is nulhomotopic. Again, setting $Y = X$ and $f = \text{id}_{X}$ shows that $X$ is contractible. Conversely, pick some $f : Y \rightarrow X$.
    Let $F$ be a homotopy between $\text{id}_{X}$ and $e_c$. Then define $G(x, t) = F(f(x), t)$. We claim that this is a homotopy between $f$ and $e_c : Y \rightarrow X$.
    \newline

    Indeed, note that $G$ is clearly continuous and well-defined, $G(x, 0) = f(x)$, and $G(x, 1) = c$.
    \newline

    This proof is also a trivial consequence of the lemma given in in Problem 0.3: $\text{id}_{X} \simeq e_c$ implies $f \circ \text{id}_{X} \simeq f \circ e_c$, and $\text{id}_{x} \circ f \simeq e_c \circ f$.

    \hrulefill

    \subsection{Problem 1.10}

    \textit{Show that loops of $\pi_1(X \times Y, x_0 \times y_0)$ whose images are contained in $X_0 \times \{y_0\}$ and $\{x_0\} \times Y_0$ respectively commute.}
    \newline

    Loops in $X \times \{y_0\}$ and $\{x_0\} \times Y$ are of the form $f = (f_1, y_0)$ and $g = (x_0, g_2)$. We can think of $f * g$ as a loop which traverses a path in $X$, while remaining stationary at
    $y_0$, then traverses a path in $Y$ while remaining stationary at $x_0$.
    \newline

    We wish to show that $[f] * [g] = [g] * [f]$. We define a homotopy between $f * g = (f_1 * x_0, y_0 * g_2)$ and $g * f = (x_0 * f_1, g_2 * y_0)$. We define a homotopy between $f_1 * x_0$ and
    $x_0 * f_1$ as follows: we let $\alpha$ and $\beta$ be defined as:

    $$\alpha(s) = \begin{cases}
      2s & s \in \left[0, \frac{1}{2} \right] \\
      1 & s \in \left[\frac{1}{2}, 1 \right]
    \end{cases} \ \ \ \ \ \text{and} \ \ \ \ \ \ \beta(s) = \begin{cases}
      0 & s \in \left[0, \frac{1}{2} \right] \\
      2s - 1 & s \in \left[\frac{1}{2}, 1 \right]
    \end{cases}$$

    We then define $F(x, t) = f_1( (1 - t) \alpha(x) + t \beta(x) )$. Clearly, this function is continuous, as it is the composition of continuous functions. Note that:

    $$F(x, 0) = f_1(\alpha(x)) = \begin{cases}
      f_1(2x) & x \in \left[0, \frac{1}{2} \right] \\
      f_1(1) = x_0 & x \in \left[\frac{1}{2}, 1\right]
    \end{cases} \ \ \ \ \ \text{and} \ \ \ \ \ F(x, 1) = f_1(\beta(x)) = \begin{cases}
      f_1(0) = x_0 & x \in \left[0, \frac{1}{2} \right] \\
      f_1(2x - 1) & x \in \left[\frac{1}{2}, 1\right]
    \end{cases}
    $$

    so $F(x, 0) = f_1 * x_0$ and $F(x, 1) = x_0 * f_1$. Finally, note that:

    $$F(0, t) = f_1(0) = x_0 \ \ \ \ \text{and} \ \ \ \ F(1, t) = f_1(1) = x_0$$

    for all $t$. Thus, $F$ is a path-homotopy between $f_1 * x_0$ and $x_0 * f_1$. Identical reasoning yields a path-homotopy $G$ between $g_2 * y_0$ and $y_0 * g_2$. Finally, we define a
    homotopy $H$ between $f * g$ and $g * f$ as $H(x, t) = (F(x, t), G(x, 1 - t))$. Clearly, $H$ is continuous as a map into $X \times Y$. In addition, we have:

    $$H(x, 0) = (F(x, 0), G(x, 1)) = (f_1 * x_0, y_0 * g_2) \ \ \ \ \text{and} \ \ \ \ H(x, 1) = (F(x, 1), G(x, 0)) = (x_0 * f_1, g_2 * y_0)$$

    $$H(0, t) = (F(0, t), G(0, 1 - t)) = (x_0, y_0) \ \ \ \ \text{and} \ \ \ \ H(1, t) = H(1, t) = (F(1, t), G(1, 1 - t)) = (x_0, y_0)$$

    so $H$ is a path-homotopy between $f * g$ and $g * f$.

    \hrulefill

    \subsection{Problem 1.16}

    \textit{Show that no retract exists in the following cases: (a) $\mathbb{R}^3$ to a space homeomorphic to $S^1$, (b) Solid torus to boundary torus, (c) Weird image in Hatcher (ignore), (d) Wedge of disks to wedge of boundary disks, (e)
      Disk with two points identified to boundary $S^1 \vee S^1$}
    \newline

    \textbf{Part A}
    \newline

    Clearly, $X = \mathbb{R}^3$ is simply connected while $A$ homeomorphic to $S^1$ has fundamental group $\mathbb{Z}$, so there can exist no injective homomorphism from $\pi_1(A)$ to $\pi_1(\mathbb{R}^3)$, and therefore no retract.
    \newline

    \textbf{Part B}
    \newline

    See practice exam.
    \newline

    \textbf{Part C}
    \newline

    This is tough to see, and frankly, I hate how ``hand-wavy'' this question is.
    Both spaces have the same fundamental group, but inclusion of loops in $A$ into the solid torus are actually trivial, as we can ``unlink'' the part of the circle at which they joint, so the
    homomorphism of fundamental groups given by inclusion is actually trivial, therefore not injective, so there is no retract.
    \newline

    \textbf{Part D}
    \newline

    Clearly, $D^2 \vee D^2$ has a trivial fundamental group, as it deform retracts to a point. We know that $\pi_1(S^1 \vee S^1) = \mathbb{Z} * \mathbb{Z}$, so there is no injection
    of fundamental groups, and therefore no retract.
    \newline

    \textbf{Part E}
    \newline

    It is easy to see (through drawings), that $X$ is homotopy equivalent to a circle, so it has fundamental group $\mathbb{Z}$. Clearly, there is no injection from $\pi_1(S^1 \vee S^1) = \mathbb{Z} * \mathbb{Z}$ to $\mathbb{Z}$.
    Let $a$ and $b$ be the generators, then for some homomorphism $\phi$, we have $\phi(ab) = \phi(a) \phi(b) = \phi(b) \phi(a) = \phi(ba)$, but $ab \neq ba$, by definition.

    \hrulefill

    \subsection{Problem 2.2}

    \textit{If $X$ is the union of $n$ convex open subsets of $\mathbb{R}^{m}$, denoted $X_1, \ ..., \ X_n$, such that $X_i \cap X_j \cap X_k \neq \emptyset$ for all $i, \ j, \ k$,
      then $X$ is simply connected.}
    \newline

    We proceed by induction. Clearly, this holds for the case of $n = 1$ and $n = 2$ (by a trivial application of Seifert van-Kampen). Suppose it holds for the case of $n$ (with $n \geq 2$). We prove
    that it holds for the case of $n + 1$.
    \newline

    Clearly, the intersection of finitely many convex sets is convex, as if we have $A$, $B$ convex, and $x, y \in A \cap B$, then all points on the line between $x$ and $y$ will lie in both
    $A$ and $B$, and therefore in their intersection.
    \newline

    Let our sets be denoted $X_1, \ ..., \ X_{n + 1}$. We know that $Y = X_1 \cup \cdots \cup X_n$ is simply connected. We also know that $X_{n + 1}$ is simply-connected, as
    it is convex. Finally, note that $X_{n + 1} \cap Y$ is path-connected: if this weren't the case then there would exist some $X_{i}$ and $X_{j}$ with $i, j \leq n$ such that
    $X_{n + 1} \cap X_{i}$ and $X_{n + 1} \cap X_{j}$ are disjoint, but this can't be the case, by assumption.
    \newline

    It follows that $X_1 \cap \cdots \cap X_{n + 1} = Y \cap X_{n + 1}$ is path-connected convex, so it is simply connected. Thus, by Seifert-van Kampen:

    $$\pi_1(X, x_0) = \pi_1(X_1 \cup \cdots \cup X_n, x_0) * \pi_1(X_{n + 1}, x_0) = \langle \emptyset \rangle$$

    as we are taking the free product of two trivial groups.

    \hrulefill

    \subsection{Problem 2.4}

    Already done in practice exam!

    \hrulefill

    \subsection{Problem 3.1}

    \textit{For a covering space $p : \tilde{X} \rightarrow X$ and a subspace $A \subset X$, let $\tilde{A} = p^{-1}(A)$. Show that the restriction $p_{\tilde{A}} : \tilde{A} \rightarrow A$ is a covering map.}
    \newline

    Clearly, $p|_{\tilde{A}}$ is still continuous and surjective. Take $x \in A$. Pick $U$ around $x$ open in $X$ that is evenly covered by $p$. We claim that $U \cap A$ open in $A$, which contains $x$, is evenly covered by $p|_{\tilde{A}}$.
    Indeed, note that:

    $$p|_{\tilde{A}}^{-1}(U \cap A) = \{a \in \tilde{A} \ | \ p(a) \in U, \ p(a) \in A \} = p^{-1}(U) \cap \tilde{A} = \displaystyle\bigcup_{\alpha} (V_{\alpha} \cap \tilde{A})$$

    where each $V_{\alpha}$ is disjoint, open in $\tilde{X}$, and locally homeomorphic to $U$ via $p$. It follows that each $V_{\alpha} \cap \tilde{A}$ is disjoint and open in $\tilde{A}$.
    \newline

    In addition,
    note that each restriction of $p|_{\tilde{A}}$ to $p|_{\tilde{A} \cap V_{\alpha}} : \tilde{A} \cap V_{\alpha} \rightarrow A$ is a local homeomorphism with $U \cap A$. Indeed, $p|_{\tilde{A} \cap V_{\alpha}}$ is injective, as $p|_{V_{\alpha}}$ is, and as a result:

    $$p|_{\tilde{A} \cap V_{\alpha}}(\tilde{A} \cap V_{\alpha}) = p(\tilde{A}) \cap p(V_{\alpha}) = A \cap U$$

    so $p|_{\tilde{A} \cap V_{\alpha}}$ bijectivelty maps to $A \cap U$. Finally, it is clear that this restriction is continuous, as $p$ is continuous, and in addition, since $p^{-1} |_{U} : U \rightarrow V_{\alpha}$ is continuous, so too is
    $p^{-1} |_{U \cap A} : U \cap A \rightarrow \tilde{A} \cap V_{\alpha}$. Clearly, this is the inverse of the restricted map. Thus, it is a local homeomorphism.
    \newline

    It follows that $p$ is a covering map.

    \hrulefill

    \subsection{Problem 3.2}

    \textit{Show that a product of covering maps is a covering map.}
    \newline

    Suppose $p_1 : \tilde{X}_1 \rightarrow X_1$ and $p_2 : \tilde{X}_2 \rightarrow X_2$ are covering maps. Clearly, their product is continuous and surjective.
    \newline

    For some $x_1 \times x_2 \in X_1 \times X_2$, we pick $U_1$ open in
    $X_1$ evenly covered by $p_1$ and $U_2$ open in $X_2$ evenly covered by $p_2$. Then $U = U_1 \times U_2$ is open in $X_1 \times X_2$. We claim that $U_1 \times U_2$ is evenly covered by $p_1 \times p_2$. Indeed,
    note that:

    $$(p_1 \times p_2)^{-1}(U_1 \times U_2) = \{ t_1 \times t_2 \in \tilde{X}_1 \times \tilde{X}_2 \ | \ p_1(t_1) \in U_1, \ p_2(t_2) \in U_2 \} = \left[ p_1^{-1}(U_1) \times \tilde{X}_2 \right] \cap \left[ \tilde{X}_1 \times p_2^{-1}(U_2) \right]$$

    $$ = \left[ \displaystyle\bigcup_{\alpha} V_{\alpha} \times \tilde{X}_1 \right] \cap \left[ \displaystyle\bigcup_{\alpha} W_{\beta} \times \tilde{X}_2 \right] = \displaystyle\bigcup_{\alpha, \beta} V_{\alpha} \times W_{\beta}$$

    Clearly, each open set in this union is disjoint. In addition, we note that $(p_1 \times p_2)|_{V_{\alpha} \times W_{\beta}}$ is a local homeomorphism from $V_{\alpha} \times W_{\beta}$ to $U_1 \times U_2$. Indeed,
    it is clear that $(p_1 \times p_2)|_{V_{\alpha} \times W_{\beta}}$ is injective and surjective, as its component functions are. Since $p_1 \times p_2$ is continuous, this restriction is as well.
    \newline

    Finally, we note that the restricted map is a homeomorphism. Since $p_1^{-1}|_{U_1} : U_1 \rightarrow V_{\alpha}$ and $p_2^{-1}|_{U_2} : U_2 \rightarrow W_{\beta}$ are continuous, so too is their product, which
    is clearly the inverse of $(p_1 \times p_2)|_{V_{\alpha} \times W_{\beta}}$. Thus, $p_1 \times p_2$ is a local homeomorphism.

    \hrulefill

    \subsection{Problem 3.4}

    Already done
    \newline



    \hrulefill

    \newpage

    \hrulefill

    \subsection{Problem 1.5}

    Note that this was done in Munkres, but I'll attempt to reconstruct the proof from scratch.
    \newline

    Suppose (a) holds. We know that there exists a continuous $F : S^{1} \times I \rightarrow X$ such that $F(x, 0) = f(x)$ and $F(x, 1) = c$. It is easy to see that the quotient space $Y$
    obtained from identifying every point of $S^{1} \times \{1\}$ is homeomorphic to $B^2$. Hence, we have a quotient map $p$ going from $S^{1} \times I$ to $B^2$.
    \newline

    We claim that there exists a well-defined, continuous map $k$ from $B^2$ to $X$ induced by $F$ and $p$. Indeed, since $F$ is constant in each point in $S^{1} \times \{1\} = p^{-1}(0)$,
    it follows that $F$ extends to a map $k$ from $B^2$ to $X$ such that $k \circ g = F$.
    \newline

    Now, we show that (b) implies (c).
    \newline

    We can now make the deduction. Suppose $X$ is simply-connected. Then we immediately have all maps $S^1 \rightarrow X$ being homotopic to a constant map, so they are homotopic to each other.
    Conversely, if all maps are homotopic, then they are all homotopic to a constant map, so the fundamental group is trivial.

    \subsection{Problem 1.6}

    \hrulefill

     \textbf{Part F}
    \newline

    Let $x_0$ be a point on the boundary circle. Let $x_1$ be a point on the ``middle circle'' $M$ of the Mobius strip. Note that $X$ deform retracts onto $M$, so the
    inclusion map $j_{*} : \pi_1(M, x_1) \rightarrow \pi_1(X, x_1)$ as an isomorphism. Since $M$ is a circle, $\pi_1(M, x_1) = \mathbb{Z}$.
    Since $X$ is path-connected, the change-of-basepoint map from $x_0$ to $x_1$ as an isomorphism of fundamental groups. Thus, we have an isomorphism $k : \pi_1(X, x_0) \rightarrow \pi_1(M, x_1)$.
    \newline

    Let $[\omega]$ be the generator of $\pi_1(\partial X, x_0)$. Let $i_{*} : \pi_1(\partial X, x_0) \rightarrow \pi_1(X, x_0)$ be the inclusion map. Suppose there exists a retract from $X$ to $\partial X$,
    so $i_{*}$ is a homomorphism.
    \newline

    We have a homomorphism $k \circ i_{*}$ from $\pi_1(\partial X, x_0)$ to $\pi_1(M, x_1)$. Taking
    $[\omega]$, we note that mapping this loop into $X$ and changing the basepoint with $k \circ i_{*}$ is path-homotopic to a loop which goes twice around $M$.
    In other words, $(k \circ i_{*})([\omega]) = [\delta]^2$, where $\delta$ is the generator of $\pi_1(M, x_1)$. Since we identify $[\omega]$ and $[\delta]$ with $1 \in \mathbb{Z}$, we get $(k \circ i_{*})(1) = 2$.
    \newline

    Note that $r_{*} k^{-1} k i_{*}$ is the identity map on $\pi_1(\partial X, x_0)$. Thus, $(r_{*} k^{-1} k i_{*})(1) = (r_{*} k^{-1})(2) = 1$. But this is a homomorphism from $\mathbb{Z}$ to $\mathbb{Z}$, so $2 (r_{*} k^{-1})(1) = 1$,
    which is impossible. Thus, no such $r$ can exist.

    \hrulefill

    \section{Seifert van-Kampen Notes and Examples}

    \subsection{Theory}

    In this section, we outline the theory behind Seifert-van Kampen theorem.
    \newline

    At a high-level, the utility of Seifert-van Kampen relates to computing the fundamental group of a space by breaking it up into smaller pieces, of which we already know the
    fundamental groups.
    \newline

    We begin with a smaller theorem, which is a special case of Seifert-van Kampen:

    \begin{thm}
      Suppose that $U$ and $V$ are open sets, with $X = U \cup V$ such that the intersection $U \cap V$ is path-connected. Then given $x_0 \in U \cap V$, with $i$ and $j$ the
      inclusions of $U$ and $V$ in $X$, then the images of the induced homomorphisms of $i$ and $j$ based at $x_0$ generate the fundamental group of $X$ at $x_0$.
    \end{thm}

    \begin{proof}
      The above statement is another way of stating that every element of $\pi_1(X, x_0)$ can be expressed as a product of elements in $\pi_1(U, x_0)$ and $\pi_1(V, x_0)$.
      \newline

      Let $f : [0, 1] \rightarrow X$ be a loop in $X = U \cup V$. Note that $f^{-1}(U)$ and $f^{-1}(V)$ are on open cover for $[0, 1]$. Thus, by the Lebesgue number lemma, there exists some $\delta$ such
      that any interval $[a, b] \subset [0, 1]$ with $b - a < \delta$ is contained in either $f^{-1}(U)$ or $f^{-1}(V)$. Take the collection of intervals:

      $$[0, \delta], \ [\delta, 2 \delta], \ ..., \ [1 - \delta, \delta]$$

      It then follows that for each interval $r_k$, we have $f(r_k) \subset U$ or $f(r_k) \subset V$. Now, we paste together intervals to get a new set of intervals
      $[s_1, s_2], \ ..., \ [s_{n - 1}, s_n]$ to get a new set of interval $q_k$, with endpoints in $U \cap V$ (which we can clearly do, as we have a finite collection).
      \newline

      Clearly, $[f] = [q_1] * \cdots * [q_k]$. Finally, since $U \cap V$ is path connected, for each path $q_k$, choose a path $\alpha_k$ from $x_0$ to the start point of $q_k$ and
      $\beta_k$ to the endpoint. Note that we can have $\alpha_{k + 1} = \beta_k$ and $\beta_{k - 1} = \alpha_k$. Hence, inserting these products between each element of the product
      gives a product of loops entirely in $U$ and $V$.
    \end{proof}

    \newpage

    \subsection{Example: The Torus}

    We begin by noting that we can express the torus as a union of two open set $U$ and $V$ with path-connected intersection: the torus with a point removed, and the torus without its boundary, respectively.
    \newline

    Clearly, $U$ deform retracts to the boundary, which is itself a wedge of circles. Hence, the inclusion-induced homomorphism $i_{*} : \pi_1(\partial U, x_1) \rightarrow \pi_1(U, x_1)$ is an isomorphism, where $x_1$
    is some point in $\partial U$. We know that $\pi_1(\partial U, x_1) \simeq \mathbb{Z} * \mathbb{Z}$. Finally, we know that the change-of-basepoint map $\hat{\alpha}$ is an isomorphism from $\pi_1(U, x_0)$ to $\pi_1(U, x_1)$.
    \newline

    We also note that $V$ is simply connected, so $\pi_1(V, x_0) \simeq \emptyset$, where $x_0$ is our base-point.
    \newline

    Finally, note that $U \cap V$ is the torus with a point and its boundary removed, which deform retracts to a circle $S$ pasing through $x_0$. Thus, $j_{*} : \pi_1(S, x_0) \rightarrow \pi_1(U \cap V, x_0)$
    is an isomorphism, so $\pi_1(U \cap V, x_0) \simeq \mathbb{Z}$.
    \newline

    Now, we can construct $\pi_1(U \cup V, x_0)$ using Seifert Van-Kampen theorem. We know that this group will be the free group of generators given by generators of $\pi_1(U, x_0)$ and $\pi_1(V, x_0)$, along
    with their relations, and the set of relations $R_S$.
    \newline

    Clearly, the generators of $\pi_1(U, x_0)$ are single loops around each loop in the wedge (up to a change of basepoint), and $\pi_1(V, x_0)$ has no generators. In addition, the only relations arise from $R_S$, in which we identify
    $(\phi_1)_{*}([s]) = (\phi_2)_{*}([s])$, where $[s] \in \pi_1(U \cap V, x_0)$, and the two $\phi_1$ and $\phi_2$ maps are inclusions of $U \cap V$ into $U$ and $V$ respectively.
    \newline

    Let $[\omega]$ be the generator of $\pi_1(U \cap V, x_0)$ send to $1$ by the bijection between this group and $\mathbb{Z}$, which is the loop that goes once around the central hole.
    Clearly, we will have $[\omega]$ sent to the trivial loop in $\pi_1(V, x_0)$, as this
    is a trivial group.
    \newline

    It is also clear that sending $[\omega]$ into $U$ will yield a loop that is path-homotopic to a loop which starts at $x_0$, goes to $x_1$, goes around the boundary once, then goes back to $x_0$.
    This is isomorphic to the loop in $\pi_1(\partial U, x_1)$ which goes around the loop $a$, around $b$, then around $a^{-1}$, then around $b^{-1}$. Thus, we can write $ab a^{-1} b^{-1} = 1$, where $1$ is the trivial loop.
    \newline

    It follows that:

    $$\pi_1(T, x_0) = \pi_1(U \cup V, x_0) = \langle a, b \ | \ ab = ba \rangle = \mathbb{Z} \times \mathbb{Z}$$

    \section{Covering Spaces Notes and Examples}

    These notes are on universal covering spaces.
    \newline

    \subsection{Steps for Drawing a Universal Covering Space}

    Below is a method for computing the universal covering space of a wedge of two spaces:

    \begin{enumerate}
    \item Let $\tilde{X}$ and $\tilde{Y}$ be universal covering spaces of $X$ and $Y$. Let $x$ be the point at which $X$ and $Y$ are wedged.
    \item Take a copy of $\tilde{X}$, and note the position of each element of $p^{-1}(x)$. At each of these points, join a copy of $\tilde{Y}$
      at some $q^{-1}(x)$ in $\tilde{Y}$.
    \item For all the unidentified points in each copy of $\tilde{Y}$, join a copy of $\tilde{X}$ at some $p^{-1}(x)$.
      \item Repeat until termination of infinity!
    \end{enumerate}

    To find the universal covering spaces of a sphere with a diameter, it is simply a matter of pushing the diameter out onto the surface of the sphere.

    \section{Other Practice Problems}

    \hrulefill

    \subsection{Munkres: Problem 57.2}

    \hrulefill

    


   \end{document}
