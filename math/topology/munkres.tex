\documentclass[10pt, oneside]{amsart} 
\usepackage{amsmath, amsthm, amssymb, wasysym, verbatim, bbm, color, graphics, geometry, hyperref, biblatex, mathtools}
\usepackage{tcolorbox}

\hypersetup{
	colorlinks=true,
	linkcolor=blue,
	urlcolor=blue
}


\geometry{tmargin=1.25in, bmargin=1.25in, lmargin=1.25in, rmargin =1.25in}
\setlength\parindent{0pt}

\tcbuselibrary{theorems}
\newtcbtheorem
    []% init options
    {problem}% name
    {Problem}% title
    {%
      fonttitle=\bfseries,
    }% options
    {prob}% prefix

    \newcommand{\R}{\mathbb{R}}
    \newcommand{\C}{\mathbb{C}}
    \newcommand{\Z}{\mathbb{Z}}
    \newcommand{\N}{\mathbb{N}}
    \newcommand{\Q}{\mathbb{Q}}
    \newcommand{\Cdot}{\boldsymbol{\cdot}}
    \newcommand{\U}{\mathcal{U}}
    \newcommand{\V}{\mathcal{V}}

    \newtheorem{thm}{Theorem}
    \newtheorem{defn}{Definition}
    \newtheorem{conv}{Convention}
    \newtheorem{rem}{Remark}
    \newtheorem{lem}{Lemma}
    \newtheorem{cor}{Corollary}
    \newtheorem{prop}{Proposition}
    \newtheorem{prob}{Problem}

    \newcommand{\tr}{\mathrm{Tr}}
    \newcommand{\bm}{\boldsymbol}
    \DeclarePairedDelimiter\floor{\lfloor}{\rfloor}


    \title{Munkres Topology: Chapter 2}
    \author{Jack Ceroni}
    \date{September 2021}

    \begin{document}

    \maketitle

    \tableofcontents

    \vspace{.25in}

    \newpage

    \section{Section 13: Bases}

    \subsection{Theorems}

    \subsection{Problems and Solutions}

    \begin{prop}
      If $\mathcal{A}$ is a basis for a topology on $X$, then the topology generated by $\mathcal{A}$ is the intersections of all topologies containing $\mathcal{A}$. Same holds true if $\mathcal{A}$ is a subbasis.
    \end{prop}

    \begin{proof}
      Let $\mathcal{T}_{\mathcal{A}}$ be the topology generated by $\mathcal{A}$. If $U \in \mathcal{T}_{\mathcal{A}}$, then it is a union of elements of $\mathcal{A}$. Clearly, all topologies containing $\mathcal{A}$ must contain
      all unions of elements of $\mathcal{A}$, so $U$ is in the intersection of all such topologies and $\mathcal{T}_{\mathcal{A}}$ is in the intersection. Similarly, note that $\mathcal{T}_{\mathcal{A}}$ contains $\mathcal{A}$, so it contains
      the intersection of all topologies containing $\mathcal{A}$. Thus, $\mathcal{T}_{\mathcal{A}}$ is equal to the intersection.
      \newline

      If $\mathcal{A}$ is a subbasis, we recylce the above proof except note that some $U \in \mathcal{T}_{\mathcal{A}}$ is a union of intersections of elements of $\mathcal{A}$, which must be in each topology containing $\mathcal{A}$, so
      it is in the intersection.
    \end{proof}

    \begin{prop}
     The topologies $\mathbb{R}_{\ell}$ and $\mathbb{R}_{K}$ are not comparable.
    \end{prop}

    \begin{proof}
      Pick the basis element $B = (-1, 1) - K$ of the $K$-topology. Every basis element of the lower-limit topology containing $0$ contains some element of $K$. Hence, the $K$-topology is not contained in the lower limit topology. Consider the
      basis element $B' = [0, 1)$ of the lower limit topology. There is no basis element of the $K$-topology containing $0$ which is in $B'$. Hence, the lower limit topology is not in the $K$-topology.
    \end{proof}

    \begin{prop}
      If $T_{\alpha}$ is a family of topologies, then $\cap T_{\alpha}$ is also a topology. Moreover, there exists a unique smallest topology containing all $T_{\alpha}$, and a unique largest topology contained in all $T_{\alpha}$.
    \end{prop}

    \begin{proof}
      Checking that the intersection of arbitrarily many topologies is easy. Clearly, $X, \emptyset \in \cap T_{\alpha}$. In addition, note that unions and finite intersections
      of some subcollection of elements contained in all $T_{\alpha}$ must also be in all $T_{\alpha}$, by definition of a topology. Thus, $\cap T_\alpha$ is a topology.
      \newline

      We assert that the unique smallest topology is the topology $B$ having $\cup T_{\alpha}$ as a subbasis. Indeed, this is a valid basis. Clearly, this topology contains all $T_{\alpha}$. Finally,
      suppose $R$ is another topology that contains each $T_\alpha$. If $U$ is open in $B$, it is a union of finite intersections of elements of $\cup T_{\alpha}$, so $U \subset R$ as well. Thus, $R$ is finer than $T_\alpha$.
      \newline

      We assert that the unique largest topology is $\cap T_\alpha$. Clearly, $\cap T_\alpha$ is contained in each $T_\alpha$. In addition, suppose $R$ is another topology contained in each $T_\alpha$. Then it follows that
      $R$ is in their intersection, so $\cap T_\alpha$ is finer than $R$.
    \end{proof}

    \begin{prop}
      Classify the standard topology, $K$-topology, finite complement, upper limit, and upper-ray basis topology.
    \end{prop}

    \begin{proof}
      Clearly, $\mathcal{T}_5 \subset \mathcal{T}_1$, where inclusion is proper, and $\mathcal{T}_1 \subset \mathcal{T}_2$, where inclusion is also proper (Consider the basis element $S = (-1, 1) - K$. The point $0$ is not contained in any basis
      element of the standard topology that is itself contained in $S$). It's also pretty clear that $\mathcal{T}_4$ is finer than $\mathcal{T}_1$ (properly). In addition, $\mathcal{T}_4$ is finer than $\mathcal{T}_2$ (properly, consider then interval $(2, 3]$, clearly
      this interval is not open in the $K$-topology).
      \newline

      Now, note that $\mathcal{T}_1$ is finer than $\mathcal{T}_3$, as if $X - U$ is a finite set of points, then $U$ can be written as a union of intervals, so $U$ is open in $\mathcal{T}_1$. This inclusion is clearly proper.
      We also assert that $\mathcal{T}_3$ and $\mathcal{T}_5$ are not comparable. Clearly, some interval of the form $(-\infty, a)$ is not in $\mathcal{T}_3$. Similarly, note that the point $2$ in the
      open set $(-\infty, 1) \cup (1, \infty)$ in $\mathcal{T}_3$ is not contained in some basis element of the form $(-\infty, a)$, so it is not open in $\mathcal{T}_5$.
      \newline

      Thus, we have:

      $$\mathcal{T}_5, \mathcal{T}_3 \subset \mathcal{T}_1 \subset \mathcal{T}_2 \subset \mathcal{T}_4$$
    \end{proof}

    \section{Section 16: Different Topologies}

    \subsection{Theorems}

    \begin{lem}
    The collection of:

    $$S = \{ \pi_1^{-1}(U) \ | \ U \ \text{open \ in} \ X\} \cup \{ \pi_2^{-1}(V) \ | \ V \ \text{open \ in} \ Y\}$$

    is a subbasis for the product topology.
    \end{lem}

    \begin{proof}
      Let $T_S$ be the topology generated by the subbasis and $T$ be the product topology. Note that each element of $S$ is in $T$, so unions of
      finite intersections of elements of $S$ will be in $T$ as well. Thus, $T_S \subset T$.
      \newline

      Now, pick basis element $U \times V$ in $T$. Note that:

      $$U \times V = \pi_1^{-1}(U) \cap \pi_2^{-1}(V)$$

      which is in $T_S$. Hence $U \times V$ is in $T_S$. It follows unions of such elements will be in $T_S$ as well, so
      every open $U$ in $T$ is also open in $T_S$. This implies that $T \subset T_S$.
      \newline

      We have inclusion both ways, so $T_S = T$.
    \end{proof}

    \begin{lem}
      The product and the subspace topologies are equivalent.
    \end{lem}

    \begin{proof}
      Let $(U \times V) \cap (A \times B)$ be a basis element for the subspace topology. We have:

      $$(U \times V) \cap (A \times B) = (U \cap A) \times (V \cap B)$$

      so this is also a basis element for the product topology on $A \times B$. Similarly, if $(U \cap A) \times (V \cap B)$
      is a basis element for the product topology, it is also a basis element for the subspace topology, by the above equation.
      \newline

      The two topologies have the same basis, so they are equal.
    \end{proof}

    \subsection{Problems and Solutions}

    \begin{prop}
      The dictionary order topology on $\mathbb{R} \times \mathbb{R}$ is the same as the product topology $\mathbb{R}_d \times \mathbb{R}$, where $\mathbb{R}_d$ is the discrete topology on
      $\mathbb{R}$. Compare this with the product topology on $\mathbb{R}^2$.
    \end{prop}

    \begin{proof}
      First, note that every basis element of $\mathbb{R}_{d} \times \mathbb{R}$ of the form $\{x\} \times (a, b)$ is an element of the order topology, so the topology $\mathbb{R}_d \times \mathbb{R}$ is in the order topology.
      Also, note that every element of the order topology is a union of elements of the above form, so is also open in $\mathbb{R}_d \times \mathbb{R}$. Thus, the topologies are the same.
      \newline

      We claim that the product topology is coarser than both of these topologies. Indeed, $\{x\} \times (a, b)$ is not open in the product topology, but any open rectangle can be expressed as a union of elements $\{x\} \times (a, b)$, so
      it is open in the order topology.
      \end{proof}

    \section{Section 17: Closures, Closed Sets}

    \begin{prop}
      Prove that $A \subset B \Rightarrow \bar{A} \subset \bar{B}$, $\overline{A \cup B} = \bar{A} \cup \bar{B}$, and $\overline{ \bigcup A_\alpha} \superset \bigcup \overline{A_\alphA}$. Give an example where equality fails for the
      last proof.
      \end{prop}


    \begin{prop}
      $\mathrm{Int}(A)$ and $\partial(A)$ are disjoint. Moreover, $\overline{A} = \mathrm{Int}(A) \cup \partial(A)$.
    \end{prop}

    \begin{proof}
      If $x \in \mathrm{Int}(A)$, then it follows that it is contained in an open set that is contained in $A$, and therefore does not intersect $X - A$, so $x \notin \partial(A)$. Thus, $\mathrm{Int}(A)$ contains no point that
      is also in $\partial(A)$, so the two sets must be disjoint.
      \newline

      Clearly, $\text{Int}(A) \cup \partial(A) \subset \overline{A}$ (this follows directly from the definitions). Moreover, picking some $x \in \overline{A}$,
      we know that every neighbourhood of $x$ intersects $A$. Note that
      either there exists a neighbourhood of $x$ is contained in $A$, so it is in the interior, or every neighbourhood of $x$ intersects $X - A$, so it is in the boundary. Thus, $\overline{A} \subset \text{Int}(A) \cup \partial(A)$. Equality follows.
    \end{proof}

    \begin{prop}
      $\partial(A) = \emptyset \Leftrightarrow A$ is both open and closed.
    \end{prop}

    \begin{proof}
      Suppose first that $\partial(A) = \emptyset$. It follows that $\overline{A} \cap \overline{X - A} = \emptyset$. Thus, every point $x \in A$ is contained in a neighbourhood that is contained entirely in $A$, so $A$ is open. In addition,
      every point $x \in X - A$ is contained in a neighbourhood entirely in $X - A$, so $X - A$ is open and $A$ is closed.
      \newline

      Conversely, suppose $A$ is both open and closed. Since $A$ is closed, $\overline{A} = A$ and since it is open, $\overline{X - A} = X - A$. Thus:

      $$\partial(A) = \overline{A} \cap \overline{X - A} = A \cap (X - A) = \emptyset$$
    \end{proof}

    \begin{prop}
      $U$ open $\Leftrightarrow \partial(U) = \overline{U} - U$.
    \end{prop}

    \begin{proof}
      Suppose $U$ is open. Then $X - U$ is closed, so:

      $$\partial(U) = \overline{U} \cap \overline{X - U} = \overline{U} \cap (X - U) = \overline{U} - U$$

      Conversely, suppose that $\partial(U) = \overline{U} - U$. We then have:

      $$\partial(U) = \left( \text{Int}(U) \cup \partial(U) \right) \cap (X - U) = \partial(U) \cap (X - U)$$

      Thus, $\partial(U) \subset X - U$, which implies that if $x \in U$, then it must be contained in a neighbourhood that does not intersect $X - U$. This implies that $U$ is open.
    \end{proof}

    \begin{prop}
      It is not necessarily true that $U = \text{Int}(\overline{U})$ when $U$ is open.
    \end{prop}

    \begin{proof}
      Consider $\mathbb{R} \times \mathbb{R}$ in the dictionary order topology. Consider the open set $U = (a, b) \times (c, d)$. It is easy to see that:

      $$\overline{(a, b) \times (c, d)} = [a, b] \times [c, d]$$

      From here, one can check that $\text{Int}([a, b] \times [c, d]) = [a, b] \times (c, d)$, when we work in the order topology, which is clearly different from $U$ itself.
    \end{proof}

    \section{Section 18: Continuous Functions}

    \begin{prop}
      If $f$ and $g$ are continuous, then the set $U = \{x \ | \ f(x) \leq g(x) \}$ is closed. Then, show that $h(x) = \text{min} \left(f(x), g(x)\right)$ is continuous.
    \end{prop}

    \begin{proof}
      
    \end{proof}

    \begin{prop}
      Let $\{A_{\alpha} \}$ be a collection of subsets of $X$ whose union is $X$. Let $f : X \rightarrow Y$, with $f | A_{\alpha}$ continuous for each $\alpha$. If the collection is finite and each
      set is closed, then $f$ is continuous.
    \end{prop}

    \begin{proof}
      The proof of this statement is similar to the proof that ``local continuity'' implies continuity. Note that for some closed set $C$ in $Y$:

      $$f^{-1}(C) =  \displaystyle\bigcup_{\alpha} \left( f^{-1}(C) \cap A_\alpha \right) = \displaystyle\bigcup_{\alpha} (f|{A_\alpha})^{-1}(C) = U$$

      Since $U$ is a finite union of closed sets, it is closed, so $f$ is continuous. Alternatively, this follows directly from the pasting lemma.
    \end{proof}

    \begin{prop}
      Find an example where $\{A_\alpha\}$ is countable and each $A_\alpha$ is closed by $f$ is not continuous.
    \end{prop}

    \begin{proof}
      For each $x$, let $U_x$ be a neighbourhood around $x$ that only intersects finitely many $A_\alpha$. Considering the finite collection of closed sets of the form $U_x \cap A_\alpha$ in the subspace
      topology for $U_x$, we note that since each $f | A_\alpha$ is continuous, so too is each $f | (U_x \cap A_\alpha)$, as this amounts to a domain restriction. By the above proposition, it follows that each
      $f | U_x$ is continuous. Then, by the local formulation of continuity, $f$ is continuous as well.
    \end{proof}

    \begin{prop}
      Suppose $F : X \times Y \rightarrow A$ is continuous. Then it is continuous in both arguments separately.
    \end{prop}

    \begin{proof}
      Pick $y_0 \in Y$. Note that the map $g$ from $X$ to $X \times Y$ such that $x \rightarrow (x, y_0)$ is continuous, as each component function is continuous (identity and constant fucntions). Thus, $F \circ g$ is continuous,
      as it is the composition of continuous functions, and it is precisely the function $f$ such that $f(x) = F(x \times y_0)$. The claim follows.
      \end{proof}

    \begin{prop}
      Let $A \subset X$. Let $f : A \rightarrow Y$ be continuous, $Y$ be Hausdorff. Show that if $f$ may be extended to continuous $g : \overline{A} \rightarrow Y$, then $g$ is uniquely determined by $f$.
    \end{prop}

    \begin{proof}
      Suppose there exist two distinct continuous functions $h$ and $g$ such that $h(x) = g(x) = f(x)$ for all $x \in A$. Let $x$ be a point such that $h(x) \neq g(x)$, so $x \notin A$. Since $Y$ is Hausdorff,
      we choose disjoint neighbourhoods $U$ and $V$ around $h(x)$ and $g(x)$ respectively. Since $h$ and $g$ are continuous, $h^{-1}(U)$ and $g^{-1}(V)$ are open and contain $x$. Since $x \in \overline{A}$, both must intersect $A$.
      Clearly, there cannot exist some point $a \in A$ that is in both $h^{-1}(U)$ and $g^{-1}(V)$, as we would have $h(a) = g(a) = f(a)$ in both $U$ and $V$, contradicting the fact that they are disjoint. Thus, $h^{-1}(U) \cap g^{-1}(V)$ does not intersect $A$,
      contradicting the fact that $x \in \overline{A}$.
      \newline

      It follows that distinct $h$ and $g$ cannot exist and the function is uniquely determined by $f$.
    \end{proof}

    \begin{prop}
      In both the box and product topolgies:

      $$\displaystyle\prod_{\alpha} \overline{A_\alpha} = \overline{\displaystyle\prod_{\alpha} A_{\alpha}}$$
    \end{prop}

    \begin{proof}
      Suppose $\bm{x} \in \prod_{\alpha} \overline{A_\alpha}$. Pick some open neighbourhood $U$ of $\bm{x}$. In both the box and product topologies, we have:

      $$\bm{x} \in B = \displaystyle\prod_{\alpha} U_{\alpha} \subset U$$

      where each $U_{\alpha}$ is open. Since $x_{\alpha} \in \overline{A_\alpha}$, it follows that $U_{\alpha}$ intersects $A_{\alpha}$ for each $\alpha$. Thus, $B$ must intersect $\prod_{\alpha} A_{\alpha}$,
      so $U$ does as well. It follows that $\bm{x} \in \overline{\prod_{\alpha} A_{\alpha}}$. Conversely, suppose $\bm{x} \in \overline{\prod_{\alpha} A_{\alpha}}$. Picking some $x_{\beta}$, we pick some neighbourhood
      $U$. Then, consider the set open in both the product and box topologies:

      $$V = \displaystyle\prod_{\alpha} V_{\alpha} \ \ \ \ V_{\alpha} = U \ \text{for} \ \alpha = \beta, \ X_{\alpha} \ \text{otherwise}$$

      This open set must intersect $\prod_{\alpha} A_{\alpha}$, so it follows that $U$ intersects $A_{\beta}$. Thus, $x_{\beta} \in \overline{A_{\beta}}$. It follows that $\bm{x} \in \prod_{\alpha} \overline{A_{\alpha}}$.
    \end{proof}

    \begin{prop}
      A component-wise map into a product space (in the product topology) is continuous if and only if each component function is continuous.
    \end{prop}

    \begin{proof}
      One direction is easy: we know that $f_{\alpha} = \pi_{\alpha} \circ f$, so if $f$ is continuous, $f_{\alpha}$ is the composition of two continuous functions, making it continuous.
      \newline

      Conversely, suppose each $f_{\alpha}$ is continuous. Picking some basis element $B$ in $\prod_{\alpha} X_{\alpha}$, we note that:

      $$B = \pi_{\alpha_1}^{-1}(U_{\alpha_1}) \cap \cdots \cap \pi_{\alpha_n}^{-1}(U_{\alpha_n}) \Rightarrow f^{-1}(B) = (f^{-1} \circ \pi_{\alpha_1}^{-1})(U_{\alpha_1}) \cap \cdots \cap (f^{-1} \circ \pi_{\alpha_n}^{-1})(U_{\alpha_n})$$
      $$ = (\pi_{\alpha_1} \circ f)^{-1}(U_{\alpha_1}) \cap \cdots (\pi_{\alpha_n} \circ f)^{-1}(U_{\alpha_n}) = f_{\alpha_1}^{-1}(U_{\alpha_1}) \cap \cdots f_{\alpha_n}^{-1}(U_{\alpha_n})$$

      which is a finite product of open sets, and is therefore open. It follows that $f$ is continuous and the proof is complete. Note that this proof will sometimes fail for the product topology, as
      the above intersection is not guaranteed to be finite (if we were to formulate basis elements of the box topology in this form), and we could end up with a set that is not open.
    \end{proof}


    \begin{prop}
      Let $\mathbb{R}^{\infty}$ be the subset of $\mathbb{R}^{\omega}$ consisting of all sequences that are eventually $0$. What is the closure of $\mathbb{R}^{\infty}$ in the box and product topologies?
    \end{prop}

    \begin{proof}
      In the box topology, the closue of $\mathbb{R}^{\infty}$ is $\mathbb{R}^{\infty}$. Picking some $\bm{x}$ in the complement of $\mathbb{R}^{\infty}$, we note that $\bm{x}$ is not eventually $0$. Thus, for each
      non-zero $x_{\alpha}$, we choose an open interval $U_{\alpha}$ around $x_{\alpha}$ that does not contain $0$. The open set $U = \prod_{\alpha} U_{\alpha}$ then does not intersect $\mathbb{R}^{\infty}$, so
      $\overline{\mathbb{R}^{\infty}} = \mathbb{R}^{\infty}$.
      \newline

      In the product topology, the closure of $\mathbb{R}^{\infty}$ is $\mathbb{R}^{\omega}$. Picking any $\bm{x} \in \mathbb{R}^{\omega}$, note that $\bm{x}$ is contained in a basis element of the form $\prod_{\alpha} U_{\alpha}$,
      where $U_{\alpha}$ eventually equals $\mathbb{R}$. Thus, since $\mathbb{R}$ contains $0$, $U$ will contain a sequence that is eventually equal to $0$, so $\bm{x} \in \overline{\mathbb{R}^{\infty}}$.
    \end{proof}

    \begin{prop}
      Let $f_{\alpha} : A \rightarrow X_{\alpha}$. Show that there exists a unique coarsest topology relative to which each $f_{\alpha}$ is continuous.
    \end{prop}

    \begin{proof}
      For fun, let's do this in a non-constructive fashion. Let $\mathcal{T}_a$ and $\mathcal{T}_b$ be two topologies such that each $f_{\alpha}$ is continuous. Let $\mathcal{T} = \mathcal{T}_a \cap \mathcal{T}_b$.
      It is easy to see that $\mathcal{T}$ is also a topology. Moreover, for some open set $V \in X_{\alpha}$, note that $f^{-1}_{\alpha}(V)$ must be in both $\mathcal{T}_a$ and $\mathcal{T}_b$, by definition, so it is in $\mathcal{T}$.
      Thus, each $f_{\alpha}$ is continuous with respect to $\mathcal{T}$.
      \newline

      Taking the intersection of all topologies with repsect to which each $f_{\alpha}$ is continuous (which we denote $\mathcal{T}'$), and we are left with the unique coarsest topology, as desired.
    \end{proof}

    \begin{prop}
      Show that the set:

      $$S = \displaystyle\bigcup_{\beta} \{ f_{\beta}^{-1}(U_{\beta}) \ | \ U_{\beta} \ \mathrm{open} \ \mathrm{in} \ X_{\beta} \}$$

      is a subbasis for the above topology.
    \end{prop}

    \begin{proof}
      Clearly, this subbasis generates a topology with respect to each $f_{\beta}$ is continuous, so $\mathcal{T}' \subset \mathcal{T}_{S}$. Conversely, note that we must have $f^{-1}_{\beta}(V)$ in $\mathcal{T}'$ for
      all $\beta$, where $V$ is open in $X_{\beta}$. Thus, all finite intersections and unions of such elements must also be in $\mathcal{T}'$. Thus, every set generated by the subbasis is in $\mathcal{T}'$, so $\mathcal{T}_{S} \subset \mathcal{T}'$.
      Equality follows.
    \end{proof}

    \begin{prop}
      Show that a map $g : Y \rightarrow A$ is continuous relative to $\mathcal{T}'$ if and only if each map $f_{\alpha} \circ g$ is continuous.
    \end{prop}

    \begin{proof}
      Suppose $g$ is continuous. Then $f_{\alpha} \circ g$ is a composition of continuous maps, so it is continuous.
      \newline

      Conversely, suppose each $f_{\alpha} \circ g$ is continuous. A general subbasis element of $A$ is of the form $f_{\beta}^{-1}(U_{\beta})$. Hence, we will have:

      $$g^{-1} (f_{\beta}^{-1}(U_{\beta})) = (f_{\beta} \circ g)^{-1}(U_{\beta})$$

      is open, as $f_{\beta} \circ g$ is continuous. It follows that $g$ is continuous as well.
    \end{proof}

    \begin{prop}
      Let $f : A \rightarrow \prod_{\alpha} X_{\alpha}$. Show that the image of each open set in $A$ under $f$ is open in $f(A)$.
    \end{prop}

    \begin{proof}
      It is enough to show that this is true for all basis elements. A general basis element of $A$ will be of the form $V = \cap_{j = 1}^{n} f^{-1}_{\alpha_j}(U_{\alpha_j})$. Note that:

      $$V = \displaystyle\bigcap_{j = 1}^{n} f^{-1}_{\alpha_j}(U_{\alpha_j}) = f^{-1} \left( \displaystyle\prod_{\alpha} U_{\alpha} \right)$$

      where $U_{\alpha} = X_{\alpha}$ for all $\alpha$ not in the set $\alpha_1, \ ..., \ \alpha_n$. It follows that:

      $$f(V) = f(A) \cap  \displaystyle\prod_{\alpha} U_{\alpha}$$

      which is open in the subspace topology on $f(A)$.
    \end{proof}

    \section{Section 20: Metric Topology}

    \begin{prop}
      The uniform topology on $\mathbb{R}^{J}$ is finer than the product topology and coarser than the box topology.
    \end{prop}

    \begin{proof}
      Recall that the uniform topology is the metric topology generated by the metric $\overline{\rho}(\bm{x}, \bm{y}) = \sup \{ \overline{d}(x_i, y_i) \ | \ i \in J \}$. Suppose we have some point $\bm{x}$ and a
      basis element containing $\bm{x}$ in the product topology of the form $B = \prod_{\alpha \in J} U_{\alpha}$, where $U_{\alpha} = \mathbb{R}$ for all but finite $\alpha$. It is easy to pick $\epsilon_{\alpha} < 1$ (for the
      finite set of $\alpha$ where $U_{\alpha} \neq \mathbb{R}$ such that $(x - \epsilon_\alpha, x + \epsilon_\alpha) \subset U_{\alpha}$. Letting $\epsilon = \min_{\alpha} \epsilon_\alpha$, we note that the basis element
      $B_{\overline{\rho}}(\bm{x}, \epsilon)$ is contained in $B$. Thus, the uniform topology is finer than the product topology.
      \newline

      Now, pick basis element $B_{\overline{\rho}}(\bm{x}, \epsilon)$ in the uniform topology. Note that the basis element:

      $$B = \displaystyle\prod_{\alpha \in J} (x_\alpha - \epsilon, x_\alpha + \epsilon)$$

      in the box topology is contained in $B_{\overline{\rho}}(\bm{x}, \epsilon)$.
    \end{proof}

    Interestingly enough, I think there is some intuition to this. Basically, the product topology ony gives us information about a finite number of the sets in some product space, whereas the box topology gives us information about all of the
    sets. The reason that the uniform metric is finer than the product topology is because it can always encode information about a finite number of sets. However, there may be cases where the distances involved in computing the uniform topology converge
    to some undesireable place, which is the reason that it is coarser than the box topology.

    \begin{prop}
      Let $\overline{d}$ be the standard bounded metric on $\mathbb{R}$. On $\mathbb{R}^{\omega}$, define the metric:

      $$D(\bm{x}, \bm{y}) = \sup \left\{ \frac{ \overline{d}(x_i, y_i)}{i} \right\}$$

      This metric gives rise to the product topology on $\mathbb{R}^{\omega}$.
    \end{prop}

    \begin{proof}
      First pick a basis element of the form $B = \prod_{\alpha} U_{\alpha}$, where $U_{\alpha} = \mathbb{R}$ for all but finite $\alpha$ containing some $\bm{x}$. We know that for each $U_{\alpha}$, we
      have $x_\alpha \in (x_\alpha - \epsilon_\alpha, x_\alpha + \epsilon_\alpha) \subset U_\alpha$, for some $\epsilon_\alpha < 1$. Now, let $\epsilon = \min_{i} \{\epsilon_i/i\}$. Consider the set $B_{D}(\bm{x}, \epsilon)$.
      Clearly, for $\bm{y} \in B_D$, we will have:

      $$D(\bm{x}, \bm{y}) < \epsilon \Rightarrow \frac{\overline{d}(x_i, y_i)}{i} < \frac{\epsilon_i}{i} \Rightarrow \overline{d}(x_i, y_i) < \epsilon_i \Rightarrow d(x_i, y_u) < \epsilon_i$$

      for all $i$. Thus, $\bm{y} \in U$, so the topology generated by $D$ is finer than the product topology. Conversely, consider the basis element $B_{D}(\bm{x}, \epsilon)$ of the metric topology. Without loss of generality, let $\epsilon < 1$. Let us choose
      natural $N$ such that $\frac{1}{N} < \epsilon$. We consider the basis element in the product topology where:

      $$B = \displaystyle\prod_{j = 1}^{N - 1} \left( x_j - \epsilon, x_j + \epsilon \right) \times \mathbb{R} \times \mathbb{R} \times \cdots$$

      and note that for $\bm{y} \in B$, for $i < N$, we will have:

      $$\frac{\overline{d}(x_i, y_i)}{i} \leq \overline{d}(x_i, y_i) = d(x_i, y_i) < \epsilon$$

      and for $i \geq N$, we will have:

      $$\frac{\overline{d}(x_i, y_i)}{i} \leq \frac{1}{i} \leq \frac{1}{N} < \epsilon$$

      Thus, $D(\bm{x}, \bm{y}) < \epsilon$, so $\bm{y} \in B_D$. This implies that the product topology is finer than the metric topology. Equality follows.
    \end{proof}

    The reason why this topology gives rise to the product topology is because beyond a certain point, the $1/i$ part of the elements of the set describing $D$ dominate over the numerator, and
    we are effectively left with a ``finite amount'' of information describing our space.

    \begin{prop}
      Let $X$ be a metric space. A metric $d : X \times X \rightarrow \mathbb{R}$ is continuous, where $X \times X$ is given the product topology.
    \end{prop}

    \begin{proof}
      The idea here is to exploit the triangle inequality in a slick way. Suppose we have some open $U \in \mathbb{R}$ with $d(x, y) \in U$. Pick some $A = \left( d(x, y) - \epsilon, d(x, y) + \epsilon \right) \subset U$.
      We then consider the open sets $B_{d}(x, \epsilon/2)$ and $B_{d}(y, \epsilon/2)$ in $X$. It follows that $B = B_{d}(x) \times B_d(y)$ is open in $X \times X$. Suppose $(a, b) \in B$. It follows that:

      $$d(a, b) \leq d(a, x) + d(x, b) \leq d(a, x) + d(x, y) + d(y, b) < d(x, y) + \frac{\epsilon}{2} + \frac{\epsilon}{2} = d(x, y) + \epsilon$$
      $$d(a, b) \geq d(x, b) - d(a, x) \geq d(x, y) - d(b, y) - d(a, x) > d(x, y) - \epsilon$$

      Thus, $d(a, b) \in A \subset U$. It follows by definition that $d : X \times X \rightarrow \mathbb{R}$ is continuous.
    \end{proof}

    \begin{prop}
      Suppose $X'$ is a topological space with the same underlying set as $X$ but possibly a different topology. If $d : X' \times X' \rightarrow \mathbb{R}$ is continuous, then the
      topology of $X'$ is finer than the metric topology imposed on $X$.
    \end{prop}

    \begin{proof}
      This follows almost immediately from the fact that if a map is continuous, then it is continuous in each of its arguments seperately. Picking some $y$, we note that $d_y : X \rightarrow \mathbb{R}$ with $d_y(x) = d(y, x)$ is
      continuous. Thus, picking some $B_{\epsilon}(y)$ (a basis element for $X$), we note that $d_y^{-1}(-\epsilon, \epsilon)$ is open in $X'$ and is equal to $B_{\epsilon}(y)$. Thus, the topology on $X'$ is finer than the metric topology.
    \end{proof}

    \begin{prop}
      If the space $\mathbb{R}^{\omega}$ is given the uniform topology, then the closure of $\mathbb{R}^{\infty} \subset \mathbb{R}^{\omega}$ is precisely all $x$ such that $x_i \rightarrow 0$ as
      elements of $\mathbb{R}$.
    \end{prop}

    \begin{proof}
      Suppose $\bm{x}$ is such that $x_i \rightarrow 0$. Pick some open neighbourhood $U$ of $\bm{x}$. We then know that we have $\epsilon < 1$ such that $\bm{x} \in B_{\overline{\rho}}(\bm{x}, \epsilon) \subset U$. By definition, there must exist some $N$ such that
      for $i > N$, we have $|x_i| < \delta$, where we choose $\delta < \epsilon$. We let $\bm{y} = (x_1, \ ..., \ x_N, 0, 0, \ ...)$, which is clearly an element of $\mathbb{R}^{n}$. Note that:

      $$\overline{p}(\bm{x}, \bm{y}) = \sup \{ \overline{d}(x_\alpha, y_\alpha) \ | \ \alpha \in \mathbb{N} \} = \sup \{ |x_\alpha - y_\alpha| \ | \ \alpha \in \mathbb{N} \}$$
      $$= \sup \left( \{|x_\alpha - y_\alpha| \ | \ 1 \leq \alpha \leq N\} \cup \{ |x_\alpha| \ | \ \alpha > N \} \right) \leq \delta < \epsilon$$

      as $|x_{\alpha} - y_{\alpha} | = 0$ for $1 \leq \alpha \leq N$ and $|x_i| < \delta$ otherwise. Thus, $\bm{y} \in B_{\overline{\rho}}$, so $\bm{x} \in \overline{\mathbb{R}^{\infty}}$.
      \newline

      Conversely, suppose $\bm{x} \in \overline{\mathbb{R}^{\infty}}$. It follows that every $B_{\overline{\rho}}(\bm{x}, \epsilon)$ around $\bm{x}$ intersects $\mathbb{R}^{\infty}$. Now, let us consider the interval $(-\delta, \delta)$ around
      $0$ in $\mathbb{R}$. Assume WLOG that $\delta < 1$. Let $\bm{y} \in B_{\overline{\rho}}(\bm{x}, \delta)$ and in $\mathbb{R}^{\infty}$. We have:

      $$\overline{\rho}(\bm{x}, \bm{y}) < \delta \Rightarrow |x_i - y_i| < \delta \ \ \ \text{for all} \ i$$

      We also know that there exists
      some $N$ such that for $i > N$, we have $y_i = 0$, by definition. Thus, for $i > N$, we have $|x_i| < \delta$, so all such $x_i$ are in $(-\delta, \delta)$. It follows by definition that $x_i \rightarrow 0$.
    \end{proof}

    \begin{prop}
      Letting $X$ be the subset of $\mathbb{R}^{\omega}$ such that $\sum_{i} x_i^{2}$ converges, show that:

      $$\mathrm{box \ topology} \supset \ \ell^2 \ \mathrm{topology} \supset \mathrm{uniform \ topology}$$
    \end{prop}

    \begin{proof}
      Suppose $B_{\overline{\rho}}(\bm{x}, \epsilon)$ is a basis element in the product topology. Note that for $\bm{y} \in B_{d}(\bm{x}, \delta)$ where we choose $\delta < \epsilon$, we have:

      $$d(\bm{y}, \bm{x}) < \delta \Rightarrow \sqrt{\displaystyle\sum_{i} (x_i - y_i)^2 } < \delta \Rightarrow |x_i - y_i| < \delta \ \text{for all} \ i$$

      Thus, the supremum over $i$ is less than or equal to $\delta$, so it less than $\epsilon$. This implies that $\bm{y} \in B_{\overline{\rho}}$, so $B_{\overline{\rho}} \supset B_{d}$. It follows from an earlier
      result in Munkres that the $\ell^2$ topology is finer than the uniform topology.
      \newline

      Consider the basis element $B$ of the $\ell^2$ topology. For some $\bm{x} \in B$, we have $\bm{x} \in B_d(\bm{x}, \epsilon) \subset B$. Consider the following basis element in the box topology:

      $$U = \displaystyle\prod_{i} \left( x_i - \frac{\epsilon}{2^i}, x_i + \frac{\epsilon}{2^i} \right)$$

      for some $\bm{y} \in U$, note that:

      $$d(\bm{x}, \bm{y}) = \sqrt{\displaystyle\sum_{i} (x_i - y_i)^2} < \sqrt{\epsilon^2 \displaystyle\sum_{i = 1}^{\infty} \frac{1}{4^i}} = \frac{\epsilon}{\sqrt{3}} < \epsilon$$

      so $\bm{y} \in B_{d} \subset B$. Thus, the box topology is finer than the $\ell^2$ topology.
    \end{proof}

    \begin{prop}
      Clearly, $\mathbb{R}^{\infty} \subset X$. Show that the subspace topology for $\mathbb{R}^{\infty}$ from the product, box, $\ell^2$, and uniform topologies are distinct.
    \end{prop}

    \begin{proof}

    \end{proof}

    \begin{prop}
      In the box topology, no function of the form $f : \mathbb{R} \rightarrow \mathbb{R}^{\omega}$ with $f(t) = (a_1 t, a_2 t, ...)$ is continuous, where $a_i > 0$.
    \end{prop}

    \begin{proof}
      Pick the open set:

      $$U = \displaystyle\prod_{n} \left( -\frac{a_n}{n}, \frac{a_n}{n} \right)$$

      in the box topology. We assert that $f^{-1}(U)$ is not open in $\mathbb{R}$. Suppose it is. Note that $0 \in f^{-1}(U)$. Thus, there must be some $(-\delta, \delta)$ such that $0 \in (-\delta, \delta) \subset f^{-1}(U)$, by definition
      of the standard topology. However, this would imply that for all $n$:

      $$(-\delta a_n, \delta a_n) \subset \left( -\frac{a_n}{n}, \frac{a_n}{n} \right)$$

      which clearly is not the case. Thus, $f^{-1}(U)$ can't be open. It follows that $f$ is not continuous.
    \end{proof}

    \begin{prop}
      In the uniform topology, a function of the form $f : \mathbb{R} \rightarrow \mathbb{R}^{\omega}$ with $f(t) = (a_1 t, a_2 t, ...)$ is continuous, where $a_i > 0$ is continuous if and only if the set of $a_i$ is bounded in $\mathbb{R}$.
    \end{prop}

    \begin{proof}
      Suppose we are given some $f$ with the set $\{a_i\}$ bounded. Let $a = \sup \{a_i\}$. Suppose $U$ is open in the uniform topology and contains $f(x)$. It follows that there is some $B_{\overline{\rho}}(f(x), \epsilon)$
      with $\epsilon < 1$ such that $f(x) \in B_{\overline{\rho}}(f(x), \epsilon) \subset U$. Choose $\delta = \frac{\epsilon}{s}$. We then note that:

      $$y \in B(x, \delta) \Rightarrow |y - x| < \delta = \frac{\epsilon}{a}$$
      $$\Rightarrow \overline{\rho}(f(x), f(y)) = \sup \{ |a_i x - a_i y | \} = \sup \{ |a_i| |x - y| \} < \sup \left\{ \frac{|a_i| \epsilon}{a} \right\} \leq \epsilon$$

      Hence, $f(y) \in B_{\overline{\rho}}(f(x), \epsilon)$, so $f(B) \subset B_{\overline{\rho}}(f(x), \epsilon)$. By definition, $f$ is continuous.
      \newline

      Conversely, suppose $f$ is continuous. Consider the open set $B_{\overline{\rho}}(f(x), \epsilon)$ in $\mathbb{R}^{\omega}$, with $\epsilon < 1$. Since $f$ is continuous, we can choose some
      $U = (x - \delta, x + \delta)$ such that $f(U) \subset B_{\overline{\rho}}$. Thus, we have:

      $$y \in U \Rightarrow \sup \left\{ \overline{d}(f(x)_i, f(y)_i) \right\} = \sup \left\{ \min (|a_i||x - y|, 1) \right\} < \epsilon < 1$$

      Since $\epsilon < 1$, we must have $|a_i| | x - y | < 1$ for all $i$ and all $y \in U$. Since we know there exists some $y_0 \neq x$ in $U$, we note that $|a_i| < \frac{1}{|x - y_0|}$ for all $i$, so the set of $a_i$ is bounded.
    \end{proof}

    From the above two propositions, we immediately arrive at the solution to Munkres Exercise 2.20.4. All of the maps are continuous with respect to the product topology, none are continuous with respect to the box topology, and the last two are
    continuous with respect to the uniform topology.

    \section{Section 21}

    \begin{prop}
      Let $X$ be a set, and let $f_n : X \rightarrow \mathbb{R}$ be a sequence of functions. Let $\overline{\rho}$ be a uniform metric on the space $\mathbb{R}^{X}$. The sequence
      $(f_n)$ converges uniformly to $f$ if and only if the sequence $(f_n)$ converges to $f$ as elements of the metric space $(\mathbb{R}^{X}, \overline{\rho})$.
    \end{prop}

    \begin{proof}
      Firstly, note that the uniform metric on $\mathbb{R}^{X}$ is given by:

      $$\overline{\rho}(f, g) = \sup \left\{ \overline{d}(f(x), g(x)) \ | \ x \in X \right\}$$

      Suppose to begin that $f_n$ converges uniformly to $f$. Pick some open $U$
      around $f$. Clearly, we will have $f \in B_{\overline{\rho}}(f, \epsilon/2) \subset U$, for some $\epsilon/2$ (we let $\epsilon/2 < 1$).
      We can choose $N$ such that for $n > N$, we have $d(f_n(x), f(x)) < \epsilon/2$ for all $x$. It then follows that:

      $$\overline{\rho}(f, f_n) = \sup \left\{ \overline{d}(f(x), f_n(x)) \ | \ x \in X \right\} \leq \frac{\epsilon}{2} < \epsilon$$

      so $f_n \in B_{\overline{\rho}}$, for $n > N$. It follows that $f_n$ converges to $f$ in the uniform metric space. Conversely, suppose
      $f_n$ converges to $f$ in the uniform metric. For some $\epsilon$ (WLOG, we let $\epsilon < 1$), we can choose $N$ such that for $n > N$, we have:


      $$\sup \left\{ \overline{d}(f(x), f_n(x)) \ | \ x \in X \right\} < \epsilon$$

      implying that $d(f(x), f_n(x)) < \epsilon$ for all $x$, for all $n > N$. The result follows.
    \end{proof}

    \begin{prop}
      Suppose $x_n \rightarrow x$ and $f_n$ is a sequence of continuous function that converges uniformly to $f$. It follows that $f_n(x_n)$ converges to $f(x)$.
    \end{prop}

    \begin{proof}
      By uniform limit theorem, $f$ is continuous. It follows that $f(x_n)$ converges to $f(x)$. Pick some open interval $U = (f(x) - \epsilon, f(x) + \epsilon)$ around $f(x)$. There exists some
      $M$ such that for $n > M$, we have $|f(x) - f(x_n)| < \epsilon/2$. By uniform continuity, we can choose $N$ such that for $n > N$, we have $|f(x_n) - f_n(x_n)| < \epsilon/2$. Thus, for $n > \max \{M, N\}$, we have:

      $$|f_n(x_n) - f(x)| < |f(x) - f(x_n)| + |f(x_n) - f_n(x_n)| < \epsilon$$

      so $f_n(x_n) \in U$. It follows that $f_n(x_n) \rightarrow f(x)$.
    \end{proof}

    Now, we will look at a few problems involving infinite series:

    \begin{prop}

      \end{prop}

    \end{document}


