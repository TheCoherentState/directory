\documentclass[10pt, oneside]{amsart} 
\usepackage{amsmath, amsthm, amssymb, wasysym, verbatim, bbm, color, graphics, geometry, hyperref, biblatex, mathtools}
\usepackage{tcolorbox}

\hypersetup{
	colorlinks=true,
	linkcolor=blue,
	urlcolor=blue
}


\geometry{tmargin=1.25in, bmargin=1.25in, lmargin=1.25in, rmargin =1.25in}
\setlength\parindent{0pt}

\tcbuselibrary{theorems}
\newtcbtheorem
    []% init options
    {problem}% name
    {Problem}% title
    {%
      fonttitle=\bfseries,
    }% options
    {prob}% prefix

    \newcommand{\R}{\mathbb{R}}
    \newcommand{\C}{\mathbb{C}}
    \newcommand{\Z}{\mathbb{Z}}
    \newcommand{\N}{\mathbb{N}}
    \newcommand{\Q}{\mathbb{Q}}
    \newcommand{\Cdot}{\boldsymbol{\cdot}}
    \newcommand{\U}{\mathcal{U}}
    \newcommand{\V}{\mathcal{V}}

    \newtheorem{thm}{Theorem}
    \newtheorem{defn}{Definition}
    \newtheorem{conv}{Convention}
    \newtheorem{rem}{Remark}
    \newtheorem{lem}{Lemma}
    \newtheorem{cor}{Corollary}
    \newtheorem{prop}{Proposition}
    \newtheorem{prob}{Problem}

    \newcommand{\tr}{\mathrm{Tr}}
    \newcommand{\bm}{\boldsymbol}
    \DeclarePairedDelimiter\floor{\lfloor}{\rfloor}


    \title{Topology Stream}
    \author{Jack Ceroni}
    \date{October 2021}

    \begin{document}

    \maketitle

    \tableofcontents

    \vspace{.25in}

    \newpage

    \section{Introduction: What is this?}

    \section{Stuff}

    \subsection{A cool problem from a previous MAT327 midterm}

    \begin{prop}
      Given a set $X$ with metric $d$ there exists a unique topology that satisfies the following criteria:

      \begin{enumerate}
      \item The function $f_x(y) = d(x, y)$ is continuous for all $x \in X$.
      \item Given another topological space $Z$ and some $g : Z \rightarrow X$, the function $h_x(y) = d(x, g(y))$ being continuous for all $x$ implies that
        $g$ is continuous.
        \end{enumerate}
    \end{prop}

    \begin{proof}
      We claim that this unique topology is exactly the metric topology induced by $d$. First, it is clear that $f_x$ is continuous for each $x$. Picking some
      $f_x(y) = d(x, y) \in \mathbb{R}$, and some open $V$ around $f_x(y)$, we note that:

      $$f_x(y) \in (f_x(y) - \epsilon, f_x(y) + \epsilon) \subset V$$

      for some $\epsilon$. Note that: $y \in B_{\epsilon}(y)$ and:
      \end{proof}

    \end{document}
