\documentclass[10pt, oneside]{amsart} 
\usepackage{amsmath, amsthm, amssymb, wasysym, verbatim, bbm, color, graphics, geometry, hyperref, biblatex, mathtools}
\usepackage{tcolorbox}

\hypersetup{
	colorlinks=true,
	linkcolor=blue,
	urlcolor=blue
}


\geometry{tmargin=1.25in, bmargin=1.25in, lmargin=1.25in, rmargin =1.25in}
\setlength\parindent{0pt}

\tcbuselibrary{theorems}
\newtcbtheorem
    []% init options
    {problem}% name
    {Problem}% title
    {%
      fonttitle=\bfseries,
    }% options
    {prob}% prefix

    \newcommand{\R}{\mathbb{R}}
    \newcommand{\C}{\mathbb{C}}
    \newcommand{\Z}{\mathbb{Z}}
    \newcommand{\N}{\mathbb{N}}
    \newcommand{\Q}{\mathbb{Q}}
    \newcommand{\Cdot}{\boldsymbol{\cdot}}
    \newcommand{\U}{\mathcal{U}}
    \newcommand{\V}{\mathcal{V}}

    \newtheorem{thm}{Theorem}
    \newtheorem{defn}{Definition}
    \newtheorem{conv}{Convention}
    \newtheorem{rem}{Remark}
    \newtheorem{lem}{Lemma}
    \newtheorem{cor}{Corollary}
    \newtheorem{prop}{Proposition}
    \newtheorem{prob}{Problem}

    \newcommand{\tr}{\mathrm{Tr}}
    \newcommand{\bm}{\boldsymbol}
    \DeclarePairedDelimiter\floor{\lfloor}{\rfloor}


    \title{MAT327 Exam Prep}
    \author{Jack Ceroni}
    \date{November 2021}

    \begin{document}

    \maketitle

    \tableofcontents

    \vspace{.25in}

    \newpage

    \section{Introduction}

    GENERAL LIFTING LEMMA WILL BE ON EXAM!!!!!
    BORSUK-ULAM WILL BE ON THE EXAM!!!!!
    \newline

    The goal of these notes is to help me prepare for the algebraic topology section of the MAT327 exam.

    \section{Covering Spaces and the Fundamental Group}

    It seems as though having a thorough understanding of covering spaces will be crucial to understanding algebraic topology. Thus, we dedicate this section of the notes to discussing them.
    \newline

    At a high-level, the idea of a covering space is to ``unravel'' loops in a bigger space, where it is easier to study them. For instance, in the case of $S^1$, it seems pretty reasonable to
    make the guess that $\pi_1(S^1)$ is isomorphic to $\mathbb{Z}$, where $n \in \mathbb{Z}$ represents the number of times we wind around $S^1$. Nevertheless, this requires a proof. We will use the
    notion of a covering space to ``unwind'' loops around $S^1$ so that we can look at them in a space where the ``string'' that we use to make circular loops doesn't intersect itself. In this case,
    we can look at the endpoints of unwinded loops to generate a correspondence between loops in $S^1$ and $\mathbb{Z}$.
    \newline

    Our next goal will be attempt to motivate the definition of a covering space and a covering map. A covering map is the tool that we will use to re-wind all of the loops that we unwinded in our covering space.
    \newline

    Now, we have to develop the machinery needed to actually unwind paths in the original base space. To do this, we will have to define the notion of a lifting map. This does exactly as it sounds.

    \section{Munkres: Section 51}

    \begin{rem}
    The goal of this section of the textbook is to develop the concept of homotopy and path-homotopy. In general, we think of homotopy as a continuous deformation of one
      continuous map to another. Path homotopy, similarly, is a continuous deformation of one path to another.
    \end{rem}

    \begin{defn}
      Let $f : X \rightarrow Y$ and $f' : X \rightarrow Y$ be continuous maps. We say that $f$ is \textbf{homotopic} to $f'$ is there exists a continuous $F : X \times I \rightarrow Y$ such that:

      $$F(x, 0) = f(x) \ \ \ \ \text{and} \ \ \ \ F(x, 1) = f'(x)$$
    \end{defn}

    Similarly, paths $f$ and $f'$ are said to be \textbf{path-homotopic} if they start and end at the same point, and the end-points of the paths are preserved during deformation.
    Note that homotopy and path-homotopy are equivalence relations, which allows us to make equivalence classes. For the most part, we will be interested in path-homotopy classes $[f]$, where
    $f$ is a path.
    \newline

    The eventual goel of this section is to endow the set of path-homotopy classes (or at least a subset of path-homotopy classes) with some kind of algebraic structure. As it turns out, this algebraic structure will
    be a \textbf{group}. Thus, we need to identify a good group operation. We define:

    $$f * g = \begin{cases}
      f(2s) & \text{for} \ s \in \left[ 0, \frac{1}{2} \right] \\
      g(2s - 1) & \text{for} \ s \in \left[\frac{1}{2}, 1\right]
    \end{cases}
    $$

    where $f$ and $g$ are paths from $x_0$ to $x_1$, and $x_1$ to $x_2$ respectively. Such an operation can be visualized as constructing a new path which effectively joins
    two paths together at their common endpoint, and ``runs over'' the path at ``twice the speed'' of the original path.
    \newline

    We now claim that we can extend $*$ to path-homotopy classes as:

    $$[f] * [g] := [f * g]$$

    Indeed, given $f'$ path-homotopic to $f$ (with homotopy $F$) and $g'$ path-homotopic to $g$ (with homotopy $G$), we define a new homotopy:

    $$H = \begin{cases}
      F(2s, t) & \text{for} \ s \in \left[ 0, \frac{1}{2} \right] \\
      G(2s - 1, t) & \text{for} \ s \in \left[ \frac{1}{2}, 1 \right]
    \end{cases}
    $$


    \section{Munkres: Section 53}

    \begin{prob}
      If $p : E \rightarrow B$ is continuous surjective and $U$ is evenly covered by $p$, and $U$ is connected, show that the partition of $p^{-1}(U)$ is unique.
    \end{prob}

    \begin{proof}
      We claim that there exists either some element of $\mathcal{V}$ that intersects more than one element of $\mathcal{U}$
    \end{proof}

\end{document}
