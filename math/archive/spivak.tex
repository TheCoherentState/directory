\documentclass[10pt, oneside]{amsart} 
\usepackage{amsmath, amsthm, amssymb, wasysym, verbatim, bbm, color, graphics, geometry, hyperref, biblatex, mathtools}
\usepackage{tcolorbox}

\hypersetup{
	colorlinks=true,
	linkcolor=blue,
	urlcolor=blue
}


\geometry{tmargin=1.25in, bmargin=1.25in, lmargin=1.25in, rmargin =1.25in}
\setlength\parindent{0pt}

\tcbuselibrary{theorems}
\newtcbtheorem
    []% init options
    {problem}% name
    {Problem}% title
    {%
      fonttitle=\bfseries,
    }% options
    {prob}% prefix

    \newcommand{\R}{\mathbb{R}}
    \newcommand{\C}{\mathbb{C}}
    \newcommand{\Z}{\mathbb{Z}}
    \newcommand{\N}{\mathbb{N}}
    \newcommand{\Q}{\mathbb{Q}}
    \newcommand{\Cdot}{\boldsymbol{\cdot}}
    \newcommand{\U}{\mathcal{U}}
    \newcommand{\V}{\mathcal{V}}

    \newcommand{\vs}{\vspace{0.1pt}}
    \newcommand{\hl}{\vspace{4pt} \hrule \vspace{4pt}}

    \newtheorem{thm}{Theorem}
    \newtheorem{defn}{Definition}
    \newtheorem{conv}{Convention}
    \newtheorem{rem}{Remark}
    \newtheorem{lem}{Lemma}
    \newtheorem{cor}{Corollary}
    \newtheorem{prop}{Proposition}
    \newtheorem{prob}{Problem}

    \newcommand{\tr}{\mathrm{Tr}}
    \newcommand{\bm}{\boldsymbol}
    \DeclarePairedDelimiter\floor{\lfloor}{\rfloor}


    \title{Calculus on Manifolds}
    \author{Jack Ceroni}
    \date{October 2021}

    \begin{document}

    \maketitle

    \tableofcontents

    \vspace{.25in}

    \newpage

    \section{Introduction}

    \section{Chapter 2}

    \subsection{Notes}

    \begin{rem}[Motivating the Chain Rule]
      The chain rule is a very natural statement. More specifically, it states that given a the composition of two functions, $g \circ f$, we will have:

      $$[D(g \circ f)](a) = [Dg](f(a)) \circ [Df](a)$$
      \vs

      which amounts to the statement that a linear approximation near the point $a$ of the function $g \circ f$ is the same as taking a linear approximation
      of $f$ near $a$, and then mapping the points given from this linear approximation using the linear approximation of $g$ near $f(a)$.
    \end{rem}

    \hl

    \begin{thm}[General Chain Rule]
      Given functions $f$ and $g$, such that $f$ is differentiable at $a$, and $g$ is differentiable at $f(a)$, then $g \circ f$ is differentiable at $a$, with:

      \begin{equation}
        [D(g \circ f)](a) = [Dg](f(a)) \circ [Df](a)
        \end{equation}
    \end{thm}

    \vspace{1pt}

    \begin{proof}
      Our goal is to show that in some neighbourhood around $a$, we have:

      $$(g \circ f)(a + h) = (g \circ f)(a) + \left( [Dg](f(a)) \circ [Df](a) \right) h + o(h)$$
      \vs

      where $o(h)$ is small. We first note that in a neighbourhood around $f(a)$, and a neighbourhood around $a$, we have:

      $$g(f(a) + h) = g(f(a)) + [Dg](f(a)) h + q(h) \ \ \ \ \ \ f(a + h) = f(a) + [Df](a) h + p(h)$$
      \vs

      where $q(h)$ and $p(h)$ are small. It follows that in this neighbourhood, we have:

      $$g(f(a + h)) = g \left( f(a) + [Df](a)h + p(h) \right)$$
      \vs

      Since $[Df](a)h + p(h) \rightarrow 0$ as $h \rightarrow 0$, we can choose $h$ small enough such that $f(a) + [Df](a)h + p(h)$ is in the neighbourhood of
      $f(a)$ for which the previous statement holds. In this neighbourhood around $a$, we have:

      $$(g \circ f)(a + h) = (g \circ f)(a) + [Dg](f(a)) \left( [Df](a)h + p(h) \right) + q\left( [Df](a)h + p(h) \right)$$
      \vs
      $$\Rightarrow (g \circ f)(a + h) - (g \circ f)(a) - \left( [Dg](f(a)) \circ [Df](a) \right) h = [Dg](f(a)) p(h) + q\left( [Df](a)h + p(h) \right)$$
      \vs

      so all that is left to do is to show the right-hand side of the above equation is small. Indeed, we have $| [Dg](f(a)) p(h) | \leq M |p(h)|$, for some $M$. Thus:

      $$\frac{ | [Dg](f(a)) p(h) | }{|h|} \leq \frac{M | p(h) |}{|h|}$$

      Clearly, the right-hand side of the inequality goes to $0$ as $h \rightarrow 0$, so the left-hand side does as well. Finally, we prove the final function is small.
      \newline

      Note that $| [Df](a)h| \leq M |h|$ for some $M$, and since $p(h)/|h| \rightarrow 0$, we can choose
      $|h| < r_1$ such that $|p(h)| < |h|$. Thus, for $|h| < r_1$, we have $|[Df](a)h + p(h)| \leq (M + 1)|h|$.
      \newline

      Given some $\epsilon > 0$, we choose $|t| < \delta$ such that $\frac{|q(t)|}{|t|} < \frac{\epsilon}{M + 1}$. We also choose $r_2$ such that for $|h| < r_2$,
      we have $|[Df](a)h + p(h)| < \delta$. We then note that for $|h| < r = \min\{r_1, r_2\}$, we have:

      $$\frac{q \left( [Df](a)h + p(h) \right)}{|h|} \leq (M + 1)  \frac{q \left( [Df](a)h + p(h) \right)}{|[Df](a)h + p(h)|} < (M + 1) \frac{\epsilon}{M + 1} = \epsilon$$

      Thus, this function is small as well.
    \end{proof}

    \hl

    \begin{thm}
      If $f$ is differentiable, then all of the partial derivatives exist and the matrix of $[Df](a)$ with respect to the standard basis is precisely the matrix of partial derivatives.
    \end{thm}

    \begin{proof}
      For some $a$, we can define a function of the form $h(x) = (a_1, \ ..., \ a_{k - 1}, x, a_{k + 1}, \ ..., \ a_n)$ We then note that the function $f \circ h$ is differentiable,
      as it is the composition of two differentiable functions. It is also easy to see that:

      $$(f \circ h) = (f_1 \circ h, \ ..., \ f_m \circ h)$$

      where each $f_j \circ h$ is differentiable. It is easy to see that $(f_j \circ h)'(a_k) = D_k f_j(a)$, from the definition of the partial derivative, so the partial derivatives exist. In addition, we have that:

      $$D_k f_j(a) = [D (f_j \circ h)](a_k) = [D f_j](a) \circ e_k$$

      which is the $k$-entry of $[D f_j](a)$. But since $[D f_j](a)$ is the $j$-th row of the matrix for $[Df](a)$, it follows that $D_k f_j(a)$ is the entry at the $k$-column and $j$-th row.
    \end{proof}

    \hl

    \subsection{Problems and Solutions}



    \end{document}
