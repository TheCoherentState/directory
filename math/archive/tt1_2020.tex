\documentclass[10pt, oneside]{amsart} 
\usepackage{amsmath, amsthm, amssymb, wasysym, verbatim, bbm, color, graphics, geometry, hyperref, biblatex, mathtools}
\usepackage{tcolorbox}

\hypersetup{
	colorlinks=true,
	linkcolor=blue,
	urlcolor=blue
}


\geometry{tmargin=1.25in, bmargin=1.25in, lmargin=1.25in, rmargin =1.25in}
\setlength\parindent{0pt}

\tcbuselibrary{theorems}
\newtcbtheorem
    []% init options
    {problem}% name
    {Problem}% title
    {%
      fonttitle=\bfseries,
    }% options
    {prob}% prefix

    \newcommand{\R}{\mathbb{R}}
    \newcommand{\C}{\mathbb{C}}
    \newcommand{\Z}{\mathbb{Z}}
    \newcommand{\N}{\mathbb{N}}
    \newcommand{\Q}{\mathbb{Q}}
    \newcommand{\Cdot}{\boldsymbol{\cdot}}
    \newcommand{\U}{\mathcal{U}}
    \newcommand{\V}{\mathcal{V}}

    \newtheorem{thm}{Theorem}
    \newtheorem{defn}{Definition}
    \newtheorem{conv}{Convention}
    \newtheorem{rem}{Remark}
    \newtheorem{lem}{Lemma}
    \newtheorem{cor}{Corollary}
    \newtheorem{prop}{Proposition}
    \newtheorem{prob}{Problem}

    \newcommand{\tr}{\mathrm{Tr}}
    \newcommand{\bm}{\boldsymbol}
    \DeclarePairedDelimiter\floor{\lfloor}{\rfloor}


    \title{MAT327 Problem Set 1}
    \author{Jack Ceroni}
    \date{September 2021}

    \begin{document}

    \maketitle

    \tableofcontents

    \vspace{.25in}

    \newpage

    \section{Problem 1}

    Since the sequence is Cauchy, it converges to some point $x$. It may be the case that $x$ is in $\{x_k\}$, or it may be the case that it is not. Either way, we can take $B = A \cup \{x\}$.
    Pick an open cover of $A$ called $\mathcal{U}$. Let $V \in \mathcal{U}$ contain $\{x\}$. Since the sequence converges to $\{x\}$, all but finitely many of the $x_k$ must be in $V$. Taking elements of
    $\mathcal{U}$ that contain all other elements of $x_k$ not in $V$, and we have a finite subcover, so $A$ is compact.
    \newline

    Take the sequence $x_k = k$. Clearly, this sequence is not Cauchy, as $d(x_m, x_n) = |m - n| \geq 1$ for all $m$ and $n$ not equal. In addition, this set is not compact
    as it is not bounded (and it will clearly not be bounded even if we add a point).

    \section{Problem 2}

    It is easy to see from the definition that $\lambda(0) = 0$.
    \newline

    First consider $\mu \circ \lambda$. We pick $\epsilon > 0$, and choose $\delta$ small enough such that $|\lambda(x)| \leq C |x|$, and $|x| < \delta \Rightarrow |\mu(x)| < \frac{\epsilon}{C}|x|$. We then note that:

    $$|x| < \min \left( \delta, \frac{\delta}{C} \right) \Rightarrow |\lambda(x)| \leq C |x| < \delta \Rightarrow |\mu(\lambda(x))| < \frac{\epsilon}{C} | \lambda(x)| \leq \epsilon |x|$$

    so $\mu \circ \lambda$ is tiny. Now, we can consider $\lambda \circ \mu$. We choose $\epsilon > 0$, and $\delta$ small enough such that $|\lambda(x)| \leq C|x|$, and $|x| < \delta$ implies $|\mu(x)| < \frac{\epsilon}{C} |x|$. It follows that:

    $$|x| < \min \left( \delta, \frac{\delta C}{\epsilon} \right) \Rightarrow |\mu(x)| < \frac{\epsilon}{C} |x| \Rightarrow |\lambda(\mu(x))| \leq C |\mu(x)| < \epsilon|x|$$

    where we know that $|\lambda(\mu(x))| \leq C |\mu(x)|$ as $|\mu(x)| < \delta$.

    \section{Problem 4}

    Recall the axis-crawl method. Let $U$ be an open set inside which all the partial derivatives exist and are bounded. We let $h = (h_1, \ ..., \ h_n)$ be in $U$, and define the sequence of points
    $t_0 = (0, \ ..., \ 0), t_1 = (h_1, 0, \ ..., \ 0), \ ..., \ t_n = h$. We then note that:

    $$f(h) - f(0) = \displaystyle\sum_{j = 1}^{n} f(t_i) - f(t_{i - 1}) = \displaystyle\sum_{i = 1}^{n} f(t_{i - 1} + h_i e_i) - f(t_{i - 1})$$

    Note that the functions obtained by setting all $h_j$ constant, except for the $k$-th spot are differentiable in $U$, as the partial derivatives exist. Hence, we know from the mean value theorem
    that there exists some $c_i \in (0, h_i)$ for which $f(t_{i - 1} + h_i e_i) - f(t_{i - 1}) = D_i f(c_i) h_i$. Thus:

    $$f(h) - f(0) = \displaystyle\sum_{j = 1}^{n} D_i f(c_i) h_i \Rightarrow |f(h) - f(0)| = \left| \displaystyle\sum_{j = 1}^{n} D_i f(c_i) h_i \right| \leq \displaystyle\sum_{j = 1}^{n} |D_i f(c_i) h_i| \leq M \displaystyle\sum_{j = 1}^{n} |h_i|$$

    as each partial derivative is bounded. Taking the limit as $h \rightarrow 0$ allows us to conclude from squeeze theorem that $\lim_{h \to 0} |f(h) - f(0)| = 0$, so by definition, $f$ is continuous at $0$.

    \section{Problem 5}

    \textbf{Part A}
    \newline

    From chain rule:

    $$[Dh](x, y) = [Df](x, y, g(x, y)) \circ [D k](x, y)$$

    where $k(x, y) = (x, y, g(x, y))$. Clearly:

    $$[Df](x, y, g(x, y)) = \begin{pmatrix} D_1 f(x, y, g(x, y)) & D_2 f(x, y, g(x, y)) & D_3 f(x, y, g(x, y)) \end{pmatrix}$$

    and:

    $$[Dk](x, y) = \begin{pmatrix} 1 & 0 \\ 0 & 1 \\ D_1 g(x, y) & D_2 g(x, y) \end{pmatrix}$$

    which implies that:

    $$[Dh](x, y) = \begin{pmatrix} D_1 f(x, y, g(x, y)) + D_3 f(x, y, g(x, y)) D_1 g(x, y) & D_2 f(x, y, g(x, y)) + D_3 f(x, y, g(x, y)) D_2 g(x, y) \end{pmatrix}$$

    \end{document}
