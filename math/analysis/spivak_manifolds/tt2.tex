\documentclass[10pt, oneside]{amsart} 
\usepackage{amsmath, amsthm, amssymb, wasysym, verbatim, bbm, color, graphics, geometry, hyperref, biblatex, mathtools}
\usepackage{tcolorbox}
\usepackage{tikz-cd} 

\hypersetup{
	colorlinks=true,
	linkcolor=blue,
	urlcolor=blue
}


\geometry{tmargin=1.25in, bmargin=1.25in, lmargin=1.25in, rmargin =1.25in}
\setlength\parindent{0pt}

\tcbuselibrary{theorems}
\newtcbtheorem
    []% init options
    {problem}% name
    {Problem}% title
    {%
      fonttitle=\bfseries,
    }% options
    {prob}% prefix

    \newcommand{\R}{\mathbb{R}}
    \newcommand{\C}{\mathbb{C}}
    \newcommand{\Z}{\mathbb{Z}}
    \newcommand{\N}{\mathbb{N}}
    \newcommand{\Q}{\mathbb{Q}}
    \newcommand{\Cdot}{\boldsymbol{\cdot}}
    \newcommand{\U}{\mathcal{U}}
    \newcommand{\V}{\mathcal{V}}

    \newtheorem{thm}{Theorem}
    \newtheorem{defn}{Definition}
    \newtheorem{conv}{Convention}
    \newtheorem{rem}{Remark}
    \newtheorem{lem}{Lemma}
    \newtheorem{cor}{Corollary}
    \newtheorem{prop}{Proposition}
    \newtheorem{prob}{Problem}

    \newcommand{\tr}{\mathrm{Tr}}
    \newcommand{\bm}{\boldsymbol}
    \DeclarePairedDelimiter\floor{\lfloor}{\rfloor}


    \title{Calculus on Manifolds: Integration}
    \author{Jack Ceroni}
    \date{December 2021}

    \begin{document}

    \maketitle

    \tableofcontents

    \vspace{.25in}

    \newpage

    \hrulefill

    https://utoronto.zoom.us/j/87576958097https://utoronto.zoom.us/j/87576958097   \section{Problem 1}

    Let the collection of points for which $f \neq g$ be denoted as $\{x_1, \ ..., \ x_n\}$. Let $m = g(x_1) + \cdots + g(x_n)$. 

    \hrulefill

    \section{Problem 2}

    \subsection{Part A} Pick some $\epsilon > 0$. Since $C$ is content-$0$, we can choose a finite collection of closed rectangles $R_k$ such that the sum of all $\text{vol}(R_k)$ is less than $\epsilon$.
    We claim that $\partial C$ is covered by the collection of $R_k$ as well. Let $R = \cup R_k$. Indeed, given some $c \in \partial C$, note that every neighbourhood of $c$ must intersect both $C$ and $C^{C}$,
    so $c \notin R^{C}$, as this is an open set that does not intersect $C$.
    \newline

    It follows by definition that $\partial C$ is also content-$0$.

    \subsection{Part B} Take the rationals in $[0, 1]$, in the context of $\mathbb{R}$. This set is countable, so it is measure-$0$, but its boundary is clearly $[0, 1]$, which is not measure-$0$.

    \hrulefill

    \section{Problem 3}

    Since $f$ is integrable, it is continuous except on a set of measure-$0$, which we denote $S$. Since $g$ is continuous, it follows that the composite $g \circ f$ is continuous for all $x \notin S$. Thus,
    $f \circ g$ is continuous except on a measure-$0$ set, so it is integrable.

    \hrulefill

    \section{Problem 4}

    Since $A$ is Jordan-measureable, the indicator function $\chi_A$ is integrable on some rectangle $K$ containing $A$. Thus, by the Reimann criteria for integration, we can pick some partition $P$ of $K$ such that:

    $$U(\chi_A, P) - L(\chi_A, P) = \displaystyle\sum_{S \in P} M_S(\chi_A) \text{vol}(S) - \displaystyle\sum_{S \in P} m_S(\chi_A) \text{vol}(S) < \frac{1}{257}$$

    Clearly, for some $S \in P$ such that $S \subset A$, we will have $M_S(\chi_A) = m_s(\chi_A) = 1$, and similarly, for $S \subset A^C$, we will have both $M_S$ and $m_S$ equal to $0$. Finally,
    for the remaining $S$, which intersect both $A$ and $A^C$, we have $M_S = 1$ and $m_S = 0$, so:

    $$\displaystyle\sum_{S \in P} M_S(\chi_A) \text{vol}(S) - \displaystyle\sum_{S \in P} m_S(\chi_A) \text{vol}(S) = \displaystyle\sum_{S \in P'} \text{vol}(S) < \frac{1}{257}$$

    where $P'$ is the set of all rectangles intersecting both $A$ and $A^C$. Thus, taking $R = P$ and $R' = P'$, the proof is complete.

    \hrulefill

    \section{Problem 5}

    \subsection{Part A} Since $\chi_B$ is integrable, it follows that $\partial B$ is measure-$0$, so the closed set $\overline{B} = B \cup \partial B$ containing $B$ is measure-$0$. Since this set is also
    bounded, it follows that it is compact, so it is content-$0$.
    \newline

    Thus, for any $\epsilon > 0$, we can pick some finite collection $S$ of rectangles covering $\overline{B}$, with sum of volumes less than $\epsilon$. We can take intersections of these rectangles, and extend the resulting set
    to a partition $P$ of a rectangle $R$ containing $\overline{B}$.
    \newline

    Clearly, the upper sum of $\chi_B$ on $R$ will be the sums of the volumes of rectangles intersecting $B$, which is precisely the subset of $P$ of rectangles obtained from intersecting elements of $S$. Clearly, the sum
    of volumes of these rectangles will be less than $\epsilon$. Since $\text{vol}(B) \leq U(\chi_B, P)$, for all $P$, it follows that $\text{vol}(B)$ must equal $0$, as we can make $U(\chi_B, P)$ arbitrarily small.

    \subsection{Part B} Let $A$ be the set of rationals in $[0, 1]$. Checking that $\chi_A$ satisfies the criteria is a simple exercise.

    \hrulefill

    \section{Problem 6}

    Suppose is not equal to $0$. Then there exists some point $a$ at which $f(a) \neq 0$. Since $f$ is continuous, there is a neighbourhood of this point on which $f > 0$. Inside this neighbourhood,
    we can pick a rectangle $R$.
    \newline

    Thus, picking a partition containing $R$, we get a lower sum that is greater than $0$, implying the integral itself must be greater than $0$, which is a contradiction. Thus, we must have $f = 0$.

    \hrulefill

    \section{Problem 8}

    \hrulefill

    Since $A$ and $B$ are Jordan-measureable, it follows that $\chi_{A}$ and $\chi{B}$ are integrable. In addition, we have assumed that the functions $f_t(x) = \chi_{A_t}(x)$ and $g_t(x) = \chi_{B_t}(x)$,
    where $A_t$ and $B_t$ are the slices of $A$ and $B$, are integrable as well.
    \newline

    Recall that for any $t$, we have:

    $$\displaystyle\int \chi_{A_t}(x) = \displaystyle\int \chi_{B_t}(x)$$

    by assumption. Clearly, for some $t$, we have $\chi_{S_t}(x) = \chi_{S}(x, t)$. Thus, by Fubini's theorem:

    $$\text{vol}(A) = \displaystyle\int \chi_{A} = \displaystyle\int_{\mathbb{R}} \displaystyle\int_{R_A} \chi_{A}(x, t) \ dx \ dt = \displaystyle\int_{\mathbb{R}} \displaystyle\int_{R_B} \chi_{B}(x, t) \ dx \ dt = \text{vol}(B)$$

    and we are done. \textbf{Note:} I'm being quite sloppy with my notation in a few places, but the idea should be clear.

    \hrulefill

    \section{Problem 11}

    Let $E$ be the ellipsoid in question. This is a standard change of vbariables. Let:

    $$f(x, y, z) = \left( \frac{x}{\sqrt{2}}, \frac{y}{\sqrt{3}}, \frac{z}{\sqrt{5}} \right)$$

    Obviously, this function is bijective, and differentiable, with its differential having a non-zero determinant:

    $$Df(x, y, z) = \text{diag} \left( \frac{1}{\sqrt{2}}, \frac{1}{\sqrt{3}}, \frac{1}{\sqrt{5}} \right)$$

    Finally, it obvious that:

    $$f(C) = f( \{(x, y, z) \ | \ x^2 + y^2 + z^2 \leq \}) = \left\{ \left( \frac{x}{\sqrt{2}}, \frac{y}{\sqrt{3}}, \frac{z}{\sqrt{5}} \right) \ \Big| \ x^2 + y^2 + z^2 \leq 1 \right\} = E$$

    Thus, by change of variables:

    $$\text{vol}(E) = \text{vol}(f(C)) = \displaystyle\int_{f(C)} 1 = \displaystyle\int_{C} | \det Df |$$

    From above, $| \det Df | = \frac{1}{\sqrt{30}}$. Thus:

    $$\text{vol}(E) = \frac{1}{\sqrt{30}} \displaystyle\int_{C} 1$$

    Since $C$ is simply a sphere with radius $1$, we know $\int_{C} 1 = \frac{4}{3} \pi$ (we could also show this with a spherical coordinate transform, but we are lazy). Thus,
    the volume of the ellipsoid is $\frac{1}{\sqrt{30}} \frac{4\pi}{3}$.

    \hrulefill

    \section{Problem 12}

    \subsection{Part A} This is an immediate consequence of the fundamental theorem of calculus. Let $g_1(x, y) = \partial_x \partial_y f(x, y)$, and let
    $g_2(x, y) = \partial_y \partial_x f(x, y)$. We know both of these functions are continuous. Hence, by Fubini's theorem:

    $$\displaystyle\int_{R} g_1(x, y) = \displaystyle\int_{[c, d]} \displaystyle\int_{[a, b]} \partial_x \partial_y f(x, y) \ dx \ dy = \displaystyle\int_{[c, d]} \left( f(b, y) - f(a, y) \right) \ dy = f(b, d) - f(a, d) - f(b, c) + f(a, c)$$

    An almost identical calculation shows that $\int_{R} g_2$ yields the same result. Thus, we have the desired equality.

    \subsection{Part B}

    Suppose there is some $(a, b)$ at which $\partial_x \partial_y f - \partial_y \partial_x f > 0$. Since this function is continuous, there must be a neighbourhood around $(a, b)$ on which it is positive.
    Taking the integral on a rectangle contained in this neighbourhood gives some positive number, but this contradicts Part A. Thus, not $(a, b)$ exists.
    \newline

    Identical logic shows that an $(a, b)$ at which the difference is negative cannot exist. Thus, $\partial_x \partial_y f = \partial_y \partial_x f$ for all points.

    \hrulefill

    \section{Problem 13}

    Let $f(x, y) = \sqrt{x^2 + y^2}$. Note that $f \circ T_{\theta} = f$:

    $$(f \circ T_{\theta})(x, y) = f(x \cos \theta - y \sin \theta, x \sin \theta + y \cos \theta) = \sqrt{(x^2 + y^2) (\sin^2 \theta + \cos^2 \theta)} = \sqrt{x^2 + y^2} = f(x, y)$$

    Since $B$ is Jordan-measureable, it bounded, so it is contained in some rectangle $R$. Since $f$ is continuous, it has a maximum value on $R$. Since $f$ is the radial distance of a point $(x, y)$ from the
    origin, it therefore follows there is some circle $C$ containing $R$, and thus $B$. It is easy to see that $T_{\theta}(C) = C$.
    \newline

    Now, all that is left to do is a change of variables. Clearly, $T_{\theta}$ is a diffeomorphism, and has determinant $1$. Thus:

    $$\text{vol}(B) = \displaystyle\int_{C} \chi_{B} = \displaystyle\int_{T_{\theta}(C)} \chi_{B} = \displaystyle\int_{C} \chi_B \circ T_{\theta} = \displaystyle\int_{C} \chi_{T_{\theta}(B)} = \text{vol}(T_{\theta} B)$$

    \hrulefill

    \section{Problem 14}

    This is a straightforward application of definitions.

    \hrulefill

    \section{Problem 15}

    Easy

    \hrulefill

    \section{Problem 16}

    Easy

    \hrulefill

    \section{Problem 17}

    Easy

    \hrulefill


    \end{document}
