\documentclass[aps,pra,showpacs,notitlepage,onecolumn,superscriptaddress,nofootinbib]{revtex4-1}
\usepackage[utf8]{inputenc}
\usepackage[tmargin=1in, bmargin=1.25in, lmargin=1.5in, rmargin=1.5in]{geometry}
\usepackage{amsmath, amssymb, amsthm}
\usepackage{graphicx}
\usepackage{xcolor}
\usepackage{enumitem}
\usepackage{datetime}
\usepackage{hyperref}
\usepackage{titlesec}
\usepackage{import}
\usepackage{mathtools}


% package for commutative diagrams
% \usepackage{tikz-cd}

%%%%%%%%%%%%%%%%%%%%%%%%%%%%%%%%%%%%%%%%%%%%%
\definecolor{crimson}{RGB}{186,0,44}
\definecolor{moss}{RGB}{0, 186, 111}
\newcommand{\pop}[1]{\textcolor{crimson}{#1}}
\newcommand{\zcom}[1]{\noindent\textcolor{crimson}{(Z): #1}}
\newcommand{\jcom}[1]{\noindent\textcolor{moss}{(J): #1}}
\newcommand{\wt}[1]{\widetilde{#1}}
\newcommand{\pqeq}{\succcurlyeq}
\newcommand{\pleq}{\preccurlyeq}

%%%%%%%%%%%%%%%%%%%%%%%%%%%%%%%%%%%%%%%%%%%%%
\hypersetup{
    colorlinks,
    linkcolor={crimson},
    citecolor={crimson},
    urlcolor={crimson}
}

\usepackage{qcircuit}

%%%%%%%%%%%%%%%%%%%%%%%%%%%%%%%%%%%%%%%%%%%%%
\theoremstyle{definition}
\newtheorem{definition}{Definition}[section]
\newtheorem{lemma}{Lemma}[section]
\newtheorem{theorem}{Theorem}[section]
\newtheorem{corollary}{Corollary}[theorem]
\newtheorem*{theorem*}{Theorem}
\newtheorem*{corollary*}{Corollary}
\newtheorem{remark}{Remark}[section]
\newtheorem{conjecture}{Conjecture}[section]
\newtheorem{example}{Example}[section]
\newtheorem{reminder}{Reminder}[section]
\newtheorem{problem}{Problem}[section]
\newtheorem{question}{Question}[section]
\newtheorem{answer}{Answer}[section]
\newtheorem{fact}{Fact}[section]
\newtheorem{claim}{Claim}[section]
\newtheorem{exercise}{Exercise}[section]

\usepackage{geometry}
\geometry{
  left=25mm,
  right=25mm,
  top=20mm,
}

%%%%%%%%%%%%%%%%%%%%%%%%%%%%%%%%%%%%%%%%%%%%%
\bibliographystyle{unsrt}

%%%%%%%%%%%%%%%%%%%%%%%%%%%%%%%%%%%%%%%%%%%%%
%%%%%%%%%%%%%%%%%%%%%%%%%%%%%%%%%%%%%%%%%%%%%
%%%%%%%%%%%%%%%%%%%%%%%%%%%%%%%%%%%%%%%%%%%%%
\begin{document}

\title{Lee's Introduction to Smooth Manifolds: Chapter 1 Assorted Notes, Proofs and Exercises}

\author{Jack Ceroni}
\email{jackceroni@gmail.com}

\date{\today}

\maketitle

%%%%%%%%%%%%%%%%%%%%%%%%%%%%%%%%%%%%%%%%%%%%%

\section{Exercise 1.1}
\label{sec:11}

\noindent
For both definitions, it is clear that the new subsumes the old, as both open balls and $\mathbb{R}^{n}$ are open neighbourhoods around every point of $\mathbb{R}^{n}$.
Conversely, if we assume the original definition, then given some point $p \in M$ and local homeomorphism $\varphi : U \rightarrow \widehat{U}$, then we can find an open ball
$B$ around $\varphi(p)$, and note that the restriction $\varphi : \varphi^{-1}(B) \rightarrow B$ is a homeomorphism, so $p$ has a neighbourhood homeomorphic to an open ball.
As for the other case, this follows from the fact that open $n$-balls are homeomorphic to $\mathbb{R}^{n}$.

\hrulefill

\section{Example 1.4}

\noindent
We wish to show that the $n$-sphere is a manifold. The Hausdorff condition and second-countability are clearly inherited from the global space. As for the local Euclidean condition,
the general strategy is to separate the sphere into different parts. Let $p = (p_1, \dots, p_{n + 1})$ be a point on the $n$-sphere. Clearly, there must exist some coordinate $p_k \neq 0$. We let
\begin{equation}
  U_p = \left\{ (x_1, \dots, x_{n + 1}) \ | \ x_{k}^2 = 1 - x_{n + 1}^2 - \cdots - x_{k + 1}^2 - x_{k - 1}^2 - \cdots - x_1^2 \ \text{and} \ \text{sign}(x_k) = \text{sign}(p_k) \right\}
  \end{equation}
Clearly, $U_p = S^{n} \cap \mathbb{R}_k^{\text{sign}(p_k)}$, which is open in the subspace topology. Let $\pi_k : U_p \rightarrow \pi_k(U_p)$ be the projection onto all coordinates \textit{except for the $k$-th}.
Define
\begin{equation}
  \pi_k^{-1}(x) = \pi_k^{-1}(x_1, \dots, x_{k - 1}, x_{k + 1}, \dots, x_{n + 1}) = (x_1, \dots, f_k(x), \dots, x_{n + 1})
\end{equation}
where $f_k(x) = \text{sign}(p_k) \sqrt{1 - x_{n + 1}^2 - \cdots - x_1^{2}}$ is clearly continuous. It is easy to check that $\pi_k^{-1}$ is the well-defined inverse of $\pi_k$, and both maps are continuous,
so $\pi_k(U_p)$ is open, and $\pi_k$ is the desired local homeomorphism.

\hrulefill

\section{Lemma 1.10}

\noindent
Note that $M$ is second-countable, let $\mathcal{B}$ be the countable basis. For each $p \in M$, note that there exists a local homeomorphism $\varphi : U \rightarrow \widehat{U}$, where $U$
contains $p$. Since there exists some $B \in \mathcal{B}$ with $p \in B \subset U$, $B$ is homeomorphic to an open subset of $\mathbb{R}^{n}$. Thus, we have a countable subcollection $\mathcal{B}'$ of elements of $\mathcal{B}$
such that each $p \in M$ is contained in some element of $\mathcal{B}'$ and each element of $\mathcal{B}'$ is homeomorphic (via some $\varphi_B$) to an open subset of $\mathbb{R}^{n}$.

For each $B \in \mathcal{B}'$, let $S_B$ denote the countable collection of all open balls of rational radius contained in $\varphi_B(B)$, centred at rational points. Let $\varphi_B^{-1}(S_B)$ denote the inverse image of
each of these balls, which are all open. Clearly, given some $p \in U$ open in $M$, $p$ is contained in some $B \in \mathcal{B}'$, and $B \cap U$ is an open subset of $B$, so $\varphi_B(B \cap U)$ is open in $\varphi_B(B)$.
We can pick an open ball in $S_B$ containing $\varphi_B(p)$, so $p$ is contained in some element of $\varphi_B^{-1}(S_B)$ which is contained in $B$. Let $S = \cup_{B \in \mathcal{B}'} S_B$. Suppose $\varphi_{B_1}^{-1}(B_1), \varphi_{B_2}^{-1}(B_2) \in S$ are two preimages of open balls with non-empty (thus open) intersection, $U$. Then the images $\varphi_{B_1}(U)$ and $\varphi_{B_2}(U)$ will intersect as and be open, so we can find some open ball $B$ at a rational point, of rational radius, in this intersection, so $\varphi_{B_1}^{-1}(B)$ (and $\varphi_{B_2}^{-1}(B)$) are in $U$. Thus, $S$ satisfies the intersection condition, and is in fact a countable basis of coordinate balls (as it is a countable intersection of countable sets).

Finally, note that since each $\varphi_B$ is a homeomorphism, $\overline{\varphi^{-1}_B(B)} = \varphi^{-1}_B(\overline{B})$. Since $\overline{B}$ is compact, so too is $\varphi^{-1}_B(\overline{B})$ and thus $\overline{\varphi^{-1}_B(B)}$.

\hrulefill

\section{Exercise 1.14}

\noindent
We begin with the first claim. Suppose $\mathcal{X}$ is locally finite, let $x$ be a point of the topological space. Then there exists a neighbourhood $U$ intersecting only $X_1, \dots, X_N$. Suppose $U$ intersects
some $\overline{X_M}$ not in this set. Then by definition, $U$ contains a point $y$ where every neighbourhood of $y$ intersects $X_M$, so $U$ intersects $X_M$, a contradiction. Thus, $\overline{\mathcal{X}}$ is locally finite.

Next, note that $\cup_{X \mathcal{X}} \overline{X} \subset \overline{\cup_{X \in \mathcal{X}} X}$ trivially. To prove inclusion the other way, given $x \in \overline{\cup_{X \in \mathcal{X}} X}$, note that every open set of $x$ intersects the union at a finite number of $X$. If there isn't some $X'$ intersected by all neighbourhoods, we can pick some neighbourhood $U$ which intersects $X_1, \dots X_M$ and neighbourhood $U_1, \dots, U_M$ where $U_k$ does not intersect $X_k$. Then $U \cap U_1 \cap \cdots U_M$ does not intersect any element of $\mathcal{X}$, which is a contradiction.

\hrulefill

\section{Proposition 1.16}

\noindent
\textit{This result is very simple when it is understood diagramatically, which is basically as far as we made it in third-year algebraic topology! Nevertheless, a formal proof is required.}
\newline

\noindent
We first require a lemma.
\begin{lemma}[Lebesgue number lemma]
Suppose $X$ is a compact set in a metric space and $\mathcal{A}$ is an open cover. Then there exists some $\delta > 0$ such that every subset of $X$ having diameter less than $\delta$ is contained in an element of $\mathcal{A}$.
\end{lemma}
\begin{proof}
  Let $A_1, \dots, A_N$ be a finite subcover. The key idea is to use extreme value theorem. Define the function $f : X \rightarrow \mathbb{R}$ as
  \begin{equation}
    f(x) = \frac{1}{N} \displaystyle\sum_{k = 1}^{N} d(x, X - A_k),
    \end{equation}
  the average distance from $x$ to the exterior of each $A_k$. Clearly, $x \in A_k$, for some $A_k$ open, so the minimum value of $f$ must be greater than $0$. Set it to $\delta$. Suppose $U$ has diameter less than
  $\delta$. Suppose $U$ is not contained in a single $A_k$. Then, there exist points $x$ and $y$ of $U$ not contained in a common $A_k$, with $d(x, y) \leq \delta$. Then, clearly, $d(x, y) \geq d(x, X - A_k)$ for each $A_k$.
  But then, $f(x) < \delta$, a contradiction. It follows that $U$ must be contained in a single $A_k$, so the lemma holds.
  \end{proof}

\noindent
Let $\mathcal{B}$ be a countable basis for $M$, let $B, B' \in \mathcal{B}$ such that $B \cap B'$ is non-empty. Since $B \cap B'$ is a manifold, it has a finite number of connected components, which
are also path-connected. Pick some $x$ from each component. Then, for all pairs of $B, B' \in \mathcal{B}$, take all such points $x$ and combine them into a set $\mathcal{X}$. For each pair of points $x, x' \in \mathcal{X}$ where $x, x' \in B$,
let $h_{x, x'}^{B}(t)$ be a path between them.

Assume $M$ is path-connected. Let us pick some $p \in \mathcal{X}$. Let $f : [0, 1] \rightarrow M$ be a loop in $\pi_1(M, p)$. The collection of $f^{-1}(B)$ for $B \in \mathcal{B}$. will clearly cover $[0, 1]$,
so there exists a finite subcover, $f^{-1}(B_1), \dots, f^{-1}(B_N)$. From the Lebesgue number lemma, we can partition $[0, 1]$ into a finite number of intervals $[a_k, a_{k + 1}]$ for $0 = a_1 < a_2 < \cdots < a_{N - 1} < a_N = 1$ such
that each interval is contained in some $f^{-1}(B_j)$.


\hrulefill

%%%%%%%%%%%%%%%%%%%%%%%%%%%%%%%%%%%%%%%%%%%%%

\end{document}
