\documentclass[aps,pra,showpacs,notitlepage,onecolumn,superscriptaddress,nofootinbib]{revtex4-1}
\usepackage[utf8]{inputenc}
\usepackage[tmargin=1in, bmargin=1.25in, lmargin=1.5in, rmargin=1.5in]{geometry}
\usepackage{amsmath, amssymb, amsthm}
\usepackage{graphicx}
\usepackage{xcolor}
\usepackage{enumitem}
\usepackage{datetime}
\usepackage{hyperref}
\usepackage{titlesec}
\usepackage{import}
\usepackage{mathtools}
\usepackage{thmtools,thm-restate}


% package for commutative diagrams
% \usepackage{tikz-cd}

%%%%%%%%%%%%%%%%%%%%%%%%%%%%%%%%%%%%%%%%%%%%%
\definecolor{crimson}{RGB}{186,0,44}
\definecolor{moss}{RGB}{0, 186, 111}
\newcommand{\pop}[1]{\textcolor{crimson}{#1}}
\newcommand{\zcom}[1]{\noindent\textcolor{crimson}{(Z): #1}}
\newcommand{\jcom}[1]{\noindent\textcolor{moss}{(J): #1}}
\newcommand{\wt}[1]{\widetilde{#1}}
\newcommand{\pqeq}{\succcurlyeq}
\newcommand{\pleq}{\preccurlyeq}

%%%%%%%%%%%%%%%%%%%%%%%%%%%%%%%%%%%%%%%%%%%%%
\hypersetup{
    colorlinks,
    linkcolor={crimson},
    citecolor={crimson},
    urlcolor={crimson}
}

\usepackage{qcircuit}

%%%%%%%%%%%%%%%%%%%%%%%%%%%%%%%%%%%%%%%%%%%%%
\theoremstyle{definition}
\newtheorem{definition}{Definition}[section]
\newtheorem{lemma}{Lemma}[section]
\newtheorem{theorem}{Theorem}[section]
\newtheorem{corollary}{Corollary}[theorem]
\newtheorem*{theorem*}{Theorem}
\newtheorem*{corollary*}{Corollary}
\newtheorem{remark}{Remark}[section]
\newtheorem{conjecture}{Conjecture}[section]
\newtheorem{example}{Example}[section]
\newtheorem{reminder}{Reminder}[section]
\newtheorem{problem}{Problem}[section]
\newtheorem{question}{Question}[section]
\newtheorem{answer}{Answer}[section]
\newtheorem{fact}{Fact}[section]
\newtheorem{claim}{Claim}[section]

\newcommand{\hhrulefill}{\hspace{-1em} \hrulefill}

\usepackage{geometry}
\geometry{
  left=25mm,
  right=25mm,
  top=20mm,
}

%%%%%%%%%%%%%%%%%%%%%%%%%%%%%%%%%%%%%%%%%%%%%
\bibliographystyle{unsrt}

%%%%%%%%%%%%%%%%%%%%%%%%%%%%%%%%%%%%%%%%%%%%%
%%%%%%%%%%%%%%%%%%%%%%%%%%%%%%%%%%%%%%%%%%%%%
%%%%%%%%%%%%%%%%%%%%%%%%%%%%%%%%%%%%%%%%%%%%%
\begin{document}

\title{Complex Analysis}
\author{Jack Ceroni}
\email{jackceroni@gmail.com}
\affiliation{Department of Mathematics, University of Toronto}

\date{\today}

\maketitle

\section{Introduction}

\noindent The goal of these notes is to follow a basic first course in complex analysis, following Edward Bierstone's course at the University of Toronto (MAT354), in the Fall of 2023. I will
also be adding information from my own readings/other resources. I hope that someone finds these notes useful. However, in the event that no one does, I certainly will!

Here is a list of other resources that I will draw from in these notes:

\begin{itemize}
  \item Edward Bierstone's lectures (MAT354, University of Toronto, Fall 2023)
  \item Cartan, \textit{Elementary theory of analytic functions of one or several complex variables}
  \item Ahlfors, \textit{Complex analysis}
  \end{itemize}

\section{Basics and notation}

\noindent We begin our discussion of complex analysis by introducing the complex plane, $\mathbb{C}$. Of course, $\mathbb{C}$ can be constructed via taking $\mathbb{R}^{2}$ and imposing a particular
complex structure over it, which tells us how to multiply by the imaginary unit $i$. We can also construct $\mathbb{C}$ via taking $\mathbb{R}[x]$ and quotienting by the ideal $(x^2 + 1)$.
For the purposes of these introductory notes, it is not particularly important to dwell on this point: all that
we need to know is that there are many different ways to construct $\mathbb{C}$, each of which are equivalent, and are topologically identical to $\mathbb{R}^2$.

We will assume basic familiarity with fundamental notions related to the manipulation of complex numbers. As for notation, we let $\overline{z}$ denote the complex conjugate of $z \in \mathbb{C}$.
We let $|z|$ denote the absolute value or modulus. We let $\Re[z]$ and $\Im[z]$ denote the real and complex components in $\mathbb{R}$ of $z \in \mathbb{C}$, respectively. Assuming that
we have already constructed the function $\sin(t)$ and $\cos(t)$, and know their basic properties, it is straightforward to show that any $(x, y) \in \mathbb{C}$ can be expressed as $(r \cos(\theta), r \sin(\theta))$:
we simply set $r = \sqrt{x^2 + y^2}$ and apply the Pythagorean identity to $\cos(\theta)$ and $\sin(\theta)$, which allows us to choose $\theta \in [0, 2\pi)$ satisfying the desired equality.

\begin{definition}[Polar coordinates]
  The map going from $\mathbb{R}^2$ to $\mathbb{R} \times [0, 2\pi)$ which sends $(x, y) \mapsto (r, \theta)$ via the above protocol is a bijection where the imagine of $(x, y)$ is refered to as the \textit{polar coordinate representation}
    of a pair of coordinates $(x, y)$.
\end{definition}

\begin{definition}[Complex exponential]
  We define the map $t \mapsto  \cos(t) + i \sin(t) \coloneqq e^{it}$ from $\mathbb{R}$ to $\mathbb{C}$ and refer to it as the \textit{complex exponential}. It is easy to demonstrate that
  the complex exponential has all the standard properties of a regular exponential function, when it comes to the inverse, multiplication, powers, etc. This definition also agrees
  with the series definition of the exponential function mapping from $\mathbb{R}$ to $\mathbb{R}$, where we substitute an imaginary number, and utilize the multiplication rule
  for repeatedly multiplying $i$ with itself.
\end{definition}

\noindent \jcom{I may revise this section in the future, if I think of any extra, basic information that must be exposited/notation that must be explained.}

\section{The Riemann sphere and complex projective space}

\noindent We are now able to move on to a discussion of different representations of the complex plane.

\begin{definition}[Riemann sphere]
  Rather than strictly working with the standard complex plane $\mathbb{C}$, we will often consider its one-point compactification $\overline{\mathbb{C}} = \mathbb{C} \cup \{\infty\}$, where $\infty$ is refered to as the ``point at infinity''.
  $\overline{\mathbb{C}}$ is called the Riemann sphere, as it is homeomorphic to a sphere (topologically, the one-point compactification of $\mathbb{R}^2$ is homeomorphic to $S^2$).

  Of course, the function addition $+ : \mathbb{C} \times \mathbb{C} \rightarrow \mathbb{C}$ and multiplication, $\cdot : \mathbb{C} \times \mathbb{C} \rightarrow \mathbb{C}$ must be extended to functions from $\overline{\mathbb{C}} \times \overline{\mathbb{C}}$ 
  to $\overline{\mathbb{C}}$ (to the greatest extent possible). We take
  \begin{align}
    a + \infty = \infty + a = \infty \ \ \ \text{for all} \ a \neq \infty \\
    b \cdot \infty = \infty \cdot b = \infty \ \ \ \text{for all} \ b \neq 0
  \end{align}
  Now, when we say ``to the greatest extent possible'' it is clear what we mean: we cannot define $0 \cdot \infty$ or $\infty + \infty$, or else we will violate the standard laws of addition and multiplication that we wish to satisfy with our new, extended arithmetic.
  For example, one may be inclined to say that $\infty + \infty = \infty$, but then if additive cancellation holds, $\infty = 0$, a contradiction.
\end{definition}

\begin{definition}[Stereographic projection]
  \end{definition}

\begin{claim}
  The one-point compactification of $\mathbb{R}^2$ is homeomorphic to $S^2$ via the stereographic projection.
\end{claim}

\begin{definition}[Complex projective space]
\end{definition}

\noindent Now, let us move back to the discussion of the Riemann sphere, which is the representation of the complex numbers we will work with the most. It is natural
to ask what the correspondence is between basic objects in $\overline{\mathbb{C}}$ and $S^2$, via the stereographic projection. We begin with a result.

\begin{theorem}
  All circles in the Riemann sphere correspond to lines or circles in $\overline{\mathbb{C}}$. Conversely, all lines and
  circles in $\overline{\mathbb{C}}$ correspond to circles in the Riemann sphere.
\end{theorem}

\section{Linear, polynomial, and rational complex functions}

\begin{theorem}[Representation of rational functions by partial fractions]

  \end{theorem}

\end{document}
