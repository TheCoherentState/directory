\documentclass[aps,pra,showpacs,notitlepage,onecolumn,superscriptaddress,nofootinbib]{revtex4-1}
\usepackage[utf8]{inputenc}
\usepackage[tmargin=1in, bmargin=1.25in, lmargin=1.5in, rmargin=1.5in]{geometry}
\usepackage{amsmath, amssymb, amsthm}
\usepackage{graphicx}
\usepackage{xcolor}
\usepackage{enumitem}
\usepackage{datetime}
\usepackage{hyperref}
\usepackage{titlesec}
\usepackage{import}
\usepackage{mathtools}
\usepackage{thmtools,thm-restate}


% package for commutative diagrams
% \usepackage{tikz-cd}

%%%%%%%%%%%%%%%%%%%%%%%%%%%%%%%%%%%%%%%%%%%%%
\definecolor{crimson}{RGB}{186,0,44}
\definecolor{moss}{RGB}{0, 186, 111}
\newcommand{\pop}[1]{\textcolor{crimson}{#1}}
\newcommand{\zcom}[1]{\noindent\textcolor{crimson}{(Z): #1}}
\newcommand{\jcom}[1]{\noindent\textcolor{moss}{(J): #1}}
\newcommand{\wt}[1]{\widetilde{#1}}
\newcommand{\pqeq}{\succcurlyeq}
\newcommand{\pleq}{\preccurlyeq}
\newcommand{\Wedge}{\bigwedge}

%%%%%%%%%%%%%%%%%%%%%%%%%%%%%%%%%%%%%%%%%%%%%
\hypersetup{
    colorlinks,
    linkcolor={crimson},
    citecolor={crimson},
    urlcolor={crimson}
}

\usepackage{qcircuit}

%%%%%%%%%%%%%%%%%%%%%%%%%%%%%%%%%%%%%%%%%%%%%
\theoremstyle{definition}
\newtheorem{definition}{Definition}[section]
\newtheorem{lemma}{Lemma}[section]
\newtheorem{theorem}{Theorem}[section]
\newtheorem{corollary}{Corollary}[theorem]
\newtheorem*{theorem*}{Theorem}
\newtheorem*{corollary*}{Corollary}
\newtheorem{remark}{Remark}[section]
\newtheorem{conjecture}{Conjecture}[section]
\newtheorem{example}{Example}[section]
\newtheorem{reminder}{Reminder}[section]
\newtheorem{problem}{Problem}[section]
\newtheorem{question}{Question}[section]
\newtheorem{answer}{Answer}[section]
\newtheorem{fact}{Fact}[section]
\newtheorem{claim}{Claim}[section]

\newcommand{\hhrulefill}{\hspace{-1.5em} \hrulefill}

\usepackage{geometry}
\geometry{
  left=25mm,
  right=25mm,
  top=20mm,
}

%%%%%%%%%%%%%%%%%%%%%%%%%%%%%%%%%%%%%%%%%%%%%
\bibliographystyle{unsrt}

%%%%%%%%%%%%%%%%%%%%%%%%%%%%%%%%%%%%%%%%%%%%%
%%%%%%%%%%%%%%%%%%%%%%%%%%%%%%%%%%%%%%%%%%%%%
%%%%%%%%%%%%%%%%%%%%%%%%%%%%%%%%%%%%%%%%%%%%%
\begin{document}

\title{A brief course in basic complex analysis}
\author{Jack Ceroni}
\email{jackceroni@gmail.com}
\affiliation{Department of Mathematics, University of Toronto}

\date{\today}

\maketitle

\tableofcontents

\section{Introduction}

\noindent The goal of these notes is to follow a basic first course in complex analysis, following Edward Bierstone's course at the University of Toronto (MAT354), in the Fall of 2023. I will
also be adding information from my own readings/other resources. I hope that someone finds these notes useful. However, in the event that no one does, I certainly will!

Here is a list of other resources that I will draw from in these notes:

\begin{itemize}
  \item Edward Bierstone's lectures (MAT354, University of Toronto, Fall 2023)
  \item Cartan, \textit{Elementary theory of analytic functions of one or several complex variables}
  \item Ahlfors, \textit{Complex analysis}
  \end{itemize}

\section{Basics and notation}

\noindent We begin our discussion of complex analysis by introducing the complex plane, $\mathbb{C}$. Of course, $\mathbb{C}$ can be constructed via taking $\mathbb{R}^{2}$ and imposing a particular
complex structure over it, which tells us how to multiply by the imaginary unit $i$.

\begin{definition}[Complex structure]
  \end{definition}

\noindent We can also construct $\mathbb{C}$ via taking $\mathbb{R}[x]$ and quotienting by the ideal $(x^2 + 1)$.
For the purposes of these introductory notes, it is not particularly important to dwell on this point: all that
we need to know is that there are many different ways to construct $\mathbb{C}$, each of which are equivalent. They are all topologically identical to $\mathbb{R}^2$, and posses the desired ``complex arithmetic'' that we are familiar with
and want to use.

We will assume basic familiarity with fundamental notions related to the manipulation of complex numbers. As for notation, we let $\overline{z}$ denote the complex conjugate of $z \in \mathbb{C}$.
We let $|z|$ denote the absolute value or modulus. We let $\Re[z]$ and $\Im[z]$ denote the real and complex components in $\mathbb{R}$ of $z \in \mathbb{C}$, respectively. Assuming that
we have already constructed the function $\sin(t)$ and $\cos(t)$, and know their basic properties, it is straightforward to show that any $(x, y) \in \mathbb{C}$ can be expressed as $(r \cos(\theta), r \sin(\theta))$:
we simply set $r = \sqrt{x^2 + y^2}$ and apply the Pythagorean identity to $\cos(\theta)$ and $\sin(\theta)$, which allows us to choose $\theta \in [0, 2\pi)$ satisfying the desired equality.

\begin{definition}[Polar coordinates]
  The map going from $\mathbb{R}^2$ to $\mathbb{R} \times [0, 2\pi)$ which sends $(x, y) \mapsto (r, \theta)$ via the above protocol is a bijection where the imagine of $(x, y)$ is refered to as the \textit{polar coordinate representation}
    of a pair of coordinates $(x, y)$.
\end{definition}

\begin{definition}[Complex exponential]
  We define the map $t \mapsto  \cos(t) + i \sin(t) \coloneqq e^{it}$ from $\mathbb{R}$ to $\mathbb{C}$ and refer to it as the \textit{complex exponential}. It is easy to demonstrate that
  the complex exponential has all the standard properties of a regular exponential function, when it comes to the inverse, multiplication, powers, etc. This definition also agrees
  with the series definition of the exponential function mapping from $\mathbb{R}$ to $\mathbb{R}$, where we substitute an imaginary number, and utilize the multiplication rule
  for repeatedly multiplying $i$ with itself.
\end{definition}

\noindent \pop{\textbf{Note}: I may revise this section in the future, if I think of any extra, basic information that must be exposited/notation that must be explained.}

\section{The Riemann sphere and complex projective space}

\noindent We are now able to move on to a discussion of different representations of the complex plane.

\begin{definition}[Riemann sphere]
  Rather than strictly working with the standard complex plane $\mathbb{C}$, we will often consider its one-point compactification $\overline{\mathbb{C}} = \mathbb{C} \cup \{\infty\}$, where $\infty$ is refered to as the ``point at infinity''.
  $\overline{\mathbb{C}}$ is called the Riemann sphere, as it is homeomorphic to a sphere (topologically, the one-point compactification of $\mathbb{R}^2$ is homeomorphic to $S^2$).

  Of course, the function addition $+ : \mathbb{C} \times \mathbb{C} \rightarrow \mathbb{C}$ and multiplication, $\cdot : \mathbb{C} \times \mathbb{C} \rightarrow \mathbb{C}$ must be extended to functions from $\overline{\mathbb{C}} \times \overline{\mathbb{C}}$ 
  to $\overline{\mathbb{C}}$ (to the greatest extent possible). We take
  \begin{align}
    a + \infty = \infty + a = \infty \ \ \ \text{for all} \ a \neq \infty \\
    b \cdot \infty = \infty \cdot b = \infty \ \ \ \text{for all} \ b \neq 0
  \end{align}
  Now, when we say ``to the greatest extent possible'' it is clear what we mean: we cannot define $0 \cdot \infty$ or $\infty + \infty$, or else we will violate the standard laws of addition and multiplication that we wish to satisfy with our new, extended arithmetic.
  For example, one may be inclined to say that $\infty + \infty = \infty$, but then if additive cancellation holds, $\infty = 0$, a contradiction.
\end{definition}

\begin{claim}[Stereographic projection]
  We can define a map $f$ (in fact, a homeomorphism) from $S^{2} - (0, 0, 1)$ to $\mathbb{R}^2$ called the \emph{stereographic projection}. This map is given by
  \begin{equation}
    f : (x, y, z) \mapsto w = \frac{1}{1 - z} (x, y)
    \end{equation}
\end{claim}
\begin{proof}
  Clearly, this is a continuous function. Moreover, given $(x, y, z) \in S^{2} - (0, 0, 1)$, note that $x^2 + y^2 + z^2 = 1$. It follows immediately that if identify $x + iy$ with $(x, y)$, then
  we will have
  \begin{equation}
    \frac{x + iy}{1 - z} = w \Longrightarrow \frac{x^2 + y^2}{(1 - z)^2} = \frac{1 - z^2}{(1 - z)^2} = \frac{1 + z}{1 - z} = |w|^2
  \end{equation}
  for some $w \in \mathbb{C}$. Or, if $w$ is in the image of the stereographic projection, then we have $z = \frac{|w|^2 - 1}{|w|^2 + 1}$. Moreover,
  \begin{equation}
    x = \frac{1}{2} (1 - z) (w + \overline{w}) = \frac{w + \overline{w}}{|w|^2 + 1} \ \ \ \text{and} \ \ \ y = \frac{1}{2i} (1 - z) (w - \overline{w}) = i \frac{\overline{w} - w}{|w|^2 + 1}.
  \end{equation}
  Thus, we define
  \begin{equation}
    f^{-1} : w \mapsto \left( \frac{w + \overline{w}}{|w|^2 + 1}, i \frac{\overline{w} - w}{|w|^2 + 1}, \frac{|w|^2 - 1}{|w|^2 + 1} \right).
  \end{equation}
  as a function from $\mathbb{R}^2$ to $S^2$, where $(x, y)$ is identified with $w = x + iy$. Clearly, this function is continuous, and it is easy to verify that it is in fact the desired inverse of $f$, so $f$ is a homeomorphism.
\end{proof}

It follows from basic point-set topology that we can extend this
homeomorphism to a homeomorphism between the one-point compactifications, $S^2$ and $\mathbb{R}^2 \cup \{\infty\}$, where $(0, 0, 1)$ is sent to $\infty$, the so-called point at infinity.

\begin{fact}
  The one-point compactification of $\mathbb{R}^2$ is homeomorphic to $S^2$ via the unique extension of the stereographic projection.
\end{fact}

\begin{definition}[Complex projective space]
  \pop{\textbf{TODO}: Fill this in}
\end{definition}

\noindent Now, let us move back to the discussion of the Riemann sphere, which is the representation of the complex numbers we will work with the most. It is natural
to ask what the correspondence is between basic objects in $\overline{\mathbb{C}}$ and $S^2$, via the stereographic projection. We begin with a result.

\begin{theorem}
  Via stereographic projection, all circles in $S^2 - (0, 0, 1)$ correspond to lines or circles in $\mathbb{R}^2$. Conversely, all lines and
  circles in $\mathbb{R}^2$ correspond to circles in the sphere minus a point.
\end{theorem}

\begin{proof}
  Let us suppose that we have a circle $C \subset S^2$. Such a circle corresponds to a collection of points which lie on the intersection of a plane with $S^2$. Thus, $C = \{(x, y, z) \in S^2 \ | \ ax + by + cz = d\}$
  for some real constants $a, b, c, d$. Let $f$ be the stereographic projection, we wish to find $f(C)$.

  Let us assume that $w \in f(C)$ with $w = x + iy$ identified with $(x, y)$ (we will often perform this identification use complex arithmetic more naturally). Moreover, suppose $f(x_0, y_0, z_0) = w$. Then we know that
  \begin{equation}
    |w|^2 = x^2 + y^2 = \frac{1 + z_0}{1 - z_0} \Longrightarrow cx^2 + cy^2 = \frac{c + d - ax_0 - by_0}{1 - z_0} \Longrightarrow cx^2 + cy^2 + ax + by = \frac{c + d}{1 - z_0}
  \end{equation}
  Recall that $1 - z_0 = \frac{2}{|w|^2 + 1}$, so we then have
  \begin{equation}
    cx^2 + cy^2 + ax + by = \frac{c + d}{2} (x^2 + y^2 + 1)
  \end{equation}
  which, upon simplification, if the equation of a circle or a line (a line when $c - d = 0$ and a circle otherwise). Conversely, given the circle or line in the plane, it is not difficult to verify that
  we can find constants $a, b, c, d$ such that the above equation yields the equations of the circle/line. Thus, the inverse of the stereographic projection takes this points to the original set $C$: a circle.
\end{proof}

Verifying that this construction extends to circles on the Riemann sphere and circles/lines in $\mathbb{R}^2 \cup \{\infty\}$ is straightforward and just requires some casework.

\section{Introduction to complex functions}

\subsection{Linear, polynomial, and rational functions}

\noindent Now that we have described the arena in which we perform calculations (the complex plane/Riemann sphere), we can move on to understanding corresponding functional transformations.

\begin{definition}[Linear functions]
  When dealing with the complex plane, we can split notions of real functions into two categories: \emph{real linear functions} and \emph{complex linear functions}. Real linear functions
  are the functions which are linear when $\mathbb{C} \simeq \mathbb{R}^2$ is treated as a two-dimensional vector space over $\mathbb{R}$ and complex linear functions are those which are linear
  when $\mathbb{C}$ is treated as a one-dimensional vector space over $\mathbb{C}$.
\end{definition}

\begin{claim}
  Any complex linear function is of the form $w(z) = az$ for some $a \in \mathbb{C}$. Any real linear function that preserves angles is of the form $w(z) = az$ for some $a \in \mathbb{C}$, or $w(z) = a\overline{z}$.
\end{claim}

\begin{proof}
  Obviously, if $f$ is complex linear, then $f(c) = f(1) \cdot c$ for any $c \in \mathbb{C}$, so $f(z) = f(1) z$.
  \newline

  \noindent Now, let $g$ be a real linear function which preserves angles. Clearly, $g(1) = w$. Let $s(z) = w^{-1} z$, so $f = s \circ g$ fixes $1$. Since $f$ preserves angles, it follows that $f(0, 1) = f(i) = c i$ as $1$ is fixed, for some $c \in \mathbb{R}$.
  Note from linearity, we must have $f(1, 1) = (1, c)$. But if the angle between $(1, 0)$ and $(1, 1)$ is to be preserved, $c = \pm 1$. So it follows that $f(0, 1) = (0, \pm 1)$ and $f(1, 0) = (1, 0)$. We know $f$ on a basis,
  so it follows that $f(z) = z$ or $f(z) = \overline{z}$. It follows that $g(z) = w f(z)$, which is either $wz$ or $w \overline{z}$, as desired.
\end{proof}

\begin{definition}
  Given a complex rational function $R : \mathbb{C} \rightarrow \overline{\mathbb{C}}$, we can define an extension of $R$ to $\overline{\mathbb{C}}$ where $R(\infty) = \lim_{z \to 0} R\left(\frac{1}{z}\right)$. Clearly, $R\left(\frac{1}{z}\right)$ is a well-defined
  rational function on $\mathbb{C}$, so $\lim_{z \to 0} R\left( \frac{1}{z} \right)$ is a well-defined point in $\overline{\mathbb{C}}$.
  \end{definition}

Given a rational function, it is possible to quickly sketch a rough portrait of its behaviour via analyzing its roots and poles.

\begin{definition}[Roots and poles]
  Given a rational function $R(z) = \frac{P(z)}{Q(z)}$ in reduced form, defined on $\overline{\mathbb{C}}$ (so $R(\infty)$ is well-defined)
  the roots of $R$ are the collection of all $z$ such that $P(z) = 0$. The poles are the collection of all $z$ such that $Q(z) = 0$.
  Provided $z \neq \infty$, both $P(z)$ and $Q(z)$ will be finite, and since they are in reduced form, we will have $R(z) = 0$ at the roots and $R(z) = \infty$ at the poles (as common roots of $P$ and $Q$ cannot exist in reduced form).
  This also holds in the case that $z = \infty$, due to the fact that $R(\infty) = \lim_{z \to 0} R\left(\frac{1}{z}\right)$, where $R\left( \frac{1}{z} \right)$ is also a rational function in reduced form.
  The \emph{order} or a finite root of $R(z)$ is the multiplicity of the root in $P(z)$. The order of a finite pole of $Q(z)$ is the multiplicity of the root in $Q(z)$. The order of a root/pole at infinite is the order of the $0$-root/pole
  for the rational function $R\left(\frac{1}{z}\right)$.
\end{definition}

\noindent Suppose we are provided a rational function $R(z) = \frac{P(z)}{Q(z)}$ in reduced form, such that $\text{deg}(P) = m$ and $\text{deg}(Q) = n$. A natural question to ask is the total number of roots and poles that
a particular rational function has (where we count roots/pole swith multiplicity). Of course, by the fundamental theorem of algebra, $R$ has $m$ finite roots and $n$ finite poles, so the only remaining question is
how many infinite roots/poles the function has. Suppose $P(z) = \sum_{j = 0}^{m} p_j z^{j}$ and $Q(z) = \sum_{j = 0}^{n} q_j z^j$. Then note that
\begin{equation}
  R\left(\frac{1}{z}\right) = \frac{P\left(\frac{1}{z}\right)}{Q\left(\frac{1}{z}\right)} = \frac{\sum_{j = 0}^{m} p_j z^{-1}}{\sum_{j = 0}^{n} q_j z^{-1}} = z^{n - m} \frac{\sum_{j = 0}^{m} p_{m - j} z^{j}}{\sum_{j = 0}^{n} q_{n - j} z^{j}}
\end{equation}
Note that $p_m$ and $q_n$ are non-zero, as we have assumed that the polynomials are simplified. Therefore, the numerator and denominator of this rational function are both non-zero at $z = 0$, so the only roots/poles at $z = 0$ are given
by the leading $z^{n - m}$ term. In particular, if $n = m$, we have no roots/poles. If $n > m$, we have $n - m$ roots and no poles. And if $n < m$, we have $m - n$ poles and no roots.

It follows that in each of the three cases, we can write down the number of roots/poles: when $n = m$, we have $m = n$ roots and poles. When $n > m$, we have $m + (n - m) = n$ roots and $n$ poles. Finally, when $n < m$,
we have $m$ roots and $(m - n) + n = m$ poles. We can easily theoremize this result:

\hhrulefill

\begin{theorem}
  Given a rational function $R = P/Q$ in reduced form, where $m = \text{deg}(P)$ and $n = \text{deg}(Q)$, the number of roots is equal to the number of poles, with both numbers being equal to $k = \max\{m, n\}$.
  We call $k$ the \emph{degree} of a rational function.
\end{theorem}

\hhrulefill

\begin{remark}
  This result pretty much matches our intuition from calculus: given a rational function of real polynomials, for large enough $x$, their quotient eventually behaves like a polynomial of degree $x^{m - n}$ when $m > n$,
  an inverse polynomial of degree $n - m$ when $m < n$, and some constant when $m = n$. Treating this asymptotic behaviour as the ``order'' of an $\infty$-root/pole yields the same casework and the same result as above.
\end{remark}

\noindent This result allows us to also prove some corollaries fairly immediately. For example:

\begin{corollary}
  \label{cor:exist}
  Given a rational function $R(z)$, the equation $R(z) = a$ for some $a \in \mathbb{C}$ has $\max\{m, n\}$ solutions (where $m$ and $n$ are the degrees of the reduced numerator and denominator).
\end{corollary}
\begin{proof}
  $R(z) - a$ is a rational function, and thus has exactly $\max\{m, n\}$ roots.
\end{proof}

\noindent Aside from the fairly low-hanging fruit of computing the number of roots/poles of a rational function, with a bit more effort, it is possible to \emph{put a rational function in a form
where its behaviour near poles is clearly exposed}. We call this form the \textbf{partial fraction decomposition} of a rational function.

\begin{theorem}[Representation of complex rational functions by partial fractions]
  Given a rational function $R$ with unique poles at $\beta_1, \dots, \beta_n$ and possibly at $\infty$, there exist polynomials $G, G_1, \dots, G_n$ such that
  \begin{equation}
    R(z) = G(z) + \displaystyle\sum_{j = 1}^{n} G_j \left( \frac{1}{z - \beta_j}\right)
  \end{equation}
\end{theorem}

\begin{proof}
  Let us begin by considering the case of a pole at $\infty$. In particular, suppose $R$ has a pole at $\infty$. It follows that if $R = \frac{P}{Q}$, then $\deg(P) > \deg(Q)$.
  Suppose we perform polynomial division, so that $P(z) = G(z) Q(z) + R(z)$, where $\deg(R) \leq \deg(Q)$ and $G$ has no constant term. It follows that $R = G + H$, where $H = \frac{R}{Q}$.
  Clearly, $H$ is finite at $\infty$, since the degree of the numerator is less than or equal to the degree of the denomninator, and $G$ is simply a polynomial (this polynomial will in fact
  dictate the ``behaviour'' of the rational function as $z \to \infty$.

  Let us now consider the case of a finite pole. As it turns out, we can easily reduce this case to the previous case of a pole at infinity.
  Let us define $z = \beta_j + \zeta^{-1}$, so that $\zeta = (z - \beta_j)^{-1}$. Clearly,
  \begin{equation}
    \widetilde{R}(\zeta) = R\left( \beta_j + \frac{1}{\zeta} \right) = \frac{\widetilde{P}(\zeta)}{\widetilde{Q}(\zeta)}
  \end{equation}
  is a rational function in the variable $\zeta$. Moreover, since $R$ has a pole at $\beta_j$, $R'$ will have a pole at $\zeta = \infty$. Thus, we can repeat the above procedure for the polynomial $P'$
  to get $P' = G_j' Q' + R'$ and $R'(\zeta) = G_j'(\zeta) + H'(\zeta)$, where $H'$ is finite at $\zeta = \infty$. We then take $G_j(z) = G_j'\left(\frac{1}{z - \beta_j}\right)$, switching back to the variable $z$.

  Our claim is now that the sum of all the $G_j$ along with $G$ is the original rational function $R$. Indeed, consider the rational function $F = R - G - \sum_{j} G_j$. Clearly, the only possible poles
  of this rational function are at the $\beta_j$ and at $\infty$. Except for each $\beta_i$, note that $F = (R - G_i) - G - \sum_{j \neq i} G_j$. We know that $R - G_i$ is finite at $\beta_i$, by construction,
  and that each of the other polynomials will also be finite. This also holds true at the $\infty$ pole.

  Thus, the rational function $F$ can have no poles, so it must be a constant $c$ (since the number of roots/poles is the maximum of the numerator and denominators' degrees). We can easily perform the shift $G \mapsto G - c$
  without changing the asymptotic behaviour of $G$, so that it is in fact true that $F = 0$ for all $z$, and we have the desired equality.
  \end{proof}

\noindent A nice feature of this theorem is that it allows us to prove the partial fraction decomposition in the case of real rational functions.

\begin{theorem}[Representation of real rational functions by partial fractions]
  Given a real rational function $R(z) = \frac{P(z)}{Q(z)}$ (so $P$ and $Q$ have real coefficients), there exists a decomposition of $R$ of the form
  \begin{equation}
    \label{eq:r_decomp}
    R(z) = G(z) + \displaystyle\sum_{i} A_i \left( \frac{1}{z - a_i} \right) + \displaystyle\sum_{j} z B_j \left( \frac{1}{z^2 + b_j z + c_j} \right) + C_j \left( \frac{1}{z^2 + b_j z + c_j} \right)
  \end{equation}
  where $G$, $A_i$, $B_j$, and $C_j$ are all real polynomials, and $a_i, b_j, c_j \in \mathbb{R}$, where $z - a_i$ and $z^2 + b_j z + c_j$ are the unique irreducible factors of $Q(z)$ as a real polynomial.
\end{theorem}

\begin{proof}
  \pop{\textbf{TODO}: Add this from problem set solutions}
  \end{proof}

\subsection{Fractional linear transformations and classifying rational functions}

\noindent As it turns out, degree-$1$ rational functions are not only particularly well-behaved, but they are a very special class of functions on the complex plane. In fact, they are
so special that we give them their own name: \emph{fractional linear transformation (FLTs)}. We begin by making note of a key fact

\begin{fact}
  Fractional linear transformations are invertible.
\end{fact}
\begin{proof}
  If $R = \frac{P(z)}{Q(z)}$ is an FLT, it is order-$1$. From Cor.~\ref{cor:exist}, for any $a \in \mathbb{C}$, the equation $R(z) = a$ has a single solution. Moreover, $R(z)$ has a single pole, so there is a unique $z_0$ sent to $\infty$
  by $R$. Thus, $R$ is bijective and invertible.
\end{proof}

\noindent Clearly, given an FLT, $R(z) = \frac{P(z)}{Q(z)}$, at least one of $P$ and $Q$ are degree-$1$ polynomials, so we can write $R(z) = \frac{az + b}{cz + d}$. However, it is not the case that every such set of $(a, b, c, d)$ will yield an FLT. Recall
that a for $\frac{az + b}{cz + d}$ to be of order-$1$, it must either have degree-$1$ numerator and denominator \emph{in reduced form}. Given arbitrary $(a, b, c, d)$, the quotient $\frac{az + b}{cz + d}$ does not necessarily have this property
when put into reduced form. This is a particularly subtle point which must be stressed: \emph{the degree of a rational function can only be defined with respect to a rational function in reduced form}.

\begin{claim}
  Given $(a, b, c, d)$, the rational function $f(z) = \frac{az + b}{cz + d}$ is an FLT if and only if $ad - bc \neq 0$.
\end{claim}
\begin{proof}
  Suppose $ad = bc$. Then either $a = b = 0$, $a = c = 0$, $d = b = 0$, $d = c = 0$, or $a, b, c, d$ are all non-zero. In each of the first four cases, it is easy to check that $f$ is not an FLT (as in reduced form, it is a constant).
  In the final case, we have
  \begin{equation}
   f(z) = \frac{1}{d} \cdot \frac{adz + bd}{cz + d} = \frac{bcz + bd}{cz + d} = \frac{b}{d} \frac{cz + d}{cz + d}.
  \end{equation}
  Thus, in reduced form, $f$ is a constant here as well. It follows that if $f$ is an FLT, $ad \neq bc$, so $ad - bc \neq 0$. Now, suppose $f$ is not an FLT, so it is degree-$0$ in reduced form (a constant). This implies
  that we must have $f(0) = f(\infty)$, so $a/c = b/d$, implying $ad - bc = 0$. Thus, if $ad - bc \neq 0$, then $f$ is an FLT. This completes the proof.
\end{proof}

\noindent Note that we can represent FLTs using matrices
\begin{equation}
 \frac{az + b}{cz + d} \Longleftrightarrow \begin{pmatrix} a & b\\ c & d \end{pmatrix}
  \end{equation}

\begin{claim}
  The composition of fractional linear transformations yields a fractional linear transformation whose associated matrix is precisely the the product of the associated matrices of the two
  constituent fractional linear transformations of the composition.
\end{claim}

\noindent \pop{This proof is just tedious computation, so I won't do it.} This result has the nice corollary:

\begin{corollary}
  The inverse of a fractional linear transform is the fractional linear transform described by the entries in the matrix obtained from inverting the matrix of the original fractional linear transform
  \end{corollary}

\hhrulefill

\noindent As has been suggested, fractional linear transformations are particularly well-behaved, so much so that it makes sense to consider rational functions which are equal,
up to composition with fractional linear transformations, to be ``equivalent in a sense''. More formally,

\begin{fact}
  The criterion of ``equal up to fractional linear transformations'' (which we will call \emph{FLT equivalent} going forward) defines an equivalence relation on the set of complex rational functions. We say that $R \sim R'$ is there
  exist fractional linear transformations $f$ and $g$ such that $R' = f \circ R \circ g$.
 \end{fact}

\noindent Of course, since we have an equivalence relation, we automatically can ask what the associated equivalence classes are. Beginning with the simple case of a rational function of order $2$, let us determine both how
many equivalence classes are generated by this relation and a ``nice'' representative of each equivalence class.

\begin{theorem}
  Every fractional linear transformation of order $2$ is FLT equivalent to $a(z) = z^2$ or $b(z) = z + z^{-1}$.
\end{theorem}

\begin{proof}
  Let us begin with the case where 
  \end{proof}

\subsection{The cross-ratio}

\noindent Now, we can move on to particularly important type of FLT: the cross-ratio. The cross-ratio is an answer to the question of whether, given three \emph{distinct} points $z_1, z_2, z_3 \in \overline{\mathbb{C}}$, we
can construct a fractional linear transformation which performs the mapping $(z_1, z_2, z_3) \mapsto (0, 1, \infty)$. Let us first suppose that $z_1, z_2$ and $z_3$ are all finite. Clearly, the FLT $\frac{z - z_1}{z - z_3}$ will
send $z_3$ to $\infty$ and $z_1$ to $0$. To ensure that $z_1$ is sent to $1$, we simply multiply by the reciprocal of this function by its evaluation at $z_2$, giving:
\begin{equation}
  S(z, z_1, z_2, z_3) = \frac{z_2 - z_3}{z_2 - z_1} \frac{z - z_1}{z - z_3}
\end{equation}
we call $S$ the \emph{cross-ratio}. We can also define the cross-ration when one of $z_1$, $z_2$ or $z_3$ is equal to $\infty$: this just requires some casework so we will leave it as an exercise. We arrive next at a very important result:

\begin{lemma}
  A fractional linear transformation is uniquely determined by where it sends any three distinct points of $\overline{\mathbb{C}}$.
\end{lemma}

\begin{proof}
  Suppose $h$ is an FLT taking $0, 1, \infty \mapsto 0, 1, \infty$. We can write $h(z) = \frac{az + b}{cz + d}$.
  Clearly, if $h(0) = 0$, then we must have $b/d = 0$, so $b = 0$. If $h(\infty) = \infty$, we must have $c = 0$. Thus, $h(z) = (a/d) z$. Since $h(1) = 1$, we must have $h(z) = z$, so $h$ is the identity

  Now, suppose $f$ is an FLT with $f(z_1), f(z_2), f(z_3) = a, b, c$. Let $p$ be the cross-ratio of $a, b, c$, such that $p(0), p(1), p(\infty) = a, b, c$. Suppose $g$ is another FLT such that $g(z_1), g(z_2), g(z_3) = a, b, c$.
  Then $p^{-1} \circ f \circ g^{-1} \circ p$ is an FLT which takes $0, 1, \infty \mapsto 0, 1 \infty$.
  From above, this FLT is the identity, so $f \circ g^{-1}$ is the identity, so $f = g$.
\end{proof}

\noindent Using this result allows us quickly prove the following result:

\begin{lemma}
  The cross-ratio is invariant under application of an FLT. In particular, if $z_1, z_2, z_3, z_4$ are distinct points, and $T$ is an FLT, then $S(Tz_1, Tz_2, Tz_3, Tz_4) = S(z_1, z_2, z_3, z_4)$.
\end{lemma}

\begin{proof}
  Note that by definition, $Q(z) = S(z, Tz_2, Tz_3, Tz_4)$ sends $(Tz_2, Tz_3, Tz_4) \mapsto (0, 1, \infty)$. Thus, $Q \circ T$ sends $(z_2, z_3, z_4) \mapsto (0, 1, \infty)$, so $Q \circ T = S$ from the above theorem.
  But clearly, $(Q \circ T)(z) = S(Tz, Tz_1, Tz_2, Tz_3)$, so the proof is complete.
\end{proof}

\begin{lemma}
 The cross ratio, $S(z_1, z_2, z_3, z_4)$ is real if and only if $z_1, z_2, z_3, z_4$ lie on a circle or a line.
\end{lemma}

\begin{proof}
  Let $T(z) = \frac{az + b}{cz + d}$. $T(z)$ is real if and only if
  \begin{align}
    T(z) = \overline{T(z)} & \Longleftrightarrow (az + b)(\overline{c} \overline{z} + \overline{d}) = (\overline{a} \overline{z} + \overline{b})(cz + d)
    \\ & \Longleftrightarrow (a\overline{c} - \overline{a} c) |z|^2 + (b\overline{c} - \overline{a} d) \overline{z} + (a \overline{d} - \overline{b} c) z + (b\overline{d} - \overline{b} d) = 0
  \end{align}
  for $z = x + iy$, we have
  \begin{equation}
    \alpha (x^2 + y^2) + (\beta - \overline{\beta}) x + i (\beta + \overline{\beta}) y + \gamma = 0
  \end{equation}
  where $\alpha = a\overline{c} - \overline{a} c$, $\beta = a \overline{d} - \overline{b} c$ and $\gamma = b\overline{d} - \overline{b} d$. Clearly, $\alpha$ and $\gamma$ are imaginary. In addition,
  $\beta - \overline{\beta}$ is imaginary, and $i (\beta + \overline{\beta})$ is as well. Thus, dividing both sides by $i$, we have
  \begin{equation}
    \alpha' (x^2 + y^2) + \beta_1' x + \beta_2' y + \gamma' = 0
  \end{equation}
  for real $\alpha', \beta_1', \beta_2'$, and $\gamma'$. This is the equation of a circle or a line (if $\alpha = 0$). Thus, an FLT (any FLT!) is real at $z$ if and only if $z$ lies on a particular circle/line determined
  by the FLT. In particular, the cross-ratio $z \mapsto S(z, z_2, z_3, z_4)$ is real at $z = z_2, z_3, z_4$, and is real if and only if $z$ lies on a particular circle/line determined by $S$. Thus, $S(z, z_2, z_3, z_4)$ is real
  if and only if $(z, z_2, z_3, z_4)$ lie on the same circle/line.
\end{proof}

\noindent Note that since FLTs are bijective, the cross-ratio $z \mapsto S(z, z_2, z_3, z_4)$ must send the circle/line determined by $z_2, z_3, z_4$ to the real line. We can now use the above two results to prove the following \emph{very important} claim:

\begin{theorem}
  Fractional linear transformations take circles and lines to circles and lines. Moreover, given any pair of circles/lines, there exists a unique FLT sending one
  to the other.
\end{theorem}

\begin{proof}
  First, note that a circle or a line in the plane can be determined by precisely three distinct points. Two points will determine a circle \emph{and} a line passing through them, a third point will then
  distinguish whether the points do lie on the corresponding circle, \emph{or} the line. Let $A$ be a circle/line determined by $z_1$, $z_2$, and $z_3$. Since $S(z, z_1, z_2, z_3)$ will be real
  if and only if $z, z_1, z_2, z_3$ lie on a circle/line, $S(z, z_1, z_2, z_3)$ will be real if and only if $z \in A$. Since $S$ is bijective, this means that $S(A) = \mathbb{R}$.

  Now, suppose $T$ is an arbitrary FLT. Consider the cross-ratio $F$ sending $T(z_1), T(z_2), T(z_3) \mapsto 0, 1 \infty$. Note that $F \circ T = S$. Thus, $T = F^{-1} \circ S$, and
  $T(A) = F^{-1}(\mathbb{R})$. Again, we know that $F$ will take the circle/line $A'$ defined by $T(z_1), T(z_2), T(z_3)$ to $\mathbb{R}$ bijectively, so $F^{-1}(\mathbb{R}) = A'$, and the proof is complete.
\end{proof}

The importance of this result lies in the fact that it allows us to think about FLTs as a very general-purpose tool which we can use to be back and forth between different regions in a particularly nice fashion.
In particular, since an FLT is bijective and continuous in both directions, it is a homeomorphism (in fact, it is a diffeomorphism which is everywhere conformal: we will discuss this later).
\newline

\noindent \pop{\textbf{TODO:} Fill in with more examples of why this result is useful.}

\section{Holomorphic functions}

\noindent Now that we have introduced some basic theory related to complex functions, we can move to explaining one of the fundamental concepts underlying complex analysis: \emph{holomorphic functions}.
Before proceeding, we make an important remark

\hhrulefill

\begin{remark}
  \label{rem:r2_vs_c}
As a high-level, when we do complex analysis, we are in some sense doing calculus in two variables, and in another sense, doing calculus in a single variable. The complex plane is defined as $\mathbb{R}^{2}$,
with a particular complex structure imposed, but we then go on to defining a collection of arithmetic operations which allow us to perform all of the standard operations one would expect to be able to apply to real numbers,
to pairs of numbers $(x, y)$ which we call \emph{complex numbers} and denote $z = x + iy$.

As a result of this fundamental fact, great care must be taken when defining particular constructions in complex analysis. Oftentimes, we will define something ``with respect to $x$ and/or $y$''. In other situations,
we will define something ``with respect to $z = x + iy$''. This vague assertion will make more sense as we begin to encounter concrete examples of it throughout the following sections, but it is worthwhile to make note
of it early, so that one may be aware and on the lookout for situations in which this slight ambiguity may arise.
\end{remark}

\hhrulefill

\subsection{Introducing and motivating holomorphic functions}

\noindent The main idea motivating the definition of a holomorphic function is the desire for a function of the variables $(x, y)$, such that standard differentiation of a two-variable function on $\mathbb{R}^2$
is ``compatible'' with the notion of differentiability that we expect to arise for a single-variable function, \emph{with the added structure of complex arithmetic}.

As was stated in Rem.~\ref{rem:r2_vs_c}, we can think about a complex function $f$ either as a two-variable function, $f(x, y)$ on $\mathbb{R}^2$, or as a single-variable function $f(z)$ on $\mathbb{C}$.
We know how to differentiate function on $\mathbb{R}^2$ from standard calculus. We do not know how to differentiate a function on $\mathbb{C}$ (yet), however, we do know how to differentiate single-variable
functions on $\mathbb{R}$. Thus, if we wish to treat the complex plane similar to how we treat $\mathbb{R}$, with the addition of complex arithmetic, it may make sense to define the derivative of a complex function on the
same way as it is defined in the real-single variable case.

\begin{definition}[Complex derivative]
  Given a function $f : \mathbb{C} \rightarrow \mathbb{C}$ written in terms of the complex variable $z = x + iy$, we say that $f$ is complex differentiable or \emph{holomorphic} at $z$ if the limit
  \begin{equation}
    \label{eq:complex_d}
    f'(z) = \lim_{|h| \to 0} \frac{f(z + h) - f(z)}{h}
  \end{equation}
  exists, calling this limit $f'(z)$ the complex derivative of $f$ in this case. Note that this limit is well-defined, as division by a complex number is well-defined.
\end{definition}

\noindent Now, suppose we are given a function of the form $g : \mathbb{R}^2 \rightarrow \mathbb{R}^2$. This function both has a well-defined \emph{real derivative} and a $2 \times 2$ Jacobian of partial
derivatives, but it also has an interpretation as a complex function, when we put the associated complex structure (and thus, complex arithmetic) on the domain and co-domain. However, it is \textbf{not} the
case that the standard, real derivative of $g$ will agree with its associated \emph{complex derivative}. In fact it is even worse than this: \emph{there will be cases where the real derivative exists but the complex derivative does not}.

\begin{example}
  Consider the function $f(x, y) = (x^2 + y^2, 0)$. When $z = x + iy$, this function can also be written as $f(z) = \overline{z} z$. This function is differentiable as a function from $\mathbb{R}^2$ to $\mathbb{R}^2$,
  but is not complex differentiable. The first fact is trivial. But, note that
  \begin{equation}
    f(z + h) - f(z) = (x + h_1)^2 + (y + h_2)^2 - x^2 - y^2 = 2h_1 x + h_1^2 + 2h_2 y + h_2^2
  \end{equation}
  As an example, in the case that $h_1 = 0$, then $\frac{f(z + h) - f(z)}{h} = 2 y + h_2$ and if $h_2 = 0$, then $\frac{f(z + h) - f(z)}{h} = 2 x + h_1$. In other words, in arbitrarily small neighbourhoods around $z$,
  there exists points such the quotient inside the limit is arbitrarily close to $2y$ and $2x$. Thus, if $x \neq y$, then the limit cannot possibly exist at $z$. In fact, using similar arguments when $x = y$, we can show that
  this function is nowhere holomorphic.
\end{example}

\noindent Examples like this lead us to the following question:

\hhrulefill

\begin{question}
If a function $g : \mathbb{R}^2 \rightarrow \mathbb{R}^2$ is differentiable in the usual, real sense, under what conditions is it also differentiable when interpreted as a complex function (holomorphic)?
\end{question}

\hhrulefill

\noindent The answer to this question leads us to the famous Cauchy-Riemann equations.

\begin{theorem}
  A function $g : \mathbb{R}^2 \rightarrow \mathbb{R}^2$ with $g(x, y) = (g_1(x, y), g_2(x, y))$ is complex differentiable/holomorphic, when interpreted as a function from $\mathbb{C}$ to $\mathbb{C}$ (in other words, when written as $g(x + iy) \coloneqq g_1(x, y) + i g_2(x, y)$),
  if and only if it is differentiable as a real function, and satisfies the \emph{Cauchy-Riemann equations}
  \begin{align}
    \frac{\partial g_1}{\partial x} &= \frac{\partial g_2}{\partial y} \\
    \frac{\partial g_1}{\partial y} &= -\frac{\partial g_2}{\partial x}
  \end{align}
  In particular, note that the condition of being holomorphic is stronger than that of being real-differentiable.
\end{theorem}

\begin{proof}
  Suppose that $g$ is complex differentiable, so that the limit of Eq.~\eqref{eq:complex_d} exists. It then follows that there exists a neighbourhood
  around $z$ for which
  \begin{equation}
    \label{eq:complex_der}
    g(z + h) = g(z) + g'(z) h + o(h)
  \end{equation}
  where $o(h)$ is a function such that $\lim_{|h| \to 0} \frac{o(h)}{h} = 0$. $g'(z) = u + i v$ and $z = x + iy$ are complex numbers. In addition, $h = h_1 + i h_2$ and when interpreted as a complex function,
  $g(x + iy) \coloneqq g_1(x, y) + i g_2(x, y)$. It follows that
  \begin{align}
    g(z + h) = g\left( (x + h_1) + i(y + h_2) \right) = g_1(x + h_1, y + h_2) + i g_2(x + h_1, y + h_2) \\
    g(z) = g_1(x, y) + i g_2(x, y) \ \ \ \text{and} \ \ \ g'(z) h = (u h_1 - v h_2) + i (u h_2 + v h_1)
  \end{align}
  Therefore, it follows that
  \begin{align}
    g_1(x + h_1, y + h_2) = g_1(x, y) + (u h_1 - v h_2) + \text{Re}[o(h)] = g_1(x, y) + \begin{pmatrix} u & -v \end{pmatrix} \begin{pmatrix} h_1 \\ h_2 \end{pmatrix} + \text{Re}[o(h)]
  \end{align}
  as well as
  \begin{align}
    g_2(x + h_1, y + h_2) = g_2(x, y) + (u h_2 + v h_1) + \text{Im}[o(h)] = g_2(x, y) + \begin{pmatrix} v & u \end{pmatrix} \begin{pmatrix} h_1 \\ h_2 \end{pmatrix} + \text{Im}[o(h)]
  \end{align}
  It follows immediately that
  \begin{equation}
    g((x, y) + (h_1, h_2)) = g(x, y) + \begin{pmatrix} u & -v \\ v & u \end{pmatrix} \begin{pmatrix} h_1 \\ h_2 \end{pmatrix} + o(h_1, h_2)
  \end{equation}
  in a neighbourhood around $(x, y)$. It follows by definition that $g$ is real-differentiable, and has (unique) derivative
  \begin{equation}
    Dg(z) = \begin{pmatrix} \partial_x g_1 & \partial_y g_1 \\ \partial_x g_2 & \partial_y g_2\end{pmatrix} = \begin{pmatrix} u & -v \\ v & u \end{pmatrix}
  \end{equation}
  which of course means that $\partial_x g_1 = \partial_y g_2$ and $\partial_y g_1 = -\partial_x g_2$, precisely the Cauchy-Riemann equations. Conversely, let us suppose that
  $g$ is real-differentiable with partial derivatives satisfying the Cauchy-Riemann equations: we effectively just trace the derivation above in reverse to recover Eq.~\eqref{eq:complex_der},
  which implies that $g$ is complex-differentiable.
\end{proof}

\begin{remark}
An import artifact of the above proof is that for a holomorphic function $f$, matrix multiplication of the standard differential matrix against some $h \in \mathbb{R}^2$
is precisely the same as multiplication of the complex derivative $f'(z)$ against the complex number $h = h_1 + i h_2$.
\end{remark}

\noindent Another important artifact of the above proof is handful of formulas for the complex derivative, namely
\begin{align}
  \label{eq:comp}
  g'(z) = u + iv &= \frac{\partial g_1}{\partial x} - i \frac{\partial g_1}{\partial y} = \frac{\partial g_2}{\partial y} + i \frac{\partial g_2}{\partial x} = \frac{1}{2} \left( \frac{\partial g_1}{\partial x} + \frac{\partial g_2}{\partial y} \right) + \frac{i}{2} \left(  \frac{\partial g_2}{\partial x} - \frac{\partial g_1}{\partial y} \right)
  \\ & =  \frac{\partial g_1}{\partial x} + i \frac{\partial g_2}{\partial x} = \frac{\partial}{\partial x} \left( g_1(x, y) + i g_2(x, y) \right) = \frac{\partial g(x, y)}{\partial x}
  \\ & =  -i \left( \frac{\partial g_1}{\partial y} + i \frac{\partial g_2}{\partial y} \right) = -i \frac{\partial}{\partial y} \left( g_1(x, y) + i g_2(x, y) \right) = -i \frac{\partial g(x, y)}{\partial y}
\end{align}
as well as its squared magnitude
\begin{equation}
  |g'(z)|^2 = u^2 + v^2 = \frac{\partial g_1}{\partial x} \frac{\partial g_2}{\partial y} - \frac{\partial g_1}{\partial y} \frac{\partial g_2}{\partial x} = \det Dg(x, y) = \left| \frac{\partial g(x, y)}{\partial x} \right|^2 =  \left| \frac{\partial g(x, y)}{\partial y} \right|^2.
\end{equation}

\noindent While being explicit and easy to use, these equations are suggestive: there are more succinct ways of expressing the holomorphic condition. The condition of a function
being holomorphic may better be expressed as \emph{a function which is constant in a particular direction in the complex plane}. In particular, let us define the following differential
operators
\begin{equation}
  \frac{\partial}{\partial z} = \frac{1}{2} \left( \frac{\partial}{\partial x} - i \frac{\partial}{\partial y} \right) \ \ \ \text{and} \ \ \  \frac{\partial}{\partial \overline{z}} = \frac{1}{2} \left( \frac{\partial}{\partial x} + i \frac{\partial}{\partial y} \right)
\end{equation}
This allows us to rewrite the complex derivative in Eq.~\eqref{eq:comp} in a much nicer way:
\begin{equation}
  g'(z) = \frac{1}{2} \left( \frac{\partial g_1}{\partial x} + \frac{\partial g_2}{\partial y} \right) + \frac{i}{2} \left(  \frac{\partial g_2}{\partial x} - \frac{\partial g_1}{\partial y} \right) = \frac{1}{2} \left( \frac{\partial}{\partial x} - i \frac{\partial}{\partial y} \right) (g_1 + i g_2) = \frac{\partial g(z)}{\partial z}
\end{equation}
It should now make sense why we defined $\frac{\partial}{\partial z}$ the way we did! In addition, note that
\begin{equation}
  \frac{\partial g}{\partial \overline{z}} = \frac{1}{2} \left( \frac{\partial}{\partial x} + i \frac{\partial}{\partial y} \right) (g_1 + i g_2) = \frac{1}{2} \left( \frac{\partial g_1}{\partial x} - \frac{\partial g_2}{\partial y} \right) + \frac{i}{2} \left(  \frac{\partial g_2}{\partial x} + \frac{\partial g_1}{\partial y} \right) = 0
\end{equation}
This leads us, easily, to the following result:
\begin{theorem}
  A differentiable function $g$ is holomorphic if and only if $\frac{\partial g}{\partial \overline{z}} = 0$.
  \end{theorem}
The definitions of these differential operators effectively arise from a change of basis in the tangent space, when we think about $\mathbb{R}^2$ as a \emph{complex manifold}. Let's think through what this means: There exists a global
identifcation $\iota : \mathbb{R}^2 \rightarrow \mathbb{C}$, which we can think of as a global coordinate chart for thinking of $\mathbb{R}^2$ as a one-dimensional complex manifold. \pop{\textbf{TODO:} Need to finish this}.

\hhrulefill

\subsection{A more sophisticated perspective}

\noindent \pop{\textbf{TODO:} This section is incomplete and needs to be finished.}
\newline

\noindent Let us begin with a brief review of differential geometry.

\begin{remark}[Tangent space notation]
Recall that for a vector space $V$, we define $\Wedge^{k}(V)$ to be the set of alternating $k$-tensors on $V$.
We define the tangent space of $\mathbb{R}^{n}$ at point $p$, $T_p \mathbb{R}^n$, to be the vector space $\{p\} \times \mathbb{R}^{n}$, with elements $(p, v)$ and $(p, v_1) + \lambda (p, v_2) \coloneqq (p, v_1 + \lambda v_2)$.
A differential $k$-form $\omega$ is a function from $\mathbb{R}^{n}$ to the disjoint union of all $\Wedge^{k}(T_p \mathbb{R}^{n})$ over $p \in \mathbb{R}^{n}$ such that $\omega(p) \in \Wedge^{k}(T_p \mathbb{R}^{n})$. Let $\Omega^{k}(\mathbb{R}^{n})$
denote the space of all $k$-forms. We define the differential $d : C^{\infty}(\mathbb{R}^{n}, \mathbb{R}) \rightarrow \Omega^1(\mathbb{R}^n)$ as taking $f$ to the differential form $\omega$ such that $\omega(p)(p, v) = [D f(p)] v$. Note that
$C^{\infty}(\mathbb{R}^{n}, \mathbb{R}) = \Omega^0(\mathbb{R}^n)$. It follows that
\end{remark}

\begin{definition}
  \end{definition}

\begin{remark}
  Another, more formal way, of stating this fact is that holomorphic functions are precisely the functions $g$ from $(\mathbb{R}^2, J)$ to $(\mathbb{R}^2, J)$ whose derivative
  commutes with the complex structure,
  \begin{equation}
    J \circ D g(z) = D g(z) \circ J.
  \end{equation}
  Intuitively, this means that the map which turns $\mathbb{R}^2$ into $\mathbb{C}$ is compatible with taking the derivative in the usual sense.
  \end{remark}

\begin{definition}[Holomorphic and anti-holomorphic functions]
\end{definition}

\hhrulefill

\subsection{Basic properties and implications}

\noindent Having introduced the notion of a holomorphic function, let us discuss some basic properties and implications related to functions of this form.

\begin{claim}
  If $f$ is differentiable, then $\frac{\partial f}{\partial \overline{z}} = \frac{\partial \overline{f}}{\partial z}$.
\end{claim}
\begin{proof}
  Note that
  \begin{equation}
    \frac{\partial f}{\partial z} = \frac{\partial f}{\partial x} - i \frac{\partial f}{\partial y} = \overline{\frac{\partial \overline{f}}{\partial x} + i \frac{\partial \overline{f}}{\partial y}} = \overline{\frac{\partial \overline{f}}{\partial \overline{z}}}.
    \end{equation}
\end{proof}

\noindent Thanfkully, we can also prove the following result:

\begin{claim}[Complex derivatives behave like normal derivatives]
The complex derivative $\frac{\partial}{\partial z}$ satisfies the chain rule and product rule.
\end{claim}
\begin{proof}
  Let's consider the chain rule, for example. We want to show that $\partial_z f(g(z)) = \partial_z f(g(z) \cdot \partial_z g(z) = f'(g(z)) \cdot g'(z)$. Recall that for a holomorphic function,
  multiplication via $f'(z)$ is the same as matrix multiplication by the differential. In other words,
  \begin{align}
    Df(g(z)) \cdot g'(z) = f'(g(z)) \cdot g'(z) &\Longrightarrow Df(g(z)) \cdot \partial_x g(x, y) = f'(g(z)) \cdot g'(z) \\& \Longrightarrow \partial_x f(g(x, y)) = \partial_z f(g(z)) = f'(g(z)) \cdot g'(z)
  \end{align}
  so we have chain rule. Of course, we used the fact that $\partial_z f(z) = \partial_x f(x,y)$ for holomorphic $f$, and the regular two-variable chain rule. Product rule is more annyoing, but we can still do it. We have
  \begin{align}
    f'(z) g(z) + g'(z) f(z) &= \partial_x f(x, y) g(z) + \partial_x g(x, y) f(z) \\ &= \left( (\partial_x f_1) g_1 - (\partial_x f_2) g_2 + (\partial_x g_1) f_1 - (\partial_x g_2) f_2 \right) + i
    \left( (\partial_x f_1) g_2  + (\partial_x f_2) g_1 + (\partial_x g_1) f_2 + (\partial_x g_2) f_1 \right) \nonumber
    \\ & = \partial_x (f_1 g_1 - f_2 g_2) + i \partial_x (f_1 g_2 + f_2 g_1) = \partial_x (fg) = (fg)'(z)
    \end{align}
  \end{proof}

\begin{claim}[Basic properties of holomorphic functions]
  Let $f$ be a holomorphic function on a connected open set $\Omega$. Then
  \begin{enumerate}
  \item If $f'(z) = 0$ on $\Omega$, then $f$ is constant.
  \item If $\text{Re}[f]$ is constant on $\Omega$, then $f$ is constant.
    \item If $|f|$ is constant, then $f$ is constant.
    \end{enumerate}
\end{claim}
\begin{proof}
  \begin{enumerate}
  \item If $f'(z) = 0$, then by definition $\frac{\partial f}{\partial z} = 0$. Since $f$ is holomorphic, $\frac{\partial f}{\partial \overline{z}} = 0$. Therefore,
    \begin{align}
      \frac{\partial f}{\partial x} = \frac{\partial f}{\partial z} + \frac{\partial f}{\partial \overline{z}} = 0
      \\  \frac{\partial f}{\partial y} = i\frac{\partial f}{\partial z} - i \frac{\partial f}{\partial \overline{z}} = 0
    \end{align}
    which implies that $Df = 0$ so $f$ is constant.
  \item Since $f$ is holomorphic, it follows from the Cauchy-Riemann equations that $\partial_x \text{Im}[f] = \partial_y \text{Im}[f] = 0$, so $\text{Im}[f]$ is constant, and $f$ is therefore constant.
  \item $|f|$ being constant is equivalent to $f \overline{f}$ being constant. Thus,
    \begin{equation}
      \frac{\partial}{\partial z} (f \overline{f}) = \overline{f} \frac{\partial f}{\partial z} + f \frac{\partial \overline{f}}{\partial z} = 0 \Longrightarrow \overline{f} \frac{\partial f}{\partial z} = 0
    \end{equation}
    where we know that $\frac{\partial \overline{f}}{\partial z} = \overline{\frac{\partial f}{\partial \overline{z}}} = 0$ as $f$ is holomorphic. Thus, at each point in $\Omega$, we either have $\overline{f}(z) = 0$ or
    $\partial_z f = 0$. If $|f|$ is constant and $\overline{f} = 0$ at a single point, then $|f| = 0$ everywhere and thus $f = 0$ everywhere. If $\overline{f}$ is nowhere $0$, then $\partial_z f = 0$ on $\Omega$, so from the first
    point, $f$ is constant.
    \end{enumerate}
\end{proof}

\noindent Now, let us look at a few more sophisiticated results.

\begin{theorem}
  Suppose $f$ is a smooth function defined on connected domain $\Omega$ such that $|\det Df(x, y)| \neq 0$ on $\Omega$. Then if $f$ is angle-preserving (i.e. the angles between tangent lines are preserved), it will either be holomorphic or anti-holomorphic ($\partial_z f = 0$).
\end{theorem}
\begin{proof}
  Because $\Omega$ is connected and $|\det D(x, y)|$ is continuous, it cannot change sign, otherwise it would equal $0$ at some point by intermediate value theorem. If $f$ preserves angles, it follows by definition that $Df(x, y)$ is an isometry.
  Thus, we must have
  \begin{equation}
    D f(x, y) = \begin{pmatrix} u(x, y) & -v(x, y) \\ v(x, y) & u(x, y) \end{pmatrix} \ \ \ \text{or} \ \ \  \begin{pmatrix} -u(x, y) & v(x, y) \\ v(x, y) & u(x, y) \end{pmatrix}
  \end{equation}
  depending on whether $\det Df(x, y)$ is positive or negative on $\Omega$, which corresponds to $f$ being holomorphic or anti-holomorphic. We could have also seen this by noting that a real linear function which preserves angles must be
  of the form $f(z) = w z$ or $f(z) = w \overline{z}$. Since $f(z) = Df(z_0) z$ preserves angles and is real-linear, it must be of either of these forms. It is easy to check that the former case corresponds to $f$ being holomorphic and the
  latter corresponds to $f$ being anti-holomorphic.
\end{proof}

\noindent Now, let's consider an even more advanced result, which should be familiar from calculus.

\begin{theorem}[Inverse function theorem]
Suppose $f$ is holomorphic on an open set $U$ containing $z_0$ with $f'(z_0) \neq 0$. Then there exists a local holomorphic inverse around $f(z_0)$
\end{theorem}
\begin{proof}
  If $f$ is holomorphic, then $f'(z_0) \neq 0 \Rightarrow |\det Df| \neq 0$, so we just need to show that if $f$ is holomorphic, then is is continuously differentiable,
  to show that we have a continuously differentiable local inverse. The fact that this local inverse is holomorphic itself follows from the fact that we can write down
  exactly what the inverse differential must be, namely
  \begin{equation}
    D f^{-1}(f(z_0)) = Df(z_0)^{-1} = \begin{pmatrix} u & -v \\ v & u \end{pmatrix}^{-1} = \frac{1}{u^2 + v^2} \begin{pmatrix} u & v \\ -v & u \end{pmatrix}
  \end{equation}
  where it is easy to check from here that $f^{-1}$ must be holomorphic. \pop{We will prove that holomorphic implies continuously differentiable \textbf{later}!}
\end{proof}

\section{Examples of holomorphic functions}

\noindent Now that we have outlined a few basic results related to holomorphic functions, let us discuss a few examples of families of functions which are holomorphic.
We can begin with the obvious: rational functions (away from their poles). To prove this, note that
\begin{align}
  \frac{\partial z}{\partial \overline{z}} = \frac{1}{2} \left( \frac{\partial}{\partial x} + i \frac{\partial}{\partial y} \right) (x + iy) = 0
\end{align}
In addition, note that since $\frac{\partial}{\partial \overline{z}}$ satisfies chain and product rule, for $z \neq 0$, note that $z^{-1} = \frac{x - iy}{x^2 + y^2}$ is (real) differentiable and
\begin{equation}
  \frac{\partial}{\partial \overline{z}} (z z^{-1}) = z^{-1} \frac{\partial z}{\partial \overline{z}} + z \frac{\partial z^{-1}}{\partial \overline{z}} = z \frac{\partial z^{-1}}{\partial \overline{z}} = 0 \Rightarrow \frac{\partial z^{-1}}{\partial \overline{z}} = 0
\end{equation}
so $z^{-1}$ is holomorphic as well. Product and chain rule imply that all rational functions are holomorphic, after proving these facts.

\subsection{Power series}

\noindent The next example of holomorphic functions that we will look at are \emph{power series}: the limits of sequences of complex polynomials.

\begin{theorem}[Abel]
  Given a complex power series $f(z) = \sum_{j = 0}^{\infty} a_j z^{j}$, there exists a \emph{radius of convergence} $R \in [0, \infty]$ such that
  \begin{enumerate}
  \item for $r < R$, $f$ is uniformly and absolutely convergent in $B_r(0)$: the closed ball of radius $r$ around $0$.
  \item For $|z| > R$, $\sum_{j = 0}^{\infty} a_j z^j$ diverges (in fact, the terms $a_j z^j$ are unbounded in magnitude).
  \item The derived series $\sum_{j = 1}^{\infty} j a_j z^{j - 1}$ has the same radius of convergence as $f$. Moreover, $f$ is holomorphic and $f'(z)$ (the complex derivative)
    is precisely the derived series. Note that this result immediately implies that a power series is smooth.
  \end{enumerate}
  In fact, we can compute the radius of convergence via the \emph{Cauchy-Hadamard formula}: $R^{-1} = \limsup \sqrt[n]{a_n}$.
\end{theorem}

\begin{proof}
  \pop{\textbf{TODO}: Fill this in. This is an important result!}
\end{proof}

\noindent Equipped with the main theorem characterizing power series as functions, we can begin to discuss manipulations of complex power series. Of course, all complex power series are determined
by a sequence of complex coefficients.

\begin{remark}[Composition of power series]
  Suppose we have two power series $f(z) = \sum_{n = 0}^{\infty} a_n z^{n}$ and $g(z) = \sum_{n = 0}^{\infty} b_n z^n$. We consider the question of when it makes sense
  to write down the composition of two power series. It turns out that this will be precisely when $b_0 = 0$, as here, formally composing the power series yields a new, well-defined power series
  where the coefficients are \emph{finite} sums rather than \emph{infinite} ones.
  \end{remark}

\begin{theorem}[Formal inverse function theorem for power series]
\end{theorem}

\begin{theorem}[Composition of power series]
  Let $f(z) = \sum_{n = 0}^{\infty} a_n z^{n}$ and $g(z) = \sum_{n = 1}^{\infty} b_n z^n$ be power series (note that $g(0) = 0$). If $f$ and $g$ both have positive radius
  of convergence, then so does $f \circ g$. In fact, suppose we take $r$ such that $\sum_{n} |b_n| r^n < R(f)$, then $R(f \circ g) \geq r$. Moreover, if we let $f \circ g$
  denote the formal composite power series defined above, we have $(f \circ g)(z) = f(g(z))$ for $|z| \leq r$.
\end{theorem}
\begin{proof}
The main 
  \end{proof}

\subsection{The complex exponential}

\noindent Let us now consider, arguably, the most important holomorphic complex function determined via power series: the complex exponential.

\begin{definition}[Complex exponential]
  We define the complex exponential function, $\exp(z)$, via the power series
  \begin{equation}
    \exp(z) = e^z = \displaystyle\sum_{j = 0}^{\infty} \frac{z^j}{j!}
  \end{equation}
  This definition agrees with the standard definition of the exponential function on $\mathbb{R}$, as this is simply the Taylor expansion: we now take it as the \emph{definition}
  of the function on all complex numbers.
\end{definition}

\begin{theorem}
  \label{thm:circ}
  The complex exponential behaves as we would expect an exponential to behave. In particular, $e^{a} e^{b} = e^{a + b}$ for all $a, b \in \mathbb{C}$.
\end{theorem}
\begin{proof}
  We will use a slick proof. Define $f(z) = e^{-z} e^{z + w} - e^{w}$ for fixed $w$. Since the exponential is defined via a power series and converges on all of $\mathbb{C}$, note that we
  can compute its (complex) derivative via the term-wise derivative, as well as use chain rule (since we showed it applies to complex derivatives). In particular, it is easy to see that $\partial_z e^{z} = e^{z}$. Thus,
  \begin{equation}
    \partial_z f(z) = -e^{-z} e^{z + w} + e^{-z} e^{z + w} = 0
  \end{equation}
  so $f(z)$ must be constant on connected domain $\mathbb{C}$. Setting $z = 0$, we can see that this constant must be $0$. Thus, for any $z$ and $w$, $e^{-z} e^{z + w} = e^{w}$, which implies that $e^{a} e^{b} = e^{a + b}$
  for any $a, b \in \mathbb{C}$.
  \end{proof}

\noindent Let us define the function $g : \mathbb{R} \rightarrow \mathbb{T}$ where $g(y) = e^{iy}$. The reason why this is a mapping into the unit circle follows from Thm.~\ref{thm:circ}, $|e^{iy}| = e^{-iy} e^{iy} = 1$.
By definition,
\begin{equation}
  e^{iy} =  \displaystyle\sum_{j = 0}^{\infty} \frac{(iy)^j}{j!} = \displaystyle\sum_{j = 0}^{\infty} \frac{(-1)^{j} y^{2j}}{(2j)!} + i \displaystyle\sum_{j = 0}^{\infty} \frac{(-1)^{j} y^{2j + 1}}{(2j + 1)!} = \cos(y) + i \sin(y)
\end{equation}
where we use the standard real-valued Taylor series for $\cos$ and $\sin$ functions. As you may have guessed, we will define $\cos$ and $\sin$ \emph{on all of $\mathbb{C}$} with these power series expansions! This allows us to
note that
\begin{equation}
  \cos(z) = \frac{e^{iz} + e^{-iz}}{2} \ \ \ \text{and} \ \ \ \sin(z) = \frac{e^{iz} - e^{-iz}}{2i}
\end{equation}
From here,

\subsection{The Arg function}

\noindent In the previous section, we defined the exponential function via a power series on the entirety of the complex domain. This allows us to define a function from $\mathbb{R}$ to $\mathbb{T}$, $y \mapsto e^{iy}$.


\noindent We will also invoke a very important result:

\begin{theorem}[Universal property of quotient topology]
  Let $g : X \rightarrow Y$ be a surjective continuous map, let $X^{*}$ be the collection of fibres of $g$, let $\pi : X \rightarrow X^{*}$ be the corresponding quotient map. $g$ induces a bijective
  continuous map $f : X^{*} \rightarrow Z$ which is a homeomorphism if and only if $g$ is a quotient map.
\end{theorem}

\begin{proof}
 \pop{ \textbf{TODO:} This is definitely worth proving, I'll do this after I finish filling in the rest of the unfinished sections.}
  \end{proof}

\noindent Let $\sim$ be the equivalence relation which identifies points differing up to integer multiple of $2\pi$. We already demonstrated that the fibres of $y \mapsto e^{iy}$ are precisely the equivalence classes
generated by $\sim$, so $X^{*} = \mathbb{R}/\sim = \mathbb{R}/2\pi \mathbb{Z}$, in the notation of the above theorem. Thus, if we show that $g : y \mapsto e^{iy}$ is a quotient map, we will have a homeomorphism between $\mathbb{R}/2\pi \mathbb{Z}$
and $\mathbb{T}$. Indeed, $g$ is continuous, open, and surjective, so it is a quotient map. Thus, the map induced by $y \mapsto e^{iy}$ such that $[y] \mapsto e^{iy}$ is a homeomorphism.

The most important implication of this result is that \emph{$[y] \mapsto e^{iy}$ has a continuous inverse}. We will call this inverse $\arg$, the argument function.

\begin{definition}[The argument function]
  The function $\arg : \mathbb{T} \rightarrow \mathbb{R}/2\pi\mathbb{Z}$ is called the argument function and is defined as the continuous inverse of the induced map $[y] \mapsto e^{iy}$ constructed above.
  In fact, we can extend $\arg$ to a function on $\mathbb{C} - (0, 0)$ by re-defining $\arg(z)$ as $\arg(z/|z|)$, where $z/|z| \in \mathbb{T}$, clearly.
\end{definition}

\noindent The argument function is effectively a function assigning an ``angle'' to points on the complex plane. However, working with equivalence classes isn't always desirable.
\newline

\noindent \pop{\textbf{TODO:} Fill in the part about branches}
\newline

\begin{definition}[Branch]
  A branch is a continuous function $f(z)$ defined on some connected set $\Omega$ such that $g(f(z)) = 0$, for some constraint $g(x) = 0$.
  \end{definition}

\subsection{The complex logarithm and power functions}

\noindent In addition to defining a complex exponential, we can also define an inverse function: a complex logarithm. We want the complex logarithm to be a function
which satisfies $e^{f(z)} = z$. If $f(z) = f_1(z) + i f_2(z)$, then we must have $e^{f_1(z)} e^{i f_2(z)} = |z| e^{i \arg(z)}$. For real $z$, we must have $e^{f_1(z)} = |z|$, so
we should set $f_1(z) = |z|$. Then, we must have $e^{i f_2(z)} = e^{i \arg(z)}$.

\begin{claim}
  If $f$ is a branch of $\log(z)$ on $\Omega$, then it is of the form $\log |z| + i \text{Arg}(z) + 2\pi i k$, where $\text{Arg}(z)$ is the principal argument (defined above).
  Moreover, $f$ is holomorphic on $\Omega$ and has complex derivative $1/z$.
\end{claim}

\begin{proof}
  Suppose $e^{f(z)} = z$. We must have $f(z) = f_1(z) + i f_2(z)$, where both $f_1$ and $f_2$ are real functions. In particular, we must have $e^{f_1(z)} e^{i f_2(z)} = z$, so $e^{f_1(z)} = |z|$.
  This immediately implies that $f_1(z) = \log |z|$. Note that $z = |z| e^{i \text{Arg}(z)}$, so we must have $e^{i f_2(z)} = e^{i \text{Arg}(z)}$. In other words, $e^{i (f_2(z) - \text{Arg}(z))} = 1$
  for all $z$. We know that $e^{i x} = 1$ if and only if $x = 2\pi i k$ for some $k \in \mathbb{Z}$. Thus, we must have $f_2(z) - \text{Arg}(z) = 2\pi i k(z)$, where $k(z)$ is a \emph{continuous}
  function from $\mathbb{C}$ to $\mathbb{R}$ which is integer-valued. Thus, it must be constant, so $f_2(z) = \text{Arg}(z) + 2\pi i k$ for some constant $k \in \mathbb{Z}$.

  To prove that a branch $f(z)$ is holomorphic with the desired derivative, note that
  \begin{align}
    \lim_{h \to 0} \frac{f(z + h) - f(z)}{(z + h) - h} &= \lim_{h \to 0} \frac{f(z + h) - f(z)}{e^{f(z + h)} - e^{f(z)}} = \left( \lim_{h \to 0} \frac{e^{f(z + h)} - e^{f(z)}}{f(z + h) - f(z)} \right)^{-1}
    \\ & = \left( \lim_{f(r) \rightarrow f(z)} \frac{e^{f(r)} - e^{f(z)}}{f(r) - f(z)} \right)^{-1} = e^{-f(z)} = \frac{1}{z}
  \end{align}
  and the proof is complete.
\end{proof}

\noindent As it turns out, $\log$ has a convergent power series representation! \pop{\textbf{TODO}: fill this in}

\subsection{Riemann surfaces}

\noindent Riemann surfaces are, at a high-level, a nice way of dealing with classes of complex-valued functions which have multiple branches.
in particular, suppose we wish to define a function which sends $z \mapsto w$ when the pair $(z, w)$ satisfies the constraint $g(z, w) = 0$. Of course,
we can immediately obtain a function when for a given $z$, there exists precisely one $w$ such that $g(z, w) = 0$. However, there will often be cases
where \emph{there exists multiple values of $w$ satisfying the constraint}. For example, suppose $g(z, w) = w^2 - z$. Both pairs $(z, z)$ and $(z, -z)$
are valid solutions. In many cases, even though we no longer have a well-defined function right away, we still wish to define some sort of mapping
using these kinds of constraints (as they come up often, for example, the constraint $g(z, w) = w^2 - z$ yields precisely the square roots of $z$).

To deal with these cases, we introduce the notion of a Riemann surface. At a high-level, Riemann surfaces allow us to define functions
with multi-valued constraints.

\section{Conformal maps}

\noindent We now move to a discussion of conformal maps, which roughly speaking, can be thought of as maps which preserve angles. We did briefly mention maps
of this form in previous sections, but here, we will elaborate.

\begin{theorem}
  Holomorphic functions are conformal.
\end{theorem}

\begin{corollary}
  Fractional linear transformations preserve angles on $\mathbb{C}$.
\end{corollary}

\noindent \textbf{TODO:} Include example of using conformality to prove mapping properties of some complicated complex mapping

\section{Analytic functions and analytic continuation}

\noindent We begin this section with a fundamental definition.

\begin{definition}[Analytic function]
  \end{definition}

\section{Complex integration}

\noindent In this section, we initiate our discussion of the next major topic in the study of introductory complex analysis: complex integration.
This discussion will conclude with a proof of \textbf{Cauchy's theorem for a disk}, arguably the most important result in all of complex analysis it will allow us to
prove the following results, essentially as corollaries:
\begin{itemize}
\item Generalized Cauchy's theorem (Thm.~\ref{thm:})
\item Cauchy's integral formula (Thm.~\ref{thm:})
\item $\text{Holomorphic} \ \Leftrightarrow \ \text{Analytic}$ (Thm.~\ref{thm:})
\item Morera's theorem (Thm.~\ref{thm:})
\item Liouville's theorem (Thm.~\ref{thm:})
\item The fundamental theorem of algebra (Thm.~\ref{thm:})
\item Schwarz' reflection principle
\item Schwarz's lemma
\end{itemize}
to name a few (there are more).

\subsection{A review of integration with differential forms}

\noindent Let us briefly review the theory of integration with differential forms.

\begin{definition}[Continuously differentiable forms]
  Let $\omega$ be a continuous form on open set $\Omega$. The form $\omega$ is said to be \emph{exact} if there exists some form $\nu$ on $\Omega$ such that $d\nu = \omega$ (in other words,
  $\omega$ has a primitive). If $\omega$ is a $C^1$-form on $\Omega$, $\omega$ is said to be \emph{closed} if $d\omega = 0$. Note that every exact $C^1$-form is closed, but it is not true that every closed form is exact.
\end{definition}

\noindent Let us now recall how to go about integrating a form over a chain (an element of the $\mathbb{Z}$-module generated by $n$-cubes).

\subsection{Integrating over a complex domain}

\noindent To begin, let us consider what it means to integrate over a complex domain. Of course, $\mathbb{C}$ is simply $\mathbb{R}^2$ with extra structure. Thus, integrating a function
over a region is simply defined to be integration as is the case in $\mathbb{R}^2$. In two-dimensions, we know that we can integrate $1$-forms over curves, and $2$-forms over areas. For example,
$\omega = dz/z$ is a $1$-form in $\mathbb{R}^2$, defined as
\begin{equation}
  \frac{dz}{z} = \frac{1}{x + iy} (dx + i dy) = \frac{x - i y}{x^2 + y^2} dx + \frac{y + i x}{x^2 + y^2} dy
\end{equation}
One may notice a small problem here: we are used to integrating $1$-forms in $\mathbb{R}^2$ of the form $p dx + q dy$, where $p$ and $q$ are scalar functions. However,
if we continue with our identification of $\mathbb{C}$ with $\mathbb{R}^2$, then the coefficient functions of this form are actually mapping into $\mathbb{R}^2$! However, we already know
that we can very naturally treat $\mathbb{C}$ as a scalar field in its own right. This leads us to \emph{define} integration over forms with coefficient functions taking values in $\mathbb{C}$
as the \emph{sum} of the usual integrals over their real and imaginary parts.
\newline

\noindent In the standard theory of differential forms, we usually only consider integrals of $1$-forms $\int_{\gamma} \omega$ when $\gamma$ is a $C^{1}$ curve. However, we can generalize integration to
the case where $\gamma$ is piecewise $C^{1}$, which is to say that it can be broken up into a finite number of components, each of which is $C^{1}$. \pop{\textbf{TODO}: Finish this}.

\begin{lemma}
  Let $\Omega$ be a connected open set. Any two points $a, b \in \Omega$ are connected by a piecewise $C^1$ path.
\end{lemma}
\begin{proof}
  Fix $a \in \Omega$. Let $U_a$ be the set of all points $x \in \Omega$ such that there exists a piecewise $C^1$ path from $a$ to $x$. Clearly, some open ball around $a$
  contained in $\Omega$ is a subset of $U_a$ (we can trace a straight line from $a$ to any point in the ball), so $U_a$ is non-empty. Note that $U_a$ is an open set, as if $x \in U_a$, then any open ball around $x$ contained in $\Omega$ is in $U_a$ as well (we can always find such a ball as $\Omega$ is open).

  Let $U_a'$ be the collection of all $x \in \Omega$ such that there \emph{doesn't} exist a peicewise $C^1$ path from $a$ to $x$. Then similar to above, it is easy to see that
  $U_a'$ must be open, as if $x \in U_a'$, then any open ball around $x$ contained in $\Omega$ must also be contained in $U_a'$, otherwise we could generate a path from $a$ to a point $y$
  in the ball, and then connect $y$ to $x$ (as they both lie in the ball).

  Clearly, $\Omega = U_a \cup U_a'$. But since $\Omega$ is connected, and $U_a$ is non-empty, we must have $U_a = \Omega$, and the proof is complete.
\end{proof}

\begin{lemma}
  If $\omega$ is a $1$-form on a connected open set $\Omega$ with the property that for any closed curve $\gamma \in \Omega$, $\displaystyle\int_{\gamma} \omega = 0$,
  then there exists some $f$ such that $df = \omega$ (i.e. $\omega$ has a primitive on $R$).
\end{lemma}
\begin{proof}
  One direction is easy: suppose that $\omega = df$. Then for $\gamma$ a closed curve,
  \begin{equation}
    \displaystyle\int_{\gamma} \omega = \displaystyle\int_{\gamma} df = f(\gamma(1)) - f(\gamma(0)) = 0
  \end{equation}
  as $\gamma(0) = \gamma(1)$. Conversely, suppose $\omega$ is a form with the property that $\int_{\gamma} \omega = 0$ for any closed
  curve $\gamma$. Let $(x_0, y_0)$ be a fixed point in $\Omega$, let $\gamma$ be a piecewise $C^{1}$ curve starting at $(x_0, y_0)$ and ending at $(x, y)$. We define
  $f(x, y) = \int_{\gamma} \omega$. Note that $f$ is well defined: if $\nu$ is another piecewise $C^1$ curve from $(x_0, y_0)$ to $(x, y)$, then by assumption
  \begin{equation}
    \displaystyle\int_{\gamma - \nu} \omega = 0 \Longrightarrow \displaystyle\int_{\gamma} \omega = \displaystyle\int_{\nu} \omega
  \end{equation}
  so that $f$ is independent of which path $\gamma$ is chosen. Now, note that
  \begin{equation}
    f(x + h, y) - f(x, y) = \displaystyle\int_{\gamma_1} \omega - \displaystyle\int_{\gamma_2} \omega = \displaystyle\int_{-\gamma_2 + \gamma_1} \omega = \displaystyle\int_{\gamma} \omega
  \end{equation}
  where $\gamma = -\gamma_2 + \gamma_1$ is a $C^1$ piecewise path from $(x + h, y)$ to $(x, y)$ (in particular, we travel along $\gamma_2$ from $(x, y)$ to $(x_0, y_0)$,
  then along $\gamma_1$ from $(x_0, y_0)$ to $(x + h, y)$. As we concluded earlier, we can replace $\gamma$ with any path from $(x, y)$ to $(x + h, y)$ in $\Omega$.
  Suppose $\gamma_x$ is the horizontal path between the two points. Let $\omega = p(x, y) dx + q(x, y) dy$. Then,
  \begin{equation}
    \displaystyle\int_{\gamma} \omega = \displaystyle\int_{\gamma_x} p(x, y) dx + q(x, y) dy = \displaystyle\int_{\gamma_x} p(x, y) dx = \displaystyle\int_{x}^{x + h} p(s, y)
  \end{equation}
  It follows from the fundamental theorem of calculus that
  \begin{equation}
    \lim_{h \to 0} \frac{1}{h} \displaystyle\int_{x}^{x + h} p(s, y) = \frac{d}{dt} \displaystyle\int_{0}^{t} p(s, y) ds \biggr\rvert_{t = x} = p(x, y)
  \end{equation}
  So, combining the above equations,
  \begin{equation}
    \frac{\partial f(x, y)}{\partial x} = \lim_{h \to 0}  \frac{f(x + h, y) - f(x, y)}{h} = p(x, y)
  \end{equation}
  A similar argument shows that $\partial_y f(x, y) = q(x, y)$. Thus, it follows immediately that $\omega = df$, and the proof is complete.
\end{proof}

\begin{corollary}
  \label{cor:rect}
  If $\omega$ is a $1$-form on an open disk $D$ with the property that for any closed curve $\gamma \in \Omega$ which is the boundary of a rectangle, $\int_{\gamma} \omega = 0$,
  then there exists some $f$ such that $df = \omega$ (i.e. $\omega$ has a primitive on $R$).
\end{corollary}

\begin{proof}
  The proof is identical to the previous case. Since the region on which we integrate is a disk, it is always possible to find a path between two points which lies on the
  boundary of a rectangle.
  \end{proof}

\noindent We now move on to proving a theorem which will be very useful

\begin{theorem}[Green's theorem for a rectangle]
  Suppose $A$ is a rectangle in the complex plane, suppose $\gamma$ is the curve describing its boundary. Suppose $\omega$ is a $C^1$
  $1$-form, with $\omega = p dx + q dy$. Then we have:
  \begin{equation}
    \displaystyle\int_{\gamma} p dx + q dy = \displaystyle\int_{A} \left( \frac{\partial q}{\partial x} - \frac{\partial p}{\partial y} \right) \ dx \ dy
    \end{equation}
\end{theorem}

\begin{remark}[Stokes' theorem]
  Note that Green's theorem is simply an immediate consequence of Stokes' theorem, which tell us that $\int_{\Omega} d\omega = \int_{\partial \Omega} \omega$, for some region $\Omega$ with a piecewise
  smooth boundary. Nevertheless, it is our goal to prove most results without invoking Stokes' theorem, because it is vastly more powerful than we require for most of our purposes.
  \end{remark}

\begin{lemma}
  Let $\Omega$ be an open set in the complex domain. Let $\omega$ be a $C^{1}$ $1$-form. If $\omega$ is exact in $\Omega$, then $\omega$ is closed.
  Moreover, if $\Omega$ is an open disk and $\omega$ is closed, then $\omega$ is exact.
\end{lemma}

\begin{proof}
  Clearly, if $\omega$ is exact with $\omega = df$, then $d \omega = d^2 f = 0$, so $\omega$ is closed. Another way to see this fact is via Green's theorem: if $\omega$ is exact, then
  the integral of the form around the boundary of a rectangle must be $0$. Let $\omega = p dx + q dy$, let $x \in \Omega$. Let us choose a rectangle $A$ around $x$ contained in $\Omega$. Then
  \begin{equation}
    \displaystyle\int_{A} \left( \frac{\partial q}{\partial x} - \frac{\partial p}{\partial y} \right) \ dx \ dy = 0.
  \end{equation}
  From here, suppose there exists some point in $A$ at which $\frac{\partial q}{\partial x} - \frac{\partial p}{\partial y} > 0$. Because the derivatives are continuous, it follows that there is a neighbourhood
  inside $A$ for which this quantity is positive. Without loss of generality, we can assume this region is a rectangle with boundary $\nu$. But this would then imply that $\int_{\nu} \omega \neq 0$, contradicting
  the exactness of $\omega$. Thus,
  \begin{equation}
    \frac{\partial q}{\partial x} - \frac{\partial p}{\partial y} = 0 \Longrightarrow \frac{\partial q}{\partial x} = \frac{\partial p}{\partial y}
  \end{equation}
  in $A$. This equality then must hold on all of $\Omega$, so it follows by definition that $\omega = p dx + q dy$ is exact, as $d \omega = (\partial_x q - \partial_y p) dx \wedge dy = 0$.
  To prove the partial converse, assume that $\Omega$ is an open disk, and $\omega$ is closed. It then follows that whenever $\gamma$ is the boundary of a rectangle in $\Omega$, $\int_{\gamma} \omega = 0$.
  But then, from Cor.~\ref{cor:rect}, it follows that $\omega$ must be exact.
\end{proof}

\noindent Let us now consider a new type of form, which will be useful to us throughout the rest of our work:

\begin{definition}[Locally exact forms]
  A continuous form $\omega$ on open set $\Omega$ is said to be \emph{locally exact} if, for each $x \in \Omega$, there exists a neighbourhood of $x$ contained in $\Omega$ such
  that in this neighbourhood, there exists $\nu$ with $d\nu = \omega$ (in other words, there is a local primitive).
\end{definition}

\noindent As it turns out, this definition agrees for nicely with some special cases of the results derived in the previous section:

\begin{corollary}
  A continuous form is locally exact at $x \in \Omega$ if and only if, when $\gamma$ is the boundary of a sufficiently small rectangle around $x$,
  $\int_{\gamma} \omega = 0$. In addition, a $C^1$ $1$-form is locally exact on $\Omega$ if and only if it is closed.
\end{corollary}
\begin{proof}
\end{proof}

\noindent To convince ourselves that there exist forms which are locally exact but not exact, let us consider the following example.

\begin{example}
  Let $D$ be $\mathbb{C} - 0$. Let $\omega = dz/z$. This is an example of a form which is locally exact on $D$ but is not exact.
  Let $\gamma$ be the circular path of radius $1$ about the origin, contained in $D$. Note that
  \begin{equation}
    \displaystyle\int_{\gamma} \frac{dz}{z} = 2 \pi i \displaystyle\int_{0}^{1} e^{-2 \pi i \theta} e^{2 \pi i \theta} \ d\theta = 2 \pi i \neq 0
  \end{equation}
  However, let us note that $\omega$ is \emph{locally} exact. Indeed, we know that each branch of $\log(z)$ is defined for some $z \neq 0$.
  Moreover, there will exist a neighbourhood of $z$ in which any branch of $\log(z)$ is continuous: it is not continuous on the whole complex domain
  because the $\text{Arg}(z)$ function is discontinuous along half of the $x$-axis. Such a branch of the logartihm has derivative $1/z$, as we proved, so
  the form is locally exact.
\end{example}

\noindent Let us now continue with our study of locally exact forms. Our goal is to prove certain results relating to integration of locally exact forms, as it turns
out that locally exact forms will be fairly ubiquitous through complex analysis. We have seen that when a form is exact, it is easy to integrate it, via the fundamental theorem
of calculus: we have $\omega = dF$ for some $F$. If $\gamma$ is a curve, we then have
\begin{equation}
  \displaystyle\int_{\gamma} \omega = \displaystyle\int_{a}^{b} \gamma^{*}(dF) = \displaystyle\int_{a}^{b} \frac{d (F \circ \gamma)(t)}{dt} \ dt = f(b) - f(a)
\end{equation}
where $f = F \circ \gamma$.

But what about the case when a form is locally exact? Integrals are local in the sense that they ``sum'' a form over a path by individually accumulating values
at individual points along a curve, so it would make sense to assume that we can also integrate locally exact forms via the fundamental theorem of calculus.

If a form is locally exact, then there exists a small region around each point in which we can find a primitive.

\begin{definition}[Primitive along a curve]
  Let $\gamma : [a, b] \rightarrow \Omega$ be a continuous path contained in connected open set $\Omega$, let $\omega$ be a locally exact differential form on $\Omega$. A continuous function
  $f$ is said to be a primitive of $\omega$ along $\gamma$ if, for each $x \in [a, b]$, there exists a primitive $F_x$ of $\omega$ in a neighbourhood around $\gamma(x)$, such that
  for all $t \in (x - \varepsilon, x + \varepsilon)$, we have $F_x(\gamma(t)) = f(t)$.
\end{definition}

\noindent This definition effectively ``localizes'' the $f$ given above, when we were discussing exact forms.

\begin{theorem}
  \label{thm:loc}
  A locally exact $1$-form always has a primitive along any curve $\gamma$, which we will call $f$. Moreover, this primitive along $\gamma$ is
  unique up to addition of a constant, and $\int_{\gamma} \omega = f(b) - f(a)$
\end{theorem}

\noindent \textbf{Note:} This proof is rather topological, be warned.

\begin{proof}
  To begin, let us first suppose that we have two functions $f_1$ and $f_2$ which are both primitives along $\gamma$ of the form $\omega$.
  It follows that for each $x \in [a, b]$, there is a neighbourhood $V$ of $x$ where $F_1(\gamma(t)) = f_1(t)$ and $F_2(\gamma(t)) = f_2(t)$ for all $t \in V$, where
  $F_1$ and $F_2$ are local primitives of $\omega$ in neighbourhood $U$ around $\gamma(x)$. Thus, the derivatives of $F_1$ and $F_2$ in $U$
  are the same, so they differ by only a constant in $U$. Without loss of generality, we can assume $\gamma(V) \subset U$. Thus, $f_1$ and $f_2$ differ only by a constant on $V$.

  It follows that $f = f_1 - f_2$ is a locally constant, continuous function on a connected domain. Suppose $f$ isn't globally constant, so it takes on
  two or more values, $c_1 < c_2$. Choose $c_1 < c < c_2$. Let $A$ be the open set consisting of all points for which there exists a neighbourhood on which the function is less than $c$,
  let $B$ be the open set consisting of all points for which there is a neighbourhood on which the function is greater than $c$. $A$ and $B$ must be disjoint and their union is the domain, a contradiction, so
  $f$ must be globally constant. It follows that $f_1 - f_2$ is a constant on $[a, b]$, so the primitive along a curve is unique up to a constant.
  \newline

  \noindent Let us now move on to the meat of this proof: demonstrating that a primitive along a curve actually exists. To do this, we proceed as follows. For each $x \in \gamma([a, b])$,
  we pick an open ball $B_x$ inside which $\omega$ has a primitive. Since $\gamma$ is continuous, $\gamma^{-1}(B_x)$ is open for each $B_x$. Moreover, these open sets clearly form an open cover
  of $[a, b]$. Pick a finite subcover. From the Lebesgue number lemma, there exists some $\delta$ such $[x - \delta, x + \delta]$ is contained in some element of the cover. Consider the sequence
  \begin{equation}
    a < a + \delta < a + 2 \delta < \cdots a + k \delta < b
  \end{equation}
  We index this sequence of points by $t_i$, so $t_0 = a$, $t_{k + 1} = b$, and $t_j = a + j \delta$. It follows that any interval $[t_{i}, t_{i + 1}]$ is contained in some $\gamma^{-1}(B_x)$, so that
  $\gamma([t_i, t_{i + 1}])$ is contained in some $B_x$. Let $B_{x_0}, \dots, B_{x_{k + 1}}$ be a finite collection of balls such that $\gamma([t_i, t_{i + 1}]) \subset B_{x_i}$. Clearly,
  these balls will cover $\gamma([a, b])$.
  \newline

  We define our function inductively. Starting with the interval $[t_0, t_1]$, we define $f(x)$ for $x \in [t_0, t_1]$ as $F_{x_0}(\gamma(x))$.
  Suppose $B = B_{x_0} \cap B_{x_1}$. It follows that $F_{x_0}$ and $F_{x_1}$ are both primitives of $\omega$ on $B$, so, using similar justification
  to above, they can only differ by a constant $c$. Thus, we define $f(x) = F_{x_1}(\gamma(x)) - c$ for $x \in [t_1, t_2]$. Thus, by the pasting lemma,
  the function $f$ on $[t_0, t_2]$ is well-defined and continuous.

  We can continue inductively, defining the function for each interval $[t_0, t_k]$, until we have a continuous function on the entire interval which satisfies
  the local primitive property (adding a constant does not change whether the function is a primitive, of course). To conclude, let $\gamma_k : [t_k, t_{k + 1}] \rightarrow \Omega$
  be the domain-restriction of $\gamma$. We have
  \begin{equation}
    \displaystyle\int_{\gamma} \omega = \displaystyle\sum_{k} \displaystyle\int_{\gamma_k} \omega = \displaystyle\sum_{k} \displaystyle\int_{\gamma_k} dF_{x_k} = \displaystyle\sum_{k} \displaystyle\int_{t_k}^{t_{k+1}}
    \frac{ d (F_{x_k} \circ \gamma)(t)}{dt} \ dt = \displaystyle\sum_{k} f(t_{k + 1}) - f(t_{k}) = f(b) - f(a)
    \end{equation}
\end{proof}

\noindent With this result, we can prove the following result, which will allow us to compute integrals of locally exact forms:

\begin{corollary}
  \label{cor:loc_ex}
  If $\omega$ is locally exact, and $\gamma : [a, b] \rightarrow \Omega$ is a piecewise $C^1$ curve in $\Omega$, with $f$ being the primitive of $\omega$ along $\gamma$, then $\int_{\gamma} \omega = f(b) - f(a)$. Note that
  it doesn't matter that $f$ is only unique up to a constant, as this will cancel in the difference $f(b) - f(a)$.
\end{corollary}

\noindent This result is very powerful, as we now have the ability to integrate locally exact forms easily, a much more generic class of differential forms than those which are globally exact.
\newline

\noindent Let us now revisit the example of $\omega = \frac{dz}{z}$. Even though, as we have seen, this form is not globally exact, it is locally exact, and so we can in fact easily integrate it on
any path that we wish. On some open neighbourhood, note that any primitive of $\frac{1}{z}$ will be a branch of $\log(z)$, which differ from each other by additive factors of $2\pi i k$, for $k \in \mathbb{Z}$.
Suppose $\gamma$ is a closed curve, so $\gamma(a) = \gamma(b) = x$.

Then, $f(b) = F(\gamma(b)) = F(x)$ and $f(a) = G(\gamma(a)) = G(x)$ for local primitives $F$ and $G$ of $\log(z)$ must differ by $2 \pi i k$, for some $k \in \mathbb{Z}$. In other words:

\begin{lemma}
  The integral of $\frac{dz}{z}$ around some closed curve $\gamma$ not passing through the origin will always be of the form $2 \pi i k$, for some $k \in \mathbb{Z}$
\end{lemma}

\begin{definition}
  Given a curve $\gamma$ in $\mathbb{C} - \{a\}$ for some point $a$, we define the \emph{winding number} of $\gamma$ around $a$ as
  \begin{equation}
    n(\gamma, a) = \frac{1}{2\pi i} \displaystyle\int_{\gamma} \frac{dz}{z - a}.
  \end{equation}
  As was proved above, this quantity will always be an integer.
  \end{definition}

\subsection{Generalizing: homotopy}

\noindent We can now complete the generalization of our results. So far, we have been considering the integration around circular curves, or the boundaries of rectangles, but as it turns out,
perturbing these curves via \emph{homotopy} will not change any of the fundamental results that we derived in the previous section.

\begin{definition}[Different notions of homotopy]

  \end{definition}

\begin{theorem}
  Let $\Omega$ be a connected open set, if $\omega$ is a locally exact form and $\gamma_1$ and $\gamma_2$ are path-homotopic, then $\int_{\gamma_1} \omega = \int_{\gamma_2} \omega$.
  Similarly, if $\gamma_1$ and $\gamma_2$ are free homotopic as loops, then $\int_{\gamma_1} \omega = \int_{\gamma_2} \omega$ as well.
\end{theorem}

\noindent \textbf{Note.} Of course this is true for exact forms, as the integral around any closed loop will simply be $0$. But for \emph{locally} exact forms, this statement is
no longer a triviality. In addition, we already know from Cor.~\ref{cor:loc_ex} that these integrals will be the same, up to a constant, as both integrals will only depend only
on the value of a primitive along each curve, evaluated at the endpoints. Such primitives will necessarily agree up to constant factor.
\newline

\noindent Both proofs will essentially be identical, so we only really have to prove one case (we will do the former). The main idea here is to use the fact that path homotopy is like a
smooth interpolation between curves. In the definition of a primitive along a path, we defined a primitive in a neighbourhood around each point of $\gamma([a, b])$, for the path
$\gamma$. If we perturb $\gamma$ ever so slightly, to curve $\gamma'$, so that it remains in the union of all such neighbourhoods, then we again have a primitive along $\gamma'$
that is identical to the primitive along $\gamma$. Repeating this procedure, we can transform from $\gamma_1$ to $\gamma_2$ in a finite number of steps (due to the compactness of the interpolation interval,
$[0, 1]$), and we will have shown that the primitives along $\gamma_1$ and $\gamma_2$ are the same.
\newline

\noindent Now, let us jump into the actual proof:

\begin{proof}

  \noindent Let $F : [a, b] \times [0, 1] \rightarrow \Omega$ be a path-homotopy between $\gamma_1$ and $\gamma_2$. Thus, $\gamma_1(x) = F(x, 0)$ and $\gamma_2(x) = F(x, 1)$.
  Note that $F(a, t) = \gamma_1(a) = \gamma_2(a)$ for all $t \in [0, 1]$, and $F(b, t) = \gamma_1(b) = \gamma_2(b)$. Let $B_{t}$ be the union of all $B_{x_k, t}$ as defined in Thm.~\ref{thm:loc},
  for the curve $\gamma(s) = F(s, t)$. Since these balls cover a compact set, we take a finite subcover containing $B_{t_0} = B_{0}$ and $B_{t_n} = B_{1}$. Now, for some $x \in [a, b]$,
  we will have $x \in B_{x_k, t_0}, B_{x_k, t_1}$ for some $x_k$.
  \end{proof}

\begin{theorem}[Winding number]
\end{theorem}

\subsection{Cauchy's theorem}

\noindent We now arrive at the first ``checkpoint'' in our study of integration over a complex domain:

\begin{theorem}[Cauchy's theorem for a rectangle]
  If $f$ is a holomorphic function on rectangle $R$, then
  \begin{equation}
    \displaystyle\int_{R} f(z) \ dz = 0
    \end{equation}
\end{theorem}

\begin{theorem}[Cauchy's theorem for a disk]
\end{theorem}

\begin{theorem}[Generalized Cauchy's theorem]
\end{theorem}

\begin{corollary}[Cauchy's integral formula]
  \end{corollary}

\subsection{The substantial implications of Cauchy's theorem}

\noindent It's very difficult to argue against the claim that Cauchy's theorem is the most important result in complex analysis. Here, we will attempt to justify this claim,
by proving several fundamental mathematical results are corollaries of Cauchy's theorem.

\section{Harmonic functions}

% Must discuss various theorems, Cauchy's inequalities, etc.

\section{Meromorphic functions, Laurent series and essential singularities}

\begin{definition}[Meromorphic function]
  \end{definition}

\end{document}
