\documentclass[aps,pra,showpacs,notitlepage,onecolumn,superscriptaddress,nofootinbib]{revtex4-1}
\usepackage[utf8]{inputenc}
\usepackage[tmargin=1in, bmargin=1.25in, lmargin=1.5in, rmargin=1.5in]{geometry}
\usepackage{amsmath, amssymb, amsthm}
\usepackage{graphicx}
\usepackage{xcolor}
\usepackage{enumitem}
\usepackage{datetime}
\usepackage{hyperref}
\usepackage{titlesec}
\usepackage{import}
\usepackage{mathtools}
\usepackage{thmtools,thm-restate}
\usepackage{comment}


% package for commutative diagrams
% \usepackage{tikz-cd}

%%%%%%%%%%%%%%%%%%%%%%%%%%%%%%%%%%%%%%%%%%%%%
\definecolor{crimson}{RGB}{186,0,44}
\definecolor{moss}{RGB}{0, 186, 111}
\newcommand{\pop}[1]{\textcolor{crimson}{#1}}
\newcommand{\zcom}[1]{\noindent\textcolor{crimson}{(Z): #1}}
\newcommand{\jcom}[1]{\noindent\textcolor{moss}{(J): #1}}
\newcommand{\wt}[1]{\widetilde{#1}}
\newcommand{\pqeq}{\succcurlyeq}
\newcommand{\pleq}{\preccurlyeq}
\newcommand{\Wedge}{\bigwedge}

%%%%%%%%%%%%%%%%%%%%%%%%%%%%%%%%%%%%%%%%%%%%%
\hypersetup{
    colorlinks,
    linkcolor={crimson},
    citecolor={crimson},
    urlcolor={crimson}
}

\usepackage{qcircuit}

%%%%%%%%%%%%%%%%%%%%%%%%%%%%%%%%%%%%%%%%%%%%%
\theoremstyle{definition}
\newtheorem{definition}{Definition}[section]
\newtheorem{lemma}{Lemma}[section]
\newtheorem{theorem}{Theorem}[section]
\newtheorem{corollary}{Corollary}[theorem]
\newtheorem*{theorem*}{Theorem}
\newtheorem*{corollary*}{Corollary}
\newtheorem{remark}{Remark}[section]
\newtheorem{conjecture}{Conjecture}[section]
\newtheorem{example}{Example}[section]
\newtheorem{reminder}{Reminder}[section]
\newtheorem{problem}{Problem}[section]
\newtheorem{question}{Question}[section]
\newtheorem{answer}{Answer}[section]
\newtheorem{fact}{Fact}[section]
\newtheorem{claim}{Claim}[section]
\newtheorem{exercise}{Exercise}[section]

\newcommand{\hhrulefill}{\hspace{-1.5em} \hrulefill}

\usepackage{geometry}
\geometry{
  left=25mm,
  right=25mm,
  top=20mm,
}

%%%%%%%%%%%%%%%%%%%%%%%%%%%%%%%%%%%%%%%%%%%%%
\bibliographystyle{unsrt}

%%%%%%%%%%%%%%%%%%%%%%%%%%%%%%%%%%%%%%%%%%%%%
%%%%%%%%%%%%%%%%%%%%%%%%%%%%%%%%%%%%%%%%%%%%%
%%%%%%%%%%%%%%%%%%%%%%%%%%%%%%%%%%%%%%%%%%%%%
\begin{document}

\title{A brief course in basic complex analysis}
\author{Jack Ceroni}
\email{jackceroni@gmail.com}
\affiliation{Department of Mathematics, University of Toronto}

\date{\today}

\maketitle

\tableofcontents

\section{Introduction}

\noindent The goal of these notes is to follow a very elementary first course in complex analysis, with similar structure to Edward Bierstone's course at the University of Toronto (MAT354), in the Fall of 2023. I will
also be adding information from my own readings/other resources, and modifying/expanding upon certain explanations as I see fit. I hope that someone finds these notes useful. However, in the event that no one does, I certainly will!
(I'm writing these notes as I am studying for the MAT354 exam!)

Here is a list of resources that I will draw from in these notes:

\begin{itemize}
\item Professor Edward Bierstone's lectures (MAT354, University of Toronto, Fall 2023)
\item The MAT354 tutorial worksheets, prepared by Vicente Marin-Marquez
  \item Cartan, \textit{Elementary theory of analytic functions of one or several complex variables}
  \item Ahlfors, \textit{Complex analysis}
\end{itemize}

\section{Basics, the complex plane and the Riemann sphere}

\section{Introduction to complex functions}

\section{Holomorphic functions}

\begin{definition}[Branch]
\end{definition}

\begin{definition}[Branch cut]
  \end{definition}

\section{Power series and analytic functions}

\begin{theorem}
  The zeros of an analytic function for a discrete set. That is, each is contained in a neighbourhood not containing any
  of the others.
\end{theorem}

\begin{proof}
  Let $f$ be an analytic function, let $a$ be a zero of the function. Of course, we can expand $f$ locally as
  $$f(z) = \sum_{k = 0}^{\infty} c_k (z - a)^{k} = (z - a)^{n} \sum_{k = 0}^{\infty} c_{k + n} (z - a)^{k} = (z - a)^{n} g(z)$$
  where we assume that $c_n \neq 0$. In particular, $g(a) \neq 0$. Therefore, since $g$ is continuous, we can choose an open neighbourhood
  around $a$ for which $g(z) \neq 0$. Thus, there is a neighbourhood around $a$ for which $a$ is the only root.
\end{proof}

\begin{theorem}[Analytic continuation]
  Let $f$ be an analytic function in connected open set $D \subset \mathbb{C}$. Let $x_0 \in D$. Then the following conditions are equivalent:
  \begin{itemize}
  \item $f^{(n)}(x_0) = 0$ for all non-negative integers $n$
  \item $f$ is identically $0$ in a neighbourhood of $x_0$
    \item $f$ is identically $0$ on the entire set $D$
    \end{itemize}
\end{theorem}
\begin{proof}

\end{proof}

\noindent This proof has a few immediate corollaries:



\subsection{Interlude: exercises}

\noindent Another exercise interlude:

\hhrulefill

\begin{exercise}[MAT354 6.2]
  Recall that if all derivatives of an analytic function are equal to $0$ anywhere, then the function is identically $0$. Since $f$ is $0$ on $S = D \cap \{\text{Im}(z) = 0\}$,
  it follows immediately that $\partial_x^{(n)} f(0) = 0$. Thus, since $f$ is holomorphic, $\partial_y^{(n)} f(0) = 0$ as well. It follows that $f$ has all of its derivatives
  equal to $0$ at the origin, so it must be identically $0$.
  \newline

  \noindent In the case that $f$ is meromorphic, it can be expressed as a quotient $g/h$ of analytic functions. If it is $0$ on $S$, then $g$ is $0$ on $S$, so
  $g = 0$ on $D$, and $f$ is $0$ on $D$ as well.
  \end{exercise}

\hhrulefill

\section{Conformal maps}

\noindent We conclude the first part of this course with a brief discussion of conformal mappings of the plane.

\subsection{Interlude: exercises}

%%%%%%%%%%%%%%%%%%%%%%%%%%%%%%%%%%%% Part 2: COMPLEX INTEGRATION %%%%%%%%%%%%%%%%%%%%%%%%%%%%%%%

\section{Complex integration}

\noindent We now move into the next major topic in our study of complex analysis: integration over complex domains. The tools that we
develop in this section will prove to be uniquely useful in proving many major results in complex analysis.

\subsection{Basics and notation}

\noindent Recall that a function $f : \mathbb{C} \rightarrow \mathbb{C}$ is, topologically, a map $f : \mathbb{R}^2 \rightarrow \mathbb{R}^2$. Of course, from our studies in standard analysis,
we already know how to integrate differential $1$-forms $\omega = p(x, y) dx + q(x, y) dy$ and $2$-forms $\eta = f(x, y) dx \wedge dy$ over chains in $\mathbb{R}^2$, with $p, q, f \in C(\mathbb{R}^2, \mathbb{R})$.
However, in the complex case, it will often be true that we will want to integrate \emph{complex-valued} forms over complex domains. This leads us to the following, somewhat obvious, definitions:
\begin{align}
  \displaystyle\int_{\gamma} p(x, y) dx + q(x, y) dy \coloneqq \displaystyle\int_{\gamma} \text{Re}[p](x, y) dx + \text{Re}[q](x, y) dy + i \displaystyle\int_{\gamma} \text{Im}[p](x, y) dx + \text{Im}[q](x, y) dy \\
  \displaystyle\int_{\Omega} f(x, y) dx \wedge dy \coloneqq  \displaystyle\int_{\Omega} \text{Re}[f](x, y) dx \wedge dy + i \displaystyle\int_{\Omega} \text{Im}[f](x, y) dx \wedge dy
\end{align}
when $p, q, f \in C(\mathbb{R}^2, \mathbb{C})$. When the component functions of a form are continuous, we call the form continuous. Generally speaking, every form that we consider in this section
will be continuous (at least). When the functions are continuously differentiable, we call
the form continuously differentiable, or a $C^1$-form.
\newline

\noindent We make one more generalization. Generally, when integrating one-forms over paths $\gamma$, we make the implicit assumption that $\gamma$ is continuously differentiable. Thus is required in the standard
definition of the integral $\int_{\gamma} \omega$, as when the integral is pulled back to an integral over an interval, we must take a derivative of $\gamma$. Let us suppose instead that $\gamma$ is piecewise-$C^1$,
meaning that it is characterized by a finite collection of $C^1$ curves $\gamma_k$ which, when appended together, form a continuous path. We define the integral over $\gamma$ as
\begin{equation}
  \displaystyle\int_{\gamma} \omega \coloneqq \displaystyle\sum_{k = 1}^{n} \displaystyle\int_{\gamma_k} \omega.
\end{equation}


\subsection{Closed and exact forms}

\noindent We begin by recalling definitions from standard analysis:
\begin{definition}[Closed and exact forms]
  A continuously differentiable form $\omega$ is \emph{closed} if $d\omega = 0$, where $d$ is the standard exterior derivative, extended linearly from the case of real forms to complex forms in the obvious way.
  A form $\omega$ is called \emph{exact} if there exists a $C^1$-form $\eta$ such that $d\eta = \omega$. In particular, when $\omega$ is a $1$-form, $\omega$ is exact if there exists a $C^1$-function $F$ such that $dF = \omega$.
  We call such a function a \emph{primitive} of $\omega$.
\end{definition}

\noindent We won't deal with closed forms too much in subsequent sections, to the exactness property will prove to be very important. We begin with the following equivalence:
\begin{lemma}
  Let $\omega$ be a $1$-form on a connected open set $\Omega$ in the complex plane. Then $\Omega$ is exact if and only if, for every piecewise-$C^1$ closed path $\gamma$ lying in $\Omega$ ($\gamma$
  has the same start and end points), we have $\int_{\gamma} \omega = 0$.
\end{lemma}

\begin{corollary}
  Let $\omega$ be a $1$-form on a disk $D$ in the complex plane. Then $\Omega$ is exact if and only if, for every piecewise-$C^1$ closed path $\eta$ lying on the boundary of a rectangle in $D$, we have $\int_{\eta} \omega = 0$.
\end{corollary}

\begin{comment}
\begin{lemma}[Green-Riemann formula]
  Let $\omega = p dx + q dy$ be a $C^1$-form on connected open set $\Omega$. Then, if $R$ is a rectangle lying in $\Omega$, and $\gamma$ is the curve describing its boundary:
  \begin{equation}
    \displaystyle\int_{\gamma} p(x, y) dx + q(x, y) dy = \displaystyle\int_{R} \left( \frac{\partial p}{\partial y} - \frac{\partial q}{\partial x} \right) dx \wedge dy
    \end{equation}
\end{lemma}
\begin{proof}
  This is an immediate consequence of Stokes' theorem.
  \newline

  \noindent Of course, we can also prove this fact without Stokes' theorem.
\end{proof}
\end{comment}

\noindent This leads us to the following result:
\begin{lemma}
  If $\omega$ defined on open connected $\Omega$ is an exact $C^1$-form, then it is closed. If $\Omega$ is an open disc, and $\omega$ is closed, then it is exact.
  \end{lemma}

\subsection{Locally exact forms}

\noindent Exact forms have useful properties, as we have seen, but exactness if a fairly strong condition. As it turns out, if we weaken the exactness condition such as to make it \emph{local} rather than global,
we achieve a much more broadly-useful class of forms, which still have some pretty nice properties.

\begin{definition}
\end{definition}

\begin{remark}
  Some authors, such as Cartan, will call locally exact forms \emph{closed forms}. As we will see below, this is because in the case of $C^1$-forms, the definitions of equvalent. However,
  for continuous forms which are not $C^1$, the definitions are different (the closed property isn't defined).
\end{remark}

\begin{lemma}
  A form $\omega$ is locally exact at $x$ if and only if, for a sufficiently small rectangular boundary around $x$, $\gamma$, we have $\int_{\gamma} \omega = 0$. In addition,
  a $C^1$-form is locally exact if and only if it is closed.
\end{lemma}
\begin{proof}
\end{proof}

\subsection{Integrating locally exact forms}

\noindent We now move on to the main technical lemma of this section, which will prove invaluable for our study of integration of locally exact forms.
At a high-level, this result will tell us that given two paths $\gamma_1$ and $\gamma_2$ which are homotopic, we can always choose primitives along
intermediate paths of the homotopy which vary continuously.

In the case of exact $1$-forms, we know that we can always find a global primitive. Thus, it is easy to integrate exact forms, via the fundamental theorem
of calculus. In particular, we have $\omega = dF$ for some $F$. If $\gamma$ is a $C^1$ curve, we then have
\begin{equation}
  \displaystyle\int_{\gamma} \omega = \displaystyle\int_{a}^{b} \gamma^{*}(dF) = \displaystyle\int_{a}^{b} \frac{d (F \circ \gamma)(t)}{dt} \ dt = f(b) - f(a)
\end{equation}
where $f = F \circ \gamma$.

But what about the case when a form is locally exact? Integrals are local in the sense that they ``sum'' a form over a path by individually accumulating values
at individual points along a curve, so it would make sense to assume that we can also integrate locally exact forms via the fundamental theorem of calculus.
We cannot find a global primitive, but, as we will see, it suffices to find local primitives which vary continuously, and sum over them.


\begin{definition}[Primitive along a path and primitive on a region]
  Let $\gamma : [a, b] \rightarrow \Omega$ be a continuous path, let $\omega$ be locally exact. We say that a function $f : [a, b] \rightarrow \mathbb{C}$ is a primitive along the path $\gamma$
  if, for each point on the curve, $\gamma(t)$, there exists a function $F$ such that in a neighbourhood of $\gamma(t)$, $F$ is a primitive of $\omega$, and for all $x$ in
  some neighbourhood of $t$, $f(x) = (F \circ \gamma)(x)$.

  
\end{definition}

\noindent This definition is effectively a ``local'' version of the $f$ in the previous paragraph, which was the composition of the global primitive with the path.

\begin{remark}
  Note that the definition of a primitive along a path is a special case of the primitive on a region, where the rectangle from which we map
  has one of its side-lengths equal to $0$.
\end{remark}

\hhrulefill

\begin{lemma}[The general primitive lemma]
  If $\omega$ is locally exact, and $\delta : [a, b] \times [a', b'] \rightarrow \Omega$ is a continuous mapping of the rectangle $R = [a, b] \times [a', b']$ into
  $\Omega$, then there always exists a primitive $f(x, y)$ on the region. Moreover, it is unique up to addition of a constant.

  Finally, if $S = \{F_1, F_2, \dots\}$ is a set of functions
  such that for any $x \in \Omega$, there exists $F_k \in S$ which is a local primitive of $\omega$ near $x$, and for any $i, j, k$, $F_i + (F_j - F_k) \in S$, then $f(x, y) = F_k(\gamma(x, y))$
  for some $k$, for each point $(x, y) \in R$.
\end{lemma}

\begin{proof}
  Clearly, $R$ is compact, so its image under the continuous function $\delta$ is compact. Let us choose an open cover $\{A_k\}$ of $\delta(R)$, its inverse image
  is an open cover of $R$. It follows from the Lebesgue number lemma that there exists $\varepsilon$ sufficiently small such that any rectangle with side lengths
  less than or equal to $\varepsilon$ is contained in an element of $\{\delta^{-1}(A_k)\}$.
  \end{proof}

\hhrulefill

\subsection{An illustrative example: winding number}

\noindent We dedicate a section of these notes to explaining a particular useful example of the integration of a locally exact form, $\omega = dz/z$.

\begin{lemma}[Winding number for $C^1$-curves]
  If $\gamma : [0, 1] \rightarrow \Omega$ is a $C^1$ curve which does not pass through point $a$, then the integral
  \begin{equation}
    w(\gamma, a) = \frac{1}{2\pi i} \displaystyle\int_{\gamma} \frac{dz}{z - a}
  \end{equation}
  is an integer. We call $w(\gamma, a)$ the winding number of $\gamma$ about $a$.
\end{lemma}

\begin{proof}
  Of course, we have
  \begin{equation}
    \displaystyle\int_{\gamma} \frac{dz}{z - a} = \displaystyle\int_{0}^{1} \frac{\gamma'(t)}{\gamma(t) - a} \ dt
    \end{equation}
Let us define $f(s) = \int_{0}^{s} \frac{\gamma'(t)}{\gamma(t) - a} \ dt$. We know that such a function is differentiable, with
\begin{equation}
  f'(s) = \frac{\gamma'(s)}{\gamma(s) - a} \Longrightarrow (\gamma(s) - a) f'(s) = \gamma'(s)
\end{equation}
Let $F(s) = e^{-f(s)} (\gamma(s) - a)$. From above, $F'(s) = 0$, so $F$ is a constant. In particular, $F(s) = F(0) = \gamma(0) - a$. Thus,
$e^{f(s)} = \frac{\gamma(s) - a}{\gamma(0) - a}$. We then have $e^{f(1)} = 1$, as $\gamma(0) = \gamma(1)$. This implies that $f$ must be an integer
multiple of $2\pi i$. Therefore, $w(\gamma, a)$ is an integer.
\end{proof}

\noindent We now prove the general version of the winding number lemma, where we only require that $\gamma$ is continuous.

\begin{lemma}[Adult winding-number lemma]
  If $\gamma : [0, 1] \rightarrow \Omega$ is a continuous curve which does not pass through point $a$, then the integral
  \begin{equation}
    w(\gamma, a) = \frac{1}{2\pi i} \displaystyle\int_{\gamma} \frac{dz}{z - a}
  \end{equation}
  is an integer. We call $w(\gamma, a)$ the winding number of $\gamma$ about $a$.
  \end{lemma}

\subsection{Cauchy's theorem and its implications}

\noindent The goal of this section is to prove the most broadly useful theorem in the study of complex integration: Cauchy's theorem. We will use this result to prove
a plethora of further results, including incredibly deep statements such as the equivalence between holomorphic and analytic functions, and by extension Liouville's theorem
and the fundamental theorem of algebra.
\newline

\noindent Let us begin by proving a baby version of the theorem:

\begin{theorem}[$C^1$-Cauchy's theorem]
  Recall that $dz \coloneqq dx + i dy$. If $f$ is a holomorphic function with continuous complex derivative, then the form $\omega = f(z) dz$ is closed.
\end{theorem}
\begin{proof}
  We have
  \begin{equation}
    d\omega = \left( \frac{\partial f}{\partial y} - i \frac{\partial f}{\partial y} \right) dx \wedge dy = 0
  \end{equation}
  where we simply used the characteristic identity of holomorphic functions, $\partial_x f = i \partial_y f$.
\end{proof}

\noindent We will see, very soon, that holomorphic functions are smooth, so this version of the proof actually does hold for any holomorphic $f$. However, it is the following proof
that we will \emph{use} to show that holomorphic functions are analytic, and thus smooth!

\begin{theorem}[Cauchy's theorem]
  If $f$ is a holomorphic function, then the form $\omega = f(z) dz$ is locally exact.
\end{theorem}
\begin{proof}
  For this proof, we must take a more direct approach.
\end{proof}

\begin{theorem}[The equivalence theorem]
  Let $f$ be a function defined on connected open set $\Omega$. The following properties are equivalent:
  \begin{enumerate}
  \item $f$ is holomorphic in $\Omega$.
  \item The form $\omega = f(z) dz$ is locally exact in $\Omega$.
  \item Cauchy's integral formula holds. That is,
    \begin{equation}
      f(z) = \displaystyle\int_{\gamma} \frac{f(t)}{z - t} dz
    \end{equation}
    for any closed piecewise $C^1$ path around $z$ in $\Omega$.
    \item $f$ is analytic on $\Omega$.
    \end{enumerate}
\end{theorem}

\subsection{Interlude: exercises}

\noindent We take a small break from presenting new material to solve a few problems from Cartan, Ahlfors, and the MAT354 tutorial worksheets. Note that we will use a few concepts
introduced in subsequent sections in these problems (i.e. the definition of a meromorphic function).

\hhrulefill

\begin{exercise}[Cartan 2.2]
  Recall the definition of an integral of a locally exact form over a continuous path $\gamma : [a, b] \rightarrow \mathbb{C}$: it is the difference $f(b) - f(a)$, where
  $f$ is the primitive along $\gamma$ (unique up to a constant).
  \newline

  \noindent Of course, if $\omega_1$ and $\omega_2$ are locally exact, then their sum $\omega_1 + \omega_2$ is as well: given $x \in \Omega$, let $F_1$ and $F_2$ be local primitives
  of $\omega_1$ and $\omega_2$ on $U_1$ and $U_2$ respectively. Then $F_1 + F_2$ is a local primitive on $U_1 \cap U_2$, an open neighbourhood of $x$. Moreover, let $f_1$
  be the primitive along $\gamma$ of $\omega_1$, let $f_2$ be the primitive along $\gamma$ of $\omega_2$. Let $f = f_1 + f_2$. By definition, given $t \in [a, b]$, we can
  choose $x \in (t - \varepsilon, t + \varepsilon)$.
\end{exercise}

\hhrulefill

\begin{exercise}[Cartan 2.3]
  Recall that if a sequence of continuous functions $f_n$ converges uniformly to $f$ (which is continuous via uniform limit theorem), then
  \begin{equation}
    \lim_{n \to \infty} \displaystyle\int_{I} f_n = \displaystyle\int_{I} f
    \end{equation}
\end{exercise}

\hhrulefill

\begin{exercise}[MAT354 8.1 and 8.2]

  Clearly, if $\gamma$ is piecewise-$C^1$ with $C^1$-components $\gamma_k$, then $f \circ \gamma$ is piecewise-$C^1$ with $C^1$-components $f \circ \gamma_k$, so this construction is well-defined. By definition, we have
  \begin{align}
    \alpha(f) & = w(f \circ \sigma, 0) = \frac{1}{2\pi i} \displaystyle\int_{f \circ \sigma} \frac{dz}{z} = \frac{1}{2\pi i} \displaystyle\int_{0}^{1} (f \circ \sigma)^{*}(z^{-1}) d (f \circ \sigma)^{*}(z)
    \\ & = \frac{1}{2\pi i} \displaystyle\int_{0}^{1} \sigma^{*}(f(z)^{-1}) d \sigma^{*}(f(z)) = \frac{1}{2\pi i} \displaystyle\int_{\sigma} \frac{df}{f} = \frac{1}{2\pi i} \displaystyle\int_{\sigma} \frac{f'}{f} dz
  \end{align}
  This immediately implies that
  \begin{equation}
    \alpha(f g) = \frac{1}{2\pi i} \displaystyle\int_{\sigma} \frac{(fg)'}{fg} dz = \frac{1}{2\pi i} \displaystyle\int_{\sigma} \frac{(f'g + f g'}{fg} dz = \frac{1}{2\pi i} \displaystyle\int_{\sigma} \frac{f'}{f} dz + \frac{1}{2\pi i} \displaystyle\int_{\sigma} \frac{g'}{g} dz = \alpha(f) + \alpha(g).
  \end{equation}
  Clearly, for $c \neq 0$, $\alpha(c) = 0$. Thus, for $f$ non-zero on $S^1$, we have
  \begin{equation}
    0 = \alpha(1) = \alpha\left( \frac{f}{f} \right) = \alpha(f) + \alpha\left(\frac{1}{f} \right) \Longrightarrow \alpha\left( \frac{1}{f} \right) = -\alpha(f).
  \end{equation}
\end{exercise}

\hhrulefill

\begin{exercise}[MAT354 8.3]
  Note that $f(z) = 2z^3 - 5z^2 + 2z = z (z - 2)(2z - 1)$. It follows that
  $$\alpha(f) = \alpha(z) + \alpha(z - 2) + \alpha(2z - 1) = \alpha(z) + \alpha(z - 2) + \frac{1}{2} \alpha\left( z - \frac{1}{2} \right).$$
  Of course, for a function of the form $p(z) = z - c$, we have $\alpha(p) = w(\sigma, c)$ (from Problem 8.2 above). $\sigma$ is a circular path
  around the origin of unit radius. Thus, if $c$ lies in the unit disk, then $w(\sigma, c) = 1$ and otherwise, $w(\sigma, c) = 0$. Therefore,
  \begin{equation}
    \alpha(z) + \alpha(z - 2) + \frac{1}{2} \alpha\left( z - \frac{1}{2} \right) = 1 + 0 + \frac{1}{2} = \frac{3}{2}.
  \end{equation}
  We can compure $\alpha(4z^2 + 1)$ in an identical fashion. In particular, note that $f(z) = 4z^2 + 1 = (2z + i)(2z - i)$. Thus,
  \begin{equation}
    \alpha(f) = \alpha(2z + i) + \alpha(2z - i) = \frac{1}{2} \alpha\left(z + \frac{i}{2} \right) + \frac{1}{2} \alpha \left( z - \frac{i}{2} \right) = \frac{1}{2} + \frac{1}{2} = 1.
  \end{equation}
  It follows that $\alpha(1/f) = -\alpha(f) = -1$. Note that $f$ is non-zero on $S^1$, so we can apply this identity.
\end{exercise}

\hhrulefill

\begin{exercise}[MAT354 8.4]
  Recall that if a form is locally exact on $\Omega$, and $\gamma_1$ and $\gamma_2$ two free homotopic loops in $\Omega$, then $\int_{\gamma_1} \omega = \int_{\gamma_2} \omega$.
  Since $f$ is non-zero on $\overline{D}$, $f'/f$ is holomorphic on $\overline{D}$. Clearly, $\sigma$ is free homotopic to the constant zero-path. By Cauchy's theorem, the form $f' dz / f$ is locally exact.
  Therefore,
  \begin{equation}
    \alpha(f) = \frac{1}{2\pi i} \displaystyle\int_{\gamma} \frac{f'}{f} dz = 0.
    \end{equation}
  \end{exercise}

\hhrulefill

\begin{exercise}[MAT354 8.5]
  Without loss of generality, we can assume that we never have $\nu_k = \mu_j$. Define the rational function
  \begin{equation}
    R(z) = \frac{\prod_{i = 1}^{s} (z - \mu_i)^{m_i}}{\prod_{j = 1}^{r} (z - \nu_i)^{n_i}}
  \end{equation}
  and let $g(z) = R(z) f(z)$. Clearly, this function is meromorphic, as it is the product of meromorphic functions. Moreover, for a given root $\nu_k$, note that
  we can write $f(z) = (z - \nu_k)^{n_k} f_k(z)$ where $f_k(z)$ does not have a root nor a pole at $\nu_k$. Thus,
  \begin{equation}
    g(z) = \frac{\prod_{i = 1}^{s} (z - \mu_i)^{m_i}}{\prod_{j = 1}^{r} (z - \nu_j)^{n_j}} (z - \nu_k)^{n_k} f_k(z) = \frac{\prod_{i = 1}^{s} (z - \mu_i)^{m_i}}{\prod_{j \neq k} (z - \nu_j)^{n_j}} f_k(z)
  \end{equation}
  Clearly, the right-hand side is a meromorphic function with no root and no pole at $\nu_k$. We can repeat this process for all $\nu_k$, and similarly for all $\mu_k$. Thus, $g$
  does not have roots or poles at any of the points $\{\nu_k\}$ or $\{\mu_k\}$. But of course, the only possible locations of roots and poles within the disk for $g(z) = R(z) f(z)$ are at these points,
  so we can conclude that $g$ has no roots or poles in the disk. It follows that
  \begin{equation}
    f(z) = \frac{\prod_{j = 1}^{r} (z - \nu_j)^{n_j}}{\prod_{i = 1}^{s} (z - \mu_i)^{m_i}} g(z)
  \end{equation}
  and, from the previous problems above, we have
  \begin{equation}
    \alpha(f) = \alpha \left( \frac{\prod_{j = 1}^{r} (z - \nu_j)^{n_j}}{\prod_{i = 1}^{s} (z - \mu_i)^{m_i}} g(z) \right) = \displaystyle\sum_{j} n_j \alpha\left((z - \nu_j)\right) - \displaystyle\sum_{j} m_j \alpha\left((z - \mu_j) \right) + \alpha(g)
  \end{equation}
  where we use $\alpha(fg) = \alpha(f) + \alpha(g)$ and $\alpha(1/f) = -\alpha(f)$. Since each $\nu_j$ and each $\mu_j$ lies in the unit disk, each of the $\alpha(\cdot)$ expression above evaluates to $1$, except for $\alpha(g)$, which is $0$ from 8.4.
  Thus,
  \begin{equation}
    \alpha(f) = \displaystyle\sum_{j} n_j - \displaystyle\sum_{j} m_j
  \end{equation}
  as desired.
  \end{exercise}

\hhrulefill

\begin{exercise}[MAT354 8.6 and 8.7]
  We will once again use the equivalence of integrals around homotopic paths. In particular, we will demonstrate that the homotopy $F_t(x) = f(x) + t g(x)$ never crosses
  the origin. Indeed, for some $t \in [0, 1]$, note that for any $x \in S^1$,
  $$|F_t(x)| \geq ||f(x)| - t |g(x)|| \geq ||f(x)| - |g(x)||.$$
  We know that $|f(x)| \geq m$, while $|g(x)| < m$, immediately implying that $|F_t(x)| > 0$. Thus, it is in fact true that the homotopy never crosses the origin. It follows
  immediately that the path $t \mapsto f(e^{2\pi i t})$ and $t \mapsto (f + g)(e^{2 \pi i t})$ are free-homotopic, so
  \begin{equation}
    \alpha(f) = w(f \circ \sigma, 0) = \frac{1}{2\pi i} \displaystyle\int_{f \circ \sigma} \frac{dz}{z} = \frac{1}{2\pi i} \displaystyle\int_{(f + g) \circ \sigma} \frac{dz}{z} = w((f + g) \circ \sigma, 0) = \alpha(f + g).
  \end{equation}
  This completes the proof.
  \newline

  \noindent Now, note that $h(z) = 7z^5 - 2e^{z}$ is holomorphic (and thus meromorphic with no poles). Clearly, $|7 z^5| = 7 |z|^5 = 7$ for $z \in S^1$, while $|-2e^{z}| \leq 2 e^{|z|} = 2e < 6$ for $z \in S^1$.
  It follows from the previous problems that the number of roots of $h$ is $\alpha(7 z^{5}) = 5 \alpha(z) = 5$.
\end{exercise}

\hhrulefill

\begin{exercise}[MAT354 8.8]
  Obviously, we have for $|z| = R$
  \begin{equation}
    \frac{|g(z)|}{R^n} \leq \displaystyle\sum_{k = 0}^{n - 1} |g_k| \frac{|z|^k}{R^n} = \displaystyle\sum_{k = 0}^{n - 1} |g_k| R^{k - n} \rightarrow 0 \ \ \ \ \text{as} \ R \to \infty
  \end{equation}
  So for large enough $R$, $|g(z)| < R^n$. It follows that $f(Rz) = (Rz)^n + g(Rz)$ satisfies the conditions to apply Problem 8.6 for sufficiently large $R$,
  so $\alpha(f(Rz)) = \alpha((Rz)^n) = n$. Since $f$ has no roots as it is a polynomial, this implies that the number of zeros is $n \geq 1$, so $f(Rz)$ has a root in the unit disk.
\end{exercise}

\hhrulefill

\subsection{The Cauchy inequalities and their implications}

\subsection{The mean value property and its implications}

%%%%%%%%%%%%%%%%%%%%%%%%%%%%%%%%%%%%%%%%%%%%%%%%%%%%%%%%%%%%%%%%%%%%%%%%%%%%%%%%%%%%%%%%%%

\section{Meromorphic functions and Laurent expansions}

\begin{definition}[Meromorphic function]
  A function $f$ taking values in the complex plane is said to be meromorphic on $\Omega$ if there exists a set of isolated points, $\{p_k\}$, such that $f$
  is well-defined and holomorphic on $\Omega - \{p_k\}$.
\end{definition}

\begin{theorem}
  If $f$ is meromorphic, then it can be expressed as the quotient of two entire functions.
\end{theorem}
\begin{proof}
  Let $f$ be meromorphic, let
\end{proof}

\begin{remark}[Functions of the Riemann sphere]
\end{remark}

\noindent This set of definitions allows us to prove theorems of the following form:

\begin{theorem}
  If an entire function has a pole at $\infty$, then it is a polynomial.
  \end{theorem}

\begin{theorem}
  If a function $f$ is meromorphic on $S^2$, then it is rational.
\end{theorem}
\begin{proof}
  \noindent Recall that $f$ is meromorphic on $S^2$ if and only if it is meromorphic on the plane, and $f(1/z)$ is meromorphic around the origin.
  First of all, $f(1/z)$ being meromorphic at the origin means that there exists some $M$ such that $z^M f(1/z)$ can be extended to an analytic function
  at the origin. In particular, this function is bounded in a neighbourhood of the origin, so we have for $|z| \leq R$,
  \begin{equation}
    \left| z^M f\left( \frac{1}{z} \right) \right| \leq C \Longrightarrow |f(z)| \leq C |z|^M \ \ \ \ \text{for} \ |z| \geq R.
  \end{equation}
  Now, let us prove that $f$ has a finite number of poles. Indeed, suppose not, so $\{\alpha_k\}$ is an infinite sequence of poles.
  Note that the set of $1/\alpha_k$ are poles of $f(1/z)$, which is meromorphic. If the sequence of unbounded, then $f(1/z)$ will have
  infinitely many poles arbitrarily close to $0$. Thus, $f(1/z)$ cannot be analytic at $0$, so it must have a pole here, but then this pole won't be isolated, a contradiction.
  Similarly, if the $\alpha_k$ are bounded, then they have a convergent subsequence. This point will necessarily be a pole, but also can't be isolated, a contradiction.

  Thus, $f$ must have a finite number of poles $\alpha_1, \dots, \alpha_n$. Let
  \begin{equation}
    g(z) = \displaystyle\prod_{j} (z - \alpha_j)^{m_j} \right) f(z)
  \end{equation}
  which is analytic at all points in the plane. For $|z|$ large enough, we have
  \begin{equation}
    |g(z)| \leq C |z|^M  \displaystyle\prod_{j} |z - \alpha_j|^{m_j}
  \end{equation}
  which implies from the Cauchy inequalities that $g$ must be a polynomial. Thus, $f$ is rational, as desired.
  \end{proof}
\subsection{Interlude: exercises}

\noindent It is time for another exercise interlude.

\begin{exercise}
  \end{exercise}

\end{document}
