\documentclass[aps,pra,showpacs,notitlepage,onecolumn,superscriptaddress,nofootinbib]{revtex4-1}
\usepackage[utf8]{inputenc}
\usepackage[tmargin=1in, bmargin=1.25in, lmargin=1.5in, rmargin=1.5in]{geometry}
\usepackage{amsmath, amssymb, amsthm}
\usepackage{graphicx}
\usepackage{xcolor}
\usepackage{enumitem}
\usepackage{datetime}
\usepackage{hyperref}
\usepackage{titlesec}
\usepackage{import}
\usepackage{mathtools}
\usepackage{thmtools,thm-restate}
\usepackage{comment}


% package for commutative diagrams
% \usepackage{tikz-cd}

%%%%%%%%%%%%%%%%%%%%%%%%%%%%%%%%%%%%%%%%%%%%%
\definecolor{crimson}{RGB}{186,0,44}
\definecolor{moss}{RGB}{0, 186, 111}
\newcommand{\pop}[1]{\textcolor{crimson}{#1}}
\newcommand{\zcom}[1]{\noindent\textcolor{crimson}{(Z): #1}}
\newcommand{\jcom}[1]{\noindent\textcolor{moss}{(J): #1}}
\newcommand{\wt}[1]{\widetilde{#1}}
\newcommand{\pqeq}{\succcurlyeq}
\newcommand{\pleq}{\preccurlyeq}
\newcommand{\Wedge}{\bigwedge}

%%%%%%%%%%%%%%%%%%%%%%%%%%%%%%%%%%%%%%%%%%%%%
\hypersetup{
    colorlinks,
    linkcolor={crimson},
    citecolor={crimson},
    urlcolor={crimson}
}

\usepackage{qcircuit}

%%%%%%%%%%%%%%%%%%%%%%%%%%%%%%%%%%%%%%%%%%%%%
\theoremstyle{definition}
\newtheorem{definition}{Definition}[section]
\newtheorem{lemma}{Lemma}[section]
\newtheorem{theorem}{Theorem}[section]
\newtheorem{corollary}{Corollary}[theorem]
\newtheorem*{theorem*}{Theorem}
\newtheorem*{corollary*}{Corollary}
\newtheorem{remark}{Remark}[section]
\newtheorem{conjecture}{Conjecture}[section]
\newtheorem{example}{Example}[section]
\newtheorem{prop}{Proposition}[section]
\newtheorem{proposition}{Proposition}[section]
\newtheorem{reminder}{Reminder}[section]
\newtheorem{problem}{Problem}[section]
\newtheorem{question}{Question}[section]
\newtheorem{answer}{Answer}[section]
\newtheorem{fact}{Fact}[section]
\newtheorem{claim}{Claim}[section]
\newtheorem{exercise}{Exercise}[section]

\newcommand{\hhrulefill}{\hspace{-1.5em} \hrulefill}

\usepackage{geometry}
\geometry{
  left=25mm,
  right=25mm,
  top=20mm,
}

%%%%%%%%%%%%%%%%%%%%%%%%%%%%%%%%%%%%%%%%%%%%%
\bibliographystyle{unsrt}

%%%%%%%%%%%%%%%%%%%%%%%%%%%%%%%%%%%%%%%%%%%%%
%%%%%%%%%%%%%%%%%%%%%%%%%%%%%%%%%%%%%%%%%%%%%
%%%%%%%%%%%%%%%%%%%%%%%%%%%%%%%%%%%%%%%%%%%%%
\begin{document}

\title{Complex analysis exam prep}
\author{Jack Ceroni}
\email{jackceroni@gmail.com}
\affiliation{Department of Mathematics, University of Toronto}

\date{\today}

\maketitle

\tableofcontents

\section{Introduction}

\noindent Solutions to problems from Cartan, Ahlfors, and more.

\section{Cartan}

\hhrulefill

\begin{problem}[Problem 1.7]
  Let $S(X) = \sum_{n \geq 0} s_n X^n$ be an infinite series. We know that $S$ converges absolutely if and only if the absolute partial sums satisfy $\sum_{0 \leq n \leq k} |s_n| |X|^n \leq M$
  for some fixed $M$.
  Let $S_k(X)$ denote the $k$-th partial sum of the series $S(X)$. If $S$ is a series, let $\widetilde{S}$ denote the absolute sum. It is clear that
  \begin{equation}
    \widetilde{U}_k(X) \leq \widetilde{S}_k(X) + |X| \widetilde{S}_{k - 1}(X) + \cdots + |X|^{k} \widetilde{S}_0(X)
  \end{equation}
  If we choose $|X| < 1$, each $\widetilde{S}_k(X)$ will converge and be bounded above by some $M$, and we will find that the absolute series $\widetilde{U}_k$
  is upper bounded by a geometric series, which converges. Thus, for $|X| < 1$, the series $U$ converges absolutely. Suppose $U$ converges absolutely for some $|X| > 1$.
  Then the absolute partials $\widetilde{U}_k(X) - |X| \widetilde{U}_{k - 1}(X) = \widetilde{S}_k(X)$ converge, so $S$ also converges absolutely for this $X$, a contradiction.
  Thus, the radius of convergence must be $1$.

  To prove that the second sequence converges with the same radius, note that $\widetilde{T}_k(X) \leq \widetilde{P}_k(X)$ (where $P$ is the sum obtained from $U$ via the same transformation
  which takes $S$ to $U$), so by direct comparison, the radius of convergence is
  at least $1$, as $P$ will have radius of convergence equal to $1$, from the above result.

  Now, suppose $|X| > 1$. We can choose $n$ large enough so that $|X|^n > n + 1$, so that for $m \geq n$,
  \begin{equation}
    |t_m| |X|^{m} = \frac{|X|^n}{n + 1} |s_0 + \cdots + s_m| |X|^{m - n} \geq |s_0 + \cdots + s_m| |X|^{m - n}
    \end{equation}
  Since $T$ converges absolutely, it follows via comparison that the sequence of $|X|^{-n} \widetilde{P}_k(X)$ converges absolutely, which would then imply
  that $P$ converges absolutely for some $|X| > 1$, which can't be, as $P$ has radius of convergence $1$. Thus, $T$ must have radius of convergence $1$.
  \newline

  To arrive at the second conclusion, note that $a_n = s_n - s_{n - 1}$ for $n \geq 1$ and $a_0 = s_0$. Thus, for $|z|$ inside the radius of convergence,
  \begin{equation}
    \displaystyle\sum_{n \geq 0} a_n z^n = \displaystyle\sum_{n \geq 0} s_n z^n - z \displaystyle\sum_{n \geq 0} s_{n} z^{n} = (1 - z) \displaystyle\sum_{n \geq 0} s_n z^n
  \end{equation}
  where we are able to move terms and re-sum due to aboslute convergence. This implies the desired result.
  \newline

  \noindent Upon further reflection, another way to arrive at the radius of convergence for the second power series is by using the fact that derivatives
  of power series will have the same radius of convergence.
\end{problem}

\hhrulefill

\begin{problem}[Problem 1.16]
  The first claim is trivial. To show that the sequence converges we will show that is Cauchy. Indeed,
  \begin{equation}
    \left| S_m(x) - S_n(x) \right| = |a_m x^{m} + a_{m - 1} x^{m - 1} + \cdots + a_{n + 1} x^{n + 1}| \leq |a_m + \cdots + a_{n + 1}| x^{n + 1} \leq |a_m + \cdots + a_{n + 1}|
  \end{equation}
  Of course, we can make $|a_m + \cdots + a_{n + 1}|$ small as $\sum_{k} a_k$ converges, \emph{over all $x \in [0, 1]$}. Thus, $S_n$ is Cauchy, so it converges to some $S(x)$.
  To show that this convergence is uniform, note that
  \begin{equation}
    |S(x) - S_n(x)| = \lim_{m \to \infty} |S_m(x) - S_n(x)|
  \end{equation}
  which can be made arbitrarily small over all $x \in [0, 1]$ for $m, n$ large enough, so we can in fact find $n$ such that $S$ and $S_n$ become $\varepsilon$-close over
  all $x \in [0, 1]$. Thus, the convergence is uniform.
  \newline

  \noindent To prove the second part of this statement, note that
\end{problem}

\hhrulefill

\subsection{Locally exact forms}

\noindent I need to dedicate some time to talking about locally exact forms. In particular, the fact that we can always choose a primitive along a curve corresponding to a locally exact form. To be more specific,
we have the following definition:

\begin{definition}[Primitive along a curve]
  Given a curve $\gamma(t)$ and $1$-form $\omega$, we say that $f$ is a primitive of $\omega$ along the curve $\gamma$ if for each $\gamma(t)$ in the curve,
  there is a neighbourhood around $\gamma(t)$ and a function $F$ which is a primitive of $\omega$ inside this neighbourhood, such that $f(s) = F(\gamma(s))$
  for all $s$ in some neighbourhood around $t$.
\end{definition}

\begin{defintion}

  \end{definition}

\begin{theorem}
  \end{theorem}

\section{Ahlfors}

\end{document}
