\documentclass[aps,pra,showpacs,notitlepage,onecolumn,superscriptaddress,nofootinbib]{revtex4-1}
\usepackage[utf8]{inputenc}
\usepackage[tmargin=1in, bmargin=1.25in, lmargin=1.5in, rmargin=1.5in]{geometry}
\usepackage{amsmath, amssymb, amsthm}
\usepackage{graphicx}
\usepackage{xcolor}
\usepackage{enumitem}
\usepackage{datetime}
\usepackage{hyperref}
\usepackage{titlesec}
\usepackage{import}
\usepackage{mathtools}
\usepackage{thmtools,thm-restate}
\usepackage{comment}


% package for commutative diagrams
% \usepackage{tikz-cd}

%%%%%%%%%%%%%%%%%%%%%%%%%%%%%%%%%%%%%%%%%%%%%
\definecolor{crimson}{RGB}{186,0,44}
\definecolor{moss}{RGB}{0, 186, 111}
\newcommand{\pop}[1]{\textcolor{crimson}{#1}}
\newcommand{\zcom}[1]{\noindent\textcolor{crimson}{(Z): #1}}
\newcommand{\jcom}[1]{\noindent\textcolor{moss}{(J): #1}}
\newcommand{\wt}[1]{\widetilde{#1}}
\newcommand{\pqeq}{\succcurlyeq}
\newcommand{\pleq}{\preccurlyeq}
\newcommand{\hhrulefill}{\hspace{-1em} \hrulefill}

%%%%%%%%%%%%%%%%%%%%%%%%%%%%%%%%%%%%%%%%%%%%%
\hypersetup{
    colorlinks,
    linkcolor={crimson},
    citecolor={crimson},
    urlcolor={crimson}
}

\usepackage{qcircuit}

%%%%%%%%%%%%%%%%%%%%%%%%%%%%%%%%%%%%%%%%%%%%%
\theoremstyle{definition}
\newtheorem{definition}{Definition}[section]
\newtheorem{lemma}{Lemma}[section]
\newtheorem{theorem}{Theorem}[section]
\newtheorem{corollary}{Corollary}[theorem]
\newtheorem*{theorem*}{Theorem}
\newtheorem*{corollary*}{Corollary}
\newtheorem{remark}{Remark}[section]
\newtheorem{conjecture}{Conjecture}[section]
\newtheorem{example}{Example}[section]
\newtheorem{reminder}{Reminder}[section]
\newtheorem{problem}{Problem}[section]
\newtheorem{question}{Question}[section]
\newtheorem{answer}{Answer}[section]
\newtheorem{fact}{Fact}[section]
\newtheorem{claim}{Claim}[section]
\newtheorem{proposition}{Proposition}[section]

\usepackage{geometry}
\geometry{
  left=25mm,
  right=25mm,
  top=20mm,
}
\usepackage{mathtools}

%%%%%%%%%%%%%%%%%%%%%%%%%%%%%%%%%%%%%%%%%%%%%
\bibliographystyle{unsrt}

%%%%%%%%%%%%%%%%%%%%%%%%%%%%%%%%%%%%%%%%%%%%%
%%%%%%%%%%%%%%%%%%%%%%%%%%%%%%%%%%%%%%%%%%%%%
%%%%%%%%%%%%%%%%%%%%%%%%%%%%%%%%%%%%%%%%%%%%%
\begin{document}

\title{Discrete Riemann surfaces and the Ising model: presentation}
\author{Jack Ceroni}
\email{jackceroni@gmail.com}

\date{\today}

\maketitle

\begin{itemize}
\item The talk will begin with a summary of the results of the previous presentation. In particular, I will briefly re-describe the notion of the cell complex/dual complex
  on some surface $\Sigma$. \emph{In addition, I will draw a picture very quickly, showing that intuitively, such a complex can be thought of as drawing coordinate axes on some arbitrary surface,
  with the elements of $\Gamma$ corresponding to horizontal lines, and the elements of $\Gamma^{*}$ corresponding to vertical lines}.
\item I won't really have to redescribe much else, except for the relevant intuition surrounding the \emph{de Rham cohomology}.
\item In particular, I will first quickly describe again what a chain is, and how the boundary operator sends some $k$-chain to a $(k - 1)$-chain, with coefficients signs
  determined by the same rules that someone like \emph{Spivak} uses in their work. I'll also make note of how this boundary operator gives us a \emph{chain complex}.
\item From here, it is easy to write down what the space of cochains is: it is simply the space $\text{Hom}(C_k(\Lambda), \mathbb{C})$. We \textbf{call this} the space of \textbf{forms}.
\item It is now our goal of defining the discrete analogue of differential forms (the usual definition of forms) on our discrete structure. In the usual formulation of differential geometry,
  we defined forms as functions which yield alternating $k$-tensors on the tangent space of a manifold. We then defined the \emph{differential map}
  acting on a form, the integral of a form, and from these definitions, we then proved Stokes' theorem.
\item In our discrete context, we don't yet have the notion of a form or the differential or an integral. However, recall that \textbf{Hodge's theorem} gives a deep, dual connection between a homology group of
  a chain complex and the de Rahm cohomology. In other words, this theorem gives an isomorphism between the dual of a homology group of a manifold and the associated de Rahm cohomology.
\item \pop{State de Rham's theorem}
\item This suggests that an equivalent way for us to \emph{define} de Rham cohomology in the discrete setting is simply via $\text{Hom}(C_k(\Lambda), \mathbb{C})$. In particular,
  we \textbf{define}, using the integral notation, $\int_{c} \omega$ to be $\omega(c)$, where $\omega \in \text{Hom}(C_k(\Lambda), \mathbb{C})$ and $c \in C_k(\Lambda)$.
  \item What we are calling ``forms'' here are actually equivalence classes of closed forms modulo exact forms.
  \end{itemize}

\end{document}
