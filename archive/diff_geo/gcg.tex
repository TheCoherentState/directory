\documentclass[aps,pra,showpacs,notitlepage,onecolumn,superscriptaddress,nofootinbib]{revtex4-1}
\usepackage[utf8]{inputenc}
\usepackage[tmargin=1in, bmargin=1.25in, lmargin=1.5in, rmargin=1.5in]{geometry}
\usepackage{amsmath, amssymb, amsthm}
\usepackage{graphicx}
\usepackage{xcolor}
\usepackage{enumitem}
\usepackage{datetime}
\usepackage{hyperref}
\usepackage{titlesec}
\usepackage{import}
\usepackage{mathtools}
\usepackage{thmtools,thm-restate}
\usepackage{comment}


% package for commutative diagrams
% \usepackage{tikz-cd}

%%%%%%%%%%%%%%%%%%%%%%%%%%%%%%%%%%%%%%%%%%%%%
\definecolor{crimson}{RGB}{186,0,44}
\definecolor{moss}{RGB}{0, 186, 111}
\newcommand{\pop}[1]{\textcolor{crimson}{#1}}
\newcommand{\zcom}[1]{\noindent\textcolor{crimson}{(Z): #1}}
\newcommand{\jcom}[1]{\noindent\textcolor{moss}{(J): #1}}
\newcommand{\wt}[1]{\widetilde{#1}}
\newcommand{\pqeq}{\succcurlyeq}
\newcommand{\pleq}{\preccurlyeq}

\newcommand{\hhrulefill}{\hspace{-1.0em}\hrulefill}


%%%%%%%%%%%%%%%%%%%%%%%%%%%%%%%%%%%%%%%%%%%%%
\hypersetup{
    colorlinks,
    linkcolor={crimson},
    citecolor={crimson},
    urlcolor={crimson}
}

\usepackage{qcircuit}

%%%%%%%%%%%%%%%%%%%%%%%%%%%%%%%%%%%%%%%%%%%%%
\theoremstyle{definition}
\newtheorem{definition}{Definition}[section]
\newtheorem{lemma}{Lemma}[section]
\newtheorem{theorem}{Theorem}[section]
\newtheorem{corollary}{Corollary}[theorem]
\newtheorem*{theorem*}{Theorem}
\newtheorem*{corollary*}{Corollary}
\newtheorem{remark}{Remark}[section]
\newtheorem{conjecture}{Conjecture}[section]
\newtheorem{example}{Example}[section]
\newtheorem{reminder}{Reminder}[section]
\newtheorem{problem}{Problem}[section]
\newtheorem{question}{Question}[section]
\newtheorem{proposition}{Proposition}[section]
\newtheorem{answer}{Answer}[section]
\newtheorem{fact}{Fact}[section]
\newtheorem{claim}{Claim}[section]

\usepackage{geometry}
\geometry{
  left=25mm,
  right=25mm,
  top=20mm,
}

%%%%%%%%%%%%%%%%%%%%%%%%%%%%%%%%%%%%%%%%%%%%%
\bibliographystyle{unsrt}
\setlength{\parindent}{0pt}

%%%%%%%%%%%%%%%%%%%%%%%%%%%%%%%%%%%%%%%%%%%%%
%%%%%%%%%%%%%%%%%%%%%%%%%%%%%%%%%%%%%%%%%%%%%
%%%%%%%%%%%%%%%%%%%%%%%%%%%%%%%%%%%%%%%%%%%%%
\begin{document}

\title{A primer on generalized complex geometry}
\author{Jack Ceroni}
\email{jackceroni@gmail.com}

\date{\today}

\maketitle

\hhrulefill

\noindent \emph{The purpose of these notes is to introduce, with minimal assumed knowledge, the field of generalized complex geometry. These notes were written in the Fall of 2023.}

\hhrulefill

\begin{proposition}
  Let $\langle \cdot, \cdot \rangle$ denote a bilinear form on $n$-dimensional vector space $V$. Then, there exists a basis $v_1, \dots, v_n$ for $V$ such that
  \begin{equation}
    \langle x_1 v_1 + \cdots + x_n v_n, x_1 v_1 + \cdots x_n v_n \rangle = x_1^2 + \cdots x_p^2 - x_{p + 1}^2 - \cdots x_{p + q}^2
  \end{equation}
  where $p + q = n$. Moreover, the pair $(p, q)$ is independent of choice of basis, and is called the \emph{signature} of the bilinear form.
\end{proposition}

\begin{proof}
  \end{proof}

\begin{definition}
  Given vector space $V$ endowed with bilinear form $\langle \cdot, \cdot \rangle$, a subspace $W$ is said to be isotropic if the form vanishes for pairs $w_1, w_2 \in W$.
\end{definition}

Given $n$-dimensional vector space $V$, we take a double $\mathcal{D}V$ to be a $2n$-dimensional vector space endowed with $\langle \cdot, \cdot \rangle$ and projection $\pi : \mathcal{D}V \rightarrow V$
such that $\text{Ker}(\pi)$ is isotropic with respect to $\langle \cdot, \cdot \rangle$. Note that $\text{Ker}(\pi)$ being isotropic implies that $\text{Ker}(\pi) \subset \text{Ker}(\pi)^{\perp}$.

\end{document}
