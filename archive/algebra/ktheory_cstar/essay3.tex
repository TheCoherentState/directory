\documentclass[aps,pra,showpacs,notitlepage,onecolumn,superscriptaddress,nofootinbib]{revtex4-1}
\usepackage[utf8]{inputenc}
\usepackage[tmargin=1in, bmargin=1.25in, lmargin=1.5in, rmargin=1.5in]{geometry}
\usepackage{amsmath, amssymb, amsthm}
\usepackage{graphicx}
\usepackage{xcolor}
\usepackage{enumitem}
\usepackage{datetime}
\usepackage{hyperref}
\usepackage{titlesec}
\usepackage{import}
\usepackage{mathtools}
\usepackage{thmtools,thm-restate}
\usepackage{tikz-cd}


% package for commutative diagrams
% \usepackage{tikz-cd}

%%%%%%%%%%%%%%%%%%%%%%%%%%%%%%%%%%%%%%%%%%%%%
\definecolor{crimson}{RGB}{186,0,44}
\definecolor{moss}{RGB}{0, 186, 111}
\newcommand{\pop}[1]{\textcolor{crimson}{#1}}
\newcommand{\zcom}[1]{\noindent\textcolor{crimson}{(Z): #1}}
\newcommand{\jcom}[1]{\noindent\textcolor{moss}{(J): #1}}
\newcommand{\wt}[1]{\widetilde{#1}}
\newcommand{\pqeq}{\succcurlyeq}
\newcommand{\pleq}{\preccurlyeq}
\newcommand{\Wedge}{\bigwedge}

%%%%%%%%%%%%%%%%%%%%%%%%%%%%%%%%%%%%%%%%%%%%%
\hypersetup{
    colorlinks,
    linkcolor={crimson},
    citecolor={crimson},
    urlcolor={crimson}
}

\usepackage{qcircuit}

%%%%%%%%%%%%%%%%%%%%%%%%%%%%%%%%%%%%%%%%%%%%%
\theoremstyle{definition}
\newtheorem{definition}{Definition}[section]
\newtheorem{lemma}{Lemma}[section]
\newtheorem{theorem}{Theorem}[section]
\newtheorem{corollary}{Corollary}[theorem]
\newtheorem*{theorem*}{Theorem}
\newtheorem*{corollary*}{Corollary}
\newtheorem{remark}{Remark}[section]
\newtheorem{conjecture}{Conjecture}[section]
\newtheorem{example}{Example}[section]
\newtheorem{reminder}{Reminder}[section]
\newtheorem{problem}{Problem}[section]
\newtheorem{question}{Question}[section]
\newtheorem{answer}{Answer}[section]
\newtheorem{fact}{Fact}[section]
\newtheorem{claim}{Claim}[section]

\newcommand{\hhrulefill}{\hspace{-1.5em} \hrulefill}

\usepackage{geometry}
\geometry{
  left=25mm,
  right=25mm,
  top=20mm,
}

%%%%%%%%%%%%%%%%%%%%%%%%%%%%%%%%%%%%%%%%%%%%%
\bibliographystyle{unsrt}

%%%%%%%%%%%%%%%%%%%%%%%%%%%%%%%%%%%%%%%%%%%%%
%%%%%%%%%%%%%%%%%%%%%%%%%%%%%%%%%%%%%%%%%%%%%
%%%%%%%%%%%%%%%%%%%%%%%%%%%%%%%%%%%%%%%%%%%%%

\begin{document}

\title{An primer on topological $K$-theory and a topological proof of Bott periodicity}
\author{Jack Ceroni}
\email{jackceroni@gmail.com}

\date{\today}

\maketitle

\section{Introduction}

\noindent \emph{This essay was written for the Fall 2023 session of MAT437: $K$-Theory and $C^{*}$-Algebras, taught by Professor George Elliott, at the University of Toronto.}
\newline

\noindent The goal of this essay is to discuss in greater detail than has been covered in class and in the text of Rordam, certain aspects of topological $K$-theory,
and their relationship to the algebraic flavour of $K$-theory to which we have dedicated greater focus.

\section{Basics}

\noindent Let us begin by recalling a very basic definition:

\begin{definition}[Complex vector bundle]
  A complex vector bundle over topological space $X$ is defined to be a triple $\xi = (E, \pi, X)$, where $E$ is a topological space and $\pi : E \rightarrow X$ is a continuous surjection
  from $E$ to $X$, the so-called \emph{base space}, where each \emph{fibre} $\pi^{-1}(x)$ has the structure of a complex finite-dimensional vector space. We require that $\xi$ has the following
  local compatibility property: for each $x \in X$, there must exist a neighbourhood $U$ of $x$ and a homeomorphism $h : \pi^{-1}(U) \rightarrow U \times \mathbb{C}^n$ for some $n \in \mathbb{N}$
  making the following diagram commute:
  % https://q.uiver.app/#q=WzAsMyxbMCwwLCJcXHBpXnstMX0oVSkiXSxbMiwwLCJVIFxcdGltZXMgXFxtYXRoYmJ7Q31ee259Il0sWzEsMSwiVSJdLFswLDEsImgiXSxbMCwyLCJcXHBpIiwyXSxbMSwyLCJcXGV0YSJdXQ==
  \[\begin{tikzcd}
	          {\pi^{-1}(U)} && {U \times \mathbb{C}^{n}} \\
	          & U
	          \arrow["h", from=1-1, to=1-3]
	          \arrow["\pi"', from=1-1, to=2-2]
	          \arrow["\eta", from=1-3, to=2-2]
  \end{tikzcd}\]
  where $\eta(x, v) = x$ is the projection onto the base, and, for each $y \in U$, the map $h : \pi^{-1}(y) \rightarrow \{y\} \times \mathbb{C}^n \simeq \mathbb{C}^n$ is a vector
  space isomorphism. Intuitively, this simply means that locally, in the base space, the corresponding bundle of fibres in $E$ must resemble a trivial product space $U \times \mathbb{C}^n$,
  both in a topological sense and an algebraic sense.
\end{definition}

\noindent The motivation for studying topological $K$-theory comes from the desire to classify all vector bundles over a given base space of a particular, fixed dimension $n$.
To be more specific, we say that bundles $\xi = (E, \pi, X)$ and $\xi' = (E', \nu, X)$ are \emph{isomorphic} if there is a homeomorphism $h : E \rightarrow E'$ making the following diagram commute:
% https://q.uiver.app/#q=WzAsMyxbMCwwLCJFIl0sWzIsMCwiRSciXSxbMSwxLCJYIl0sWzAsMSwiaCJdLFsxLDIsIlxcbnUiXSxbMCwyLCJcXHBpIiwyXV0=
\[\begin{tikzcd}
E && {E'} \\
& X
\arrow["h", from=1-1, to=1-3]
\arrow["\nu", from=1-3, to=2-2]
\arrow["\pi"', from=1-1, to=2-2]
\end{tikzcd}\]
such that the restricted map $h : \pi^{-1}(x) \rightarrow \nu^{-1}(x)$ is a vector space isomorphism for each $x \in X$. The task of computing the isomorphism classes
of vector bundles (which we denote $\langle \xi \rangle$, where $\xi$ is an equivalence class representative) is a highly non-trivial task, and can only be done in very basic, often low-dimensional cases.
Thus, in order to make partial progress on solving this profoundly difficult question, a natural question to ask is: \emph{are there weaker equivalence relations than the above notion of isomorphism which can more easily be computed, at the cost
of ending up with a more crude classification?} The answer to this question is a resounding yes, via the construction of topological $K$-groups.

\hhrulefill

\noindent Treating vector bundle $\xi$ now as fundamental objects, let us define algebraic operations between them.

\begin{lemma}[Direct sum of vector bundles]

\end{lemma}

\begin{lemma}[Inner products on vector bundles]
\end{lemma}

\begin{lemma}[Tensor products of vector bundles]
\end{lemma}

\begin{lemma}

\end{lemma}

\begin{theorem}[Equivalence of algebraic and topological $K$-theory]
If $X$ is a compact Hausdorff space, then $K_0(C(X))$, the algebraic $K_0$-group of the $C^{*}$-algebra of functions $C(X)$, and $K(X)$, the topological $K^{0}$-group, are isomorphic as Abelian groups.
\end{theorem}
\begin{proof}

  \end{proof}

\section{Bott periodicity}

\section{Characteristic classes}

\end{document}
