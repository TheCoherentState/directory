\documentclass[aps,pra,showpacs,notitlepage,onecolumn,superscriptaddress,nofootinbib]{revtex4-1}
\usepackage[utf8]{inputenc}
\usepackage[tmargin=1in, bmargin=1.25in, lmargin=1.5in, rmargin=1.5in]{geometry}
\usepackage{amsmath, amssymb, amsthm}
\usepackage{graphicx}
\usepackage{xcolor}
\usepackage{enumitem}
\usepackage{datetime}
\usepackage{hyperref}
\usepackage{titlesec}
\usepackage{import}
\usepackage{mathtools}
\usepackage{thmtools,thm-restate}
\usepackage{tikz-cd}


% package for commutative diagrams
% \usepackage{tikz-cd}

%%%%%%%%%%%%%%%%%%%%%%%%%%%%%%%%%%%%%%%%%%%%%
\definecolor{crimson}{RGB}{186,0,44}
\definecolor{moss}{RGB}{0, 186, 111}
\newcommand{\pop}[1]{\textcolor{crimson}{#1}}
\newcommand{\zcom}[1]{\noindent\textcolor{crimson}{(Z): #1}}
\newcommand{\jcom}[1]{\noindent\textcolor{moss}{(J): #1}}
\newcommand{\wt}[1]{\widetilde{#1}}
\newcommand{\pqeq}{\succcurlyeq}
\newcommand{\pleq}{\preccurlyeq}
\newcommand{\Wedge}{\bigwedge}

%%%%%%%%%%%%%%%%%%%%%%%%%%%%%%%%%%%%%%%%%%%%%
\hypersetup{
    colorlinks,
    linkcolor={crimson},
    citecolor={crimson},
    urlcolor={crimson}
}

\usepackage{qcircuit}

%%%%%%%%%%%%%%%%%%%%%%%%%%%%%%%%%%%%%%%%%%%%%
\theoremstyle{definition}
\newtheorem{definition}{Definition}[section]
\newtheorem{lemma}{Lemma}[section]
\newtheorem{theorem}{Theorem}[section]
\newtheorem{corollary}{Corollary}[theorem]
\newtheorem*{theorem*}{Theorem}
\newtheorem*{corollary*}{Corollary}
\newtheorem{remark}{Remark}[section]
\newtheorem{conjecture}{Conjecture}[section]
\newtheorem{example}{Example}[section]
\newtheorem{reminder}{Reminder}[section]
\newtheorem{problem}{Problem}[section]
\newtheorem{question}{Question}[section]
\newtheorem{answer}{Answer}[section]
\newtheorem{fact}{Fact}[section]
\newtheorem{claim}{Claim}[section]

\newcommand{\hhrulefill}{\hspace{-1.5em} \hrulefill}

\usepackage{geometry}
\geometry{
  left=25mm,
  right=25mm,
  top=20mm,
}

%%%%%%%%%%%%%%%%%%%%%%%%%%%%%%%%%%%%%%%%%%%%%
\bibliographystyle{unsrt}

%%%%%%%%%%%%%%%%%%%%%%%%%%%%%%%%%%%%%%%%%%%%%
%%%%%%%%%%%%%%%%%%%%%%%%%%%%%%%%%%%%%%%%%%%%%
%%%%%%%%%%%%%%%%%%%%%%%%%%%%%%%%%%%%%%%%%%%%%

\begin{document}

\title{An invitation to cyclic cohomology}
\author{Jack Ceroni}
\email{jackceroni@gmail.com}

\date{\today}

\maketitle

\section{Introduction}

\noindent \emph{This essay was written for the Fall 2023 session of MAT437: $K$-Theory and $C^{*}$-Algebras, taught by Professor George Elliott, at the University of Toronto.}
\newline

\noindent The goal of this essay is to introduce, explain, and place in broader context the basic theory of cyclic cohomology. The main references
for this essay are the book by Moriyoshi and Natsume (Ref.~\cite{}), the standard reference of Connes (Ref.~\cite{}), as well as a handful of other cites articles.

\section{Introduction and motivation}

\noindent The $K_0$-group of a $C^{*}$-algebra $A$ is a broadly powerful invariant which encodes the underlying structure of $A$, in such a way that lends
itself well to being analyzed via a large toolbox of different techniques. One such technique involves construction of \emph{positive trace maps}.

\begin{definition}[Trace maps]
  Recall that a positive trace map on a $C^{*}$-algebra $A$ is simply a bounded linear function $\tau : A \rightarrow \mathbb{C}$ such that $\tau(ab) = \tau(ba)$ for all $a, b \in A$. In addition, $\tau(a) \geq 0$
  for positive $a$. We can similarly define trace maps $\tau_n$ on each matrix algebra $M_n(A)$, via
\begin{equation}
  \tau_n(M) = \displaystyle\sum_{j = 1}^{n} \tau(M_{jj}), \ \ \ \ M \in M_n(A).
\end{equation}
Via the trace property, note that if $p \sim q$ via the Murray-von Neumann equivalence in $\mathcal{P}_n(A)$, then $p = v v^{*}$ and $q = v^{*} v$, so that $\tau_n(p) = \tau_n(q)$. Thus, if $p \sim_h q$
then $\tau_n(p) = \tau_n(q)$. It follows via the universal property for $K_0$ that there is a unique group homomorphism $K_0(\tau) : K_0(A) \rightarrow \mathbb{C}$ such that $K_0(\tau)([p]_0) = \tau(p)$,
where in this context, $\tau$ is the induced trace of the collection of $\tau_n$ on $\mathcal{P}_{\infty}(A)$.
\end{definition}

\noindent $K_0(\tau)$ is called an induced trace map on $K_0(A)$: functions of this form are vital to the analysis of $K_0$-groups because they allow us to extract numerical invariants and map
elements of the $K_0$-group to complex numbers. For example, study of induced trace maps yield simple proofs of the underlying group structure of commonly-encountered $K_0$-groups,
such as $K_0(M_n(\mathbb{C}))$, $K_0(B(H))$, and $K_0(C(X))$.

However, as it turns out, despite their usefulness, one will often encounter $C^{*}$-algebras which \emph{admit no non-trivial positive trace}. Thus, we have no hope of defining a induced trace on the $K_0$-group.
As an instructive example, we consider one of the most important types of $C^{*}$-algebras are the so-called \emph{Cuntz algebras}, which are defined as follows.

\begin{definition}[Cuntz algebra]
  Let $H$ be an infinite-dimensional separable Hilbert space, let $n$ be a positive integer greater than or equal to $2$. Suppose we have $s_j \in B(H)$ such that
  \begin{equation}
    s_1^{*} s_1 = \cdots = s_n^{*} s_n = 1 = \displaystyle\sum_{j = 1}^{n} s_j s_j^{*}
  \end{equation}
  (as it turns out, we can always find a collection of isometries of this form, but this is unimportant for our discussion). Let $\mathcal{O}_n = C^{*}(s_1, \dots, s_n) \subset B(H)$.
  We call the sub-$C^{*}$-algebra $\mathcal{O}_n$ a \emph{Cuntz algebra}.
\end{definition}

\noindent The Cuntz algebra is a simple example of a $C^{*}$-algebra admitting no non-trivial positive trace. The proof is delighfully straightforward.

\begin{theorem}
  The Cuntz algebra admits no non-trivial positive trace.
\end{theorem}
\begin{proof}
  Suppose $\tau$ is a trace on $\mathcal{O}_n$, note that we then have
  \begin{equation}
    \tau(1) = \tau \left( \displaystyle\sum_{j = 1}^{n} s_j s_j^{*} \right) =  \displaystyle\sum_{j = 1}^{n} \tau(s_j^{*} s_j) = n \tau(1) \Longrightarrow (n - 1) \tau(1) = 0.
  \end{equation}
  Thus, since $n \geq 2$, $\tau(1) = 0$. Note that if $\tau(a) \geq 0$ for positive $a$, then for negative $a$, $\tau(-a) = - \tau(a) \geq 0 \Rightarrow \tau(a) \leq 0$.
  Let $a$ be an arbitrary operator. Then clearly, $a - ||a|| \cdot 1$ is negative and $a + ||a|| \cdot 1$ is positive. But we have
  \begin{equation}
    \tau(a) = \tau(a \pm ||a|| \cdot 1 \mp ||a|| \cdot 1) = \tau(a \pm ||a|| \cdot 1) \mp ||a|| \tau(1) = \tau(a \pm ||a|| \cdot 1).
  \end{equation}
  But since $\tau(a + ||a|| \cdot 1) \geq 0$ and $\tau(a - ||a|| \cdot 1) \leq 0$, we must have $\tau(a) = a$, so $\tau$ is just the zero-map.
  \end{proof}

\noindent This potential non-existence of useful trace maps is a fundamental problem, as in such cases, it is very much unclear how one should go about extracting numerical invariants from $K_0$-groups.
\emph{Cyclic cohomology}, discovered by Alain Connes, can be thought of as a remedy to this situation. To be more specific, recall the following definition

\begin{definition}[Cohomology]
  A sequence of mdoules $A = \{A_n\}$ connected by homomorphisms $d^n : A_n \rightarrow A_{n + 1}$ is said to be a \emph{cochain complex} if $d^{n + 1} \circ d^{n} = 0$ for all $n$.
  The corresponding \emph{cohomology theory} of the cochain complex is the sequence of quotient module $H^{n} = \text{Ker}(d^{n}) / \text{Im}(d^{n - 1})$
  \end{definition}

Arguably, the most well-known cohomology theory is the de Rham cohomology of the sequence of differential $k$-forms on manifold $M$, connected by the exterior derivative map:
% https://q.uiver.app/#q=WzAsNSxbMCwwLCJcXGNkb3RzIl0sWzEsMCwiXFxPbWVnYV57ayAtIDF9KE0pIl0sWzIsMCwiXFxPbWVnYV57a30oTSkiXSxbMywwLCJcXE9tZWdhXntrICsgMX0oTSkiXSxbNCwwLCJcXGNkb3RzIl0sWzAsMSwiZCJdLFsxLDIsImQiXSxbMiwzLCJkIl0sWzMsNCwiZCJdXQ==
\[\begin{tikzcd}
\cdots & {\Omega^{k - 1}(M)} & {\Omega^{k}(M)} & {\Omega^{k + 1}(M)} & \cdots
\arrow["d", from=1-1, to=1-2]
\arrow["d", from=1-2, to=1-3]
\arrow["d", from=1-3, to=1-4]
\arrow["d", from=1-4, to=1-5]
\end{tikzcd}\]
On the other hand, we can also consider the \emph{singular cohomology} on a manifold: where the modules considered are the spaces of $k$-cochains on the manifold, $\text{Hom}(C^{k}(M), \mathbb{R})$, and
the connecting homomorphisms are the coboundary maps $\partial^{*}$, which are the dual maps of the boundary operations $\partial$ taking a $k$-chain to its boundary (a $(k - 1)$-chain):
% https://q.uiver.app/#q=WzAsNSxbMCwwLCJcXGNkb3RzIl0sWzEsMCwiXFx0ZXh0e0hvbX0oQ157ayAtIDF9KE0pKSJdLFsyLDAsIlxcdGV4dHtIb219KENee2t9KE0pKSJdLFszLDAsIlxcdGV4dHtIb219KENee2sgKyAxfShNKSkiXSxbNCwwLCJcXGNkb3RzIl0sWzAsMSwiXFxwYXJ0aWFsXnsqfSJdLFsxLDIsIlxccGFydGlhbF57Kn0iXSxbMiwzLCJcXHBhcnRpYWxeeyp9Il0sWzMsNCwiXFxwYXJ0aWFsXnsqfSJdXQ==
\[\begin{tikzcd}
\cdots & {\text{Hom}(C^{k - 1}(M))} & {\text{Hom}(C^{k}(M))} & {\text{Hom}(C^{k + 1}(M))} & \cdots
\arrow["{\partial^{*}}", from=1-1, to=1-2]
\arrow["{\partial^{*}}", from=1-2, to=1-3]
\arrow["{\partial^{*}}", from=1-3, to=1-4]
\arrow["{\partial^{*}}", from=1-4, to=1-5]
\end{tikzcd}\]
One of the most fundamental
results in differential geometry is \emph{de Rham's theorem}, which states that de Rham cohomology is isomorphic to the singular cohomology, $H^{k}_{\text{de Rham}} \simeq H^{k}_{\text{singular}}$. Thus, when studying the de Rham cohomology, we can always simply
think of $k$-cochains. From here, we make a key observation: $\text{Hom}(C^{k}(M))$ has more structure than just that of a generic module, \emph{it also has algebra structure, and moreover, it is commutative}. In other words, we
can multiply cochains. Thus, singular cohomology and by the isomorphism, de Rham cohomology, can be thought of as a ``commutative'' cohomology theory. The idea behind cyclic cohomology 


\section{Basics of cyclic cohomology}

\noindent To begin our discussion of cyclic cohomology, let us define some of the key constructions of which we will make heavy use through the subsequent sections. Let $A$ be an algebra over $\mathbb{C}$.
We define $C_{\lambda}^{n}(A)$ to be the space of cyclic $n$-cochains on $A$, that is, the set of all $(n + 1)$-linear functionals satisfying
\begin{equation}
  \phi(a^{0}, a^{1} \dots, a^{n - 1}, a^{n}) = (-1)^{n} \phi(a^{1}, \dots, a^{n - 1}, a^n, a^{0})
\end{equation}
We also define a family of \emph{Hochschild coboundary maps}, $b : C_{\lambda}^{n} \rightarrow C_{\lambda}^{n + 1}$, such that
\begin{equation}
  (b \phi)(a^0, \dots, a^{n + 1}) = (-1)^{n + 1} \phi(a^{n + 1} a^{0}, \dots, a^{n}) + \displaystyle\sum_{j = 0}^{n} (-1)^{j} \phi(a^{0}, \dots, a^j a^{j + 1}, \dots, a^{n + 1}).
\end{equation}
The coboundary operator obeys the properties that it should. In particular,
\begin{lemma}
  It is true that $b(C_{\lambda}^{n}) \subset C_{\lambda}^{n + 1}$. In addition, $b^2 = 0$.
\end{lemma}
\begin{proof}
Indeed, let $b_j$ be the map such that $(b_j \phi)(a^{0}, \dots, a^{n + 1}) = \phi(a^{0}, \dots, a^{j} a^{j + 1}, \dots, a^{n + 1})$.
Let $b_{n + 1}$ be such that $(b_{n + 1} \phi)(a^{0}, \dots, a^{n + 1}) = \phi(a^{n + 1} a^{0}, \dots, a^{n})$. For $j$ and $k$
\end{proof}

\noindent It follows that we are able to define a cohomology with the complex $(\{C_{\lambda}^{n}\}, \{b_n\})$, where $b_n : C_{\lambda}^{n} \rightarrow C_{\lambda}^{n + 1}$: we take
\begin{equation}
  H^{n}(A) = \text{Ker}(b_n) / \text{Im}(b_{n + 1})
\end{equation}
to be the corresponding cohomology classes, and we call the corresponding collection the \emph{cyclic cohomology} of $A$.

\subsection{Examples}

\noindent Let us now consider examples of simple cyclic cohomology theories. Suppose $A = \mathbb{C}$.

\noindent It is important to note that in the commutative case, cyclic cohomology reduces to de Rham cohomology!

\section{The Connes-Chern Character}

\section{Further implications and conclusion}

\end{document}
