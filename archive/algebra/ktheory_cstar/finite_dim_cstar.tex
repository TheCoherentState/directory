\documentclass[aps,pra,showpacs,notitlepage,onecolumn,superscriptaddress,nofootinbib]{revtex4-1}
\usepackage[utf8]{inputenc}
\usepackage[tmargin=1in, bmargin=1.25in, lmargin=1.5in, rmargin=1.5in]{geometry}
\usepackage{amsmath, amssymb, amsthm}
\usepackage{graphicx}
\usepackage{xcolor}
\usepackage{enumitem}
\usepackage{datetime}
\usepackage{hyperref}
\usepackage{titlesec}
\usepackage{import}
\usepackage{mathtools}
\usepackage{thmtools,thm-restate}
\usepackage{comment}


% package for commutative diagrams
% \usepackage{tikz-cd}

%%%%%%%%%%%%%%%%%%%%%%%%%%%%%%%%%%%%%%%%%%%%%
\definecolor{crimson}{RGB}{186,0,44}
\definecolor{moss}{RGB}{0, 186, 111}
\newcommand{\pop}[1]{\textcolor{crimson}{#1}}
\newcommand{\zcom}[1]{\noindent\textcolor{crimson}{(Z): #1}}
\newcommand{\jcom}[1]{\noindent\textcolor{moss}{(J): #1}}
\newcommand{\wt}[1]{\widetilde{#1}}
\newcommand{\pqeq}{\succcurlyeq}
\newcommand{\pleq}{\preccurlyeq}
\newcommand{\Wedge}{\bigwedge}

%%%%%%%%%%%%%%%%%%%%%%%%%%%%%%%%%%%%%%%%%%%%%
\hypersetup{
    colorlinks,
    linkcolor={crimson},
    citecolor={crimson},
    urlcolor={crimson}
}

\usepackage{qcircuit}

%%%%%%%%%%%%%%%%%%%%%%%%%%%%%%%%%%%%%%%%%%%%%
\theoremstyle{definition}
\newtheorem{definition}{Definition}[section]
\newtheorem{lemma}{Lemma}[section]
\newtheorem{theorem}{Theorem}[section]
\newtheorem{corollary}{Corollary}[theorem]
\newtheorem*{theorem*}{Theorem}
\newtheorem*{corollary*}{Corollary}
\newtheorem{remark}{Remark}[section]
\newtheorem{conjecture}{Conjecture}[section]
\newtheorem{example}{Example}[section]
\newtheorem{reminder}{Reminder}[section]
\newtheorem{problem}{Problem}[section]
\newtheorem{question}{Question}[section]
\newtheorem{answer}{Answer}[section]
\newtheorem{fact}{Fact}[section]
\newtheorem{claim}{Claim}[section]

\newcommand{\hhrulefill}{\hspace{-1.5em} \hrulefill}

\usepackage{geometry}
\geometry{
  left=25mm,
  right=25mm,
  top=20mm,
}

%%%%%%%%%%%%%%%%%%%%%%%%%%%%%%%%%%%%%%%%%%%%%
\bibliographystyle{unsrt}

%%%%%%%%%%%%%%%%%%%%%%%%%%%%%%%%%%%%%%%%%%%%%
%%%%%%%%%%%%%%%%%%%%%%%%%%%%%%%%%%%%%%%%%%%%%
%%%%%%%%%%%%%%%%%%%%%%%%%%%%%%%%%%%%%%%%%%%%%
\begin{document}

\title{Classification of finite-dimensional $C^{*}$-algebras}
\author{Jack Ceroni}
\email{jackceroni@gmail.com}

\date{\today}

\maketitle

\section{Introduction}

\noindent The goal of these lecture notes is to provide supporing material for a talk that I will be giving at the Fields Institute operator algebra seminar in January 2024.

\section{A few preliminary results}

\noindent Before jumping into the main proof, we have to prove a few preliminary lemmas. The eventual classification theorem will show that
all finite-dimensional $C^{*}$-algebras can be written as direct sums of matrix algebras over $\mathbb{C}$. To prove this, we will have
to show that any finite-dimensional algebra behaves in a way analogous to a direct sum of matrix algebras.

The first lemma we will prove is in this general spirit: it proves that a ``matrix-like'' property holds for $C^{*}$-algebras.

\begin{definition}[Matrix units]
  
\end{definition}

\begin{lemma}
If $v$ is a partial isometry (so $v^{*} v$ is a projection) then $v v^{*} v = v$.
\end{lemma}
\begin{proof}
  Let $z = v - v v^{*} v = (1 - v v^{*})v$. Note that
  \begin{equation}
    z^{*} z = v^{*} (1 - v v^{*})^{*} (1 - v v^{*}) = v^{*} (1 - 2 v v^{*} +  v v^{*} v v^{*}) v = v^{*} v - 2 v^{*} v + v^{*} v = 0
  \end{equation}
  where we use that $(v^{*} v)^2 = v^{*} v$. Thus, $||z||^2 = ||z^{*} z|| = 0$, so $||z|| = 0$ and $v = v v^{*} v$, as desired.
  \end{proof}

\noindent Now, another lemma:

\begin{lemma}
  Suppose that $\{f_{ii}^{(k)} \ | \ 1 \leq k \leq r, 1 \leq i \leq n_k\}$ is a set of mutually orthogonal projections in $C^{*}$-algebra $B$ and that
  \begin{equation}
    f_{11}^{(k)} \sim f_{22}^{(k)} \sim \cdots \sim f_{n_k n_k}^{(k)}
  \end{equation}
  for each $k$. Then there is a system of matrix units $\{f_{ij}^{(k)}\}$ extending $\{f_{ii}^{(k)}\}$.
\end{lemma}

\noindent The idea behind constructing systems of matrix units is, essentially, to have a ``basis'' for each component of the direct sum
that we will eventually demonstrate characterizes the $C^{*}$-algebra $B$. Each of the sets $\{f_{ii}^{(k)}\}$ are analogous to matrix projections
with $1$ at the $i$-th slot on the diagonal, at the $k$-th slot in the direct sum. Let us now prove the lemma.

\begin{proof}
  Of course, here, we will make us of the Murray-von Neumann equivalence. Namely,
  \begin{equation}
    f_{11}^{(k)} \sim f_{jj}^{(k)} \Longrightarrow f_{11}^{(k)} = f_{1j}^{(k)} f_{1j}^{(k)^{*}} \ \ \ \text{and} \ \ \ f_{jj}^{(k)} = f_{1j}^{(k)^{*}} f_{1j}^{(k)}
  \end{equation}
  This notation is consistent, as $f_{11}^{(k)}$ is self-adjoint, so setting $j = 1$ above causes no problems.
  Our claim is that if we set $\widetilde{f}_{ij}^{(k)} = f_{1i}^{(k)^{*}} f_{1j}^{(k)}$ then we will have the desired system of matrix units. This is in fact an extension of the system we are already provided. Namely, we have
  \begin{equation}
    \widetilde{f}_{jj}^{(k)} = f_{1j}^{(k)^{*}} f_{1j}^{(k)} = f_{jj}^{(k)}
  \end{equation}
  by definition. In fact, we might as well denote $\widetilde{f}_{ij}$ by $f_{ij}$, as for $i = 1$, we have
  \begin{equation}
    \widetilde{f}_{1j} = f_{11}^{(k)^{*}} f_{1j}^{(k)} = f_{1j}^{(k)} f_{1j}^{(k)^{*}} f_{1j}^{(k)} = f_{1j}^{(k)}
    \end{equation}
  where we use the above lemma and the fact that $f_{1j}^{(k)^{*}} f_{1j}^{(k)} = f_{jj}^{(k)}$ is a projection. Let us now complete our verification. Of course, we have
  \begin{equation}
    f_{pq}^{(k)} f_{qr}^{(k)} = f_{1p}^{(k)^{*}} f_{1q}^{(k)} f_{1q}^{(k)^{*}} f_{1r}^{(k)} = f_{1p}^{(k)^{*}} f_{11}^{(k)} f_{1r}^{(k)} = f_{1p}^{(k)^{*}} f_{1r}^{(k)} f_{1r}^{(k)^{*}} f_{1r}^{(k)} = f_{1p}^{(k)^{*}} f_{1r}^{(k)} = f_{pr}^{(k)}
  \end{equation}
  where we use the first lemma to note that $f_{1r}^{(k)} f_{1r}^{(k)^{*}} f_{1r}^{(k)} = f_{1r}^{(k)}$, as $f_{1r}^{(k)^{*}} f_{1r}^{(k)} = f_{rr}^{(k)}$ is a projection. Next, note that
  \begin{equation}
    f_{pq}^{(k)} f_{rs}^{(\ell)} = f_{1p}^{(k)^{*}} f_{1q}^{(k)} f_{1r}^{(\ell)^{*}} f_{1s}^{(\ell)}
  \end{equation}
  Once again using the first lemma, we have $f_{1q}^{(k)} = f_{1q}^{(k)} f_{1q}^{(k)^{*}} f_{1q}^{(k)} =  f_{1q}^{(k)}  f_{qq}^{(k)}$ and $f_{1r}^{(\ell)} = f_{1r}^{(\ell)} f_{1r}^{(\ell)^{*}} f_{1r}^{(\ell)} =  f_{1r}^{(\ell)} f_{rr}^{(\ell)}$
  so that $f_{1r}^{(\ell)^{*}} = f_{rr}^{(\ell)} f_{1r}^{(\ell)^{*}}$. We then use the fact that the projections in our set are mutually orthogonal to conclude that
  \begin{equation}
    f_{1p}^{(k)^{*}} f_{1q}^{(k)} f_{1r}^{(\ell)^{*}} f_{1s}^{(\ell)} = f_{1p}^{(k)^{*}} f_{1q}^{(k)} (f_{qq}^{(k)}  f_{rr}^{(\ell)}) f_{1r}^{(\ell)^{*}} f_{1s}^{(\ell)} = 0
  \end{equation}
  which is $0$ when $q \neq r$ or $k \neq \ell$, as in these cases, $f_{qq}^{(k)}  f_{rr}^{(\ell)} = 0$. It is very immediately clear that $f_{ij}^{(k)^{*}} = f_{1j}^{(k)^{*}} f_{1i}^{(k)} = f_{ji}^{(k)}$,
  so we have verified the third condition, and it follows that our set of $f_{ij}^{(k)}$ is in fact a system of matrix units in $B$ extending $\{f_{ii}^{(k)}\}$.
\end{proof}

\noindent Now we will state of a useful result which we will not prove, for the sake of breivty:



\section{The main proof}

\noindent Now, let us move on to the main proof: the classification theorem for finite-dimensional $C^{*}$-algebras.

\end{document}
