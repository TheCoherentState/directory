\documentclass[aps,pra,showpacs,notitlepage,onecolumn,superscriptaddress,nofootinbib]{revtex4-1}
\usepackage[utf8]{inputenc}
\usepackage[tmargin=1in, bmargin=1.25in, lmargin=1.5in, rmargin=1.5in]{geometry}
\usepackage{amsmath, amssymb, amsthm}
\usepackage{graphicx}
\usepackage{xcolor}
\usepackage{enumitem}
\usepackage{datetime}
\usepackage{hyperref}
\usepackage{titlesec}
\usepackage{import}
\usepackage{mathtools}
\usepackage{thmtools,thm-restate}
\usepackage{comment}


% package for commutative diagrams
% \usepackage{tikz-cd}

%%%%%%%%%%%%%%%%%%%%%%%%%%%%%%%%%%%%%%%%%%%%%
\definecolor{crimson}{RGB}{186,0,44}
\definecolor{moss}{RGB}{0, 186, 111}
\newcommand{\pop}[1]{\textcolor{crimson}{#1}}
\newcommand{\zcom}[1]{\noindent\textcolor{crimson}{(Z): #1}}
\newcommand{\jcom}[1]{\noindent\textcolor{moss}{(J): #1}}
\newcommand{\wt}[1]{\widetilde{#1}}
\newcommand{\pqeq}{\succcurlyeq}
\newcommand{\pleq}{\preccurlyeq}
\newcommand{\Wedge}{\bigwedge}

%%%%%%%%%%%%%%%%%%%%%%%%%%%%%%%%%%%%%%%%%%%%%
\hypersetup{
    colorlinks,
    linkcolor={crimson},
    citecolor={crimson},
    urlcolor={crimson}
}

\usepackage{qcircuit}

%%%%%%%%%%%%%%%%%%%%%%%%%%%%%%%%%%%%%%%%%%%%%
\theoremstyle{definition}
\newtheorem{definition}{Definition}[section]
\newtheorem{lemma}{Lemma}[section]
\newtheorem{theorem}{Theorem}[section]
\newtheorem{corollary}{Corollary}[theorem]
\newtheorem*{theorem*}{Theorem}
\newtheorem*{corollary*}{Corollary}
\newtheorem{remark}{Remark}[section]
\newtheorem{conjecture}{Conjecture}[section]
\newtheorem{example}{Example}[section]
\newtheorem{reminder}{Reminder}[section]
\newtheorem{problem}{Problem}[section]
\newtheorem{question}{Question}[section]
\newtheorem{answer}{Answer}[section]
\newtheorem{fact}{Fact}[section]
\newtheorem{claim}{Claim}[section]

\newcommand{\hhrulefill}{\hspace{-1.5em} \hrulefill}

\usepackage{geometry}
\geometry{
  left=25mm,
  right=25mm,
  top=20mm,
}

%%%%%%%%%%%%%%%%%%%%%%%%%%%%%%%%%%%%%%%%%%%%%
\bibliographystyle{unsrt}

%%%%%%%%%%%%%%%%%%%%%%%%%%%%%%%%%%%%%%%%%%%%%
%%%%%%%%%%%%%%%%%%%%%%%%%%%%%%%%%%%%%%%%%%%%%
%%%%%%%%%%%%%%%%%%%%%%%%%%%%%%%%%%%%%%%%%%%%%
\begin{document}

\title{MAT437 problem set 8}
\author{Jack Ceroni}
\email{jackceroni@gmail.com}

\date{\today}

\maketitle

\section{Problem 1}

\noindent \pop{\emph{The proof of the following lemma included in RLL excludes many key parts, so for one of my exercises, I complete it}}
\newline

\noindent We start by proving a collection of technical lemma, which will be required for the following proof:

\begin{lemma}
If $v$ is a partial isometry (so $v^{*} v$ is a projection) then $v v^{*} v = v$.
\end{lemma}
\begin{proof}
  Let $z = v - v v^{*} v = (1 - v v^{*})v$. Note that
  \begin{equation}
    z^{*} z = v^{*} (1 - v v^{*})^{*} (1 - v v^{*}) = v^{*} (1 - 2 v v^{*} +  v v^{*} v v^{*}) v = v^{*} v - 2 v^{*} v + v^{*} v = 0
  \end{equation}
  where we use that $(v^{*} v)^2 = v^{*} v$. Thus, $||z||^2 = ||z^{*} z|| = 0$, so $||z|| = 0$ and $v = v v^{*} v$, as desired.
  \end{proof}

\noindent Now, another lemma:

\begin{lemma}
  Suppose that $\{f_{ii}^{(k)} \ | \ 1 \leq k \leq r, 1 \leq i \leq n_k\}$ is a set of mutually orthogonal projections in $C^{*}$-algebra $B$ and that
  \begin{equation}
    f_{11}^{(k)} \sim f_{22}^{(k)} \sim \cdots \sim f_{n_k n_k}^{(k)}
  \end{equation}
  for each $k$. Then there is a system of matrix units $\{f_{ij}^{(k)}\}$ extending $\{f_{ii}^{(k)}\}$.
\end{lemma}

\noindent The idea behind constructing systems of matrix units is, essentially, to have a ``basis'' for each component of the direct sum
that we will eventually demonstrate characterizes the $C^{*}$-algebra $B$. Each of the sets $\{f_{ii}^{(k)}\}$ are analogous to matrix projections
with $1$ at the $i$-th slot on the diagonal, at the $k$-th slot in the direct sum. Let us now prove the lemma.

\begin{proof}
  Of course, here, we will make us of the Murray-von Neumann equivalence. Namely,
  \begin{equation}
    f_{11}^{(k)} \sim f_{jj}^{(k)} \Longrightarrow f_{11}^{(k)} = f_{1j}^{(k)} f_{1j}^{(k)^{*}} \ \ \ \text{and} \ \ \ f_{jj}^{(k)} = f_{1j}^{(k)^{*}} f_{1j}^{(k)}
  \end{equation}
  This notation is consistent, as $f_{11}^{(k)}$ is self-adjoint, so setting $j = 1$ above causes no problems.
  Our claim is that if we set $\widetilde{f}_{ij}^{(k)} = f_{1i}^{(k)^{*}} f_{1j}^{(k)}$ then we will have the desired system of matrix units. This is in fact an extension of the system we are already provided. Namely, we have
  \begin{equation}
    \widetilde{f}_{jj}^{(k)} = f_{1j}^{(k)^{*}} f_{1j}^{(k)} = f_{jj}^{(k)}
  \end{equation}
  by definition. In fact, we might as well denote $\widetilde{f}_{ij}$ by $f_{ij}$, as for $i = 1$, we have
  \begin{equation}
    \widetilde{f}_{1j} = f_{11}^{(k)^{*}} f_{1j}^{(k)} = f_{1j}^{(k)} f_{1j}^{(k)^{*}} f_{1j}^{(k)} = f_{1j}^{(k)}
    \end{equation}
  where we use the above lemma and the fact that $f_{1j}^{(k)^{*}} f_{1j}^{(k)} = f_{jj}^{(k)}$ is a projection. Let us now complete our verification. Of course, we have
  \begin{equation}
    f_{pq}^{(k)} f_{qr}^{(k)} = f_{1p}^{(k)^{*}} f_{1q}^{(k)} f_{1q}^{(k)^{*}} f_{1r}^{(k)} = f_{1p}^{(k)^{*}} f_{11}^{(k)} f_{1r}^{(k)} = f_{1p}^{(k)^{*}} f_{1r}^{(k)} f_{1r}^{(k)^{*}} f_{1r}^{(k)} = f_{1p}^{(k)^{*}} f_{1r}^{(k)} = f_{pr}^{(k)}
  \end{equation}
  where we use the first lemma to note that $f_{1r}^{(k)} f_{1r}^{(k)^{*}} f_{1r}^{(k)} = f_{1r}^{(k)}$, as $f_{1r}^{(k)^{*}} f_{1r}^{(k)} = f_{rr}^{(k)}$ is a projection. Next, note that
  \begin{equation}
    f_{pq}^{(k)} f_{rs}^{(\ell)} = f_{1p}^{(k)^{*}} f_{1q}^{(k)} f_{1r}^{(\ell)^{*}} f_{1s}^{(\ell)}
  \end{equation}
  Once again using the first lemma, we have $f_{1q}^{(k)} = f_{1q}^{(k)} f_{1q}^{(k)^{*}} f_{1q}^{(k)} =  f_{1q}^{(k)}  f_{qq}^{(k)}$ and $f_{1r}^{(\ell)} = f_{1r}^{(\ell)} f_{1r}^{(\ell)^{*}} f_{1r}^{(\ell)} =  f_{1r}^{(\ell)} f_{rr}^{(\ell)}$
  so that $f_{1r}^{(\ell)^{*}} = f_{rr}^{(\ell)} f_{1r}^{(\ell)^{*}}$. We then use the fact that the projections in our set are mutually orthogonal to conclude that
  \begin{equation}
    f_{1p}^{(k)^{*}} f_{1q}^{(k)} f_{1r}^{(\ell)^{*}} f_{1s}^{(\ell)} = f_{1p}^{(k)^{*}} f_{1q}^{(k)} (f_{qq}^{(k)}  f_{rr}^{(\ell)}) f_{1r}^{(\ell)^{*}} f_{1s}^{(\ell)} = 0
  \end{equation}
  which is $0$ when $q \neq r$ or $k \neq \ell$, as in these cases, $f_{qq}^{(k)}  f_{rr}^{(\ell)} = 0$. It is very immediately clear that $f_{ij}^{(k)^{*}} = f_{1j}^{(k)^{*}} f_{1i}^{(k)} = f_{ji}^{(k)}$,
  so we have verified the third condition, and it follows that our set of $f_{ij}^{(k)}$ is in fact a system of matrix units in $B$ extending $\{f_{ii}^{(k)}\}$.
\end{proof}

\begin{comment}
\section{Secondary technical lemmas: masas}

\noindent Now, we prove a handful of technical results about masas (maximal Abelian sub-algebras).

\begin{lemma}
  A sub-$C^{*}$-algebra $D$ of a $C^{*}$-algebra $A$ is a masa if and only if $D' = D$, where $D'$ is the set of all elements in $A$ for such $da = ad$ for all $d \in D$.
\end{lemma}
\begin{proof}
  We suppose first that $D$ is a masa. Note that $D'$ clearly contains $D$, as every element of $D$ commutes with every other element of $D$, since $D$ is Abelian.
  Suppose $y \in D'$ and $y \notin D$. Then $yd = dy$ for all elements of $D$. It follows that $C(D \cup \{y\})$ is an Abelian sub-$C^{*}$-algebra which contains $D$ properly,
  a clear contradiction, so we must have $D' \subset D$. We have inclusion both ways, so $D = D'$.

  Conversely, suppose $D = D'$. Pick $x, y \in D$. Note that $x \in D'$ commutes with every element of $D$, so $xy = yx$. Thus, $D$ is an Abelian sub-$C^{*}$-algebra. Via Zorn's lemma,
  we know that every Abelian sub-$C^{*}$-algebra is in a masa. Let $F$ be a masa containing $D$. Given some $f \in F$, note that $f$ will commute with every element of $D$, so that $f \in D' = D$. Thus,
  we must have $D = F$, and the proof is complete.
\end{proof}

\noindent We have to prove one more lemma:

\begin{lemma}
  Let $D$ be a masa in a $C^{*}$-algebra $A$, then
  \begin{enumerate}
  \item If $a$ is an element in $A$ that commutes with every element in $D$, then $a$ belongs to $D$.
  \item If $D$ is unital, then $A$ is unital, and the unit of $D$ is equal to the unit of $A$.
  \item If $p$ is a projection in $D$ satisfying $pDp = \mathbb{C}p$, then $pAp = \mathbb{C}p$. In other words,
    a minimal projection in $D$ is also a minimal projection in $A$.
    \end{enumerate}
\end{lemma}
\begin{proof}
  The first claim follows trivially from the proof of the previous lemma. Suppose $1_D$ is a unit in $D$. Then
  \end{proof}

\section{Main result}

\noindent With our preliminary lemmas, we can now attempt to prove the main result.

\begin{theorem}

\end{theorem}
\end{comment}

\section{RLL Problem 7.4 (Suggested Problem 1)}

\noindent \textbf{Part 1.} Suppose $A$ and $B$ both have the cancellation property, which means that the semigroups $\mathcal{D}(A)$ and $\mathcal{D}(B)$ have the cancellation property.
Equivalently, some $X$ has the cancellation property if andf only if, for each $p, q \in \mathcal{P}_{\infty}(X)$, then
\begin{equation}
  [p]_0 = [q]_0 \Longleftrightarrow p \sim_0 q
\end{equation}
Recall from earlier in RLL that if $i_A$ and $i_B$ are canonical inclusion maps of $A$ and $B$ into $A \oplus B$, then $\Phi = K_0(i_A) \oplus K_0(i_B)$ is a group isomorphism.
Suppose $p, q \in \mathcal{P}_{\infty}(A \oplus B)$. Of course, $p \sim_0 q$ implies $[p]_{\mathcal{D}} = [q]_{\mathcal{D}}$, which implies that $[p]_{0} = [q]_{0}$.

Conversely, suppose
that $[p]_0 = [q]_0$, where $p, q$ are projections in $\mathcal{P}_{\infty}(A \oplus B)$ It is easy to see in this case that $p = p_1 \oplus p_2$ and $q = q_1 \oplus q_2$ for $p_1, p_2 \in \mathcal{P}_{\infty}(A)$ and $q_1, q_2 \in \mathcal{P}_{\infty}(B)$.
Then note that $\Phi^{-1}([p]_0) = [p_1]_0 \oplus [p_2]_0 = \Phi^{-1}([q]_0) = [q_1]_0 \oplus [q_2]_0$, so that $[p_1]_0 = [q_1]_0$ in $K_0(A)$ and $[p_2]_0 = [q_2]_0$ in $K_0(B)$. Thus, since
we have the canellation property in these algebras, $p_1 \sim_0 q_1$ and $p_2 \sim_0 q_2$. It follows that $p_1 \oplus p_2 = p \sim_0 q_1 \oplus q_2 = q$, clearly. This completes the proof.
\newline

\noindent \textbf{Part 2.} Suppose the sequence of $C^{*}$-algebras has the the cancellation property. We will use continuity of $K_0$ to show that their inductive limit has the cancellation property.
In particular, note that
\begin{equation}
  \lim_{\longrightarrow} K_0(A_n) \simeq K_0(\lim_{\longrightarrow} A_n) = K_0(A)
\end{equation}
where the isomorphism is clearly of Abelian groups. Suppose $p, q \in \mathcal{P}_{\infty}(A)$ with $[p]_0, [q]_0 \in K_0(A)$ are such that $[p]_0 = [q]_0$.
Recall that from the continuity, we know that:
\begin{equation}
  K_0(A) = \displaystyle\bigcup_{n = 1}^{\infty} K_0(\mu_n)(K_0(A_n))
\end{equation}
so it follows that we can writen $[p]_0 = [\mu_n(p_n)]_0$ and $[q]_0 = [\mu_m(q_m)]_0$ for some $n$ and $m$, as well as $p_n \in \mathcal{P}_{\infty}(A_n)$ and $q_m \in \mathcal{P}_{\infty}(A_m)$,
where $(A, \{\mu_n\})$ is ther inductive limit of the algebras. Without loss of generality, we can assume $m = n$, as if $m < n$ for example, we can always note that
\begin{equation}
  \mu_n(p_n) = (\varphi_{n, m} \circ \mu_m)(p_n)
\end{equation}
via the connection maps. Thus, we will have $K_0(\mu_n)([p_n]_0 - [q_m]_0)$. Again from continuity, we know that $[p_n]_0 - [q_m]_0$ must be in the kernel of some connection map $\Phi = K_0(\varphi_{a, n})$,
so that $[\Phi(p_n)]_0 = [\Phi(q_m)]_0$. Since each algebra $A_n$ has the cancellation property, $\Phi(p_n) \sim_0 \Phi(q_m)$. Thus, $\mu_n(p_n) \sim_0 \mu_n(q_m)$, which implies that $p \sim_0 q$, and we have the
desired cancellation property.
\newline

\noindent \textbf{Part 3.} Let us prove that any matrix algebra over $\mathbb{C}$ has the cancellation property. If we can do this, then
since AF algebras are simply direct limits of direct sums of matrix algebras, we will have proved that all AF algebras have the cancellation property.
\newline

\begin{lemma}
  The algebra $M_n(\mathbb{C})$ has the cancellation property.
\end{lemma}

\begin{proof}
  Recall that if $p, q \in M_n(\mathbb{C})$ are projections, then $p \sim q$ if and only if $\text{Tr}(p) = \text{Tr}(q)$. Moreover, recall that the trace map
  induces a map $K_0(\text{Tr}) : K_0(M_n(\mathbb{C})) \rightarrow \mathbb{Z}$ which is in fact a group isomorphism. Given $[p]_0 = [q]_0$, it follows immediately that
  $$K_0(\text{Tr})([p]_0) = \text{Tr}(p) = K_0(\text{Tr})([q]_0) = \text{Tr}(q)$$
  Note that $p, q \in \mathcal{P}_{\infty}(M_n(\mathbb{C}))$, which means that they are projections in some $M_{n_{1}}(\mathbb{C})$ and $M_{n_{2}}(\mathbb{C})$ repsectivelty. Without
  loss of generality, suppose $n_1 \geq n_2$. We have $\text{Tr}(q) = \text{Tr}(q \oplus 0_{n_{1} - n_{2}})$, so $q \sim q \oplus 0_{n_{1} - n_{2}}$ from the first sentence,
  which means that $p \sim_0 q$. This completes the proof: the complex matrix algebra has the cancellation property.
\end{proof}

\noindent To reiterate, if we have an AF algebra $A$, then
$$A = \lim_{\longrightarrow} A_n = \lim_{\longrightarrow} M_{j(n)_1}(\mathbb{C}) \oplus \cdots \oplus M_{j(n)_k}(\mathbb{C})$$
Since each $M_{j(n)_k}$ has the cancellation property, so does each direct sum, and thus so too does the direct limit, implying $A$ has the cancellation property.

\end{document}
