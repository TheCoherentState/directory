\documentclass[aps,pra,showpacs,notitlepage,onecolumn,superscriptaddress,nofootinbib]{revtex4-1}
\usepackage[utf8]{inputenc}
\usepackage[tmargin=1in, bmargin=1.25in, lmargin=1.5in, rmargin=1.5in]{geometry}
\usepackage{amsmath, amssymb, amsthm}
\usepackage{graphicx}
\usepackage{xcolor}
\usepackage{enumitem}
\usepackage{datetime}
\usepackage{hyperref}
\usepackage{titlesec}
\usepackage{import}
\usepackage{mathtools}
\usepackage{thmtools,thm-restate}
\usepackage{comment}


% package for commutative diagrams
% \usepackage{tikz-cd}

%%%%%%%%%%%%%%%%%%%%%%%%%%%%%%%%%%%%%%%%%%%%%
\definecolor{crimson}{RGB}{186,0,44}
\definecolor{moss}{RGB}{0, 186, 111}
\newcommand{\pop}[1]{\textcolor{crimson}{#1}}
\newcommand{\zcom}[1]{\noindent\textcolor{crimson}{(Z): #1}}
\newcommand{\jcom}[1]{\noindent\textcolor{moss}{(J): #1}}
\newcommand{\wt}[1]{\widetilde{#1}}
\newcommand{\pqeq}{\succcurlyeq}
\newcommand{\pleq}{\preccurlyeq}
\newcommand{\Wedge}{\bigwedge}

%%%%%%%%%%%%%%%%%%%%%%%%%%%%%%%%%%%%%%%%%%%%%
\hypersetup{
    colorlinks,
    linkcolor={crimson},
    citecolor={crimson},
    urlcolor={crimson}
}

\usepackage{qcircuit}

%%%%%%%%%%%%%%%%%%%%%%%%%%%%%%%%%%%%%%%%%%%%%
\theoremstyle{definition}
\newtheorem{definition}{Definition}[section]
\newtheorem{lemma}{Lemma}[section]
\newtheorem{theorem}{Theorem}[section]
\newtheorem{corollary}{Corollary}[theorem]
\newtheorem*{theorem*}{Theorem}
\newtheorem*{corollary*}{Corollary}
\newtheorem{remark}{Remark}[section]
\newtheorem{conjecture}{Conjecture}[section]
\newtheorem{example}{Example}[section]
\newtheorem{reminder}{Reminder}[section]
\newtheorem{problem}{Problem}[section]
\newtheorem{question}{Question}[section]
\newtheorem{answer}{Answer}[section]
\newtheorem{fact}{Fact}[section]
\newtheorem{claim}{Claim}[section]

\newcommand{\hhrulefill}{\hspace{-1.5em} \hrulefill}

\usepackage{geometry}
\geometry{
  left=25mm,
  right=25mm,
  top=20mm,
}

%%%%%%%%%%%%%%%%%%%%%%%%%%%%%%%%%%%%%%%%%%%%%
\bibliographystyle{unsrt}

%%%%%%%%%%%%%%%%%%%%%%%%%%%%%%%%%%%%%%%%%%%%%
%%%%%%%%%%%%%%%%%%%%%%%%%%%%%%%%%%%%%%%%%%%%%
%%%%%%%%%%%%%%%%%%%%%%%%%%%%%%%%%%%%%%%%%%%%%
\begin{document}

\title{MAT437 problem set 11}
\author{Jack Ceroni}
\email{jackceroni@gmail.com}

\date{\today}

\maketitle

\section{RLL Problem 9.3}

\noindent \textbf{Part 1.} By Stone-Weirestrass theorem, for some continuous function $f : \mathbb{R}^{+} \rightarrow \mathbb{C}$, we can approximate $f$ to arbitrary precision
with a polynomial, $p$. In other words, we can choose $p$ such that $|f(x) - p(x)| < \varepsilon$ for self-adjoint $x \in A$ with $||x|| \leq M$ (which holds, as we are given $||x|| \leq 1$).
For some $a \in A$, $a a^{*}$ and $a^{*} a$ are self-adjoint. We then note that
\begin{align}
  \left| a f(a^{*} a) - f(a a^{*}) a \right| & \leq \left| a f(a^{*} a) - a p(a^{*} a) \right| + \left| a p(a^{*} a) - p(a a^{*}) a \right| + \left| p(a a^{*}) a - f(a a^{*}) a \right|
  \\ & \leq 2 ||a|| \varepsilon + \left| a p(a^{*} a) - p(a a^{*}) a \right|
  \\ & \leq 2 \varepsilon + \left| \displaystyle\sum_{k = 0}^{n} p_k a (a^{*} a)^k - \displaystyle\sum_{k = 0}^{n} p_k (a a^{*})^k a \right|
  \\ & = 2 \varepsilon + \left| \displaystyle\sum_{k = 0}^{n} p_k a (a^{*} a)^k - \displaystyle\sum_{k = 0}^{n} p_k a (a^{*} a)^k \right| = 2\varepsilon
\end{align}
Thus, for any $\varepsilon > 0$, we have $ \left| a f(a^{*} a) - f(a a^{*}) a \right| < \varepsilon$, so it follows that this difference must be $0$, and we have $a f(a^{*} a) = f(a a^{*}) a$, as desired.
\newline

\noindent From here, it follows immediately that if $f(x) = (1 - x)^{1/2}$, we have $a (1 - a^{*} a)^{1/2} = (1 - a a^{*})^{1/2} a$. Let
\begin{equation}
  v = \begin{pmatrix} a &  (1 - a a^{*})^{1/2} \\ -(1 - a^{*} a)^{1/2} & a^{*} \end{pmatrix} \ \ \ \ \text{so that} \ \ \ \ v^{*} = \begin{pmatrix} a^{*} &   -(1 - a^{*} a)^{1/2} \\ (1 - a a^{*})^{1/2}  & a \end{pmatrix}
\end{equation}
which immediately gives
\begin{equation}
  vv^{*} = \begin{pmatrix} 1 & f(a a^{*}) a - a f(a^{*} a) \\ a^{*} f(a a^{*}) - f(a^{*} a) a^{*} \\ 1 \end{pmatrix} = \begin{pmatrix} 1 & 0 \\ 0 & 1 \end{pmatrix}
\end{equation}
where we note that $a^{*} f(a a^{*}) = (f(a a^{*}) a)^{*} = (a f(a^{*} a))^{*} = f(a^{*} a) a^{*}$. It is easy to see, via identical logic that $v^{*} v$ is also the identity,
so that $v$ is a unitary element, as desired.
\newline

\noindent \textbf{Part 2.} Given an ideal $I$ in $A$, we let $\pi : A \rightarrow A/I$ be the quotient map. We let $u$ be a unitary in $A/I$. We know, from Rordam,
that there is an element $a$ in $A$ such that $||a|| = 1$ and $\pi(a) = u$. We define $v$ as in Part 1. We have
\begin{equation}
  \pi(v) =  \begin{pmatrix} \pi(a) &  \pi((1 - a a^{*})^{1/2}) \\ \pi(-(1 - a^{*} a)^{1/2}) & \pi(a^{*}) \end{pmatrix} =  \begin{pmatrix} u &  \pi((1 - a a^{*})^{1/2}) \\ -\pi((1 - a^{*} a)^{1/2}) & u^{*} \end{pmatrix}
\end{equation}
Since $u = u_0 + I$ is unitary in $A/I$, it follows that $u^{*} u = u_0^{*} u_0 + I = 1 + I$. Similarly, $u u^{*} = u_0 u_0^{*} + I = 1 + I$. Thus,
both $1 - u_0^{*} u_0$ and $1 - u_0 u_0^{*}$ are in $I$, so that
\begin{equation}
  (1 - u_0^{*} u_0) + I = (1 - u_0 u_0^{*}) + I = 0
\end{equation}
In addition, via the same logic as before (Stone-Weirestrass), continuous functions of these elements will be equal to $0$ as well:
\begin{equation}
  g(1 - u_0^{*} u_0) + I = g(1 - u_0 u_0^{*}) + I = 0
\end{equation}
Thus, $\pi( (1 - a^{*} a)^{1/2}) = (1 - u_0^{*} u_0)^{1/2} + I = 0$, and the same holds for $(1 - a a^{*})^{1/2}$. Thus,
\begin{equation}
  \begin{pmatrix} u &  \pi((1 - a a^{*})^{1/2}) \\ -\pi((1 - a^{*} a)^{1/2}) & u^{*} \end{pmatrix} = \begin{pmatrix} u &  0 \\ 0 & u^{*} \end{pmatrix}
\end{equation}
as desired.
\newline

\noindent \textbf{Part 3.} We let $\delta_1 : K_1(A/I) \rightarrow K_0(I)$ be the index map associated with
\begin{equation}
  0 \longrightarrow I \xrightarrow[]{\text{inclusion}} A \xrightarrow[]{\pi} A/I \longrightarrow 0
\end{equation}
We let $u$ be some unitary in $A/I$. Let us recall the standard picture of the index map: if we have a
short exact sequence of the form
\begin{equation}
  0 \longrightarrow I \xrightarrow[]{\varphi} A \xrightarrow[]{\psi} B \longrightarrow 0
\end{equation}
and we have $u \in \mathcal{U}_n(\widetilde{B})$, $v \in \mathcal{U}_{2n}(\widetilde{A})$ and $p \in \mathcal{P}_{2n}(\widetilde{I})$ satifying
\begin{equation}
  \widetilde{\varphi}(p) = v \begin{pmatrix} 1_n & 0 \\ 0 & 0 \end{pmatrix} v^{*} \ \ \ \ \text{and} \ \ \ \ \widetilde{\psi}(v) = \begin{pmatrix} u & 0 \\ 0 & u^{*} \end{pmatrix}
\end{equation}
then $\delta_1([u]_1) = [p]_0 - [s(p)]_0$. It follows that if we choose $a$ as before, with $\pi(a) = u$, then we have already shown
that the latter condition holds for the quotient map $\pi$. We should set
\begin{align}
  p = j(p) &= v \begin{pmatrix} 1_n & 0 \\ 0 & 0 \end{pmatrix} v^{*} = \begin{pmatrix} a &  (1 - a a^{*})^{1/2} \\ -(1 - a^{*} a)^{1/2} & a^{*} \end{pmatrix} \begin{pmatrix} 1_n & 0 \\ 0 & 0 \end{pmatrix} \begin{pmatrix} a^{*} &   -(1 - a^{*} a)^{1/2} \\ (1 - a a^{*})^{1/2}  & a \end{pmatrix}
  \\ & = \begin{pmatrix} a &  (1 - a a^{*})^{1/2} \\ -(1 - a^{*} a)^{1/2} & a^{*} \end{pmatrix} \begin{pmatrix} a^{*} &   -(1 - a^{*} a)^{1/2} \\ 0  & 0 \end{pmatrix}
  \\ & = \begin{pmatrix} a a^{*} & -a(1 - a^{*} a)^{1/2} \\ -(1 - a^{*} a)^{1/2} a^{*} & 1 - a^{*} a \end{pmatrix}
\end{align}
We can easily verify that this is a projection: indeed it is self-adjoint, and moreover,
\begin{align}
  p^2 &= \begin{pmatrix} a a^{*} & -a(1 - a^{*} a)^{1/2} \\ -(1 - a^{*} a)^{1/2} a^{*} & 1 - a^{*} a \end{pmatrix}^2
  \\ & = \begin{pmatrix} a a^{*} a a^{*} + a(1 - a^{*} a) a^{*} & -aa^{*} a (1 - a^{*} a)^{1/2} - a(1 - a^{*} a)^{3/2} \\ -(1 - a^{*} a)^{1/2} a^{*} a a^{*} - (1 - a^{*} a)^{3/2} a^{*} & (1 - a^{*} a)^{1/2} a^{*} a (1 - a^{*} a)^{1/2} + (1 - a^{*} a)^{2} \end{pmatrix}
  \\ & = \begin{pmatrix} a a^{*} & -a(1 - a^{*} a)^{1/2} \\ -(1 - a^{*} a)^{1/2} a^{*} & 1 - a^{*} a \end{pmatrix} = p
\end{align}
Thus, this is precisely the desired projection: the index map is given by $[\delta_1(u)]_0 = [p] - [s(p)]_0$. Recall, from RLL, that $s(p) = \text{diag}(1, 0)$,
so
\begin{equation}
  [\delta_1(u)]_0 = [p] - [s(p)]_0 = \left[ \begin{pmatrix} a a^{*} - 1 & -a(1 - a^{*} a)^{1/2} \\ -(1 - a^{*} a)^{1/2} a^{*} & 1 - a^{*} a \end{pmatrix} \right]_0.
  \end{equation}
\newline

\noindent \textbf{Part 4.} We once again let $u$ be a unitary in $A/I$, and let $a$ be the lift of $u$ with $||a|| = 1$. We let $v$ be the partial isometry in $M_2(A)$
such that $v$ lifts $\text{diag}(u, 0)$ (we know this exists from RLL). We know the explicit form of $v$, from the proof of Lemma 9.2.1 in RLL:
\begin{equation}
  v = \begin{pmatrix} a & 0 \\ (1 - a^{*} a)^{1/2} & 0 \end{pmatrix}
\end{equation}
It follows that we can easily compute $p = j(p) = 1 - v^{*} v$ and $q = j(q) = 1 - v v^{*}$ (where $j$ is the inclusion map). In particular,
\begin{align}
  p = 1 - v^{*} v &= \mathbb{I} - \begin{pmatrix} a^{*} & (1 - a^{*} a)^{1/2} \\ 0 & 0 \end{pmatrix} \begin{pmatrix} a & 0 \\ (1 - a^{*} a)^{1/2} & 0 \end{pmatrix}
  \\ &= \mathbb{I} - \begin{pmatrix} a^{*} a + (1 - a^{*} a) & 0 \\ 0 & 0 \end{pmatrix} = \begin{pmatrix} 0 & 0 \\ 0 & 1 \end{pmatrix}
\end{align}
and, in addition,
\begin{align}
  q = 1 - v v^{*} &= \mathbb{I} - \begin{pmatrix} a & 0 \\ (1 - a^{*} a)^{1/2} & 0 \end{pmatrix} \begin{pmatrix} a^{*} & (1 - a^{*} a)^{1/2} \\ 0 & 0 \end{pmatrix}
  \\ &= \mathbb{I} - \begin{pmatrix} a a^{*} & a (1 - a^{*} a)^{1/2} \\ (1 - a^{*} a)^{1/2} a^{*} & 1 - a^{*} a \end{pmatrix} =  \begin{pmatrix} 1 - a a^{*} & a (1 - a^{*} a)^{1/2} \\ (1 - a^{*} a)^{1/2} a^{*} & a^{*} a \end{pmatrix} 
  \end{align}
Recall the alternative definition of the index map, $\delta_1([u]_1) = [p]_0 - [q]_0$. From the above calculations, we have
\begin{align}
  [p]_0 - [q]_0 &= \left[ \begin{pmatrix} 0 & 0 \\ 0 & 1 \end{pmatrix} \right]_0 - \left[ \begin{pmatrix} 1 - a a^{*} & a (1 - a^{*} a)^{1/2} \\ (1 - a^{*} a)^{1/2} a^{*} & a^{*} a \end{pmatrix}  \right]_0
  \\ & = \left[ \begin{pmatrix} a a^{*} - 1 & -a(1 - a^{*} a)^{1/2} \\ -(1 - a^{*} a)^{1/2} a^{*} & 1 - a^{*} a \end{pmatrix} \right]_0
\end{align}
which agrees with the previously derived value of $[\delta_1(u)]_0$.

\section{Problem Set 10 Suggested Problem 4 (RLL Problem 10.1)}

\noindent \textbf{Part 1.} We let $\mathbb{T}A = C(\mathbb{T}, A)$. We must construct a split exact sequence
\begin{equation}
  0 \longrightarrow SA \longrightarrow \mathbb{T} A \longleftrightarrow A \longrightarrow 0
\end{equation}
Recall that $SA$ is the suspension of $A$, $SA = \{f \in C([0, 1], A) \ | \ f(0) = f(1) = 0\}$. Of course, a map
is this form can be naturally sent to an element of $\mathbb{T} A$: define $\phi : SA \rightarrow \mathbb{T} A$ as
\begin{equation}
  \phi(f)(e^{2 \pi i \theta}) = f \left( \theta \right)
\end{equation}
If course, this is an injective $*$-homomorphism. Now, let us define $\pi : \mathbb{T} A \rightarrow A$ as $\pi(f) = f(1)$.
Of course, this is a surjective $*$-homomorphism, as we can always find some $f \in \mathbb{T} A$ whose value at $1 \in \mathbb{T}$
is any $a \in A$ that we desire.
\newline

\noindent We must now show that this sequence is exact, and that it splits (i.e. $\pi$ has a right-inverse $*$-homomorphism). It is easy
to check that it is exact, note that $\text{Ker}(\pi)$ is precisely all $f \in \mathbb{T} A$ such that $f(1) = 0$. In addition,
note that $\text{Im}(\phi)$ is precisely all maps of the unit circle into $A$ (which can all be written as $g(\theta) = f(e^{2 \pi i \theta})$),
such that $g(0) = g(1) = f(1) = 0$, by definition of $SA$. Thus, $\text{Im}(\phi) = \text{Ker}(\pi)$, and the sequence of exact.
\newline

\noindent To show that it splits, define $\lambda : A \rightarrow \mathbb{T} A$ as $\lambda(a)(e^{2 \pi i \theta}) = a$ for all $\theta$.
Note that $(\pi \circ \lambda)(a) = \lambda(a)(1) = a$ for all $a$, so $\lambda$ is a right-inverse for $\pi$ and the sequence splits.
\newline

\noindent \textbf{Part 2.} Because the above sequence of split exact, it follows immediately that mapping everything under the $K_n$-functor
will yield a split exact sequence as well. In particular, it follows that
\begin{equation}
  K_n(\mathbb{T} A) \simeq K_n(SA) \oplus K_n(A)
\end{equation}
By Bott periodicity, we know that $K_{n + 1}(A) \simeq K_n(SA)$. Thus, $K_n(\mathbb{T} A) \simeq K_{n + 1}(A) \oplus K_n(A)$ as desired.
\newline

\noindent \textbf{Part 3.} To start, we must show that $\mathbb{T}^n \mathbb{C}$ is isomorphic to $C(\mathbb{T}^n)$. In the case of $n = 1$,
we have $\mathbb{T}^n \mathbb{C} = \mathbb{T} \mathbb{C} = C(\mathbb{T}, \mathbb{C}) = C(\mathbb{T})$. Let us assume the claim
holds for the case of $n - 1$. For the case of $n$, we have
\begin{equation}
  \mathbb{T}^n \mathbb{C} = \mathbb{T}(\mathbb{T}^{n - 1} \mathbb{C}) \simeq \mathbb{T} C(\mathbb{T}^{n - 1}) = C(\mathbb{T}, C(\mathbb{T}^{n - 1})) \simeq C(\mathbb{T}^n)
\end{equation}
and we are done: the claim holds by induction. It follows immediately from this fact and Part 2 that we have expressions for $K_0(C(\mathbb{T}^n))$ and $K_1(C(\mathbb{T}^n))$.
In particular, we note that for some $m$,
\begin{equation}
  K_m(C(\mathbb{T}^n)) \simeq K_m(\mathbb{T}^n \mathbb{C}) = K_{m}(\mathbb{T}^{n - 1} \mathbb{C}) \oplus K_{m + 1}(\mathbb{T}^{n - 1} \mathbb{C}) \simeq K_m(C(\mathbb{T}^{n - 1})) \oplus K_{m + 1}(C(\mathbb{T}^{n - 1}))
\end{equation}
It follows that we have a recursive relation for $K_{m}(C(\mathbb{T}^{n}))$ for some $n$. If we repeatedly use this recursion, we will eventually be able to express $K_m(C(\mathbb{T}^n))$
as a direct sum of $K_m(\mathbb{C})$, for $m \in \mathbb{Z}^{+} \cup \{0\}$. In particular, it is easy to see (via induction) that
\begin{equation}
  K_m((C(\mathbb{T}^n))) \simeq K_{m}(\mathbb{C}) \oplus \left( \bigoplus_{k = 1}^{n} K_{m + k}(\mathbb{C})^{n - k + 1} \right)
\end{equation}
where we are taking the $(n - k + 1)$-fold direct sum/Cartesian product of $K_{m + k}(\mathbb{C})$). It follows immediately that
\begin{equation}
  K_0((C(\mathbb{T}^n))) \simeq K_0(\mathbb{C}) \oplus \left( \bigoplus_{k = 1}^{n} K_{k}(\mathbb{C})^{n - k + 1} \right)
\end{equation}
and
\begin{equation}
  K_1((C(\mathbb{T}^n))) \simeq K_1(\mathbb{C}) \oplus \left( \bigoplus_{k = 1}^{n} K_{k + 1}(\mathbb{C})^{n - k + 1} \right).
\end{equation}

\section{RLL Problem 11.6}

\noindent We have
\begin{equation}
  a = \begin{pmatrix} 0 & 1 \\ 0 & 0 \end{pmatrix}.
\end{equation}
Note that $a$ is nilpotent with $a^2 = 0$. Thus, for all $k \geq 2$, $a^k = 0$. Since $f \in \mathcal{H}(\Omega)$ (it is holomorphic in the domain $\Omega$), it has a power series development about $0$ (as $\text{sp}(a) = \{0\}$, so
we take $\Omega$ a neighbourhood about $0$) $f(z) = \sum_{k = 0}^{\infty} \frac{1}{k!} f^{(k)}(0) z^k$. It follows from the holomorphic function calculus that
\begin{equation}
  f(a) = \sum_{k = 0}^{\infty} \frac{1}{k!} f^{(k)}(0) a^k = f(0) \mathbb{I} + f'(0) a = \begin{pmatrix} f(0) & f'(0) \\ 0 & f(0) \end{pmatrix}
\end{equation}
as all the higher-order terms vanish, and we are done.

\end{document}
