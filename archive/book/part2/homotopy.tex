\documentclass[10pt, oneside]{amsart} 
\usepackage{amsmath, amsthm, amssymb, calrsfs, wasysym, verbatim, bbm, color, graphics, geometry, hyperref, biblatex, mathtools}
\usepackage[framemethod=TikZ]{mdframed}
\usepackage{tcolorbox}

\hypersetup{
	colorlinks=true,
	linkcolor=blue,
	urlcolor=blue
}

\addbibresource{ref.bib}

\geometry{tmargin=1.25in, bmargin=1.25in, lmargin=1.25in, rmargin =1.25in}

\tcbuselibrary{theorems}

    \newcommand{\R}{\mathbb{R}}
    \newcommand{\C}{\mathbb{C}}
    \newcommand{\Z}{\mathbb{Z}}
    \newcommand{\N}{\mathbb{N}}
    \newcommand{\Q}{\mathbb{Q}}
    \newcommand{\Cdot}{\boldsymbol{\cdot}}

    \newtheorem{definition}{Definition}
    \newtheorem{lemma}{Lemma}
    \newtheorem{theorem}{Theorem}
    \newtheorem{corollary}{Corollary}
    \newtheorem{conjecture}{Conjecture}
    \newtheorem{problem}{Problem}
    \newtheorem{reminder}{Reminder}
    \newtheorem{question}{Question}
    \newtheorem{answer}{Answer}
    \newtheorem{fact}{Fact}
    \newtheorem{claim}{Claim}
    \newtheorem{proposition}{Proposition}

    \theoremstyle{definition}
    \newtheorem{example}{Example}
    \newtheorem{remark}{Remark}

    \newcommand{\Tr}{\mathrm{Tr}}
    \DeclarePairedDelimiter\floor{\lfloor}{\rfloor}


 \title{Chapter 2.1: Basics of homotopy}
 \author{Jack Ceroni}
 \email{jackceroni@gmail.com}
 \date{\today}

\begin{document}

\maketitle

\tableofcontents

\section{Introduction}

\noindent One of the central concepts in algebraic topology is that of \emph{homotopy}: a continuous deformation of one continuous map into another.
The notion of homotopy will allow us to define notions of ``equivalence'' for topological spaces that are weaker than homeomorphism, but still
imply preservation of many nice properties, such as the \emph{fundamental group}: something of fundamental (pun intended) importance that will be discussed
in a later chapter.

\section{Homotopy}

\noindent The definition of homotopy is pretty much what was stated above:

\begin{definition}[Homotopy]
  A homotopy is a continuous map of the form $F : X \times [0, 1] \rightarrow Y$, where $X$ and $Y$ are topological spaces. Maps $f, g : X \rightarrow Y$
  are said to be \emph{homotopic} if there exists a homotopy between them, that is, some homotopy $F : X \times [0, 1] \rightarrow Y$ such that $f(x) = F(x, 0)$
  and $g(x) = F(x, 1)$. If $f$ and $g$ are homotopic, we use notation $f \simeq g$. It is easy to check that homotopy is an equivalence relation.
\end{definition}

\begin{example}[Path homotopy]
  We will often be interested in homotopy of paths (i.e. when $X = [0, 1]$, and we have two paths $f, g : [0, 1] \rightarrow Y$). Such a homotopy
  will be a continuous map of the form $F : [0, 1] \times [0, 1] \rightarrow Y$ with $F(x, 0) = f(x)$ and $F(x, 1) = g(x)$. We say that
  two paths $f$ and $g$ which share beginning and endpoints are \emph{path homotopic} if they are homotopic with a homotopy which fixes the endpoints
  (i.e. $F(0, t) = f(0) = g(0)$ and $F(1, t) = f(1) = g(1)$ for all $t \in [0, 1]$).
\end{example}

\begin{remark}[Relative homotopy]
  More generally, a homotopy $F$ which fixes some $A \subset X$ is said to be a \emph{homotopy relative to $A$}. A path homotopy is a homotopy relative to
  the two-point set $\{0, 1\} \subset [0, 1]$.
\end{remark}

\noindent Note that a map $f : X \rightarrow Y$ that is homotopic to a constant map is said to be \emph{nullhomotopic}.

\section{Deformation retracts and the mapping cylinder}

\noindent We will dedicate the first part of this section to discussion of a particular, important kind of homotopy: a deformation retract.

\begin{definition}[Retraction]
  A continuous map $r : X \rightarrow A$ is said to be a \emph{retraction} of $X$ onto some subspace $A \subset X$ if $r|_{A} = \text{id}$.
\end{definition}

\noindent In this sense, a retraction can be thought of as a map which ``compresses'' a space down to some principal subset.

\begin{definition}
  Given a space $X$ and a subspace $A \subset X$, a deformation retract of $X$ onto $A$ is a homotopy relative to $A$ of the identity map $\text{id} : X \rightarrow X$
  with a retraction $r$ of $X$ onto $A$ (actually, $j \circ r$, the retraction embedded in the ambient space $X$, where $j : A \rightarrow X$ is inclusion).
  Given a space $X$ and a subspace $A$, if a deformation retract exists, we say that $X$ deformation retracts to $A$.
\end{definition}

\noindent In this sense, a deformation retraction can be thought of a continuous, gradual ``compression'' of a space down to some principal subset.
There is another way in which we can think of deformation retracts, which provides some nice geometric intuition: this is the mapping cylinder.

\begin{definition}[Mapping cylinder]
  Given a continuous map $f : X \rightarrow Y$, we define the mapping cylinder $M_f$ associated with $f$ to be the quotient space
  $(X \times [0, 1]) \sqcup Y / \sim$, where we identify $(x, 1) \sim f(x)$.
\end{definition}

\noindent Intuitively, the points at the ``end'' of the ``cylinder'' $X \times [0, 1]$ are glued to the point in $Y$ to which they are sent by $X$.
We have the following nice result, which makes intuitive sense if we think about ``sliding'' points of $X \times [0, 1]$ along their trajectories,
parameterized by $[0, 1]$, to their corresponding points on $Y$:

\begin{proposition}
  Let $f : X \rightarrow Y$ be a continuous map, let $M_f$ be the mapping cylinder. Then $M_f$ deformation retracts to $Y$.
\end{proposition}
\begin{proof}
  We simply have to write down such a deformation retract. Let $Z = (X \times [0, 1]) \sqcup Y$. We start by defining $F : Z \times [0, 1] \rightarrow M_f$
  as
  \begin{equation}
    F((x, s), t) = [(x, 1 - (1 - s)(1 - t))] = [(x, s + t - st)]
  \end{equation}
  on $X \times [0, 1] \times [0, 1]$ and $F(y, t) = [y]$ on $Y \times [0, 1]$. Note that $[\cdot]$ denotes the equivalence classes of $\sim$. $F$ is of course the
  composition of a continuous map from $Z \times [0, 1]$ to $Z$ with the quotient map $[\cdot] : Z \rightarrow M_f$: both maps are continuous, and thus $F$ is as well.

  By the universal property of the quotient, $F$ descends to a map $F : M_f \rightarrow M_f$ which is also continuous. Moreover, note that $F([(x, s)], 0) = [(x, s)]$,
  so $F(z, 0) = \text{id}(z)$, and in addition,
  \begin{equation}
    F([(x, s)], 1) = [(x, 1)] = [f(x)],
  \end{equation}
  so $F(z, 1)$ takes $M_f$ to the copy of $Y$ embedded in $M_f$. Note that
  $F([(x, 1)], 1) = [(x, 1)]$, so that $F(z, 1)$ is the identity on the embedded copy of $Y$. Finally, note that $F$
  fixes the copy of $Y$ embedded in $M_f$, so it is a homotopy relative to this subset.

  Thus, by definition, $F$ is a deformation retract of $M_f$ to $Y$, and we are done.
\end{proof}

\noindent The previous proof is one of the rare instances where we will take great care to actual explicitly write down a homotopy/check that it is continuous/etc.
Standard practice is algebraic topology is usually to provide diagrammatic justification, and this is good enough (provided you know what you are doing, and
what you are saying is correct).

We know that $M_f$ deformation retracts to $Y$, what about $X$? The property of $M_f$ deformation retracting to $X$ definitely does \textbf{not} hold in general!
  For instance, consider the ``dunce cap'', where we define $f : S^1 \rightarrow \{x\}$
  as the constant map, and take the mapping cylinder $M_f$. It is clear that $M_f$ will deformation retract to the single point at the top of the cap, but we \emph{cannot}
  deformation retract this space to a circle.

  Why is this the case? Well, note that $M_f = (S^1 \times [0, 1]) \sqcup \{x\}$ is homeomorphic to a disk. A retraction of the disk onto its boundary, the circle, cannot exist
  (we will show this later), so there certainly cannot be a \emph{deformation} retract.
  \section{Homotopy equivalence}

  \noindent Now, let us move on to \emph{homotopy equivalence}: the weaker notion of ``equivalence of topological spaces'' alluded to earlier.

  \begin{definition}[Homotopy equivalence]
    A continuous map $f : X \rightarrow Y$ is a homotopy equivalence if there exists $g : Y \rightarrow X$ such that $f \circ g$ and $g \circ f$ are homotopic
    to the identity on $Y$ and $X$ respectively. If, for two spaces $X$ and $Y$, there exists a homotopy equivalence $f : X \rightarrow Y$, we say that
    $X$ and $Y$ are homotopy equivalent.
  \end{definition}

  \begin{example}[Deformation retract and homotopy equivalence]
    Let us consider deformation retracts in the context of homotopy equivalence. Suppose $X$ deformation retracts to $A$.
    Let $j : A \rightarrow X$ be the inclusion map of $A$ in $X$, let $r : X \rightarrow A$ be the retraction. Note that, by
    definition, $r \circ j = \text{id}|_{A}$. Moreover, note that by definition, since $X$ deformation retracts to $A$, $j \circ r \simeq \text{id}|_X$.

    Thus, $X$ and $A$ are homotopy equivalent.
    \end{example}

\begin{example}
  A space $X$ is said to be \emph{contractible} if it is homotopy equivalent to a single point. Note that $X$ being contractible does not necessarily
  imply that $X$ deformation retracts to a point. However, the converse is true (as we showed above that a deformation retract gives a homotopy equivalence).
\end{example}

\noindent The main idea with homotopy equivalence is that we can go back and forth between
two spaces in such a way that this action of ``going back and forth'' can be continuously deformed to the identity map.
The notion of homotopy equivalence gives rise to \emph{homotopy type}, where we say that two spaces $X$ and $Y$ are of the samer homotopy type
if there exists a homotopy equivalence between them. The spaces of the same homotopy type, as it turns out, are equivalence classes: homotopy equivalence
is an equivalence relation. Reflexivity and symmetry are obvious, so it remains to show that it is transitive.

\begin{proposition}
  Composition of homotopy equivalences is a homotopy equivalence. It follows immediately that if $X$ and $Y$ are homotopy equivalent,
  and $Y$ and $Z$ are homotopy equivalent, then $X$ and $Z$ are homotopy equivalent.
\end{proposition}

\begin{proof}
  Suppose $f : X \rightarrow Y$ and $g : Y \rightarrow Z$ are homotopy equivalences, let $f'$ and $g'$ be the corresponding maps in the other direction.
  Note that
  \begin{equation}
    (f \circ g) \circ (g' \circ f') = f \circ (g \circ g') \circ f' \simeq f \circ f' \simeq \text{id}
  \end{equation}
  and
  \begin{equation}
    (g' \circ f') \circ (f \circ g) = g' \circ (f' \circ f) \circ g \simeq g' \circ g \simeq \text{id}
  \end{equation}
  so the claim follows immediately.
\end{proof}

If you found the definition of a homotopy equivalence a bit opaque, that's no problem, we are about to provide (half of) an
alternate, equivalent definition of homotopy equivalence which is easier to visualize.

\begin{proposition}
  Let $X$ and $Y$ be topological spaces, let $Z$ be a common space in which $X$ and $Y$ are embedded via embeddings $j_X : X \rightarrow Z$ and $j_Y : Y \rightarrow Z$.
  If $Z$ deformation retracts to both $j_X(X)$ and $j_Y(Y)$, then $X$ and $Y$ are homotopy equivalent.
\end{proposition}
\begin{proof}
  This follows immediately from the fact that deformation retracts are homotopy equivalences. In particular, $j_X(X)$ is homotopy equivalent to $Z$, which
  is homotopy equivalent to $j_Y(Y)$. Thus, $j_X(X)$ and $j_Y(Y)$ are homotopy equivalent. $X$ and $Y$ are homeomorphic to these spaces respectively,
  and are thus homotopy equivalent (it is trivial to check homotopy equivalence persists under homeomorphism).
  \end{proof}

\begin{comment}
  Let $r_X$ be the retraction onto $X$ and let $r_Y$ be the retraction onto $Y$. Take $f = r_X \circ j_Y$ and $g = r_Y \circ j_X$. Note that since $Z$ deformation retracts to $X$ and $Y$,
  $j_X \circ r_X \simeq \text{id}$ and $j_Y \circ r_Y \simeq \text{id}$. Thus,
  \begin{equation}
    f \circ g = r_X \circ (j_Y \circ r_Y) \circ j_X \simeq r_X \circ j_X = \text{id}_X
  \end{equation}
  and
  \begin{equation}
    g \circ f = r_Y \circ (j_X \circ r_X) \circ j_Y \simeq r_Y \circ j_Y = \text{id}_Y
  \end{equation}
  so $f : Y \rightarrow X$ is a homotopy equivalence, and the proof is complete.
\end{comment}

\noindent As it turns out, the converse holds as well, in the following sense:

\begin{claim}
  \label{claim:1}
  Two topological spaces $X$ and $Y$ are homotopy equivalent if and only if there exists a common space $Z$ and embeddings $j_X : X \rightarrow Z$ and $j_Y : Y \rightarrow Z$
  such that $Z$ deformation retracts to both $j_X(X)$ and $j_Y(Y)$.
\end{claim}

\noindent We will prove this in the next section.

\section{Homotopy extension property}

\noindent The goal of this (final) section is to prove a condition for homotopy equivalence which as a consequence will
allow us to prove Claim.~\ref{claim:1}. The main definition we will require is that of the \emph{homotopy extension property}.

\begin{definition}[Homotopy extension property]
  Suppose $f : X \rightarrow Y$ is a continuous map, and suppose that $f_t : A \rightarrow Y$ is a homotopy with $f_0 = f|_A$,
  where $A \subset X$. If $(X, A)$ has the property that for any $f$ and any $f_t$ as above, $f_t$ can be extended to a homotopy
  on all of $X$, $f_t : X \rightarrow Y$, then $(X, A)$ has the homotopy extension property.
\end{definition}

\begin{lemma}
a pair $(X, A)$ has the homotopy extension property if and only if $Z = X \times \{0\} \sqcup A \times I$ is a retract of $X \times I$.
\end{lemma}
\begin{proof}
  Suppose $Z$ is a retract of $X \times I$. Let $f : X \rightarrow Y$ be a map, let $F : A \times I \rightarrow Y$ be a homotopy with $F(x, 0) = f$. Let $r$ be the retraction.
  We claim that $F \circ r : X \times I \rightarrow Y$ is the desired extension. Indeed, this map is well-defined and continuous. Moreover,
  \begin{equation}
    (F \circ r)(x, 0) = F(x, 0) = f(x)
  \end{equation}

  \end{proof}

\vspace{20pt}
    
\end{document}
