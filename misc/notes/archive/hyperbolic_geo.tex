\documentclass[aps,pra,showpacs,notitlepage,onecolumn,superscriptaddress,nofootinbib]{revtex4-1}
\usepackage[utf8]{inputenc}
\usepackage[tmargin=1in, bmargin=1in, lmargin=1in, rmargin=1in]{geometry}
\usepackage{amsmath, amssymb, amsthm}
\usepackage{graphicx}
\usepackage[svgnames]{xcolor}
\usepackage{enumitem}
\usepackage{titlesec}
\usepackage{datetime}
\usepackage{hyperref}
\usepackage{import}
\usepackage{mathtools}
\usepackage{tikz}
% package for commutative diagrams
\usepackage{tikz-cd}
% package for referencing thms in appx
\usepackage{thmtools,thm-restate}
% for dingbats in correction figure caption
\usepackage{pifont}
% for labelled block matrices
\usepackage{blkarray}

%%%%%%%%%%%%%%%%%%%%%%%%%%%%%%%%%%%%%%%%%%%%%
\definecolor{crimson}{RGB}{186,0,44}
\definecolor{moss}{RGB}{0, 186, 111}
\newcommand{\pop}[1]{\textcolor{crimson}{#1}}
\newcommand{\zcom}[1]{\noindent\textcolor{crimson}{(Z): #1}}
\newcommand{\jcom}[1]{\noindent\textcolor{moss}{(J): #1}}
\newcommand{\wt}[1]{\widetilde{#1}}
\newcommand{\pgeq}{\succcurlyeq}
\newcommand{\pleq}{\preccurlyeq}
\newcommand{\Wedge}{\bigwedge}

%%%%%%%%%%%%%%%%%%%%%%%%%%%%%%%%%%%%%%%%%%%%%
\hypersetup{
    colorlinks,
    linkcolor={crimson},
    citecolor={crimson},
    urlcolor={crimson}
}

\tikzset{
    every node/.style={font=\sffamily\small},
    main node/.style={thick,circle ,draw},
    visible node/.style={thick,rectangle ,draw}
}
\usepackage{qcircuit}

%%%%%%%%%%%%%%%%%%%%%%%%%%%%%%%%%%%%%%%%%%%%%
\theoremstyle{definition}
\newtheorem{definition}{Definition}[section]
\newtheorem{lemma}{Lemma}[section]
\newtheorem{theorem}{Theorem}[section]
\newtheorem{corollary}{Corollary}[theorem]
\newtheorem*{theorem*}{Theorem}
\newtheorem*{corollary*}{Corollary}
\newtheorem{remark}{Remark}[section]
\newtheorem{conjecture}{Conjecture}[section]
\newtheorem{example}{Example}[section]
\newtheorem{reminder}{Reminder}[section]
\newtheorem{problem}{Problem}[section]
\newtheorem{question}{Question}[section]
\newtheorem{answer}{Answer}[section]
\newtheorem{fact}{Fact}[section]
\newtheorem{claim}{Claim}[section]
\newtheorem{exercise}{Exercise}[section]

\usepackage{geometry}
\geometry{
  left=25mm,
  right=25mm,
  top=20mm,
}


%%%%%%%%%%%%%%%%%%%%%%%%%%%%%%%%%%%%%%%%%%%%%
\bibliographystyle{unsrt}

%%%%%%%%%%%%%%%%%%%%%%%%%%%%%%%%%%%%%%%%%%%%%
%%%%%%%%%%%%%%%%%%%%%%%%%%%%%%%%%%%%%%%%%%%%%
%%%%%%%%%%%%%%%%%%%%%%%%%%%%%%%%%%%%%%%%%%%%%
\begin{document}

\title{Hyperbolic geometry: preparing for MAT1305}
\author{Jack Ceroni}
\affiliation{University of Toronto}
\date{\today}

%%%%%%%%%%%%%%%%%%%%%%%%%%%%%%%%%%%%%%%%%%%%%

\maketitle

\section{Introduction}

\noindent
These notes will be in much more of a ``stream-of-concious'' style than many of my others.

\subsection{Brief reivew on differential forms}

\noindent
\textit{The material for this section is mostly drawn from Spivak's Calculus on Manifolds. This section is a work-in-progress.}

\hrulefill

\noindent
To begin, we review some simple notions in differential geometry.
\begin{definition}[Differential Form]
  A \emph{differential $k$-form} on a manifold $M$ is a map $\omega : M \rightarrow \Wedge^{k}(TM)$, where $\Wedge^{k}(TM)$ denotes the disjoint union of all collections of alternating $k$-tensors
  at each tangent space of the manifold, such that $\omega(x) \in \Wedge^{k}(T_x M)$. Let $\Omega^{k}(M)$ denote the collection of all $k$-forms on manifold $M$.
\end{definition}

\noindent
Recall that we are able to write a particular $k$-form in terms of a sum over wedge products of the standard coordinate $1$-forms,
\begin{equation}
  \label{eq:expansion}
  \omega(p) = \displaystyle\sum_{I} f_{I}(p) dx_{i_1} \wedge \cdots \wedge dx_{i_k}
\end{equation}
where the sum is over all ascending $k$-tuples drawn from $\{1, \dots, n\}$, and we recall that coordinate one-forms are defined locally via pullback of the coordinate one-forms in $\mathbb{R}^{n}$
via the chat at a particular point.

\begin{definition}[Differential]
  The \emph{differential} is a map $d : \Omega^k(M) \rightarrow \Omega^{k + 1}(M)$ such that $d\omega$ takes the differential of each coefficients function
  when $\omega$ is expanded, as in Eq.~\eqref{eq:expansion},
  \begin{equation}
    d\omega = \displaystyle\sum_{I} \displaystyle\sum_{k = 1}^{n} \frac{\partial f_I}{\partial x_k} dx_k \wedge dx_{i_1} \wedge \cdots \wedge dx_{i_k}
    \end{equation}
\end{definition}

\subsection{Chapter 1}

\begin{lemma}
  Any complex analytic map with non-singular derivative is conformal.
\end{lemma}
\begin{proof}
  TODO
\end{proof}

\begin{exercise}[Series 1.12]
  We summarize in a list:
  \begin{itemize}
  \item This is clear in the case of $\hat{\mathbb{C}}$: take the rotation action about point $a$, and note that it fixes $a$.
  \item In the case of two-variable tuples of distinct points, the transitive property is obvious: given $(z_1, z_2)$ and $(z_1', z_2')$, we rotate so that $z_1$ is real non-zero 
    we can
  take the action which performs the exchange $z \mapsto 1/z$, followed by rotation of $\pi$ about a point $\frac{1}{2} \left(a + \frac{1}{a} \right)$. It is easy to check that this action will
  fix $(a, 1/a)$, and is not the identity.
  \end{itemize}
\end{exercise}

\begin{proposition}[Series 1.13]
  $\text{Aut}(\hat{\mathbb{C}})$ acts transitively and freely on ordered, distinct triples in $\hat{\mathbb{C}}$. More specifically, given $(z_1, z_2, z_3)$ and $(w_1, w_2, w_3)$, there is a unique $T \in \text{Aut}(\hat{\mathbb{C}})$ such
  that $T(z_1, z_2, z_3) = (w_1, w_2, w_3)$.
\end{proposition}
\begin{proof}
  We will 
  \end{proof}

\end{document}





