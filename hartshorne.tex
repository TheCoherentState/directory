\documentclass[aps,pra,showpacs,notitlepage,onecolumn,superscriptaddress,nofootinbib]{revtex4-1}
\usepackage[utf8]{inputenc}
\usepackage[tmargin=1in, bmargin=1.25in, lmargin=1.5in, rmargin=1.5in]{geometry}
\usepackage{amsmath, amssymb, amsthm}
\usepackage{graphicx}
\usepackage{xcolor}
\usepackage{enumitem}
\usepackage{datetime}
\usepackage{hyperref}
\usepackage{titlesec}
\usepackage{import}
\usepackage{mathtools}
\usepackage{thmtools,thm-restate}
\usepackage{tikz-cd}
\usepackage[many]{tcolorbox}

% package for commutative diagrams
% \usepackage{tikz-cd}

%%%%%%%%%%%%%%%%%%%%%%%%%%%%%%%%%%%%%%%%%%%%%
\definecolor{crimson}{RGB}{186,0,44}
\definecolor{moss}{RGB}{0, 186, 111}
\newcommand{\pop}[1]{\textcolor{crimson}{#1}}
\newcommand{\zcom}[1]{\noindent\textcolor{crimson}{(Z): #1}}
\newcommand{\jcom}[1]{\noindent\textcolor{moss}{(J): #1}}
\newcommand{\wt}[1]{\widetilde{#1}}
\newcommand{\pqeq}{\succcurlyeq}
\newcommand{\pleq}{\preccurlyeq}

%%%%%%%%%%%%%%%%%%%%%%%%%%%%%%%%%%%%%%%%%%%%%
\hypersetup{
    colorlinks,
    linkcolor={crimson},
    citecolor={crimson},
    urlcolor={crimson}
}

\usepackage{qcircuit}
\usepackage{comment}

%%%%%%%%%%%%%%%%%%%%%%%%%%%%%%%%%%%%%%%%%%%%%
\theoremstyle{definition}
\newtheorem{definition}{Definition}[section]

\newtheorem{lemma}{Lemma}[section]

\newtheorem{theorem}{Theorem}[section]

\newtheorem{corollary}{Corollary}[theorem]
\newtheorem*{theorem*}{Theorem}
\newtheorem*{corollary*}{Corollary}

\newtheorem{remark}{Remark}[section]

\newtheorem{conjecture}{Conjecture}[section]
\newtheorem{example}{Example}[section]
\newtheorem{reminder}{Reminder}[section]
\newtheorem{problem}{Problem}[section]
\newtheorem{question}{Question}[section]
\newtheorem{answer}{Answer}[section]
\newtheorem{fact}{Fact}[section]
\newtheorem{claim}{Claim}[section]
\newtheorem{prop}{Proposition}[section]

\newtheorem{solution}{Solution}[section]

\usepackage{geometry}
\geometry{
  left=25mm,
  right=25mm,
  top=20mm,
}

\newcommand{\hhrulefill}{\hspace{-1.5em} \hrulefill}
\renewcommand{\baselinestretch}{1.1} 

%%%%%%%%%%%%%%%%%%%%%%%%%%%%%%%%%%%%%%%%%%%%%
\bibliographystyle{unsrt}

%%%%%%%%%%%%%%%%%%%%%%%%%%%%%%%%%%%%%%%%%%%%%
%%%%%%%%%%%%%%%%%%%%%%%%%%%%%%%%%%%%%%%%%%%%%
%%%%%%%%%%%%%%%%%%%%%%%%%%%%%%%%%%%%%%%%%%%%%
\begin{document}

\title{Every exercise in the first three chapters of Hartshorne}
\author{Jack Ceroni}
\email{jceroni@uchicago.edu}
\date{\today}
\maketitle

\tableofcontents

\section{Introduction}

\noindent Every exercise in the first three chapters of Hartshorne, no excuses (``every'' means \textbf{every single one}). I'm also trying to do this with zero assistance, as
it is summer break, and I have time to ponder things.
\newline

\noindent \textbf{Current tally}
\begin{itemize}
  \item Chapter 1: 15/90
    \item Chapter 2: 1/134
      \item Chapter 3: 0/88
  \end{itemize}

\section{Chapter 1}

\subsection{Section 1.1}

\begin{solution}[Problem 1.1.1]
There are a few parts:
\begin{enumerate}
  \item Of course, $A(Y) = k[x, y]/(y - x^2)$. We can define $\varphi : k[x, y] \rightarrow k[x]$ as $\varphi(p)(x) = p(x, x^2)$. Verification that this is a ring homomorphism
    is trivial. It is obviously surjective, as $k[x] \subset k[x, y]$, and $\varphi|_{k[x]} = \text{id}$. In addition, $\varphi(y - x^2) = 0$. Moreover, if $\varphi(p) = 0$, then
    $p(x, x^2) = 0$. Define $h(x, y) = p(x, y + x^2) \in k[x, y] = k[x][y]$. Of course, we may write $h(x, y) = h_0(x) + h_1(x) y + h_2(x) y^2 + \cdots$, by definition, and $p(x, x^2) = h(x, 0) = h_0(x)$,
    so $h(x, y) = y g(x, y)$ for some $g$. Thus, $p(x, y) = (y - x^2) g(x, y - x^2)$, which means $p \in (y - x^2)$. It follows that $\text{Ker}(\varphi) = (y - x^2)$, so by the first isomorphism
    theorem, $k[x, y]/(y - x^2) \simeq \text{Im}(\varphi) = k[x]$, so $A(Y) \simeq k[x]$, as desired.
    \item Suppose $\varphi : k[x, y]/(xy - 1) \rightarrow k[x]$ is a ring homomorphism. Since $[x][y] = [xy] = [1]$ in the domain, $\varphi([x]) \varphi([y]) = \varphi([x][y]) = \varphi([1]) = 1$. Thus,
      $\varphi([x]) = a \in k$ and $\varphi([y]) = a^{-1} \in k$. Given some $b \in k$, with $b \neq 0$, note that we must also have $\varphi([b]) \varphi([b^{-1}]) = 1$, so $\varphi([b])$ is a unit in $k[x]$,
      thus in $k$. It follows that $\text{Im}(\varphi) = k$, which means that $\varphi$ cannot be an isomorphism.
     \item \textbf{(Starred)} The idea is to make use of a sequence of automorphisms of $k[x, y]$, which descend to automorphisms of the quotient. In particular, a general quadratic polynomial is of the form
       \begin{equation}
         p(x, y) = a x^2 + b y^2 + cxy + dx + ey + f
         \end{equation}
       Since we are working over an algebraically closed field, we are allowed to take square roots. For this to be a quadratic, at least one of $a, b$ or $c$ must be non-zero. Assume $c$ is non-zero, and
       assume that $4ab = c^2$. It then follows that either $a$ or $b$ is non-zero (we can assume WLOG it is $a$), so the map $(x, y) \mapsto (\sqrt{a} x + \sqrt{b} y, y)$ is an isomorphism. Note that
       we can write
       \begin{equation}
         ax^2 + by^2 + cxy + dx + ey + f = (\sqrt{a} x + \sqrt{b} y)^2 + dx + ey + f
         \end{equation}
       so it follows that we may reduce to the case that our quadratic is of the form $x^2 + dx + ey + f$. From here, note that $e \neq 0$, or else our polynomial in $x$ would be reducible. In addition,
       we can re-write $x^2 + dx + f$ in the form $-(i x - d')^2 + f'$, so that
       \begin{equation}
         x^2 + dx + ey + f = (ey + f') - (i x - d')^2
         \end{equation}
       and note that $(x, y) \mapsto (ix - d', ey + f')$ is an isomorphism, so we have reduced to the case where our quadratic is $y - x^2$.

       Next, consider the case where $4ab \neq c^2$, still with $c$ non-zero and also with $a \neq 0$ or $b \neq 0$.
       The idea is that we can transform $ax^2 + by^2 + cxy$ to $xy$. In particular, assume WLOG that $a \neq 0$, so we can factor out a unit
       and assume the degree-$2$ part of our quadratic is of the form
       \begin{equation}
         x^2 + by^2 + cxy = (x - r_{+} y)(x - r_{-} y)
         \end{equation}
       with $4b \neq c^2$, where
       \begin{equation}
         r_{\pm} = \frac{c \pm \sqrt{c^2 - 4b}}{2}
         \end{equation}
       Note that $r_{+} \neq r_{-}$, as their difference is $\sqrt{c^2 - 4b}$, which is non-zero. Thus, at least one of them is non-zero, so that $(x, y) \mapsto (x - r_{+} y, x - r_{-} y)$
       is an isomorphism. It follows that we can reduce our quadratic to the form $xy + dx + ey + f$. This is also the form of the quadratic, after factoring out $c$, if we didn't first assume
       that either $a$ or $b$ is non-zero. Of course, $xy + dx + ey + f = (x + e)(y + d) + f$, so we can reduce to the case $xy + f$. If $f = 0$, our quadratic is reducible, which is a contradiction,
       so $f \neq 0$. Factoring out a unit reduces to the case $xy - 1$.

       The final case that we must check is when $c = 0$. Again, either $a$ or $b$ must be non-zero, or we don't have a quadratic. If one of these \emph{is} zero, then our quadratic can be assumed
       of the form $x^2 + dx + ey + f$: the same as in the first case worked out, so this will reduce to the case of $y - x^2$. If both are non-zero, then we have
       \begin{equation}
         ax^2 + by^2 = (\sqrt{a} x - i \sqrt{b} y)(\sqrt{a} x + i \sqrt{b} y)
         \end{equation}
       with $(x, y) \mapsto (\sqrt{a} x - i \sqrt{b} y, \sqrt{a} x + i \sqrt{b} y)$ being an isomorphism, reducing us to the case of $xy + dx + ey + f$: another case which we already worked out.

       Thus, to summarize, given some irreducible quadratic, we can always find a sequence of ring automorphisms of $k[x, y]$ and multiplication by units which takes the quadratic to
       either $y - x^2$ or $xy - 1$. Given a ring automorphism $\varphi : k[x, y] \rightarrow k[x, y]$, this will descend to a ring automorphism $\widetilde{\varphi} : k[x, y]/(f) \rightarrow k[x, y]/(\varphi(f))$
       where $\widetilde{\varphi}(g + (f)) = \varphi(g) + (\varphi(f))$. In addition, if $c$ is a unit, $(f)$ and $(cf)$ are the same ideal. It follows that via our sequence of transformations, we can exhibit
       an isomorphism which takes $k[x, y]/(f)$ to $k[x, y]/(y - x^2)$ or $k[x, y]/(xy - 1)$, with the individual criteria for each case explained in the course of the solution above.
  \end{enumerate}
  \end{solution}

\begin{solution}[Problem 1.1.2]
  Clearly, $Y = V(y - x^2, z - x^3)$. We want to show that $(y - x^2, z - x^3)$ is prime. To do so, define the map $\varphi : k[x, y, z] \rightarrow k[x]$ which takes $p(x, y, z)$ to $p(x, x^2, x^3)$. It is
  clear that this is a surjective ring homomorphism and $(y - x^2, z - x^3) \subset \text{Ker}(\varphi)$. Moreover, suppose $\varphi(p) = 0$. Define $h(x, y, z) = p(x, y + x^2, z + x^3)$. Of course,
  $h(x, 0, 0) = 0$, so we can write
  \begin{equation}
    h(x, y, z) = y g_1(x, y, z) + z g_2(x, y, z)
    \end{equation}
  as each term is divisible by $z$ or $y$. We then have
  \begin{equation}
    p(x, y, z) = h(x, y - x^2, z - x^3) = (y - x^2) g_1(x, y - x^2, z - x^3) + (z - y^3) g_2(x, y - x^2, z - x^3)
    \end{equation}
  which means that $p \in (y - x^2, z - x^3)$. Thus, $k[x] \simeq k[x, y, z]/(y - x^2, z - x^3)$ so $(y - x^2, z - x^3)$ is prime, so $Y$ is an affine variety and is $1$-dimensional as $\dim(k[x]) = 1$.
  We have $I(Y) = (y - x^2, z - x^3)$, so we automatically have a pair of generators (this ideal clearly cannot be generated by one of these generators).
  \end{solution}

\begin{solution}[Problem 1.1.3]
  We have
  \begin{align}
    V(x^2 - yz, xz - x) &= V(x^2 - yz) \cap [V(x) \cup V(z - 1)]
    \\ &= V(x^2 - yz, x) \cup V(x^2 - yz, z - 1)
    \\ &= V(yz, x) \cup V(x^2 - y, z - 1)
    \\ &= ([V(y) \cup V(z)] \cap V(x)) \cup V(x^2 - y, z - 1)
    \\ &= V(y, x) \cup V(z, x) \cup V(y - x^2, z - 1)
    \end{align}
  Showing that each of the ideals $(y, x)$, $(z, x)$ and $(y - x^2, z - 1)$ can be done easily via the same method as Problem 1.1.2. Thus, $Y$ is the union of three irreducible components.
  It is easy to see that they each are $1$-dimensional (as their coordinate rings are isomorphic to polynomials in a single variable).
  \end{solution}

\begin{solution}[Problem 1.1.4]
  Every open set in the product topology for $\mathbb{A}^1 \times \mathbb{A}^1$ can be written as the union of a collection of $U \times V$, with $U, V \subset \mathbb{A}^1$ open. Of course, $U$ and $V$
  will be $\mathbb{A}^1$ with some finite collection of points removed (when they are non-empty). Thus, $U \times V$ is $\mathbb{A}^2$ with some finite collection of vertical/horizontal lines removed.
  All of these sets are open in the Zariski topology for $\mathbb{A}^2$. However, the Zariski topology is strictly finer than the product topology.

  In particular, consider the open set $Y = V(x - y)^{C}$ in the Zariski topology. Given some point $p$ in $Y$, if $Y$ is open in the product topology we must be able to choose some $U \times V$
  which is a neighbourhood of $p$ and contained in $Y$. However, any complement of a finite collection of horizontal/vertical lines will clearly contain some point on the diagonal $V(x - y)$.
  \end{solution}

\begin{solution}[Problem 1.1.5]
  First, note that the coordinate ring of some algebraic set $V$ is of the form $A(V) = k[x_1, \dots, x_n]/I(V)$, where $I(V)$ is a radical ideal. It follows immediately that if $[p]^n = [p^n] = 0$ in $A(V)$,
  then $p^n \in I(V)$, so $p \in I(V)$ and $[p] = 0$. This means $A(V)$ has no nilpotent elements. Moreover, it is immediately clear that $A(V)$ is a finitely-generated $k$-algebra, generated by $[x_1], \dots, [x_n]$.

  Conversely, suppose $B$ is a finitely-generated $k$-algebra with no nilpotent elements. Let $b_1, \dots, b_n$ be a generating set for $B$, then the evaluation map $\varphi : k[x_1, \dots, x_n] \rightarrow B$
  sending $p$ to $p(b_1, \dots, b_n)$ is surjective. It follows that $B \simeq k[x_1, \dots, x_n]/\text{Ker}(\varphi)$. Denote the ideal $\text{Ker}(\varphi)$ by $I$. Since $B$ has no nilpotent elements, it follows
  that if $p^n \in I$ so that $p(b_1, \dots, b_n)^n = 0$, then $p(b_1, \dots, b_n) = 0$, so $p \in I$. Thus, $I$ is a radical ideal and $I(V(I)) = I$. It follows that $B \simeq A(V(I))$.
  \end{solution}

\begin{solution}[Problem 1.1.6]
If $X$ is a topological space with non-empty open set $U$ and $\overline{U}$ is proper, then $X = \overline{U} \cup (X - U)$, so $X$ is reducible. Thus, if $X$ is irreducible, every non-empty open
set is dense. For the second part, suppose $C_1$ and $C_2$ are closed in $\overline{Y}$ with $C_1 \cup C_2 = \overline{Y}$. Then since $\overline{Y}$ is closed in $X$, $C_1$ and $C_2$ are as well. Thus,
$C_1 \cap Y$ and $C_2 \cap Y$ is closed in $Y$. Since $Y$ is irreducible and the union of these closed sets is $Y$, either $Y \subset C_1$ or $Y \subset C_2$. Thus, $\overline{Y} \subset C_1$ or $\overline{Y} \subset C_2$,
so $\overline{Y}$ is irreducible.
  \end{solution}

\begin{solution}[Problem 1.1.7]
  This is a long problem:
  \begin{enumerate}
    \item Given some non-empty family $\mathcal{F}$ of closed sets, suppose there is no minimal element. Let $C_1 \in \mathcal{F}$, we can choose
      $C_2 \in \mathcal{F}$ which is a proper subset of $C_1$. We continue on inductively, choosing $C_{j + 1} \in \mathcal{F}$ a proper subset of $C_j$. At follows that
      we have a descending chain in $X$ which does not terminate, so $X$ is not Noetherian. On the other hand, given a descending chain $C_1 \supset C_2 \supset \cdots$, there must
      be a minimal element (as such a chain is a family), hence some $C_n$ such that if $C_j \subset C_n$, then $C_n = C_j$, so the chain terminates after a finite number of steps. The open set conditions can easily
      be seen to be equivalent to the conditions above by taking complements.
      \item Let $\{U_{\alpha}\}_{\alpha}$ be an open cover for $X$. Suppose there is no finite subcover. Pick $U_1$ in the cover arbitrarily. Then pick some $x_2 \in U_1^{C}$ and pick a neighbourhood
        of $x_2$ in the cover, label it $U_2$. We proceed inductively, choosing $x_{n + 1} \in (U_1 \cup \cdots \cup U_n)^{C}$, letting $U_{n + 1} \in \{U_{\alpha}\}$ being a neighbourhood of $x_{n + 1}$. Then the ascending
        chain of open set $V_n = U_1 \cup \cdots \cup U_n$ never terminates, as $V_{n + 1}$ contains $x_{n + 1}$ while $V_n$ does not. Thus, $X$ is not Noetherian. It follows that if $X$ is Noetherian, then it must be
        quasi-compact.
        \item Any descending chain $C_1 \supset C_2 \supset \cdots$ of closed sets of $Y \subset X$ implies that $C_j = K_j \cap Y$ with $K_j$ a closed set. We then define
          $K'_n = K_1 \cap \cdots \cap K_n$, which gives a descending chain $K_1' \supset K_2' \supset \cdots$ of $X$. This chain eventually terminates, so $K_n' = K_{n + 1}' = \cdots$.
          Note that
          \begin{equation}
          K_j' \cap Y = (K_1 \cap Y) \cap \cdots \cap (K_j \cap Y) = C_1 \cap \cdots \cap C_j = C_j
          \end{equation}
          which implies that $C_n = C_{n + 1} = \cdots$, so $Y$ is Noetherian.
          \item Recall that any quasi-compact subset of a quasi-compact Hausdorff space is closed. If $X$ is Noetherian, then every subset is Noetherian, thus quasi-compact, thus closed.
           Every subset being closed implies that $X$ has the discrete topology. Quasi-compactness of the whole space then implies that it is a finite set of points (if we had an infinite number of points,
           with each single-point set open, we could not have a finite subcover).
    \end{enumerate}
  \end{solution}

\begin{solution}[Problem 1.1.8]
  Before jumping into this solution, we must remark on something: when we decompose an affine algebraic set into its irreducible components, $Y = Y_1 \cup \cdots \cup Y_m$, each $Y_j$ is maximal
  in the sense that it is not contained in any strictly larger irreducible algebraic subset $Y' \subset Y$. Clearly, $Y' = \bigcup_{j} (Y' \cap Y_j)$, so if $Y'$ is irreducible, $Y' \cap Y_j = Y'$ for at least
  one of the $j$. It can't be $j = 2, \dots, m$ as we would then have $Y_1 \subset Y' \subset Y_j$. Thus, $Y' = Y_1$.
  This means that when given some affine algebraic set $V$ with irreducible decomposition $V = V(\mathfrak{p}_1) \cup \cdots \cup V(\mathfrak{p}_m)$ with each $\mathfrak{p}_j$ a prime ideal, then each of
  these primes is minimal in the sense that they contain no smaller prime which contains $I(V)$.

  We write $Y = V(I)$, where $I$ is prime, and $H = V(f)$, where $f$ is some irreducible polynomial, as well as $Y \cap H = V(I, f) = V(J)$ where $J = \text{Rad}(I, f)$.
  Let $V(\mathfrak{p}_j)$ be one of the irreducible components of $V(J)$, so $\mathfrak{p}_j \supset (I, f)$. We note that
  \begin{align}
    \dim V(\mathfrak{p}_j) = \dim k[x_1, \dots, x_n]/\mathfrak{p}_j &= \dim (k[x_1, \dots, x_n]/I)/(\mathfrak{p}_j/I)
    \\ &= \dim(k[x_1, \dots, x_n]/I) - \text{height}(\mathfrak{p}_j/I)
    \\ &= \dim V(I) - \text{height}(\mathfrak{p}_j/I)
    \\ &= r - \text{height}(\mathfrak{p}_j/I)
    \end{align}
  where $\mathfrak{p}_j/I$ is a prime ideal of $k[x_1, \dots, x_n]/I$. Suppose $\mathfrak{q}$ were another prime ideal such that $\mathfrak{q} \subset \mathfrak{p}_j/I$, then $I \subset \pi^{-1}(\mathfrak{q}) \subset \mathfrak{p}_j$,
  so by minimality of $\mathfrak{p}_j$, $\mathfrak{q} = \mathfrak{p}_j/I$. In addition, $f + I \in \mathfrak{p}_j/I$. We can't have $f \in I$, as then $V(f) \supset V(I)$,
  which we assumed is not the case. Clearly $f + I$ is not a zero-divisor as $k[x_1, \dots, x_n]/I$ is an integral domain, as $I$ is prime. Moreover, if $fg = 1 + I$, so $fg + i = 1$ for some $i \in I$ and $g \in k[x_1, \dots, x_n]$,
  then $(f, I) = (1)$, so $V(J) = \emptyset$. Thus, we actually require the addition assumption that $Y$ and $H$ intersect at all!

  With this new assumption, we can apply Krull's Haupidealsatz to see that $\text{height}(\mathfrak{p}_j/I) = 1$, so that $\dim V(\mathfrak{p}_j) = r - 1$ for all $j$.
  \end{solution}

\begin{solution}[Problem 1.1.9]
  We can do this problem inductively, assuming $\mathfrak{a}$ is proper. Obviously if $\mathfrak{a}$ is generated by a single, non-zero element $f$ which isn't a unit, since $k[x_1, \dots, x_n]$ is a UFD, we can
  factor $f = f_1 \cdots f_m$ where each $f_j$ is irreducible, so $V(f) = V(f_1) \cup \cdots \cup V(f_m)$. We then know from Proposition 1.13 that each $V(f_j)$ is a variety (thus irreducible) of dimension $n - 1$,
  so it follows that every irreducible component of $V(f)$ has dimension $n - 1$. If $\mathfrak{a} = (0)$, then $V(\mathfrak{a}) = \mathbb{A}^n(k)$, which is dimension $n$, so we have proved the base case.

  From here, suppose $\mathfrak{a} = (a_1, \dots, a_r)$. We then have
  \begin{equation}
    V(\mathfrak{a}) = V(a_1, \dots, a_r) = V(a_1, \dots, a_{r - 1}) \cap V(a_r) = \bigcup_{j} V(\mathfrak{p}_j) \cap V(a_r)
    \end{equation}
  where $V(\mathfrak{p}_j)$ are the irreducible components of $V(a_1, \dots, a_{r - 1})$, each of which have dimension greater than or equal to $n - r + 1$. If $V(\mathfrak{p}_j) \subset V(a_r)$, then
  $V(\mathfrak{p}_j) \cap V(a_r) = V(\mathfrak{p}_j)$. If $V(\mathfrak{p}_j)$ is not a subset of $V(a_r)$, then $a_r$ cannot be $0$. We have assumed $\mathfrak{a}$ is proper, so $a_r$ is also not a unit, and
  we can decompose it into a union of $V(f_i)$, where each $V(f_i)$ is a hypersurface of dimension $n - 1$. Then, from Problem 1.1.8 above, every irreducible component of $V(\mathfrak{p}_j) \cap V(f_i)$ has dimension
  greater than or equl to $n - r$. All together, we can write $V(\mathfrak{a})$ as a union of irreducible algebraic sets, all of dimension greater than or equal to $n - r$. This completes the proof.
  \end{solution}

\begin{solution}[Problem 1.1.10]
  This problem has multiple parts:
  \begin{enumerate}
    \item Of course, any chain $Z_0 \subset \cdots \subset Z_n$ of distinct, closed, irreducible subsets of $Y$ is a chain of distinct, closed, irreducible subsets of $X$, so $\dim(Y) \leq \dim(X)$.
    \item From the first part, $\dim U_i \leq \dim X$ for all $i$, so $\sup_i \dim U_i \leq \dim X$. Conversely, let $Z_0 \subset \cdots \subset Z_n$ be a chain of distinct, closed, irreducible subsets of $X$.
     For each $Z_{j - 1} \subset Z_{j}$, there must exist some $U_{i_j}$ such that $Z_{j - 1} \cap U_{i_j}$ is a proper subset of $Z_{j} \cap U_{i_j}$. Thus, if we let $U = U_{i_1} \cup \cdots \cup U_{i_n}$, then
     \item
     \item 
    \end{enumerate}
  \end{solution}

\begin{solution}[Problem 1.1.11]
  Our first claim is that $Y = V(x^4 - y^3, x^5 - z^3, y^5 - z^4)$. It is very clear that $Y \subset V(x^4 - y^3, x^5 - z^3, y^5 - z^4)$. To show the reverse inclusion, pick some $(x, y, z) \in V$.
  We can take roots, so pick some $t$ so that $t^3 = x$, so $y^3 = t^{12}$ and $z^3 = t^{15}$. As polynomials in $y$ and $z$, each has exactly three roots (as, again, we are working in an algebraically closed field).
  We denote cubic roots of unity as $j^0 = 1, j, j^2$, so that we must have $y = j^{\alpha} t^4 \in \{ t^4, j t^4, j^2 t^4\}$ and $z = j^{\beta} t^5 \in \{t^5, j t^5, j^2 t^5\}$. From here, we must also have
  $y^5 = z^4$, so $j^{5 \alpha - 4 \beta} = 1$. It follows that $5 \alpha - 4 \beta$ must be a multiple of $3$, for $\alpha, \beta \in \{0, 1, 2\}$. It is easy to verify that the only choice is $(\alpha, \beta) = (0, 0)$,
  so $y = t^4$ and $z = t^5$, which means that $(x, y, z) = (t^3, t^4, t^5)$.

  \pop{\textbf{TODO:} Finish this}
\end{solution}

\begin{solution}[Problem 1.1.12]
  Consider
  \begin{equation}
    \label{eq:1}
    (x^2 + i y^2 - 1)(x^2 - iy^2 - 1) = x^4 + y^4 - 2x^2 + 1
    \end{equation}
  in $\mathbb{R}[x, y]$, which vanishes only at points $(\pm 1, 0) \in \mathbb{A}^2(\mathbb{R})$. Each of these single points is itself a closed proper subset of the vanishing
  set of $p(x, y) = x^4 + y^4 - 2x^2 + 1$, so the vanishing set is not irreducible. To see that this polynomial is irreducible in $\mathbb{R}[x, y]$, suppose that it was reducible.
  Since any facotirzation of $p$ in $\mathbb{R}[x, y]$ would be a factorization in $\mathbb{C}[x, y]$, and $\mathbb{C}[x, y]$ is a UFD, it follows that one of the factors in Eq.~\eqref{eq:1}
  must be reducible. Each factor must be degree-$1$, so it follows that if a factor is reducible, it will vanish precisely on the union of two straight lines. However, note that $x^2 \pm i y^2 - 1$
  vanishes on the embedded circle $(\cos(\theta), (\mp i)^{1/2} \sin(\theta))$, where given any two lines, we can always find a point on the circle not contained in these lines.

  Thus, the decomposition in Eq.~\eqref{eq:1} is into irreducible factors, so $p$ is irreducible in $\mathbb{R}[x, y]$.
  \end{solution}

\subsection{Section 1.2}

\noindent \pop{Even after taking a course in algebraic curves, my relative comfort-level when working with projective and quasi-projective varieties is significantly lower than when working with
  their affine counterparts. Hopefully I'll fix this deficiency in the process of doing these problems.}

\begin{solution}[Problem 1.2.1]
  Let $\pi : \mathbb{A}^{n + 1}(k) - (0, \dots, 0) \rightarrow \mathbb{P}^n(k)$ be the quotient map. Recall that
  \begin{equation}
    V_p(\mathfrak{a}) = \{ y \in \mathbb{P}^{n} \ | \ \text{there exists} \ x \in \mathbb{A}^{n + 1} - (0, \dots, 0) \ \text{where} \ \pi(x) = y, f(x) = 0 \ \text{for all} \ f \in \mathfrak{a}^h\}
    \end{equation}
  where $\mathfrak{a}_h$ is the set of all homogeneous elements of $\mathfrak{a}$. It follows that if $x \in V_a(\mathfrak{a}) - (0, \dots, 0)$, then $\pi(x) \in V_p(\mathfrak{a})$. Similarly,
  if $x \in \pi^{-1}(V_p(\mathfrak{a}))$, then all homogeneous element of $\mathfrak{a}$ vanish at some $z \in \pi^{-1}(\pi(x))$, thus $x$ itself, and since $\mathfrak{a}$ is generated by homogeneous elements, it follows
  that all $f \in \mathfrak{a}$ vanish at $x$, so $x \in V_a(\mathfrak{a}) - (0, \dots, 0)$. We have therefore shown that
  \begin{equation}
    V_a(\mathfrak{a}) - (0, \dots, 0) = \pi^{-1}(V_p(\mathfrak{a}))
    \end{equation}
  It follows that if $f$ is homogeneous and vanishes on $V_p(\mathfrak{a})$, it vanishes on any homogeneous coordinates for any point in $V_p(\mathfrak{a})$, thus any point in $\pi^{-1}(V_p(\mathfrak{a})) = V_a(\mathfrak{a}) - (0, \dots, 0)$.
  If $\deg(f) > 0$, then $f$ in fact vanishes on $V_a(\mathfrak{a})$, and $f \in I_a(V_a(\mathfrak{a})) = \text{Rad}(\mathfrak{a})$. We have thus shown that $I_p(V_p(\mathfrak{a})) \cap S_{> 0} \subset \text{Rad}(\mathfrak{a})$. Note that
  any element of $S_0 = k$ clearly cannot be in $I_p(V_p(\mathfrak{a}))$ if $V_p(\mathfrak{a}) \neq \emptyset$, so in this case, $I_p(V_p(\mathfrak{a})) \subset \text{Rad}(\mathfrak{a})$.
  \end{solution}

\begin{solution}[Problem 1.2.2]
Of course, if $V_p(\mathfrak{a}) = \emptyset$, then $S \supset \text{Rad}(\mathfrak{a}) \supset I_p(V_p(\mathfrak{a})) \cap S_{> 0} = I_p(\emptyset) \cap S_{> 0} = S_{> 0}$.
It follows that if $\text{Rad}(\mathfrak{a})$ does not contain some $c \neq 0$ in $S_0 = k$, then it is $S_{> 0}$, and if it does, it is all of $S$. If $\text{Rad}(\mathfrak{a})$ is either $S$ or
$S_{> 0}$, then in either case, it contains each of the elements $x_0, \dots, x_{n}$. It follows that for each $j$ from $0$ to $n$, there is some $d_j$ where $x_j^{d_j} \in \mathfrak{a}$. Then if $D = d_0 \cdots d_n$,
each $x_j^{D}$ is in $\mathfrak{a}$. Let $M = (n + 1) D$, and note that any generator $x_0^{\alpha_0} \cdots x_n^{\alpha_n}$ will have some $\alpha_j \geq D$, so the generator is in $\mathfrak{a}$. Therefore, $S_M \subset \mathfrak{a}$.
If $S_M \subset \mathfrak{a}$ for some $M$, then $V_p(\mathfrak{a}) \subset V_p(S_M)$. If $[x_0, \dots, x_n] \in V_p(S_M)$, then $x_0^M = \cdots = x_n^M = 0$, which is not the case for any point of projective space, so $V_p(S_M) = \emptyset$
and $V_p(\mathfrak{a}) = \emptyset$ as well.
\end{solution}

\begin{solution}[Problem 1.2.3]
We go point-by-point:
\begin{enumerate}
  \item This is a trivial application of the definitions.
    \item Again, trivial.
      \item Clearly, if $f$ is a homogeneous polynomial which vanishes on $Y_1 \cup Y_2$, it vanishes on $Y_1$ and $Y_2$ individually. The
        converse is also true, this proves the claim.
        \item From Problem 1.2.1, we already know that if $V_p(\mathfrak{a}) \neq \emptyset$, then $I_p(V_p(\mathfrak{a})) \subset \text{Rad}(\mathfrak{a})$. To show the reverse inclusion, let us first show that $I_p(X)$: the ideal generated
          by all homogeneous polynomials vanishing on $X \subset \mathbb{P}^n$, is radical. Suppose $f^m \in I_p(X)$, where $f = f_0 + f_1 + \cdots + f_d$ is the decomposition into homogeneous components. Since $I_p(X)$ is a homogeneous ideal, $f_d^{m} \in I_p(X)$, so
          $f_d \in I_p(X)$, as it vanishes on $X$. It follows that $(f - f_d)^m \in I_p(X)$ (use the binomial expansion) and $f - f_d = f_0 + \cdots + f_{d - 1}$. Continue this process inductively to see that $f_j \in I_p(X)$ for each $j$, so $f \in I_p(X)$ as well.

          Clearly, if $f \in \mathfrak{a}$, with $f = f_0 + \cdots + f_d$, then each homogeneous component $f_j$ is also in $\mathfrak{a}$ (as it is a homogeneous ideal). Thus, obviously $f_j$ is homogeneous and vanishes on $V_p(\mathfrak{a})$, so $f_j \in I_p(V_p(\mathfrak{a}))$,
          so $f \in I_p(V_p(\mathfrak{a}))$ as well. Then, since $I_p(V_p(\mathfrak{a}))$ is radical, $\text{Rad}(\mathfrak{a}) \subset I_p(V_p(\mathfrak{a}))$, and we have the desired equality.
          \item Clearly $V_p(I_p(Y))$ is a closed set. Any homogeneous polynomial in $I_p(Y)$ will vanish on $Y$, so $Y \subset V_p(I_p(Y))$, thus implying that $\overline{Y} \subset V_p(I_p(Y))$. To show that this inclusion is an equality, suppose $Y \subset V_p(T)$,
            where $T$ is some collection of homogeneous polynomials. Any $f \in T$ vanishes on $Y$, so $f \in I_p(Y)^{h}$, the set of homogeneous elements in $I_p(Y)$. Thus, $T \subset I_p(Y)^{h}$, and
            \begin{equation}
              V_p(I_p(Y)) = V_p(I_p(Y)^{h}) \subset V_p(T)
              \end{equation}
            It follows that $\overline{Y} = V_p(I_p(Y))$, as desired.
  \end{enumerate}
  \end{solution}

\begin{solution}[Problem 1.2.4]
  Another multi-part question, another test of my will:
  \begin{enumerate}
    \item Note that if $V_p(\mathfrak{a})$ is an algebraic/closed subset of $\mathbb{P}^n$, in the case that $V_p(\mathfrak{a}) = \emptyset$, then $I_p(V_p(\mathfrak{a})) = S$ and $V_p(S) = \emptyset$.
      If $V_p(\mathfrak{a}) \neq \emptyset$, then $I_p(V_p(\mathfrak{a})) = \text{Rad}(\mathfrak{a})$ is a radical homogeneous ideal. It is not equal to $S_{+}$, as if this were the case, then by Problem 1.2.2, $V_p(\mathfrak{a}) = \emptyset$.
      We also have
      \begin{equation}
      V_p(I_p(V_p(\mathfrak{a}))) = \overline{V_p(\mathfrak{a})} = V_p(\mathfrak{a})
      \end{equation}
      Now, assume that $\mathfrak{a}$ is a radical homogeneous ideal not equal to $S_{+}$, then if $\mathfrak{a} = S$, $V_p(\mathfrak{a}) = \emptyset$, and $I_p(V_p(\mathfrak{a})) = I_p(\emptyset) = S$. Otheriwse, $V_p(\mathfrak{a})$ is a
      non-empty closed subset of $\mathbb{P}^n$, and $I_p(V_p(\mathfrak{a})) = \text{Rad}(\mathfrak{a}) = \mathfrak{a}$, as desired.
      \item If $Y \subset \mathbb{P}^n$ is irreducible, then given $fg \in I_p(Y)$, let us decompose into homogeneous components $f = f_0 + \cdots + f_r$ and $g = g_0 + \cdots + g_q$. Then $f_r g_q \in I_p(Y)$, so $Y \subset V_p(f_r g_q) = V_p(f_r) \cup V_p(g_q)$.
        We then have $Y = (Y \cap V_p(f_r)) \cup (Y \cap V_p(g_q))$, so either $Y \subset V_p(f_r)$ or $Y \subset V_p(g_q)$, so either $f_r \in I_p(Y)$ or $g_q \in I_p(Y)$. Then, either $(f - f_r) g \in I_p(Y)$ or $f (g - g_q) \in I_p(Y)$. We repeat
        this same argument inductively by looking at the top-degree homogeneous component of the new polynomial, until we eventually conclude that either $f_0, \dots, f_r \in I_p(Y)$ or $g_0, \dots, g_q \in I_p(Y)$. Thus, $f \in I_p(Y)$ or $g \in I_p(Y)$,
        and $I_p(Y)$ is therefore prime.
        On the other hand, suppose $Y = V_p(T_1) \cup V_p(T_2)$ where both closed sets are proper. Since there is some $x \in V_p(T_1)$ which is not in $V_p(T_2)$,
        there must exist some $f \in T_2$ which does not vanish at $x$, and is thus not in $I_p(V_p(T_1))$. We can choose a similar $g \in T_1$. Note that $fg$ is homogeneous
        and vanishes on $Y$, so is in $I_p(Y)$. However, neither $f$ nor $g$ is in $I_p(Y) = I_p(V_p(T_1)) \cap I_p(V_p(T_2))$, so $I_p(Y)$ is not prime.
        \item Note that $V_p(0) = \mathbb{P}^n$, where $(0)$ is prime homogeneous, so $I_p(V_p(0)) = \text{Rad}(0) = (0)$ is prime, so from above, $\mathbb{P}^n$ is irreducible.
    \end{enumerate}
  \end{solution}

\begin{solution}[Problem 1.2.5]
  Multiple parts. In both cases, the proofs are similar to their corresponding proofs in affine space:
  \begin{enumerate}
    \item Given a descending sequence of closed sets $V_p(\mathfrak{a}_1) \supset V_p(\mathfrak{a}_2) \supset \cdots$, we have an increasing sequence of homogeneous ideals $\mathfrak{a}_1 \subset \mathfrak{a}_2 \subset \cdots$
      in $k[x_0, \dots, x_n]$, a Noetherian ring, so this sequence eventually terminates, and thus the sequence of closed sets must as well.
      \item Consider the family $\mathcal{F}$ of all closed sets which cannot be written as a finite union of irreducible closed sets. Since $\mathbb{P}^n$ is Noetherian, this family has a minimal element $V$. $V$ cannot
        be irreducible itself, so $V = V_1 \cup V_2$, where $V_1$ and $V_2$ are proper closed subsets. It can't be the case that we can write both $V_1$ and $V_2$ as a finite union of irreducible closed subsets, as then
        we could write $V$ in this form, so either $V_1 \in \mathcal{F}$ or $V_2 \in \mathcal{F}$, which contradicts minimality.

        Suppose $V = V_1 \cup \cdots \cup V_n$ and $V = W_1 \cup \cdots \cup W_m$, where the $V_k$ and $W_k$ are irreducible closed, and no closed set contains another in each of the decompositions (obviously,
        such a decomposition exists). Note that $W_k = \cup_j (W_k \cap V_j)$, and since $W_k$
        is irreducible, and each $W_k \cap V_j$ is closed, we must have $W_k \cap V_j = W_k$ for some $j$, so $W_k \subset V_j$. Thus, each $W_k$ in the decomposition of contained in one of the
        $V_j$. We can use the same argument to show that each $V_j$ is contained in one of the $W_k$. In other words, given $W_k$, we have $V_j$ and $W_{\ell}$ such that $W_k \subset V_j \subset W_{\ell}$, and
        the no-subsets condition implies that $W_k = V_j = W_{\ell}$. This means that each of the $W_k$ is one of the $V_j$, so $W_1 \cup \cdots \cup W_m = V_{k_1} \cup \cdots \cup V_{k_m}$, with $W_j = V_{k_j}$,
        with each of the $V_{k_j}$ distinct. The fact that each $V_j$ is contained in one of the $W_{i} = V_{k_i}$ then implies that these decompositions are actually equal.
    \end{enumerate}
  \end{solution}

\begin{solution}[Problem 1.2.6]
  Following the advice of the hint, let $\varphi_i : U_i \rightarrow \mathbb{A}^n$ be the homeomorphism from $U_i \subset \mathbb{P}^n$ to affine space given by sending $[x_0, \dots, x_n]$ to $(x_0/x_i, \dots, \widehat{x_i/x_i}, \dots, x_n/x_i)$.
  Consider the set $\varphi_i(Y \cap U_i)$, where $Y$ is closed and irreducible, so $Y \cap U_i$ is closed in $U_i$. It is also irreducible, as if $(C_1 \cap U_i) \cup (C_2 \cap U_i) = Y \cap U_i$, where $C_1$ and $C_2$ are closed, then $Y = C_1 \cup C_2 \cup (Y \cap U_i^{C})$.
  Thus, either $Y = C_1$ or $Y = C_2$ or $Y = Y \cap U_i^{C} = Y \cap V(x_i)$, so $Y \cap U_i = C_1 \cap U_i, C_2 \cap U_i, \ \text{or} \ \emptyset$. We assume $Y \cap U_i$ is non-empty, so
  it follows that $\varphi_i(Y \cap U_i)$ is closed and irreducible in $\mathbb{A}^n$, thus an affine variety.

  To be more specific, $Y = V_p(\mathfrak{a})$ for some homogeneous prime ideal $\mathfrak{a}$. Then
  \begin{align}
    \varphi_i(Y \cap U_i) &= \left\{ \left( \frac{x_0}{x_i}, \dots, \frac{x_n}{x_i} \right) \ \Big| \ x_i \neq 0, \ f(x_0, \dots, x_n) = 0 \ \text{for all} \ f \in \mathfrak{a} \right\}
    \\ &= \left\{ (y_1, \dots, y_n) \ | \ f_{*}(y_1, \dots, y_n) = 0 \ \text{for all} \ f \in \mathfrak{a} \right\}
    \\ &= \left\{ (y_1, \dots, y_n) \ | \ g(y_1, \dots, y_n) = 0 \ \text{for all} \ g \in \mathfrak{a}_{*} \right\}
    \end{align}
  where $f_{*}$ is the polynomial in $k[y_1, \dots, y_n]$ where we have substituted $x_j$ for $1$, and $\mathfrak{a}_{*}$ is the set of all $f_{*}$ for $f \in \mathfrak{a}$.
  It is easy to see that $\mathfrak{a}_{*}$ is an ideal, so the final line is irreducible algebraic set $V_a(\mathfrak{a}_{*})$.

  From here, let $Y_i = \varphi_i(Y \cap U_i) = V_a(\mathfrak{a}_{*})$, and let $A(Y_i)$ denote the coordinate ring. We construct a map $\phi : A(Y_i) \rightarrow S(Y)_{x_i + I_p(Y)}$ as follows.
  Given some $f + I(Y_i)$ in $A(Y_i)$, we write $f = f_0 + \cdots + f_d$ in terms of its homogeneous components. We then define
  \begin{equation}
    \phi(f(y_1, \dots, y_n) + I(Y_i)) = \sum_{k = 0}^{d} (x_i + I_p(Y))^{-k} (f_k(x_0, \dots, x_{i -  1}, x_{i + 1}, \dots, x_n) + I_p(Y))
    \end{equation}
  To prove that this map is well-defined, suppose $g \in I(Y_i) = \mathfrak{a}_{*}$, so there exists some $F \in \mathfrak{a}$ where $F_{*} = g$.
  Note that if $G$ is homogeneous, then $X^d (G_{*})^{*} = G$ for some $d$. Moreover,
  \begin{equation}
    (A + B)^{*} = \sum_{k = 0}^{\deg(A + B)} x^{\deg(A + B) - k}_i (A_k + B_k) = \sum_{k = 0}^{\deg(A)} x^{\deg(A + B) - k}_i A_k + \sum_{k = 0}^{\deg(B)} x^{\deg(A + B) - k}_i B_k = x^{\deg(A + B) - \deg(A)}_i A^{*} + x^{\deg(A + B) - \deg(B)}_i B^{*}
    \end{equation}
  which immediately means that
  \begin{equation}
  g^{*} = (F_*)^{*} = \left( \sum_{k = 0}^{d} (F_k)_{*} \right)^{*} = \sum_{k = 0}^{d} x_i^{d_k} ((F_k)_{*})^{*}
  \end{equation}
  It follows that is we multiply both sides by some $x_i^D$ for large enough $D$, we will have
  \begin{equation}
    x^D_i g^{*} = \sum_{k = 0}^{d} x_i^{D_k} F_k
    \end{equation}
  for some numbers $D_k$. Note that since $\mathfrak{a}$ is homogeneous, each $F_k$ is in $\mathfrak{a}$, so the above sum is also in $\mathfrak{a}$, which means
  $x^D_i g^{*} \in \mathfrak{a} = I_p(Y)$. If we had $x_i^D \in \mathfrak{a}$, then $Y = V_p(\mathfrak{a}) \subset V_p(x_i)$, which we already assumed is not the case, so since $\mathfrak{a}$ is prime,
  $g^{*} \in \mathfrak{a}$.
  \end{solution}

\begin{solution}[Problem 1.2.7]
  Point-by-point:
  \begin{enumerate}
    \item We have already seen that $\mathbb{P}^n$ is a projective variety, $V_p(0)$, so $S(\mathbb{P}^n) = k[x_0, \dots, x_n]$. Thus,
      from the previous problem,
      \begin{equation}
        \dim(\mathbb{P}^n) = \dim k[x_0, \dots, x_n] - 1 = (n + 1) - 1 = n
        \end{equation}
      as desired.
      \item Recall that in the process of solving Problem 1.2.6, we showed that if $\varphi_i : U_i \rightarrow \mathbb{A}^n$ is the usual homeomorphism
        from $U_i$ to affine space, then 
    \end{enumerate}
  \end{solution}

\begin{solution}[Problem 1.2.8]

  \end{solution}

\subsection{Section 1.3}

\section{Chapter 2}

\subsection{Section 2.1}

\noindent \pop{Something that should be noted about Section 2.1 is that it isn't particularly conceptually deep, its main challenge is in getting acclimatized to some subtle definitions.
If one is comfortable with all of the definitions, there should be no issue solving all of the problems.}
\newline

\begin{solution}[Problem 2.1.1]
  Let $\mathcal{F}$ be a presheaf, recall that its sheafification $\mathcal{F}^{+}$ is obtained by setting $\mathcal{F}^{+}(U)$ to be all functions $s : U \rightarrow \sqcup_{p \in U} \mathcal{F}_p$
  such that $s(p) \in \mathcal{F}_p$ and for each $p \in U$, there is a neighbourhood $V$ of $p$ and $t \in \mathcal{F}(V)$ such that $s(q) = t_q$ for each $q \in V$ (where $t_q$ is the germ of $t$ at $q$).
  The arrows are simply restriction of functions. The morphism $\theta(U) : \mathcal{F}(U) \rightarrow \mathcal{F}^{+}(U)$ of Abelian groups is given by $\theta(U)(s)(p) = s_p$.

  If $\mathcal{F}$ is the constant presheaf associated to $A$ on $X$, note that $\mathcal{F}(U) = A$, so all of the stalks of $\mathcal{F}$ is isomorphic to $A$, and the Abelian group $\mathcal{F}^{+}(U)$
  is isomorphic to the group of functions $s : U \rightarrow A$ such that for each $p \in U$, there is a neighbourhood $V$ of $p$ and $t \in A$ such that $s(q) = t$ for each $q \in V$. In other words, $s : U \rightarrow A$
  is a locally constant function, which is the case if and only if it is continuous when $A$ is given the discrete topology. Hence, $\mathcal{F}^{+}(U)$ is isomorphic to $\mathcal{G}(U)$, where $\mathcal{G}$ is the
  constant sheaf associated to $A$ on $X$. We have such an isomorphism for all $U$, and the isomorphisms are compatible with the restriction maps, so we have an isomorphism of sheaves \pop{One can imagine working out the details of
    the isomorphism described in more detail, but I personally can't be bothered as it is clear what's going on}.
  \end{solution}

\begin{solution}[Problem 2.1.2]
  There are three parts to this question:
  \begin{enumerate}
    \item Once again, recall that the kernel presheaf of a presheaf morphism $\varphi : \mathcal{F} \rightarrow \mathcal{G}$ is given by $U \mapsto \text{Ker}(\varphi(U)) \subset \mathcal{F}(U)$. Of course, it follows that every section
      of $\text{Ker}(\varphi(U))$ is a section of $\mathcal{F}(U)$, so it follows that if $\varphi$ is a morphism of \emph{sheaves}, and $s \in \text{Ker}(\varphi(U))$ restricts to $0$ on a cover of $U$, then $s = 0$. Moreover, given
      a collection of sections $s_i \in \text{Ker}(\varphi(V_i))$ for a cover $\{V_i\}$ for $U$ which agree on overlaps, we can construct $s \in \mathcal{F}(U)$ which restricts to
      each of these sections, so $s|_{V_i} = s_i$. Then, we note that
      \begin{equation}
        \varphi(U)(s)|_{V_i} = \varphi(V_i)(s|_{V_i}) = \varphi(V_i)(s_i) = 0
        \end{equation}
      for each $V_i$, so since $\mathcal{G}$ is a sheaf, it follows that $\varphi(U)(s) = 0$, which means that $s \in \text{Ker}(\varphi(U))$. Thus, in this case, the kernel presheaf is a sheaf.

      From here, we need to show that $(\text{Ker}(\varphi))_p = \text{Ker}(\varphi_p)$ at each $p \in X$. Recall that the map $\varphi_p : \mathcal{F}_p \rightarrow \mathcal{G}_p$ of stalks
      is defined as $\varphi_p(U, s) = (U, \varphi(U)(s))$, so the kernel of $\varphi_p$ will consist precisely of germs $(U, s)$ around $p$ such that $\varphi(U)(s) = 0$ (i.e. $s \in \text{Ker}(\varphi(U))$).
      On the other hand, the stalk at $p$ of $\text{Ker}(\varphi)$ consists of germs $(U, s)$ around $p$ where $s \in \text{Ker}(\varphi)(U) = \text{Ker}(\varphi(U))$, so the two are obviously equal.

      Similarly, $\text{Im}(\varphi_p)$ is all germs $(U, \varphi(U)(s))$ around $p$ in $\mathcal{G}_p$. On the other hand, $\text{Im}(\varphi)$ is the sheafification of the presheaf $U \mapsto \text{Im}(\varphi(U))$, which
      we denote $\text{Im}(\varphi)^{\text{pre}}$.
      We know that stalks of the sheaf and the sheafification can be identified with each other. In particular, a germ of $\text{Im}(\varphi)_p$ is $(U, s)$, where $s : U \rightarrow \bigcup_{q \in U} \text{Im}(\varphi)^{\text{pre}}_q$
      is a function such that if $U$ is chosen to be small enough, then $s(q) = t_q$ for some section $t \in \text{Im}(\varphi(U)) \subset \mathcal{G}(U)$. It follows that we can define an isomorphism
      $\text{Im}(\varphi)_p \rightarrow \text{Im}(\varphi^{\text{pre}}_p$ which takes $(s, U) \mapsto (t, U)$.

      So, $\text{Im}(\varphi)_p$ is isomorphic collection of germs $(U, s)$ around $p$ where $s \in \text{Im}(\varphi(U))$, which is what we want.
      \item If $\varphi$ is injective, then by definition the sheaf $\text{Ker}(\varphi)$ is trivial, so $\text{Ker}(\varphi)_p = \text{Ker}(\varphi_p)$ is trivial for all $p$, so every homomorphism of
        stalks $\varphi_p$ is injective. On the other hand, if each $\varphi_p$ is trivial, then each stalk $\text{Ker}(\varphi)_p$ is trivial. It follows that if $s \in \text{Ker}(\varphi(U))$ is a section,
        then $s|_V = 0$ for some $V \subset U$ around $p$. We can cover $U$ by $V$ around each $p$ where $s|_V = 0$, so since $s$ is a section of a \emph{sheaf}, $s = 0$
        as well, and $\text{Ker}(\varphi)$ is trivial.

        Proving the similar result for surjectivity follows the same steps. In particular, we want to show that $\text{Im}(\varphi) = \mathcal{G}$ if and only if $\varphi_p : \mathcal{F}_p \rightarrow \mathcal{G}_p$ is surjective
        for all $p$. If we assume that the former holds, then $\text{Im}(\varphi_p) = \text{Im}(\varphi)_p = \mathcal{G}_p$ for all $p$ (these equalities are actually natural isomorphisms, but once again, we can
        permit ourselves to be a bit sloppy). On the other hand, if $\text{Im}(\varphi_p)$ and thus $\text{Im}(\varphi)_p$ is equal to $\mathcal{G}_p$ for all $p$, then we need to show that $\text{Im}(\varphi)(U) \simeq \mathcal{G}(U)$
        for all $U$ (or are at least naturally isomorphic). 
      \item 
    \end{enumerate}
  \end{solution}

\subsection{Section 2.2}



\section{Chapter 3}

\end{document}
