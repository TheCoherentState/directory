\documentclass[aps,pra,showpacs,notitlepage,onecolumn,superscriptaddress,nofootinbib]{revtex4-1}
\usepackage[utf8]{inputenc}
\usepackage[tmargin=1in, bmargin=1.25in, lmargin=1.5in, rmargin=1.5in]{geometry}
\usepackage{amsmath, amssymb, amsthm}
\usepackage{graphicx}
\usepackage{xcolor}
\usepackage{enumitem}
\usepackage{datetime}
\usepackage{hyperref}
\usepackage{titlesec}
\usepackage{import}
\usepackage{mathtools}
\usepackage{thmtools,thm-restate}
\usepackage{tikz-cd}
\usepackage[many]{tcolorbox}

% package for commutative diagrams
% \usepackage{tikz-cd}

%%%%%%%%%%%%%%%%%%%%%%%%%%%%%%%%%%%%%%%%%%%%%
\definecolor{crimson}{RGB}{186,0,44}
\definecolor{moss}{RGB}{0, 186, 111}
\newcommand{\pop}[1]{\textcolor{crimson}{#1}}
\newcommand{\zcom}[1]{\noindent\textcolor{crimson}{(Z): #1}}
\newcommand{\jcom}[1]{\noindent\textcolor{moss}{(J): #1}}
\newcommand{\wt}[1]{\widetilde{#1}}
\newcommand{\pqeq}{\succcurlyeq}
\newcommand{\pleq}{\preccurlyeq}

%%%%%%%%%%%%%%%%%%%%%%%%%%%%%%%%%%%%%%%%%%%%%
\hypersetup{
    colorlinks,
    linkcolor={crimson},
    citecolor={crimson},
    urlcolor={crimson}
}

\usepackage{qcircuit}
\usepackage{comment}

%%%%%%%%%%%%%%%%%%%%%%%%%%%%%%%%%%%%%%%%%%%%%
\theoremstyle{definition}
\newtheorem{definition}{Definition}[section]

\newtheorem{lemma}{Lemma}[section]

\newtheorem{theorem}{Theorem}[section]

\newtheorem{corollary}{Corollary}[theorem]
\newtheorem*{theorem*}{Theorem}
\newtheorem*{corollary*}{Corollary}

\newtheorem{remark}{Remark}[section]

\newtheorem{conjecture}{Conjecture}[section]
\newtheorem{example}{Example}[section]
\newtheorem{reminder}{Reminder}[section]
\newtheorem{problem}{Problem}[section]
\newtheorem{question}{Question}[section]
\newtheorem{answer}{Answer}[section]
\newtheorem{fact}{Fact}[section]
\newtheorem{claim}{Claim}[section]
\newtheorem{prop}{Proposition}[section]

\newtheorem{solution}{Solution}[section]

\usepackage{geometry}
\geometry{
  left=25mm,
  right=25mm,
  top=20mm,
}

\newcommand{\hhrulefill}{\hspace{-1.5em} \hrulefill}
\renewcommand{\baselinestretch}{1.1} 

%%%%%%%%%%%%%%%%%%%%%%%%%%%%%%%%%%%%%%%%%%%%%
\bibliographystyle{unsrt}

%%%%%%%%%%%%%%%%%%%%%%%%%%%%%%%%%%%%%%%%%%%%%
%%%%%%%%%%%%%%%%%%%%%%%%%%%%%%%%%%%%%%%%%%%%%
%%%%%%%%%%%%%%%%%%%%%%%%%%%%%%%%%%%%%%%%%%%%%
\begin{document}

\title{Lectures in Differential Topology}
\author{Jack Ceroni}
\email{jceroni@uchicago.edu}
\date{\today}
\maketitle

\tableofcontents

\section{Introduction}

\noindent Some results in differential geometry and topology that I never want to forget. These notes are not exactly meant to be used as an introduction to the subject, although
they contain many notions which one would cover in an introductory course. I'm merely trying to write down things which I don't think are articulated quite in the way that I prefer (or at all)
in most of the textbooks on the subject. I also expect to be teaching geometry/topology one day, and these notes will be a good reference.

\section{Basic results}

\section{Smooth manifolds}

\begin{remark}
  A good (perhaps essential) mantra that one should always repeat when doing differential geometry is that we can produce many examples of the main objects of interest
  to starting with a collection of local things, and identifying them in a consistent way to produce something global. This will be evidenced in the next two lemmas, which
  give two techniques for producing smooth manifolds. The first one involves picking a candidate smooth atlas for an arbitrarily set, and imposing a compatible topology/smooth structure.
  The second one involves picking some 
\end{remark}

\begin{lemma}[Chart lemma for smooth manifolds]
  Let $M$ be a set, let $(U_{\alpha}, \varphi_{\alpha})$ be a collection of subsets, such that as countable subcollection covers $M$, where each map $\varphi_{\alpha} : U_{\alpha} \rightarrow \varphi_{\alpha}(U_{\alpha})$ is a bijection
  of $U_{\alpha}$ with an open set in $\mathbb{H}^n$. Moreover, suppose all transition maps $\varphi_{\alpha} \circ \varphi_{\beta}^{-1}$ defined on intersections of the cover are smooth, with $\varphi_{\beta}(U_{\alpha} \cap U_{\beta})$
  and $\varphi_{\alpha}(U_{\alpha} \cap U_{\beta})$ both open sets in $\mathbb{H}^n$. Finally, suppose that for $p \neq q$, either there exists
  some $U_{\alpha}$ containing both $p$ and $q$, or disjoint $U_{\alpha}$ and $U_{\beta}$ containing $p$ and $q$
  respectively (this is to ensure that we get the Hausdorff property). Then there exists a unique topology and smooth structure on $M$ turning it into a smooth $n$-manifold.
  \end{lemma}
\begin{proof}
 Our first claim is that the collection of all $\varphi_{\alpha}^{-1}(V)$ for $V$ open in $\varphi_{\alpha}(U_{\alpha})$ forms a basis for a topology on $M$. Indeed, note that each $p$ is
 contained in $\varphi_{\alpha}^{-1}(\varphi_{\alpha}(U_{\alpha}))$. Moreover, given $\varphi_{\alpha}^{-1}(V)$ and $\varphi_{\beta}^{-1}(W)$ with non-empty intersection,
 \begin{align}
   \varphi_{\alpha}^{-1}(V) \cap \varphi_{\beta}^{-1}(W) &= \varphi_{\alpha}^{-1}(V) \cap \varphi_{\beta}^{-1}(W) \cap (U_{\alpha} \cap U_{\beta})
   \\ &= \varphi_{\alpha}^{-1}\left( V \cap (\varphi_{\alpha} \circ \varphi_{\beta}^{-1})(W \cap \varphi_{\beta}(U_{\alpha} \cap U_{\beta})) \right)
   \end{align}
 which is of course an element of our collection of sets, so we do in fact have a basis. From here, second-countability follows from the fact that a countable collection of the $U_{\alpha}$ cover $M$, and
 each is homeomorphic to a second-countable set (an open subset of $\mathbb{H}^n$). We also immediately have the local Euclidean property: each $(U_{\alpha}, \varphi_{\alpha})$ is a chart and the transition
 functions are smooth. To finally prove the Hausdorff property, we note that any two points are either in disjoint $U_{\alpha}$ and $U_{\beta}$, or are contained in $U_{\alpha}$ which is homeomorphic to a Hausdorff space.

 To see uniqueness, note that if $M$ had any other topology and smooth structure making the $(U_{\alpha}, \varphi_{\alpha})$ smooth charts, then each set $\varphi_{\alpha}^{-1}(V)$ would necessarily have
 to be open, so that the topology defined above is coarser than the new topology. However, given open set $U$ in the new topology, note that given $p \in U$, we can pick $U_{\alpha}$ with $p \in U_{\alpha}$, and note that
 $p \in \varphi_{\alpha}^{-1}(\varphi_{\alpha}(U_{\alpha})) \cap U \subset U$, which means that $U$ is open in the topology generated by the basis (the old topology). Hence, the two topologies are equal. A choice
 of smooth atlas uniquely determines a smooth structure, so we immediately also have uniqueness of smooth structure as well.
  \end{proof}

\begin{example}[Circle]
  Let us construct the circle via the gluing construction. Let $M = \mathbb{R} \sqcup \mathbb{R}$: two disjoint copies of the real line.
  \end{example}

\begin{lemma}[Gluing lemma for smooth manifolds]
  Let $U_{\alpha}$ be a countable collection of open subsets of $\mathbb{R}^n$. Let $S$ be a collection of $\varphi_{\alpha \beta} : U_{\alpha} \rightarrow U_{\beta}$: smooth maps
  between pairs $U_{\alpha}$ and $U_{\beta}$ (not necessarily between all of the open sets) which satisfies the following criteria:
  \begin{enumerate}
    \item 
    \end{enumerate}
  \end{lemma}
\begin{proof}
  \end{proof}

\begin{prop}[Constant rank theorem]

  \end{prop}

\section{Bordism}

\section{Bundles}

\begin{lemma}[Chart lemma for vector bundles]

  \end{lemma}

\begin{lemma}[Gluing lemma for vector bundles]
  \end{lemma}

\begin{remark}
  A smooth fibre bundle $\pi : E \rightarrow B$ is a submersion. We have $\pi = \text{proj} \circ \varphi$ locally, where $\varphi$ is a diffeomorphism and $\text{proj}$ is obviously a
  submersion (as a projection $U \times F \rightarrow U$), so it follows that $\pi$ is a submersion.
  \end{remark}

\subsection{Subbundles}

\begin{lemma}[Smooth local coframe criterion]

  \end{lemma}

\subsection{Fibre integration}

\noindent Consider an $m$-dimensional fibre bundle $\pi : E \rightarrow B$ over $n$-dimensional manifold $B$, with compact oriented fibres,
in the smooth category so $\pi$ is a smooth map of manifolds and we have a trivialization of $B$ by open sets $U$
such that $\pi^{-1}(U)$ is diffeomorphic to $U \times F$.

\begin{lemma}
  Let $\pi : E \rightarrow B$ be a smooth submersion, let $X$ be a smooth vector field on $B$. Then there exists a smooth vector field $\widetilde{X}$ on $E$ such that $\pi_{*, p}(\widetilde{X}_p) = X_{\pi(p)}$
  for all $p \in E$.
  \end{lemma}
\begin{proof}
  Given $p \in E$, pick coordinates $(U, \varphi) = (U, x^1, \dots, x^m)$ and $(V, \psi) = (V, y^1, \dots, y^m)$ so that $\psi \circ \pi \circ \varphi^{-1}$
  is a projection. Write $X$ in $V$ as $\sum_{j = 1}^{m} X^j \frac{d}{dy^j}$ and define $\widetilde{X}|_{U}$ as $\sum_{j = 1}^{m} X^j \frac{d}{d x^j}$. We do this for some open cover $U_{\alpha}$
  and then combine via a partition of unity subordinate to the cover $\varphi_{\alpha}$ to obtain $\widetilde{X}$.
  \end{proof}

\noindent From here, let $\alpha$ be a $k$-form on $E$. We can define a $k - (m - n)$ form on $B$ by ``integrating $\alpha$ along the fibres'' (note that $m - n$ is the dimension of each fibre as a submanifold).
In particular, using the above lemma, given some tangent vectors $v_1, \dots, v_{k - m + n} \in T_p B$, we can always extend to a vector field on $B$ pull-back to smooth vector fields on the fibre $\pi^{-1}(b)$, as we can extend these
tangent vectors to a vector field and then use the lifting lemma (fibre bundles, locally, are projections, and thus submersions). We are then able to define a 

\section{Differential forms}

\begin{definition}[Lie derivative]
  \end{definition}

\section{Riemannian geometry}

\bibliography{refs}

\end{document}
