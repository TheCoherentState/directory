\documentclass[aps,pra,showpacs,notitlepage,onecolumn,superscriptaddress,nofootinbib]{revtex4-1}
\usepackage[utf8]{inputenc}
\usepackage[tmargin=1in, bmargin=1.25in, lmargin=1.5in, rmargin=1.5in]{geometry}
\usepackage{amsmath, amssymb, amsthm}
\usepackage{graphicx}
\usepackage{xcolor}
\usepackage{enumitem}
\usepackage{datetime}
\usepackage{hyperref}
\usepackage{titlesec}
\usepackage{import}
\usepackage{mathtools}
\usepackage{thmtools,thm-restate}
\usepackage{comment}


% package for commutative diagrams
% \usepackage{tikz-cd}

%%%%%%%%%%%%%%%%%%%%%%%%%%%%%%%%%%%%%%%%%%%%%
\definecolor{crimson}{RGB}{186,0,44}
\definecolor{moss}{RGB}{0, 186, 111}
\newcommand{\pop}[1]{\textcolor{crimson}{#1}}
\newcommand{\zcom}[1]{\noindent\textcolor{crimson}{(Z): #1}}
\newcommand{\jcom}[1]{\noindent\textcolor{moss}{(J): #1}}
\newcommand{\wt}[1]{\widetilde{#1}}
\newcommand{\pqeq}{\succcurlyeq}
\newcommand{\pleq}{\preccurlyeq}
\newcommand{\Wedge}{\bigwedge}

%%%%%%%%%%%%%%%%%%%%%%%%%%%%%%%%%%%%%%%%%%%%%
\hypersetup{
    colorlinks,
    linkcolor={crimson},
    citecolor={crimson},
    urlcolor={crimson}
}

\usepackage{qcircuit}

%%%%%%%%%%%%%%%%%%%%%%%%%%%%%%%%%%%%%%%%%%%%%
\theoremstyle{definition}
\newtheorem{definition}{Definition}[section]
\newtheorem{lemma}{Lemma}[section]
\newtheorem{theorem}{Theorem}[section]
\newtheorem{corollary}{Corollary}[theorem]
\newtheorem*{theorem*}{Theorem}
\newtheorem*{corollary*}{Corollary}
\newtheorem{remark}{Remark}[section]
\newtheorem{conjecture}{Conjecture}[section]
\newtheorem{example}{Example}[section]
\newtheorem{reminder}{Reminder}[section]
\newtheorem{problem}{Problem}[section]
\newtheorem{question}{Question}[section]
\newtheorem{answer}{Answer}[section]
\newtheorem{fact}{Fact}[section]
\newtheorem{claim}{Claim}[section]

\usepackage{tikz-cd}

\newcommand{\hhrulefill}{\hspace{-1.5em} \hrulefill}

\usepackage{geometry}
\geometry{
  left=25mm,
  right=25mm,
  top=20mm,
}

%%%%%%%%%%%%%%%%%%%%%%%%%%%%%%%%%%%%%%%%%%%%%
\bibliographystyle{unsrt}

%%%%%%%%%%%%%%%%%%%%%%%%%%%%%%%%%%%%%%%%%%%%%
%%%%%%%%%%%%%%%%%%%%%%%%%%%%%%%%%%%%%%%%%%%%%
%%%%%%%%%%%%%%%%%%%%%%%%%%%%%%%%%%%%%%%%%%%%%
\begin{document}

\title{Motivating non-commutative geometry via non-commutative vector bundles}
\author{Jack Ceroni}
\email{jackceroni@gmail.com}

\date{\today}

\maketitle

\section{Introduction}

\noindent The goal of this essay is to introduce some of the core ideas which motivate \emph{non-commutative geometry}. The majority of this essay will be
dedicated to constructing a non-commutative version of a vector bundle, and proving the Serre-Swan theorem. Much of the work in non-commutative geometry began with Connes,
and has been extended in recent years by several other individuals. According to Connes (at the very least, according to the talk he gave at the Fields institute in December
of 2023), the origins of non-commutative geometry take their inspiration from quantum mechanics, in particular, the non-commutative character of operators which makes
this physical theory markedly distinct from its classical counterparts.

\section{Introduction: the space/algebra correspondence}

\noindent Given some \emph{topological space} $X$, it is often true that the best way to study this space is via an \emph{algebra} of well-chosen functions on $X$. For example, one may be interested
in continuous functions from $X$ to some field, $C(X)$, continuous functions which become arbitrarily small outside compact sets, $C_0(X)$, or perhaps in $X$ is a smooth manifold,
and has differential structure, the space of smooth function $C^{\infty}(X)$. When speaking generally, we will let $F(X)$ denote \emph{some} space of continuous/continuous vanishing/smooth
functions from $X$ to a field, usually
the reals or complex numbers.

As it turns out, topological spaces are algebras are connected in a more formal fashion than the broad strokes outlined above. \emph{Gelfand duality}, is a pair of contravariant functor, $C$ and $\text{Spec}$, the first of which
goes from the category of compact topological spaces (with morphisms as continuous functions) to $C^{*}$-algebras (with morphisms as $*$-homomorphisms), with the other functor going in the opposite direction.
In particular, space $X$ is taken to the $C^{*}$-algebra $C(X)$ and $f : X \rightarrow Y$ is taken to $f^{*}(g) = g \circ f$. In the other direction, a $C^{*}$-algebra $A$ is taken to the space $\text{Spec}(A)$,
its space of characters, which are non-zero homomorphism $\mu : A \rightarrow \mathbb{C}$ with the weak-$*$ topology. A $*$-homomorphism $\phi : A \rightarrow B$ is taken to $\phi^{*}(\mu) = \mu \circ \phi$.

It is crucial to note that no information is lost when these functors are composed (i.e. we have an equivalence of categories). To see this, let us note that for a given topological space $X$,
if $x \in X$, then we can define a map $\text{eval}_x \in \text{Spec}(C(X))$ such that $\text{eval}_x(f) = f(x)$. In fact, this mapping $\Phi$ from $x \mapsto \text{eval}_x$ is a homeomorphism of topological spaces.
Moreover, given $a \in A$, where $A$ is a \emph{commutative} $C^{*}$-algebra,
the map $\widehat{a} : \text{Spec}(A) \rightarrow \mathbb{C}$ such that $\mu$ is taken to $\mu(a)$ is continuous, which immediately implies that the mapping $\Gamma$ of $a \mapsto \widehat{a}$ is a $*$-isomorphism.
It is straightforward to verify that the mappings are ``functorial'', in the sense that the diagrams commute:
% https://q.uiver.app/#q=WzAsOCxbMCwwLCJYIl0sWzIsMCwiWSJdLFsyLDIsIk0oQyhZKSkiXSxbMCwyLCJNKEMoWCkpIl0sWzQsMCwiQSJdLFs0LDIsIkMoTShBKSkiXSxbNiwyLCJDKE0oQikpIl0sWzYsMCwiQiJdLFswLDEsImYiXSxbMSwyLCJcXFBoaSJdLFswLDMsIlxcUGhpIiwyXSxbMywyLCJNKEMoZikpIiwyXSxbNCw3LCJcXHBoaSJdLFs0LDUsIlxcR2FtbWEiLDJdLFs1LDYsIkMoTShcXHBoaSkpIiwyXSxbNyw2LCJcXEdhbW1hIl1d
\[\begin{tikzcd}
X && Y && A && B \\
\\
	{M(C(X))} && {M(C(Y))} && {C(M(A))} && {C(M(B))}
	\arrow["f", from=1-1, to=1-3]
	\arrow["\Phi", from=1-3, to=3-3]
	\arrow["\Phi"', from=1-1, to=3-1]
	\arrow["{M(C(f))}"', from=3-1, to=3-3]
	\arrow["\phi", from=1-5, to=1-7]
	\arrow["\Gamma"', from=1-5, to=3-5]
	\arrow["{C(M(\phi))}"', from=3-5, to=3-7]
	\arrow["\Gamma", from=1-7, to=3-7]
\end{tikzcd}\]

\noindent This fact leads us to the primary starting-point of non-commutative geometry. Notice that the above constructions yield as a consequence many strong correspondences between spaces and $C^{*}$-algebras.
In particular, note that two commuative $C^{*}$-algebras $A$ and $B$ are isomorphic if and only if their character spaces $M(A)$ and $M(B)$ are homeomorphic. This
follows from the fact that if $\phi : A \rightarrow B$ is an algebra homomorphism, then $M\phi$ is a continuous map from $M(B)$ to $M(A)$, so if $\phi$ is an isomorphism, then
we can apply the same logic to the inverse function to conclude that $M\phi$ and $M\phi^{-1}$ yield the desired homeomorphism.

Because there is such a strong correspondence between topological spaces and \emph{commutative}
algebras (as an algebra of the form $C(X)$ is commutative when we define function multiplication in the usual way, $(f_1 \cdot f_2)(x) = f_1(x) \cdot f_2(x)$), can one
use the functors in their most general sense to consider topological spaces associated to \emph{non-commutative} $C^{*}$-algebras? As it turns out, the answer to this question
is yes, as we will see in the following sections.

\section{Non-commutative vector bundles}

\noindent As a further example of the geometry-algebra correspondence, let us attempt to construct a ``non-commutative vector bundle''.
\newline

\noindent In order to accomplish anything interesting with the hypothesized correspondence between algebras and spaces, we must define more ``non-commutative analogues''
of standard structures in topology and geometry. We have seen that ``non-commutative spaces'' correspond to non-commutative $C^{*}$-algebras. Our goal now
is to consider the specific case where the space is a smooth manifold, and construct the non-commutative analogue of a vector bundle structure on the ``non-commutative manifold''.
It is very import to make note of the following fact:

\begin{fact}
  Given manifold $M$, and $A = C^{\infty}(M)$ (the natural set of functions on $M$), it is only possible to recover $M$ \emph{topologically} from the algebaric properties of $A$.
  In particular, we require \emph{extra} information to re-deduce the geometric structure of $M$ (such as the notion of differential forms, the ability to integrate, etc.).
  Our definition of non-commutative vector bundles will be a first step in this direction, as it will surely end up being ``extra information'' which is specified on
  top of a given $C^{*}$-algebra.
\end{fact}

First, let $B = C^{\infty}(M)$, which is a $*$-algebra. This is of course a dense subalgebra of $A = C(M)$, which follows from Stone-Weierstrass (as every polynomial
function is smooth). A character $\mu : B \rightarrow \mathbb{C}$ can be uniquely extended to a character on $\overline{B} = A$, so that we don't lose any
information by working with $M(B)$ rather than $M(A)$.

Now, consider a vector bundle with base space $M$, and projection $\pi : E \rightarrow M$. We define our bundle morphisms $\phi : E \rightarrow E'$ in the usual sense
as continuous maps satisfying the compatibility condition $\pi' \circ \phi = \pi$ where the fibre-restricted maps $\phi_x : E_x \rightarrow E_x'$ are required to be linear maps between vector spaces.
Let $\Gamma(E) = C^{\infty}(M, E)$, the space of smooth sections (for $f \in \Gamma(E)$ and $x \in M$, $f(x) \in E_x$). For a given bundle map $\phi : E \rightarrow E'$, we can define $\Gamma \phi$ as the conjugate map from $\Gamma(E)$ to $\Gamma(E')$, with $\Gamma \phi(f) = \phi \circ f$.
It follows that for $a \in B$,
\begin{equation}
  \Gamma \phi(a \cdot s)(x) = (\phi \circ (a \cdot s))(x) = \phi(a(x) \cdot s(x)) = \phi_x(a(x) \cdot s(x)) = a(x) \cdot \phi(s(x)) = (a \cdot \Gamma \phi(s))(x)
\end{equation}
so $\Gamma \phi(a \cdot s) = a \cdot \Gamma \phi(s)$. Note that the multiplication between $a(x) \in \mathbb{C}$ and $s(x) \in E$ is done in the exepct way: $s(x) \in E_x$, which has the structure of a vector
space over $\mathbb{C}$, so we just do multiplication in this sense. Similarly, $\phi(s(x)) \in E'_x$, so the exterior multiplication by $a(x)$ works here as well. It follows that $\Gamma \phi$ is always
a morphism of $B$-modules, for a given bundle map $\phi$. It follows that $\Gamma$ can be thought of as a functor between vector bundles and $B$-modules (clearly, $C^{\infty}(M, E)$ is in fact such a module)!

From here, note that many standard operations which can be carried out on and between vector bundles factor through $\Gamma$ nicely. For example, the tensor product of vector bundles $E$ and $E'$
is merely defined as $\pi \otimes \pi' : E \otimes E' \rightarrow B$. It is straightforward to verify that
\begin{equation}
  \Gamma(E \otimes E') \simeq \Gamma(E) \otimes_B \Gamma(E')
\end{equation}
where $\otimes_B$ is all finite sums $\sum_j s_j \otimes s_j'$ subject to $as \otimes s' - s \otimes as' = 0$ for $a \in B$, $s \in \Gamma(E)$, $s' \in \Gamma(E')$.

Of course, $\Gamma(\textbf{Vector bundles})$ (the image of the functor), will (likely) not be \emph{all} $B$-modules (of which there are a lot), it will be some subcollection.
To figure out our desired algebra-geometry correspondence, it is necessary to identify precisely what this set is. To this end, we make use of the \emph{Serre-Swan theorem}.
First, we need a definition:

\begin{definition}
  An $A$-module $P$ is said to be \emph{projective} if for any surjective $A$-module homomorphism $f : M \rightarrow N$ and any $A$-module homomorphism $g : P \rightarrow N$,
  there exists an $A$-module homomorphism $h : P \rightarrow M$ such that $f \circ h = g$.
  \end{definition}

\begin{theorem}[Serre-Swan]
  Let $X$ be a compact Hausdorff space. Then a $C(X)$-module $P$ is isomorphic to a $C(X)$-module of the form $\Gamma(E)$ (a space of sections for some vector bundle $E$ over $X$) if and only if
  $P$ is finitely generated and projective.
\end{theorem}

\begin{remark}
  Note that in the case of $X$ being a manifold $M$, the above theorem carries over to the specific case when $P$ is a $C^{\infty}(M)$-module, and $\Gamma(E)$ is
  a space of \emph{smooth} sections.
\end{remark}

To prove this result, we will require a collection of lemmas. To get a feel for the main ideas of the proof, let us prove an easier version
of the theorem, for the case of trivial bundles

\begin{theorem}[Trivial Serre-Swan]
  For $X$ a topological space, a $C(X)$-module $P$ is isomorphic to a $C(X)$-module $\Gamma(E)$ where $E$ is a \emph{trivial bundle} (of the form $X \times \mathbb{F}^k$) if and only if
  $P$ is finitely generated and \emph{free}.
\end{theorem}

\begin{proof}
  Suppose that $E = X \times \mathbb{F}^{k}$ for some $k$, then the collection $\{(x, e_1), \dots, (x, e_k)\}$ for the $e_j$ are
  the standard basis for $\mathbb{F}^k$ are a global free basis for the \emph{module} $\Gamma(E)$, and thus $\Gamma(E)$ is free (and finitely generated).
  Conversely, if $P$ is finitely generated and free, $P = \langle e_1, \dots, e_r \rangle$ for some $r$. Thus, a generic element of $P$ (as a $C(X)$-module)
  is of the form
  \begin{equation}
    p = c_1(x) e_1 + \cdots + c_r(x) e_r \simeq (c_1(x), \dots, c_r(x))
  \end{equation}
  which immediately implies that $P$ is isomorphic to $C(X, \mathbb{F}^r) \simeq \Gamma(X \times \mathbb{F}^r)$.
\end{proof}

\begin{remark}
  To prove one direction of the more general theorem, the main idea will be to demonstrate that a vector bundle $E$ can be written as a direct summand of a trivial bundle (i.e. there is another bundle $\widetilde{E}$ such that
  the Whitney sum $E \oplus \widetilde{E}$ is trivial). Then, we can use the above result to conclude that $\Gamma(E \oplus \widetilde{E}) \simeq \Gamma(E) \oplus \Gamma(\widetilde{E})$ is isomorphic to the finitely-generated free $C(X)$-module,
  which will in turn allow us to show that $\Gamma(E)$ is projective.
\end{remark}

\begin{lemma}
  Suppose $P$ is a finitely-generated $A$-module. Then, there exists a finitely-generated $A$-module $Y$ such that $P \oplus Y$ is free if and only if $P$ is a projective module.
\end{lemma}
\begin{proof}
  To begin, assume $P$ is a projective $A$ module with the usual definition. Let $\{p_i\}_{i = 1}^{n}$ be a set of generators for $P$. Let $F = \langle e_1, \dots, e_n \rangle$.
  Define $f : F \rightarrow P$ as $f(e_i) = p_i$ extended linearly. This is clearly a surjective homomorphism. Thus, this map gives rise to the following short exact sequence:
  \begin{equation}
    0 \longrightarrow \text{Ker}(f) \longrightarrow F \longrightarrow P \longrightarrow 0
  \end{equation}
  where the first map is an inclusion $j$ and the second is $f$. In particular, $\text{Im}(f) = P$, so the final mapping sending all of $P$
  to $0$ satisfies the exactness condition. From the definition of $P$ being projective, with $N = P, M = F$ and $g = \text{id}$, there
  exists a map $h : P \rightarrow F$ such that $f \circ h = \text{id}$, or in other words, $f$ has a right-inverse. When we put $h$
  on the left, and $h \circ f$ is a map from $F$ to $F$, we notice that
  \begin{equation}
    (h \circ f) \circ (h \circ f) = h \circ (f \circ h) \circ f = h \circ f
  \end{equation}
  so that $h \circ f$ is a \emph{projection mapping}. It follows immediately that $h \circ f$ splits the space $F$ as $F = \text{Im}(h \circ f) \oplus \text{Ker}(h \circ f)$.
  This simply follows from the fact that for $m \in M$,
  \begin{equation}
    m = (h \circ f)(m) \oplus (1 - (h \circ f))(m)
  \end{equation}
  where $(h \circ f)(m)$ is in the image, and $(1 - (h \circ f))(m)$ is clearly in the kernel. Since $f$ is surjective, $\text{Im}(h \circ f) = \text{Im}(h)$. Of course, $h : P \rightarrow h(P)$
  has been shown to have an inverse: $f$ restricted to $\text{Im}(h)$, so $h(P) \simeq P$. Moreover, since $h$ is invertible, $\text{Ker}(h \circ f) = \text{Ker}(f)$. Thus, $F \simeq P \oplus \text{Ker}(f)$,
  where $F$ is free, by definition.

  Conversely, suppose that there exists $Y$ such that $F = P \oplus Y$ is free. Let $M, N$, $f : M \rightarrow N$ and $g : P \rightarrow N$ be
  as they are in the definition of a projective module. Let $k : F \rightarrow N$ be an arbitrary function such that $k|_P = g$. If
  we take $\{e_i\}_{i = 1}^{n}$ as the generating set for $F$, then $k(e_i) \in N$ for each $e_i$. Since $f$ is surjective, we can
  find for each $e_i$ some $t_i \in M$ where $f(t_i) = k(e_i)$. We define $q : F \rightarrow M$ as $q(e_i) = t_i$, and extend linearly, which is obviously a module homomorphism.
  If we let $h = q|_{P}$ be a map from $P$ to $M$, then we immediately see that for $\sum_{j} c_j e_j \in P$, then
  \begin{equation}
    (f \circ h)\left(\sum_{j} c_j e_j\right) = \sum_{j} c_j f(t_j) = \sum_{j} c_j k(e_j) = g \left( \sum_{j} c_j e_j \right)
  \end{equation}
  so that $f \circ h = g$, and by definition, $P$ is a projective module.
\end{proof}

Now, let us prove the desired result which will allow us to conclude one direction of the Serre-Swan theorem: that every vector bundle will be a direct summand of a trivial bundle.
Let us state a technical lemma related to vector bundles which we will make use of, but will skip for the sake of brevity (as it is fairly involved):

\begin{lemma}
  If $X$ is a normal topological space, then given any $C(X)$-module homomorphism $F : \Gamma(E) \rightarrow \Gamma(E')$ between spaces of sections, there
  is a unique bundle map $f : E \rightarrow E'$ (i.e. a continuous map where the restirction to the fibres is an homomorphism of vector spaces) such that $F = \Gamma(f)$ (where
  we are using $\Gamma$ in the functorial sense here).
\end{lemma}

Making use of this result, we can prove two technical lemmas:

\begin{lemma}
  Let $X$ be a compact Hausdorff space (thus normal), let $E$ be a vector bundle. Then there exists a trivial bundle $\Pi$
  and a surjective bundle map $f : \Pi \rightarrow E$.
\end{lemma}

\begin{proof}
  The main idea here is to choose a local basis $s(x) = (s_1(x), \dots, s_n(x))$ for each fibre in $E$ (we can do this as $E$ is locally trivial). Since $X$ is normal,
  we can find a partition of unity which combines all of the $s(x)$ in a continuous (or smooth, if it is a smooth manifold) fashion, which yields a global
  basis $\widetilde{s}(x)$ for the fibre over each $x \in X$. We let $\Pi = X \times \mathbb{F}^n$, where $\mathbb{F}^n$ always has a basis $\{e_i\}_{i = 1}^{n}$. Define $F : \Gamma(\Pi) \rightarrow \Gamma(E)$
  as $F(e_i) = \widetilde{s}_i$, extended linearly. From the previous theorem, there is a unique bundle map $f : \Pi \rightarrow E$ such that $F = \Gamma(f)$, so $f \circ e_i = \widetilde{s}_i$.
  It follows that $\widetilde{s}_i(x)$ is in the image of $f$, so any $x$, and since this collection spans each fibre, the map $f$ is surjective.
  \end{proof}

\begin{lemma}
  Let $E$ be a vector bundle with an inner product, and let $E'$ be a subbundle. Then, any exact sequence
  \begin{equation}
    0 \longrightarrow E' \longrightarrow E \longrightarrow G \longrightarrow 0
  \end{equation}
  splits, with $E \simeq E' \oplus G$.
\end{lemma}
\begin{proof}
  \textbf{Note:} This proof largely follows the exact same logic as Lemma III.1, so we leave out some details.
  \newline

  Let $j$ be the inclusion map from $E'$ to $E$, let $b$ be the map from $E$ to $G$. We define $p : E \rightarrow E'$ as the orthogonal projection via the inner product onto
  $E'$ (as $E'$ is a subbundle). Note that because the sequence is exact, $\text{Ker}(b) = j(E') = E' = \text{Im}(p)$.
  The restriction of $b$ to $\text{Ker}(p)$ will induce an isomorphism $G \simeq \text{Im}(b)$, and will induce the desired splitting via the same logic as was applied in Lemma III.1.

  From here, since the sequence splits, $b$ has a right-inverse, $b \circ a = 1$, with $a : G \rightarrow E$. It is easy to see that $a \circ b$ is a projection (again, via the same
  logic as Lemma III.1), so $E = \text{Im}(a \circ b) \oplus \text{Ker}(a \circ b)$. Again, via the same logic as Lemma III.1, $\text{Im}(a \circ b) \simeq G$ and $\text{Ker}(a \circ b) \simeq \text{Ker}(b) = E'$,
  so $E \simeq E' \oplus G$ as desired.
  \end{proof}

From these two results, the following result holds immediately:

\begin{theorem}[Serre-Swan, first direction]
  If $X$ is a compact Hausdorff space, and $\Gamma(E)$ is the space of sections over the space treated as a $C(X)$-module, then it is finitely generated and projective.
\end{theorem}

\begin{proof}
  Let $\Pi$ be a trivial bundle, from above, we can find a surjective bundle map $f : \Pi \rightarrow E$. The sequence $\text{Ker}(f) \to \Pi \to E$
  is clearly an exact sequence, implying (from above), that $\Pi \simeq \text{Ker}(f) \oplus E$. Thus, $\Gamma(\Pi) \simeq \Gamma(\text{Ker}(f)) \oplus \Gamma(E)$.
  $\Gamma(E)$ is a direct summand of the free-module $\Gamma(\Pi)$ (we proved this module is free earlier), so $\Gamma(E)$ is finitely-generated and projective.
  \end{proof}

Proving the other direction requires a little bit more work, but we can do it right away:

\begin{theorem}[Serre-Swan, other direction]
If $P$ is a finitely-generated projective $C(X)$-module, for some compact Hausdorff space $X$, then $P \simeq \Gamma(E)$, for some vector bundle $E$ over $X$.
\end{theorem}

\begin{proof}
  Let $P$ be a finitely generated projective $C(X)$-module. We already proved that $P$ is then a direct summand of a finitely-generated free $C(X)$-module $F$.
  Of course, then $P$ is the image of a projection map $p$ from $F$ to itself, which takes $P$ to $P$ and the other direct summand of the free group to $0$.
  We know that $F = \Gamma(\Pi)$, for a trivial bundle $\Pi$. Thus, $p$ is actually a map from $\Gamma(\Pi)$ to itself, so by Lemma III.2, $p$ lifts
  to a bundle map $\widetilde{p} : \Pi \rightarrow \Pi$ with $\Gamma(\widetilde{p}) = p$. But then $\Gamma(\text{Im}(\widetilde{p})) = \text{Im}(p) = P$,
  so that $P$ is in fact $\Gamma(E)$, with $E = \text{Im}(\widetilde{p})$, and we are done.
  \end{proof}

\hhrulefill

\noindent Now that we have proved Serre-Swan, let us return to the idea of a non-commutative vector bundle. In particular, recall
that we were concerned with finding the image of the function $\Gamma$. The above theorem has shown a direct correspondence between
smooth sections and finitely-generated projective modules. Thus, we arrive at the pentultimate result:

\begin{claim}
  Just as a ``non-commutative topological space'' is essentially just a non-commutative $C^{*}$-algebra, a ``non-commutative vector bundle''
  is simply a finitely-generated projective module. This is due to the fact that $\Gamma$ is a covariant (as it does not reverse the direction of morphisms)
  functor from the category of vector bundles over $M$ to the category of projective modules over $C^{\infty}(M)$.
\end{claim}

As it turns out, there are many more results just like this, which yield non-commutative variants of spin structure, the Dirac operator,
and many other important constructs in differential geometry and topology. All of these constructions, combined together under a unified
framework, is the motivation and the starting point for the rich field of \emph{non-commutative geometry}.

\end{document}
