\documentclass[aps,pra,showpacs,notitlepage,onecolumn,superscriptaddress,nofootinbib]{revtex4-1}
\usepackage[utf8]{inputenc}
\usepackage[tmargin=1in, bmargin=1.25in, lmargin=1.5in, rmargin=1.5in]{geometry}
\usepackage{amsmath, amssymb, amsthm}
\usepackage{graphicx}
\usepackage{xcolor}
\usepackage{enumitem}
\usepackage{datetime}
\usepackage{hyperref}
\usepackage{titlesec}
\usepackage{import}
\usepackage{mathtools}
\usepackage{thmtools,thm-restate}
\usepackage{comment}


% package for commutative diagrams
% \usepackage{tikz-cd}

%%%%%%%%%%%%%%%%%%%%%%%%%%%%%%%%%%%%%%%%%%%%%
\definecolor{crimson}{RGB}{186,0,44}
\definecolor{moss}{RGB}{0, 186, 111}
\newcommand{\pop}[1]{\textcolor{crimson}{#1}}
\newcommand{\zcom}[1]{\noindent\textcolor{crimson}{(Z): #1}}
\newcommand{\jcom}[1]{\noindent\textcolor{moss}{(J): #1}}
\newcommand{\wt}[1]{\widetilde{#1}}
\newcommand{\pqeq}{\succcurlyeq}
\newcommand{\pleq}{\preccurlyeq}

\newcommand{\hhrulefill}{\hspace{-1.0em}\hrulefill}


%%%%%%%%%%%%%%%%%%%%%%%%%%%%%%%%%%%%%%%%%%%%%
\hypersetup{
    colorlinks,
    linkcolor={crimson},
    citecolor={crimson},
    urlcolor={crimson}
}

\usepackage{qcircuit}

%%%%%%%%%%%%%%%%%%%%%%%%%%%%%%%%%%%%%%%%%%%%%
\theoremstyle{definition}
\newtheorem{definition}{Definition}[section]
\newtheorem{lemma}{Lemma}[section]
\newtheorem{theorem}{Theorem}[section]
\newtheorem{corollary}{Corollary}[theorem]
\newtheorem*{theorem*}{Theorem}
\newtheorem*{corollary*}{Corollary}
\newtheorem{remark}{Remark}[section]
\newtheorem{conjecture}{Conjecture}[section]
\newtheorem{example}{Example}[section]
\newtheorem{reminder}{Reminder}[section]
\newtheorem{problem}{Problem}[section]
\newtheorem{question}{Question}[section]
\newtheorem{proposition}{Proposition}[section]
\newtheorem{answer}{Answer}[section]
\newtheorem{fact}{Fact}[section]
\newtheorem{claim}{Claim}[section]

\usepackage{geometry}
\geometry{
  left=25mm,
  right=25mm,
  top=20mm,
}

%%%%%%%%%%%%%%%%%%%%%%%%%%%%%%%%%%%%%%%%%%%%%
\bibliographystyle{unsrt}

%%%%%%%%%%%%%%%%%%%%%%%%%%%%%%%%%%%%%%%%%%%%%
%%%%%%%%%%%%%%%%%%%%%%%%%%%%%%%%%%%%%%%%%%%%%
%%%%%%%%%%%%%%%%%%%%%%%%%%%%%%%%%%%%%%%%%%%%%
\begin{document}

\title{A primer on cyclic cohomology}
\author{Jack Ceroni}
\email{jackceroni@gmail.com}

\date{\today}

\maketitle

\section{Introduction}

\noindent \emph{This essay was written for the Fall 2023 session of MAT437 at the University of Toronto.}
\newline

\noindent The goal of this essay is to introduce, explain, and place in broader context the basic theory of cyclic cohomology. The main reference
for this essay is the book by Moriyoshi and Natsume (Ref.~\cite{}), as well as review articles [CITE]. This book provides a brief introduction to non-commutative geometry: we attempt
to expand upon some of their main points in this work.

\hhrulefill

\section{Motivation and background}

\subsection{Trace maps}

\noindent The $K_0$-group of a $C^{*}$-algebra $A$ is a broadly powerful invariant which encodes the underlying structure of $A$, in such a way that lends
itself well to being analyzed via a large toolbox of different techniques. Among such techniques are extracting numerical invariants from $K_0(A)$ via \emph{trace maps}.

\begin{definition}[Trace map on a $C^{*}$-algebra]
\end{definition}

\noindent Using the functoriality of $K_0$, it is possible to then obtain a trace map $K_o(\tau) : K_0(A) \rightarrow \mathbb{C}$.

\begin{claim}[Induced trace map]
\end{claim}
\begin{proof}
  \end{proof}

\noindent Trace maps of this form are vital to the analysis of $K_0$-groups. However, as it turns out, one will often encounter $C^{*}$-algebras which \emph{admit no non-trivial trace}.
For example, one of the most important types of $C^{*}$-algebras are the so-called \emph{Cuntz algebras}, which are defined as follows.

\begin{definition}[Cuntz algebra]
  \end{definition}

\noindent This is a problem, as in such cases, it is very much unclear how one should go about extracting numerical invariants from $K_0(A)$. Cyclic cohomology, discovered by Alain Connes, can
be thought of as a remedy to this situation: it provides a different technique for mapping from $K_0$-groups to numbers.

\appendix

\section{Proof of Lem.~\ref{lem:}}

\end{document}
