\documentclass[aps,pra,showpacs,notitlepage,onecolumn,superscriptaddress,nofootinbib]{revtex4-1}
\usepackage[utf8]{inputenc}
\usepackage[tmargin=1in, bmargin=1.25in, lmargin=1.5in, rmargin=1.5in]{geometry}
\usepackage{amsmath, amssymb, amsthm}
\usepackage{graphicx}
\usepackage{xcolor}
\usepackage{enumitem}
\usepackage{datetime}
\usepackage{hyperref}
\usepackage{titlesec}
\usepackage{import}
\usepackage{mathtools}
\usepackage{thmtools,thm-restate}
\usepackage{comment}


% package for commutative diagrams
% \usepackage{tikz-cd}

%%%%%%%%%%%%%%%%%%%%%%%%%%%%%%%%%%%%%%%%%%%%%
\definecolor{crimson}{RGB}{186,0,44}
\definecolor{moss}{RGB}{0, 186, 111}
\newcommand{\pop}[1]{\textcolor{crimson}{#1}}
\newcommand{\zcom}[1]{\noindent\textcolor{crimson}{(Z): #1}}
\newcommand{\jcom}[1]{\noindent\textcolor{moss}{(J): #1}}
\newcommand{\wt}[1]{\widetilde{#1}}
\newcommand{\pqeq}{\succcurlyeq}
\newcommand{\pleq}{\preccurlyeq}

\newcommand{\hhrulefill}{\hspace{-1.0em}\hrulefill}


%%%%%%%%%%%%%%%%%%%%%%%%%%%%%%%%%%%%%%%%%%%%%
\hypersetup{
    colorlinks,
    linkcolor={crimson},
    citecolor={crimson},
    urlcolor={crimson}
}

\usepackage{qcircuit}

%%%%%%%%%%%%%%%%%%%%%%%%%%%%%%%%%%%%%%%%%%%%%
\theoremstyle{definition}
\newtheorem{definition}{Definition}[section]
\newtheorem{lemma}{Lemma}[section]
\newtheorem{theorem}{Theorem}[section]
\newtheorem{corollary}{Corollary}[theorem]
\newtheorem*{theorem*}{Theorem}
\newtheorem*{corollary*}{Corollary}
\newtheorem{remark}{Remark}[section]
\newtheorem{conjecture}{Conjecture}[section]
\newtheorem{example}{Example}[section]
\newtheorem{reminder}{Reminder}[section]
\newtheorem{problem}{Problem}[section]
\newtheorem{question}{Question}[section]
\newtheorem{proposition}{Proposition}[section]
\newtheorem{answer}{Answer}[section]
\newtheorem{fact}{Fact}[section]
\newtheorem{claim}{Claim}[section]

\usepackage{geometry}
\geometry{
  left=25mm,
  right=25mm,
  top=20mm,
}
\usepackage{tikz-cd}

%%%%%%%%%%%%%%%%%%%%%%%%%%%%%%%%%%%%%%%%%%%%%
\bibliographystyle{unsrt}

%%%%%%%%%%%%%%%%%%%%%%%%%%%%%%%%%%%%%%%%%%%%%
%%%%%%%%%%%%%%%%%%%%%%%%%%%%%%%%%%%%%%%%%%%%%
%%%%%%%%%%%%%%%%%%%%%%%%%%%%%%%%%%%%%%%%%%%%%
\begin{document}

\title{Fall 2023 MAT437 problem set 7}
\author{Jack Ceroni}
\email{jackceroni@gmail.com}

\date{\today}

\maketitle

\hhrulefill

\section{Problem 1}

\noindent \emph{\pop{This is a theorem in the book: I thought that I would try to prove it myself \textbf{without} looking at the proof provided in the book (until close to the end), to make sure that I understand the inductive limit construction well.}}
\newline

  \noindent \emph{\pop{I also tried my best to explain my thought process, including a couple incorrect initial guesses I had while doing this proof, and why they eventually failed.}}
\newline

\begin{proposition}
  Every inductive sequence of $C^{*}$-algebras, $A_1 \xrightarrow{\varphi_1} A_2 \xrightarrow{\varphi_2} A_3 \xrightarrow{\varphi_3} \cdots$, has an inductive limit $(A, \{\mu_n\})$.
\end{proposition}

\begin{proof}
  \noindent First, let us consider the following naive, \emph{incorrect} approach:
  \newline

  \noindent \textbf{\pop{The wrong approach:}} Consider the sequence $\{a_{n, m}\}_{m = n + 1}^{\infty}$ where $a_{n, m} = \varphi_{m, n}(a)$, for some $a \in A_n$. We may be inclined, at first glance, to
  take $\mu_n(a)$ to be the ``limit'' of this sequence. However, this wouldn't really make sense: each element of this sequence may be an element of a different algebra,
  with a different norm, meaning there isn't a good way to compare elements of this sequence. Even worse, the norms of the individual elements of this sequence may go to infinity!
  \newline

  \noindent We have to begin by defining the space $A$ at the end of the sequence. Intuitively, we want this space to reflect what each element of the complex of $C^{*}$-algebras eventually becomes
  as it is mapped through the infinite sequence of $\varphi_k$ maps. This implies that sequences which \emph{eventually} become identical, should be identified together in the ``eventual space'', $A$.
  This leads us to define
  \begin{equation}
    \mathcal{A} = \displaystyle\prod_{n \in \mathbb{N}} A_n \Big/ \displaystyle\sum_{n \in \mathbb{N}} A_n
  \end{equation}
  as RLL showed that $\sum_{n} A_n$ is a closed ideal in $\prod_{n} A_n$, so $\mathcal{A}$ is a valid $C^{*}$-algebra. We define $\mu_n : A_n \rightarrow \mathcal{A}$ as follows: we simply map $a \in A_n$
  to $[(0, \dots, 0, a, \varphi_{n + 1, n}(a), \varphi_{n + 2, 2}(a), \dots)]$: an equivalence class. Note that $\xi \in \mathcal{A}$ because $\alpha = (0, \dots, 0, a, \varphi_{n + 1, n}(a), \varphi_{n + 2, 2}(a), \dots) \in \prod_{n} A_n$.
  This is because $\sup_n \alpha_n$ is finite: recall that we proved in an earlier problem set that if $\varphi$ is a $*$-homomorphism between $C^{*}$-algebras $A$ and $B$, then $||\varphi(a)|| \leq ||a||$. It follows that for some $m > n$,
  $||\varphi_{m, n}(a)|| \leq ||a||$.
  \newline

  \noindent Our claim now is that the
  complex $(\mathcal{A}, \{\mu_n\})$ constitutes the desired inductive limit (\pop{This is actually slightly incorrect, as we will show}).
  \newline

  \noindent The first step is to show that the diagram:
  % https://q.uiver.app/#q=WzAsMyxbMCwwLCJBX24iXSxbMiwwLCJBX3tuKzF9Il0sWzEsMSwiQSJdLFswLDEsIlxcdmFycGhpX24iXSxbMSwyLCJcXG11X3tuKzF9Il0sWzAsMiwiXFxtdV9uIiwyXV0=
  \[\begin{tikzcd}
	          {A_n} && {A_{n+1}} \\
	          & \mathcal{A}
	          \arrow["{\varphi_n}", from=1-1, to=1-3]
	          \arrow["{\mu_{n+1}}", from=1-3, to=2-2]
	          \arrow["{\mu_n}"', from=1-1, to=2-2]
  \end{tikzcd}\]
  commutes. Indeed, this is a triviality, as $(\mu_{n + 1} \circ \varphi_n)(a) = [(0, \dots, 0, \varphi_n(a), \varphi_{n + 1, n}(a), \dots)]$ and $\mu_n(a) = [(0, \dots, 0, a, \varphi_n(a), \dots)]$
  which are clearly the same equivalence class as their difference is $(0, \dots, 0, a, 0, \dots)$: a sequence which is eventually $0$. As the second and final step, we must show that if $(B, \{\lambda_n\})$
  is a system in which $\lambda_n : A_n \rightarrow B$ with $\lambda_n = \lambda_{n + 1} \circ \varphi_n$, then there is precisely one $\lambda$ making the diagram:
  % https://q.uiver.app/#q=WzAsMyxbMSwwLCJBX24iXSxbMCwxLCJBIl0sWzIsMSwiQiJdLFswLDEsIlxcbXVfbiIsMl0sWzAsMiwiXFxsYW1iZGFfbiJdLFsxLDIsIlxcbGFtYmRhIiwyXV0=
  \[\begin{tikzcd}
	& {A_n} \\
	\mathcal{A} && B
	\arrow["{\mu_n}"', from=1-2, to=2-1]
	\arrow["{\lambda_n}", from=1-2, to=2-3]
	\arrow["\lambda"', from=2-1, to=2-3]
  \end{tikzcd}\]
  commute for each $n$. As it turns out, \emph{we won't be able to do this}, because $\mathcal{A}$ is ``too big'' to guarantee uniqueness of $\lambda$. Indeed, the collection of all $\mu_n(A_n)$
  could be a small subset of $\mathcal{A}$, and $\lambda$ could vary drastically on the remainder of the codomain $\mathcal{A}$. This leads us to make the codomain ``as small as possible''. We define
  \begin{equation}
    A = \overline{\displaystyle\bigcup_{n \in \mathbb{N}} \mu_n(A_n)} = \overline{A'} \subset \mathcal{A}
  \end{equation}
  Clearly, $A$ is a $C^{*}$-algebra (note that it is closed in the metric topology). Note that replacing $\mathcal{A}$ with $A$ in the first commutative diagram does not alter the proof. We will define our map $\lambda$ as follows.
  \newline

  \noindent Let's assume that such a system $(B, \{\lambda_n\})$ exists. Suppose $x \in A_n$ and $y \in A_m$ are two elements which are mapped to the same element of $A$. In other words,
  there will exist some $N$ such that for $k \geq N$, $\varphi_{k, n}(x) = \varphi_{k, m}(y)$ (the sequences generated by these elements must eventually become equal). It follows that for some such $m$
  \begin{equation}
    \lambda_n(x) = (\lambda_{n + 1} \circ \varphi_n)(x) = (\lambda_{k + 1} \circ \varphi_{k} \circ \cdots \circ \varphi_n)(x) = \lambda_{k + 1}(\varphi_{k, n}(x)) = \lambda_{k + 1}(\varphi_{k, m}(y)) = \lambda_m(y)
  \end{equation}
  It follows that $\lambda = \lambda_n \circ \mu_n^{-1}$ constitutes a well-defined mapping: given some $a \in \bigcup_{n} \mu_n(A_n) = A'$, we
  note that $a = \mu_n(a_n)$ for some $n \in \mathbb{N}$ and some $a_n \in A_n$, and we take $\lambda(a) = \lambda_n(a_n)$. The above logic showed that $\lambda$ is in fact independent of the choice of both $n$ and $a_n$.
  \newline

  \noindent We have successfully defined a map from $A'$ to $B$. Note that $A'$ is a sub-$*$ algebra of the $C^{*}$ algebra $A$. In addition, note that $\lambda : A' \rightarrow B$ is a $*$-homomorphism
  from the sub-$*$ algebra $A'$ into $B$.  For $a, b \in A$, note that $\mu_n^{-1}(a) + \gamma \mu_n^{-1}(b) \subset \mu_n^{-1}(a + \gamma b)$, so
  that $\lambda_n (\mu_n^{-1}(a) + \gamma \mu_n^{-1}(b)) = \lambda(a) + \gamma \lambda(b) \subset \lambda(a + \gamma b)$. But of course these are both one-point sets, so we must have $\lambda(a) + \gamma \lambda(b) \subset \lambda(a + \gamma b)$.
  Using the same argument for the multiplicativity and $*$-preserving properties, we can conclude that $\lambda$ is a $*$-homomorphism.
  \newline

  \noindent We have shown that there exists a $*$-homomorphism from $A'$ to $B$ which satisfies the desired properties. Note that this $*$-homomorphism is the unique one from $A'$ to $B$ satisfying the desired properties: suppose $\lambda' : A' \rightarrow B$
  also satisfied $\lambda' \circ \mu_n = \lambda_n$ for each $n$. Since $\lambda = \lambda_n \circ \mu_n^{-1}$, we would have $\lambda = \lambda' \circ \mu_n \circ \mu_n^{-1} = \lambda'$. If there \emph{did} exist an extension of $\lambda$ to a $*$-homomorphism
  from $A$ (the closure of $A'$) to the closure of $B$, then note that it would necessarily be unique, as such a function would be continuous, and continuous functions agreeing on a dense subset are equal.
  \newline

  \noindent Thus, the only remaining task is to show that \textbf{we can extend $\lambda : A' \rightarrow B$ to a $*$-homomorphism from $A = \overline{A'}$ to $B$}. \pop{At this point, I couldn't quite finish the proof myself,
    so I prompted myself by consulting the book. Annoyingly, the book had practically nothing to say on this matter, it simply asserted that ``we can extend $\lambda : A' \rightarrow B$ by uniform continuity to $\lambda : A \rightarrow B$''.}

\end{proof}

\begin{comment}
  Let $A$ and $B$ be $C^{*}$-algebras, let $A' \subset A$ be a sub-$*$ algebra which is dense in $A$, with the subspace topology inherited from the metric topology on $A$.
  Suppose $\varphi : A' \rightarrow B$ is a $*$-homomorphism. Then there exists a unique extension of $\varphi$ to a $*$-homomorphism $\Phi : A \rightarrow B$ with $\varphi(a) = \Phi(a)$
  for all $a \in A'$.

  To begin, let $\widetilde{A}$ and $\widetilde{B}$ denote the unitizations of $A$ and $B$. Clearly, $\widetilde{A'} = \{ (a, \alpha) \ | \ a \in A', \alpha \in \mathbb{C} \}$
  equipped with the algebraic operations of $\widetilde{A}$ is a sub-$*$ algebra of $\widetilde{A}$. Define $\widetilde{\varphi} : \widetilde{A'} \rightarrow \widetilde{B}$
  as $\widetilde{\varphi}(a, \alpha) = (\varphi(a), \alpha)$. It is easy to verify this is a $*$-homomorphism.
\end{comment}

\begin{proposition}
Given an inductive sequence of $C^{*}$-algebras $A_1 \xrightarrow{\varphi_1} A_2 \xrightarrow{\varphi_2} A_3 \xrightarrow{\varphi_3} \cdots$ with inductive limit $(A, \{\mu_n\})$, the following facts holds:
\begin{enumerate}
\item $||\mu_n(a)|| = \lim_{m \to \infty} ||\varphi_{m, n}(a)||$ for all $n \in \mathbb{N}$ and $a \in A_n$. This
  immediately implies that $\text{Ker}(\mu_n) = \{a \in A_n \ | \ \lim_{m \to \infty} ||\varphi_{m, n}(a)|| = 0\}$.
\item If $(B, \{\lambda_n\})$ and $\lambda : A \rightarrow B$ are as in Def. 6.2.2 of RLL, then $\text{Ker}(\mu_n) \subset \text{Ker}(\lambda_n)$
  for all $n \in \mathbb{N}$, $\lambda$ is injective if and only if $\text{Ker}(\lambda_n) \subset \text{Ker}(\mu_n)$ for all $n \in \mathbb{N}$,
  and $\lambda$ is surjective if and only if $B = \overline{\bigcup_{n} \lambda_n(A_n)}$.
\end{enumerate}
\end{proposition}

\begin{proof}
  To prove the first claim, lt us recall Lemma 6.1.3 of RLL, namely that if we let $\pi : \prod_{n} A_n \rightarrow \prod_{n} A_n \Big/ \sum_{n} A_n$ be the quotient map, then $||\pi(a)|| = \limsup_{n \to \infty} ||a_n||$.
  Note that
  $$\mu_n(a) = [(0, \dots, 0, a, \varphi_n(a), \varphi_{n + 1, n}(a), \dots)] = \pi(0, \dots, 0, a, \varphi_n(a), \varphi_{n + 1, n}(a), \dots).$$
  Thus,
  \begin{equation}
    ||\mu_n(a)|| = \limsup_{m \to \infty} || (0, \dots, 0, a, \varphi_n(a), \varphi_{n + 1, n}(a), \dots)_m || = \limsup_{m \to \infty} ||\varphi_{m, n}(a)||
  \end{equation}
  Note that the sequence $\varphi_{m, n}(a)$ is strictly decreasing in norm, as for $*$-homomorphisms, we know that $||\varphi_{m + 1, n}(a)|| = ||(\varphi_{m + 1} \circ \varphi_{m, n})(a)|| \leq ||\varphi_{m, n}(a)||$.
  Thus, the limit superior is actually just the limit (in particular, the limit exists), and we have the desired result.
  \newline

  \noindent To prove the second claim, it is clear that if $\mu_n(a) = 0$, then $\lambda_n(a) = (\lambda \circ \mu_n)(a) = \lambda(0) = 0$, so $\text{Ker}(\mu_n) \subset \text{Ker}(\lambda_n)$.
  Clearly, a $*$-homomorphism is injective if and only if it has trivial kernel, so $\lambda(a) = 0$ must imply $a = 0$. In particular, if $\lambda_n(a) = \lambda(\mu_n(a)) = 0$, then $\mu_n(a) = 0$,
  so $\text{Ker}(\lambda_n) \subset \text{Ker}(\mu_n)$.
  \newline

  \noindent Finally, note that since $\lambda$ is continuous, we have
  \begin{equation}
    \text{Im}(\lambda) = \lambda(A) = \lambda \left( \overline{\displaystyle\bigcup_{n \in \mathbb{N}} \mu_n(A_n)} \right) \subset \overline{\displaystyle\bigcup_{n \in \mathbb{N}} (\lambda \circ \mu_n)(A_n)} = \overline{\displaystyle\bigcup_{n \in \mathbb{N}} \lambda_n(A_n)} \subset B
  \end{equation}
  Of course, if $\lambda$ is surjective, then $\text{Im}(\lambda) = B$, so since it lies between the two sets, we must have $\overline{\bigcup_{n \in \mathbb{N}} \lambda_n(A_n)} = B$ as well. \pop{I got stuck at this point again (proving the converse). From consulting the book again,
    the converse seems to follow from the extension theorem invoked earlier in the proof.}
  \end{proof}

\hhrulefill

\section{RLL Problem 6.1 (Suggested Problem 1)}

\noindent \emph{\pop{Equipped with the basic knowledge of inductive limits, let us answer a few questions regarding the calculation of simple inductive limits.}}
\newline

\noindent \textbf{Part 1.} Of course, in this exercise, we are tasked with computing the inductive limit of a sequence of Abelian groups.
Homomorphism from the group of integers to some other group are completely determined by where $1$ is sent. Let us define $\mu_n : \mathbb{Z} \rightarrow \mathbb{Q}$
as $\mu_n(1) = \frac{1}{(n - 1)!}$. We can easily verify that $(\{\mu_n\}, \mathbb{Q})$ is the inductive limit of the sequence of $\mathbb{Z}$ with $\varphi_n = n$.
\newline

\noindent First, note that
\begin{equation}
  (\mu_{n + 1} \circ \varphi_n)(k) = \frac{nk}{n!} = \frac{k}{(n - 1)!} = \mu_n(k)
\end{equation}
so we have verified the first condition. Moreover, given $B$ and $\lambda_n$ such that $\lambda_{n + 1} \circ \mu_n = \lambda_n$, we can simply define $\lambda : \mathbb{Q} \rightarrow B$ as
$\lambda(k) = \lambda_n((n - 1)! k)$. Obviously, $\lambda \circ \mu_n = \lambda_n$. In addition, $\lambda$ does not depend on $n$ because we have
\begin{equation}
 \lambda_n((n - 1)! k) = (\lambda_{m} \circ \varphi_{m - 1, n})((n - 1)! k) = \lambda_m((m - 1)! k)
\end{equation}
for some $m > n$. To verify uniqueness, note that if $\lambda'$ also satisfies the desired property, then
\begin{align}
  \lambda' \left( \frac{a}{b} \right) = \lambda' \left( \frac{(b - 1)!}{(b - 1)!} \frac{a}{b} \right) &= \lambda' \left( \frac{(b - 1)! a}{b!} \right) = (\lambda' \circ \mu_{b + 1})((b - 1)! a)
  \\ & = \lambda_{b + 1}((b - 1)! a) = (\lambda \circ \mu_{b + 1})((b - 1)! a) = \lambda \left( \frac{a}{b} \right)
\end{align}
Thus, by definition, $(\{\mu_n\}, \mathbb{Q})$ is in fact the inductive limit of this sequence.
\newline

\noindent \textbf{Part 2.} We take a similar approach to before. In particular, we define $\mu_n(1) = \frac{1}{2^{n - 1}}$. In this case, we will clearly have $(\varphi_n \circ \mu_{n + 1})(k) = \varphi_n\left( \frac{k}{2^{n}} \right) = \frac{k}{2^{n - 1}} = \mu_{n}(k)$.
If the maps $\mu_n$ are indeed the morphisms giving rise to the inductive limit, then we know that we must have
\begin{equation}
  G_2 = \displaystyle\bigcup_{n \in \mathbb{N}} \mu_n(\mathbb{Z}) =\left\{ \frac{p}{2^{n}} \ \Big| \ p, n \in \mathbb{Z} \right\} \subset \mathbb{Q}
\end{equation}
It is clear that $G_2$ is an additive subgroup of $\mathbb{Q}$. We claim that $(\{\mu_n\}, G_2)$ is in fact the desired inductive limit. We verified the first required property of $\mu_n$ above. To verify the second, we proceed similarly to Part 1.
Given $B$ and $\lambda_n$ such that $\lambda_{n + 1} \circ \mu_n = \lambda_n$, we can simply define $\lambda : G_2 \rightarrow B$ as
$\lambda(k) = \lambda_n(2^{n - 1} k)$. Obviously, $\lambda \circ \mu_n = \lambda_n$. In addition, $\lambda$ does not depend on $n$ because we have
\begin{equation}
  \lambda_n(2^{n - 1} k) = (\lambda_{m} \circ \varphi_{m - 1, n})(2^{n - 1} k) = \lambda_m(2^{m - 1} k)
\end{equation}
for some $m > n$. To verify uniqueness, note that if $\lambda'$ also satisfies the desired property, then
\begin{align}
  \lambda' \left( \frac{p}{2^n} \right) = (\lambda' \circ \mu_{n + 1}) \left( p \right) &= \lambda_{n + 1} \left( p \right) = (\lambda \circ \mu_{n + 1})(p) = \lambda \left( \frac{p}{2^n} \right)
\end{align}
Thus, by definition, $(\{\mu_n\}, G_2)$ is in fact the inductive limit of this sequence.

\hhrulefill

\section{Topological Inverse Limit and Inductive Limit Duality (Suggested Problem 5)}

\noindent \pop{Note that the definition of an inverse limit provided in the problem set is incorrect. I will complete the problem assuming the correct definition.}
\newline

\noindent \textbf{Part 1.} Let us take
$$X = \left\{ (x_1, x_2, \dots) \ \Big| (x_1, x_2, \dots) \in \prod_{n \in \mathbb{N}} X_n, \ f_i(x_{i + 1}) = x_i \ \forall i \in \mathbb{N} \right\}$$
where we give $\prod_{n \in \mathbb{N}} X_n$ the usual product topology, and $X$ inherits the subspace topology. We simply take $\pi_i : X \rightarrow X_i$
to be the projection onto the $i$-th coordinate. Automatically, it is clear that $(f_{i - 1} \circ \pi_i)(x_1, x_2, \dots) = f_{i - 1}(x_{i}) = x_{i - 1} = \pi_{i - 1}(x_1, x_2, \dots)$.

In addition, suppose continuous functions $\lambda_i : Y \rightarrow X_i$ satisfy $f_{i - 1} \circ \lambda_i = \lambda_{i - 1}$. We wish to define continuous $\lambda : B \rightarrow A$ such that
$\lambda_i = \pi_i \circ \lambda$. For a particular $y \in Y$, we define $\lambda(y) = (\lambda_1(y), \lambda_2(y), \dots)$. Note that $\lambda(y) \in X$ because we have assumed that
$$f_i(\lambda(y)_{i + 1}) = f_i(\lambda_{i + 1}(y)) = \lambda_i(y) = \lambda(y)_i$$
Clearly, $\lambda$ is a continuous function, as it is a product of continuous functions, and clearly $\lambda_i = \pi_i \circ \lambda$. To show uniqueness of $\lambda$, note that if $\lambda'$ is also a continuous function with
the same properties, we have
\begin{equation}
  \lambda'(y)_i = (\pi_i \circ \lambda')(y) = \lambda_i(y) = \lambda(y)_i
\end{equation}
for each $i$, so $\lambda'(y) = \lambda(y)$. Thus, we have shown that inverse limits exist. To show that an inverse limit is unique, suppose $(\{\pi_i'\}, X')$ is also an inverse limit. It follows from the
definition of the inverse limit that the following diagram will commute:
% https://q.uiver.app/#q=WzAsNCxbMSwwLCJYIl0sWzAsMCwiWCciXSxbMiwwLCJYJyJdLFsxLDEsIlhfaSJdLFsxLDAsIlxcUGlfMSJdLFswLDIsIlxcUGlfMiJdLFswLDMsIlxccGlfaSJdLFsxLDMsIlxccGlfaSciLDJdLFsyLDMsIlxccGlfaSciXV0=
\[\begin{tikzcd}
	          {X'} & X & {X'} \\
	          & {X_i}
	          \arrow["{\Pi_1}", from=1-1, to=1-2]
	          \arrow["{\Pi_2}", from=1-2, to=1-3]
	          \arrow["{\pi_i}", from=1-2, to=2-2]
	          \arrow["{\pi_i'}"', from=1-1, to=2-2]
	          \arrow["{\pi_i'}", from=1-3, to=2-2]
\end{tikzcd}\]
where $\Pi_1 : X' \rightarrow X$ and $\Pi_2 : X \rightarrow X'$ are continuous. Thus, $X'$ and $X$ are homeomorphic, so up to homeomorphism, the inverse limit is unique.
\newline

\noindent \textbf{Part 2.} Since each $X_j$ is Hausdorff, it follows that $\prod_{n \in \mathbb{N}} X_n$ is Hausdorff (given $x \neq y$, $x$ and $y$ must differ at coordinate $j$, so choose disjoint open set $U$ and $V$ in $X_j$ of $x_j$ and $y_j$, then
note that $x \in X_1 \times \dots U \times X_{j + 1} \times \cdots$ and $y \in X_1 \times \dots V \times X_{j + 1} \times \cdots$ are disjoint). Thus, $X$, and subspace of $\prod_{n \in \mathbb{N}} X_n$ is also Hausdorff.

Tychonoff's theorem tells us that arbitrary products of compact spaces are compact, so $\prod_{n \in \mathbb{N}} X_n$ is compact. We will show that $X$ is closed in the product space. Indeed, by definition, note that:
\begin{equation}
  X = \displaystyle\bigcap_{n \in \mathbb{N}} \left\{ (x_1, x_2, \dots) \ \Big| \ f_n(x_{n + 1}) = x_n \right\} = \displaystyle\bigcap_{n \in \mathbb{N}} X_1 \times \cdots \times X_{n - 1} \times \Gamma_{f_n}(X_{n + 1}) \times X_{n + 2} \times \cdots = \displaystyle\bigcap_{n \in \mathbb{N}} C_n
\end{equation}
where $\Gamma_{f_n}(X_{n + 1}) = \{ (f_n(x), x) \ | \ x \in X_{n + 1} \}$ is the graph of $f_n$. If we can show that each $C_n$ is closed, then $X$ will also be closed, as arbitrary intersections of closed sets are closed. From here, we can conclude that $X$ is compact,
as closed subspaces of compact spaces are compact.
\newline

\noindent In fact, it is easy to show that the graph is closed. Given a point $(y, x) \in X_n \times X_{n + 1}$ in the graph's complement, we simply take open set $U$ in $X_n$ containing $y$ and open set $W$ containing $f(x)$, such that $U$ and $W$ are disjoint (we
can make this choice as $X_n$ is Hausdorff). Note that the open set $U \times f^{-1}(W)$ contains $(y, x)$, and does not contain any point of the graph, as any $x \in f^{-1}(W)$ is mapped by $f$ to $W$ disjoint from $U$. Thus, the complement of the graph is open, so the graph is closed.
\newline

\noindent \textbf{Part 3.} This result follows from carefully making use of the contravariant nature of $C$ (I guess this could be considered a functor? I'm not completely sure.) Also note that all the spaces we
are working with are compact, so in this context $C = C_0$. To begin, let
$$X = \lim_{\longleftarrow} X_i.$$
By definition,
the following diagram commutes:
% https://q.uiver.app/#q=WzAsMyxbMCwwLCJYIl0sWzIsMCwiWF9pIl0sWzEsMSwiWF97aS0xfSJdLFswLDIsIlxccGlfe2ktMX0iLDJdLFsxLDIsImZfe2ktMX0iXSxbMCwxLCJcXHBpX2kiXV0=
\[\begin{tikzcd}
X && {X_i} \\
& {X_{i-1}}
\arrow["{\pi_{i-1}}"', from=1-1, to=2-2]
\arrow["{f_{i-1}}", from=1-3, to=2-2]
\arrow["{\pi_i}", from=1-1, to=1-3]
\end{tikzcd}\]
which immediately implies that this diagram commutes:
% https://q.uiver.app/#q=WzAsNCxbMCwwLCJDKFgpIl0sWzIsMCwiQyhYX3tpKzF9KSJdLFsxLDEsIkMoWF97aX0pIl0sWzIsMV0sWzIsMCwiXFxwaV97aX1eKiJdLFsyLDEsImZfe2l9XioiLDJdLFsxLDAsIlxccGlfe2krMX1eKiIsMl1d
\[\begin{tikzcd}
	          {C(X)} && {C(X_{i+1})} \\
	          & {C(X_{i})} & {}
	          \arrow["{\pi_{i}^*}", from=2-2, to=1-1]
	          \arrow["{f_{i}^*}"', from=2-2, to=1-3]
	          \arrow["{\pi_{i+1}^*}"', from=1-3, to=1-1]
\end{tikzcd}\]
where we shift $i$ to be $1$. Thus, we have maps $\varphi_i = f_i^{*} : C(X_i) \rightarrow C(X_{i+1})$ and $\mu_i = \pi_i^{*}$ which satisfy the first condition required for $(C(X), \{\pi_i^{*}\})$ to be the inductive limit
of the sequence of $f_i^{*} : C(X_i) \rightarrow C(X_{i + 1})$. If we can verify that the second condition holds, it will follows immediately that $(C(X), \{\pi_i^{*}\})$ is the inductive limit, and by uniqueness, we then must have
$$C(X) = C(\lim_{\longleftarrow} X_i) \simeq \lim_{\longrightarrow} C(X_i).$$
\newline

\noindent Indeed, it is easy to show that the second condition holds. Suppose $(B, \{\lambda_i\})$ is another system where $\lambda_i : C(X_i) \rightarrow B$ and $\lambda_{i} = \lambda_{i + 1} \circ f_i^{*}$. By Gelfand's theorem,
$B$ is isomorphic to $C_0(Y)$ for some locally compact Hausdorff space $Y$. Thus, WLOG, we can assume $\lambda_i : C(X_i) \rightarrow C_0(Y)$. Moreover, again from Gelfand, we know that there exists some continuous $\gamma_i : Y \rightarrow X_i$
such that $\gamma_i^{*} = \lambda_i$, so that $\gamma_i^{*} = \gamma_{i + 1}^{*} \circ f_i^{*}$, which implies that $\gamma_i = f_i \circ \gamma_{i + 1}$

From the characterization of the inverse limit, we know that there is a unique continuous function $\gamma : Y \rightarrow X$ such that the following diagram commutes:
% https://q.uiver.app/#q=WzAsMyxbMSwwLCJYIl0sWzAsMCwiWSJdLFsxLDEsIlhfaSJdLFsxLDAsIlxcbGFtYmRhIl0sWzAsMiwiXFxwaV9pIl0sWzEsMiwiXFxsYW1iZGFfaSIsMl1d
\[\begin{tikzcd}
Y & X \\
& {X_i}
\arrow["\gamma", from=1-1, to=1-2]
\arrow["{\pi_i}", from=1-2, to=2-2]
\arrow["{\gamma_i}"', from=1-1, to=2-2]
\end{tikzcd}\]

which implies immediately that there is a unique $*$-homomorphism (again following from Gelfand's theorem) $\gamma^{*}: C(X) \rightarrow C_0(Y) \simeq B$ making the diagram commute:
% https://q.uiver.app/#q=WzAsMyxbMSwwLCJDKFgpIl0sWzAsMCwiQiJdLFsxLDEsIkMoWF9pKSJdLFswLDEsIlxcbGFtYmRhXnsqfSIsMl0sWzIsMCwiXFxwaV9pXnsqfSIsMl0sWzIsMSwiXFxsYW1iZGFfaSJdXQ==
\[\begin{tikzcd}
B & {C(X)} \\
& {C(X_i)}
\arrow["{\gamma^{*}}"', from=1-2, to=1-1]
\arrow["{\pi_i^{*}}"', from=2-2, to=1-2]
\arrow["{\lambda_i}", from=2-2, to=1-1]
\end{tikzcd}\]
where we use the fact that $\gamma_i^{*} = \lambda_i$. Thus, it follows from the definition that $C(X)$ is isomorphic to the inverse limit of the sequence of $C(X_i)$, and the proof is complete.
\newline

\noindent \textbf{Part 4.} This result follows from Part 3, as well as the fact that the $K_0$ functor ``commutes'' with inverse limits, and that if $A \simeq B$, then $K_0(A) \simeq K_0(B)$. In particular, note that
\begin{equation}
  K_0(C(\Sigma_2)) \simeq K_0(\lim_{\longrightarrow} (C(S^1), \varphi_n^{*}) ) \simeq \lim_{\longrightarrow} (K_0\left( C(S^1) \right), K_0(\varphi_n^{*})) \simeq \lim_{\longrightarrow}(\mathbb{Z}, K_0(\varphi_n^{*}))
\end{equation}
where the morphisms $\varphi_n$ are simply $\varphi_n(z) = z^2$ for all $n$. This means that $\varphi_n^{*}(f) = f \circ z^2$ for $f \in C(S^1)$. 
\end{document}
