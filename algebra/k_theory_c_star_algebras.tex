\documentclass[aps,pra,showpacs,notitlepage,onecolumn,superscriptaddress,nofootinbib]{revtex4-1}
\usepackage[utf8]{inputenc}
\usepackage[tmargin=1in, bmargin=1.25in, lmargin=1.5in, rmargin=1.5in]{geometry}
\usepackage{amsmath, amssymb, amsthm}
\usepackage{graphicx}
\usepackage{xcolor}
\usepackage{enumitem}
\usepackage{datetime}
\usepackage{hyperref}
\usepackage{titlesec}
\usepackage{import}
\usepackage{mathtools}
\usepackage{thmtools,thm-restate}


% package for commutative diagrams
% \usepackage{tikz-cd}

%%%%%%%%%%%%%%%%%%%%%%%%%%%%%%%%%%%%%%%%%%%%%
\definecolor{crimson}{RGB}{186,0,44}
\definecolor{moss}{RGB}{0, 186, 111}
\newcommand{\pop}[1]{\textcolor{crimson}{#1}}
\newcommand{\zcom}[1]{\noindent\textcolor{crimson}{(Z): #1}}
\newcommand{\jcom}[1]{\noindent\textcolor{moss}{(J): #1}}
\newcommand{\wt}[1]{\widetilde{#1}}
\newcommand{\pqeq}{\succcurlyeq}
\newcommand{\pleq}{\preccurlyeq}

%%%%%%%%%%%%%%%%%%%%%%%%%%%%%%%%%%%%%%%%%%%%%
\hypersetup{
    colorlinks,
    linkcolor={crimson},
    citecolor={crimson},
    urlcolor={crimson}
}

\usepackage{qcircuit}

%%%%%%%%%%%%%%%%%%%%%%%%%%%%%%%%%%%%%%%%%%%%%
\theoremstyle{definition}
\newtheorem{definition}{Definition}[section]
\newtheorem{lemma}{Lemma}[section]
\newtheorem{theorem}{Theorem}[section]
\newtheorem{corollary}{Corollary}[theorem]
\newtheorem*{theorem*}{Theorem}
\newtheorem*{corollary*}{Corollary}
\newtheorem{remark}{Remark}[section]
\newtheorem{conjecture}{Conjecture}[section]
\newtheorem{example}{Example}[section]
\newtheorem{reminder}{Reminder}[section]
\newtheorem{problem}{Problem}[section]
\newtheorem{question}{Question}[section]
\newtheorem{answer}{Answer}[section]
\newtheorem{fact}{Fact}[section]
\newtheorem{claim}{Claim}[section]

\usepackage{geometry}
\geometry{
  left=25mm,
  right=25mm,
  top=20mm,
}

%%%%%%%%%%%%%%%%%%%%%%%%%%%%%%%%%%%%%%%%%%%%%
\bibliographystyle{unsrt}

%%%%%%%%%%%%%%%%%%%%%%%%%%%%%%%%%%%%%%%%%%%%%
%%%%%%%%%%%%%%%%%%%%%%%%%%%%%%%%%%%%%%%%%%%%%
%%%%%%%%%%%%%%%%%%%%%%%%%%%%%%%%%%%%%%%%%%%%%
\begin{document}

\title{Introduction to K-theory and C$^{*}$-algebras}
\author{Jack Ceroni}

\date{\today}

\maketitle

\section{Basics of C$^{*}$ algebra theory}

\noindent We will begin with a recap of some basic definitions.

\begin{definition}[Algebra]
  A ring $A$ is said to be a $R$-algebra, for commutative ring $R$, if $A$ is an $R$-module when $A$ is treated as an Abelian group, and the multiplication in $A$
  is compatible and commutes with the action of $R$ on $A$. That is, for $a, b \in A$ and $r \in R$, $r \cdot (ab) = (r \cdot a)b = a(r \cdot b)$.
\end{definition}

The first interesting result that we present is
due to Gelfand and Naimark. Recall that a Hilbert space $H$ is an inner product space which is complete with respect to the topology induced by the metric induced by the inner product.

\begin{theorem}[Gelfand-Naimark]
  For each $C^{*}$-algebra, there exists a Hilbert space $H$ and an isometric $*$-homomorphism from $A$ into $B(H)$ (the space of bounded linear operators on $H$, which itself
  has a canonical involution (the adjoint) and a canonical norm ($||v|| = \sqrt{\langle v, v \rangle}$). If $A$ is topologically separable, then $H$ can be chosen to be separable as well. 
\end{theorem}

\section{A collection of exercises: RLL chapter 1}

\noindent We now present solution to some relevant exercises, which make use of much of the theory developed in previous section of these notes.



\end{document}
